\documentclass[12pt]{amsart}
\usepackage[T1]{fontenc}
\usepackage[utf8]{inputenc}

\usepackage[top=1.95cm, bottom=1.95cm, left=2.35cm, right=2.35cm]{geometry}

\usepackage{amsmath}
\usepackage[french]{babel}
\usepackage{lymath}

\DeclareMathOperator{\taille}{\tau}

\newtheorem{fact}{Fait}
\newtheorem*{proof*}{Preuve}

\setlength\parindent{0pt}


\begin{document}

\title{À propos de l'exercice de SPÉ MATHS du BAC S de Juin 2018}
\author{Christophe BAL}
\date{25 Juin 2018}
\maketitle

Dans le BAC S de Juin 2018, la partie A de l'exercice de SPÉ MATHS s'intéressait à l'équation diophantienne \textbf{[ED]} : $x^2 - 8 y^2 = 1$ sur $\NN^2$. 


\medskip

On a une solution évidente $(x \,; y) = (3 \,; 1)$. L'exercice introduit alors une matrice "magique"
$A =
\begin{pmatrix} 
  3 & 8  \\ 
  1 & 3 
\end{pmatrix}$
pour ensuite construite des solutions $(x_n \,; y_n)$ de façon récursive linéairement comme suit :
$\begin{pmatrix} 
  x_{n+1} \\ 
  y_{n+1} 
\end{pmatrix}
=
A
\begin{pmatrix} 
  x_{n} \\ 
  y_{n} 
\end{pmatrix}
$.


\medskip

Très bien mais voyons comment découvrir la matrice "magique" $A$. Une idée élémentaire est de noter que
$x^2 - 8 y^2
=
\begin{pmatrix} 
  x & y 
\end{pmatrix}
Q
\begin{pmatrix} 
  x \\ 
  y 
\end{pmatrix}$
en posant
$Q
=
\begin{pmatrix} 
  1 & 0  \\ 
  0 & -8 
\end{pmatrix}$.


\medskip

En notant 
$\begin{pmatrix} 
  X \\ 
  Y 
\end{pmatrix}
=
A
\begin{pmatrix} 
  x \\ 
  y 
\end{pmatrix}$,
nous avons :
$\begin{pmatrix} 
  X & Y 
\end{pmatrix}
Q
\begin{pmatrix} 
  X \\ 
  Y 
\end{pmatrix}
=
\begin{pmatrix} 
  x & y 
\end{pmatrix}
A^T Q A
\begin{pmatrix} 
  x \\ 
  y 
\end{pmatrix}$.


\medskip

Essayons de trouver $A$ vérifiant $A^T Q A = Q$ car d'une solution
$\begin{pmatrix} 
  x \\ 
  y 
\end{pmatrix}$
on pourra passer à une "autre"
$\begin{pmatrix} 
  X \\ 
  Y 
\end{pmatrix}
=
A \begin{pmatrix} 
  x \\ 
  y 
\end{pmatrix}$.


\medskip

Utilisant le déterminant, nous avons comme contrainte $\det A = \pm 1$ donc $A$ doit être inversible. Imposons $\det A = 1$.
Notant
$A = \begin{pmatrix} 
  a & b \\ 
  c & d
\end{pmatrix}$,
nous avons alors
$A^{-1} = \begin{pmatrix} 
  d  & -b \\ 
  -c & a
\end{pmatrix}$
d'où :
\begin{flalign*}
	A^T Q A = Q & \Longleftrightarrow  A^T Q = Q A^{-1} & \\
	            & \Longleftrightarrow 
	\begin{pmatrix} 
	  a & c \\ 
	  b & d
	\end{pmatrix}
	\begin{pmatrix} 
	  1 & 0  \\ 
	  0 & -8 
	\end{pmatrix}
	=
	\begin{pmatrix} 
	  1 & 0  \\ 
	  0 & -8 
	\end{pmatrix}
	\begin{pmatrix} 
	  d  & -b \\ 
	  -c & a
	\end{pmatrix}
	                                                    & \\
	            & \Longleftrightarrow 
	\begin{pmatrix} 
	  a & -8 c \\ 
	  b & -8 d
	\end{pmatrix}
	=
	\begin{pmatrix} 
	  d  & -b \\ 
	  8c & -8a
	\end{pmatrix}
	                                                    & \\
	            & \Longleftrightarrow a = d \text{ et } b = 8c 
\end{flalign*}


\medskip

$\det A = 1$ pour
$A
=
\begin{pmatrix} 
  a & b \\ 
  c & d
\end{pmatrix}
=
\begin{pmatrix} 
  d & 8c \\ 
  c & d
\end{pmatrix}$
nous donne
$d^2 - 8c^2 = 1$. Que c'est joli !


\medskip

La matrice du sujet de BAC utilise donc la solution élémentaire
$(c \,; d) = (1 \,; 3)$.
\end{document}
