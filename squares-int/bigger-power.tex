Pour finir, nous allons analyser ce qu'il se passe si l'on somme à la puissance $p \geqslant 3$ au lieu d'élever au carré.
Nous reprenons des notations similaires à celles de la section \ref{proof}.
\begin{itemize}[label = \textbullet]
	\item Pour un naturel $n =  \left[ \, c_{d-1} c_{d-2} \cdots c_1 c_0 \, \right]_{10}$ avec $c_{d-1} \neq 0$,
	on pose
	$\displaystyle pw(n) = \sum_{k=0}^{d-1} (c_k)^{\,p}$
	et
	$\taille(n) = d$.
	
	\item Pour $(n \,; k) \in \NN^2$, on définit 
	$\sqseq{n}{0} = n$
	et
	$\sqseq{n}{k+1} = pw \left( \, \sqseq{n}{k} \right)$.
\end{itemize}

 

\bigskip

\begin{fact}
	$\forall n \in \NN$, $pw(n) \leqslant 9^{\,p} \, d$ où $d = \taille(n)$.
\end{fact}

\begin{proof*}
	Si $n = \left[ \, c_{d-1} c_{d-2} \cdots c_1 c_0 \, \right]_{10}$
	alors 
	$\displaystyle pw(n) = \sum_{k=0}^{d-1} (c_k)^{\,p} \leqslant \sum_{k=0}^{d-1} 9^{\,p} = 9^{\,p} \, d $.
\end{proof*}




\medskip

\begin{fact}\label{magicmajo}
	Il existe $d_0 \in \NN$ tel que $\forall n \in \NN$,	
	$[ \, \taille(n) \geqslant d_0 \Rightarrow pw(n) < n \, ]$.
\end{fact}

\begin{proof*}\label{magicmajo-proof}
	Notons $d = \taille(n)$ de sorte que $n \geqslant 10^{d-1}$.
	Compte tenu du fait précédent, nous cherchons à comparer $10^{d-1}$ et $9^{\,p} \, d$.
	
	
	\medskip
	
	Nous allons procéder de façon analogue au cas $p = 2$ démontré dans la section \ref{proof} mais en étant ici plus rigoureux dans la rédaction.
	
	
	\medskip
	
	Il existe $d\,^\prime \in \NN$ tel que $10^{d\,^\prime  - 1} \geqslant 9^{\,p  - 1}$.
	Il suffit de choisir $d\,^\prime = 2 + \floor{ \log \left( 9^{\,p-1} \right) }$ où $\log$ désigne le logarithme décimal et $\floor{x}$ la partie entière de $x$.
	Ensuite la croissance comparée des fonctions $10^{x - 1}$ et $9^{\,p} x$ donne $d_0 \in \ZintervalCO{d\,^\prime}{+\infty}$ tel que $10^{d_0 - 1} > 9^{\,p} d_0$ .



	\medskip
	
	Montrons par récurrence sur $d \geqslant d_0$ que $10^{d - 1} > 9^{\,p} d$ .
	Ceci donnera $n \geqslant 10^{d - 1} > 9^{\,p} d \geqslant pow(n)$ d'où $n > pow(n)$ dès que $d \geqslant d_0$ comme souhaité.

	\begin{itemize}[label=\small\textbullet]
		\item \emph{Initialisation.}
		Par choix de $d_0$ , nous avons $10^{d-1} > 9^{\,p} d$ si $d = d_0$.

		\item \emph{Hérédité.}
		Faisons l'hypothèse que $10^{d-1} > 9^{\,p} d$ est vérifiée pour un naturel $d \geqslant d_0$ \emph{\og fixé quelconque \fg}.

		\smallskip
		
		\noindent
		Nous avons : $10^{(d+1)-1} = 10\times10^{d-1} = 10^{d-1} + 9\times10^{d-1} \geqslant 10^{d-1} + 9^{\,p}$ car $10^{d-1} \geqslant 9^{\,p  - 1}$ puisque $d \geqslant d_0 \geqslant d\,^\prime$.

		\smallskip
		
		\noindent
		Comme $10^{d-1} > 9^{\,p} d$, nous avons : $10^{(d+1)-1} \geqslant 10^{d-1} + 9^{\,p} > 9^{\,p} d + 9^{\,p} = 9^{\,p} (d+ 1)$ .
		L'inégalité est donc vérifiée au rang suivant $(d+1)$. 

		\item \emph{Conclusion.}
		Par récurrence sur $d \geqslant d_0$ , nous avons $10^{d - 1} > 9^{\,p} d$ pour tout naturel $d$ tel que $d \geqslant d_0$ .
	\end{itemize}
\end{proof*}
	

\medskip

\begin{remark}
	Informatiquement une valeur de $d_0$ peut s'obtenir en testant $10^{d - 1} > 9^{\,p} d$ successivement pour les naturels $d$ tels que $d \geqslant d^{\,\prime}$ où $d^{\,\prime} = 2 + \floor{ \log \left( 9^{\,p-1} \right)}$.
	
	\medskip
	
	Pour gagner du temps, on peut tester les valeurs successives de $2^k d^{\,\prime}$ pour $k = 0, 1, 2, \dots$ pour obtenir $D$ tel que $10^{D - 1} > 9^{\,p} D$ . Si la valeur de $D$ est trop grande, on peut chercher la valeur minimale de $d$ tel que $10^{d - 1} > 9^{\,p} d$ en utilisant une recherche de type dichotomique. 
\end{remark}


\medskip

\begin{fact}\label{beautifulproof}
	$\forall n \in \NN$, la suite $\left( \, \sqseq{n}{k} \right)_{k \in \NN}$ est ultimement périodique.
\end{fact}

\begin{proof*}
	Tout est en fait contenu dans le fait \ref{magicmajo}, dont on reprend la signification de $d_0$. Expliquons pourquoi.
	\begin{itemize}[label = \textbullet]
		\item Le fait \ref{magicmajo} donne l'existence d'un indice $k_0 \in \NN$ tel que $\taille\left( \, \sqseq{n}{k_0} \right) < d_0$ \emph{(dans le cas contraire, on pourrait construire une suite strictement décroissante de naturels)}.

		\item Si pour tout naturel $k \in \ZintervalCO{k_0}{+\infty}$ , $\taille\left( \, \sqseq{n}{k} \right) < d_0$ , nous avons l'ultime périodicité via le principe des tiroirs \emph{(si besoin revoir la fin de la section \ref{proof})}.

		\item Sinon il existe $k\,^\prime_0 \in \ZintervalO{k_0}{+\infty}$ tel que $\taille\left( \, \sqseq{n}{k\,^\prime_0} \right) \geqslant d_0$. Comme dans le premier point, nous pouvons alors trouver $k_1 \in \ZintervalO{k\,^\prime_0}{+\infty}$ tel que $\taille\left( \, \sqseq{n}{k_1} \right) < d_0$.
		
		\item En répétant notre raisonnement,
		on peut aboutir à une situation similaire au 2\ieme{} point, et c'est gagné. 
		
		\noindent
		Sinon on arrive à construire une suitre strictement croissante $\left( k_i \right)_i$ d'indices tels que $\forall i \in \NN$, $\taille\left( \, \sqseq{n}{k_i} \right) < d_0$. Le principe des tiroirs s'applique ici aussi !
	\end{itemize}
\end{proof*}



\medskip

\begin{remark}
	La preuve précédente montre que pour rechercher toutes les périodes il \emph{\og suffit \fg} d'étudier les naturels appartenant à $\ZintervalCO{0}{10^{d_0}}$ .
\end{remark}
