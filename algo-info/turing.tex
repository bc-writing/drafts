% !TEX encoding = UTF-8 Unicode
\documentclass[a4paper, 12pt]{scrartcl}

\usepackage[utf8]{inputenc}
\usepackage[T1]{fontenc}

\usepackage[top=2.5cm, bottom=2.5cm, left=1.95cm, right=1.95cm]{geometry}

\usepackage[french]{babel}
\usepackage{listings}

\usepackage[fr, apmep]{lyxam}
\usepackage{lymath}
\usepackage{fancyvrb}
\usepackage{bera}
\usepackage{enumitem}
\usepackage{multicol}

\usepackage{lastpage}
\usepackage{xcolor}
\usepackage{graphicx}
\usepackage{setspace}

\usepackage{tikz}
\usetikzlibrary{arrows,positioning, calc} 


\usepackage[
    type={CC},
    modifier={by-nc-sa},
	version={4.0},
]{doclicense}


\newcommand\boxit[1]{\fbox{\makebox[.85em]{#1}\vphantom{$pX^M$}}}
\newcommand\fboxit[1]{\fcolorbox{black}{yellow}{\makebox[.85em]{#1}\vphantom{$pX^M$}}}
\newcommand\nboxit[1]{\fcolorbox{black}{lightgray}{\makebox[.85em]{#1}\vphantom{$pX^M$}}}
\newcommand\wboxit[1]{\fcolorbox{black}{red}{\makebox[.85em]{#1}\vphantom{$pX^M$}}}
\newcommand\noboxit[1]{\fcolorbox{white}{white}{\makebox[.85em]{#1}\vphantom{$pX^M$}}}

\newcommand\emptybox{\boxit{\phantom{A}}}

\newcommand\head{\noboxit{$\uparrow$}}


\begin{document}

\title{BROUILLON - Quelques machines de Turing}
\author{Christophe BAL}
\date{7 Février 2020}

\maketitle

\begin{center}
	\itshape
	Document, avec son source \LaTeX, disponible sur la page
	
	\url{https://github.com/bc-writing/drafts}.
\end{center}


\bigskip


\begin{center}
	\hrule\vspace{.3em}
	{
		\fontsize{1.35em}{1em}\selectfont
		\textbf{Mentions \og légales \fg}
	}
			
	\vspace{0.45em}
	\doclicenseThis
	\hrule
\end{center}


\bigskip
\setcounter{tocdepth}{2}
\tableofcontents



\newpage
\section{Détecter les palindromes}

	\subsection{Un exemple avec le mot abbca}

		Voici les grandes étapes présentées sur deux colonnes avec une 1\iere{} phase de duplication du mot puis une 2\ieme{} testant la présence ou non d'un palindrome.


\begin{multicols}{2}

% GESTION DE LA LETTRE LA PLUS À DROITE

\emptybox\emptybox%
	\boxit{a}\boxit{b}\boxit{b}\boxit{c}\boxit{a}%
\emptybox\emptybox

\phantom{\emptybox\emptybox}%
	\head


\medskip % RECHERCHE DE LA DERNIÈRE LETTRE
\emptybox\emptybox%
	\fboxit{a}\boxit{b}\boxit{b}\boxit{c}\boxit{a}%
\emptybox\emptybox

\phantom{\emptybox\emptybox%
	\emptybox\emptybox\emptybox\emptybox}%
	\head


\medskip % BONNE LETTRE
\emptybox\emptybox%
	\fboxit{a}\boxit{b}\boxit{b}\boxit{c}\fboxit{a}%
\emptybox\emptybox

\phantom{\emptybox\emptybox%
	\emptybox\emptybox\emptybox\emptybox}%
	\head


\medskip % EFFACEMENT DES LETTRES
\emptybox\emptybox%
	\emptybox\boxit{b}\boxit{b}\boxit{c}\emptybox%
\emptybox\emptybox

\phantom{\emptybox\emptybox%
	\emptybox}%
	\head

\vfill\null
\columnbreak

\medskip % GESTION DE LA NOUVELLE LETTRE LA PLUS À DROITE
\emptybox\emptybox%
	\emptybox\fboxit{b}\boxit{b}\boxit{c}\emptybox%
\emptybox\emptybox

\phantom{\emptybox\emptybox%
	\emptybox}%
	\head


\medskip % RECHERCHE DE LA DERNIÈRE LETTRE
\emptybox\emptybox%
	\emptybox\fboxit{b}\boxit{b}\boxit{c}\emptybox%
\emptybox\emptybox

\phantom{\emptybox\emptybox%
	\emptybox\emptybox\emptybox}%
	\head


\medskip % MAUVAISE LETTRE
\emptybox\emptybox%
	\emptybox\fboxit{b}\boxit{b}\wboxit{c}\emptybox%
\emptybox\emptybox

\phantom{\emptybox\emptybox%
	\emptybox\emptybox\emptybox}%
	\head

\end{multicols}




	\subsection{Un exemple avec le mot abbcbba}

		On fait comme précédemment en devant tout parcourir !
Voici les grandes étapes.

\begin{multicols}{2}

% ????

\phantom{\emptybox\emptybox}%
	\deah

\emptybox\emptybox%
	\boxit{a}\boxit{b}\boxit{b}\boxit{c}\boxit{b}\boxit{b}\boxit{a}%
\emptybox\emptybox

\emptybox\emptybox%
	\emptybox\emptybox\emptybox\emptybox\emptybox\emptybox\emptybox%
\emptybox\emptybox

\phantom{\emptybox\emptybox}%
	\head


\medskip % ????

\phantom{\emptybox\emptybox\emptybox\emptybox\emptybox\emptybox\emptybox\emptybox\emptybox}%
	\deah

\emptybox\emptybox%
	\boxit{a}\boxit{b}\boxit{b}\boxit{c}\boxit{b}\boxit{b}\boxit{a}%
\emptybox\emptybox

\emptybox\emptybox%
	\nboxit{a}\nboxit{b}\nboxit{b}\nboxit{c}\nboxit{b}\nboxit{b}\nboxit{a}%
\emptybox\emptybox

\phantom{\emptybox\emptybox\emptybox\emptybox\emptybox\emptybox\emptybox\emptybox\emptybox}%
	\head


\vfill\null
\columnbreak % ????

\phantom{\emptybox}%
	\deah

\emptybox\emptybox%
	\boxit{a}\boxit{b}\boxit{b}\boxit{c}\boxit{b}\boxit{b}\boxit{a}%
\emptybox\emptybox

\emptybox\emptybox%
	\nboxit{a}\nboxit{b}\nboxit{b}\nboxit{c}\nboxit{b}\nboxit{b}\nboxit{a}%
\emptybox\emptybox

\phantom{\emptybox\emptybox\emptybox\emptybox\emptybox\emptybox\emptybox\emptybox\emptybox}%
	\head


\medskip % ????

\phantom{\emptybox\emptybox\emptybox\emptybox\emptybox\emptybox\emptybox\emptybox\emptybox}%
	\deah

\emptybox\emptybox%
	\fboxit{a}\fboxit{b}\fboxit{b}\fboxit{c}\fboxit{b}\fboxit{b}\fboxit{a}%
\emptybox\emptybox

\emptybox\emptybox%
	\fboxit{a}\fboxit{b}\fboxit{b}\fboxit{c}\fboxit{b}\fboxit{b}\fboxit{a}%
\emptybox\emptybox

\phantom{\emptybox}%
	\head

\vfill\null
\end{multicols}



	\subsection{La table de transition} \label{duplicate-table}

		Une méthode simple et généraliste consiste à modifier les tables des transitions précédentes mécaniquement en respectant les deux règles suivantes.
\begin{enumerate}
	\item Toutes les cellules qui contiennent l'état final sont vidées.

	\item Toutes les cases vides, sauf celle correspondant à la case vide et l'état initial, sont remplies pour indiquer un passage à l'état final
		  \footnote{
			Ne pas oublier de traiter les états bloquants non indiqués sur la table initiale !
		  }.
	      \textbf{La cellule non traitée l'est afin d'éviter d'accepter le mot vide.} 
\end{enumerate}



Ceci nous donne la table des transitions ci-après pour les naturels impairs.
\begin{center}
	\begin{tabular}{|c||c|c|c|}
		\hline
		$\delta$ 
			& $0$ 
			& $1$
			& $B$ \\
		\hline
		\hline
		$q_0$
			& \transition{\ell_0}{0}{D}
			& \transition{\ell_1}{1}{D}
			&                           \\
		\hline
		\hline
		$\ell_0$
			& \transition{\ell_0}{0}{D}
			& \transition{\ell_1}{1}{D}
			&                           \\
		\hline
		$\ell_1$
			& \transition{\ell_0}{0}{D}
			& \transition{\ell_1}{1}{D}
			& \transition{f     }{B}{I} \\
		\hline
	\end{tabular}
\end{center}


Pour les non multiples de $3$, on obtient la table des transitions suivante.
\begin{center}
	\begin{tabular}{|c||c|c|c|}
		\hline
		$\delta$ 
			& $0$ 
			& $1$
			& $B$ \\
		\hline
		\hline
		$q_0$ 
			& \transition{d}{0}{D} 
			& \transition{d}{1}{D}
			&                      \\
		\hline
		$d$ 
			& \transition{d}{0}{D} 
			& \transition{d}{1}{D}
			& \transition{sp_0}{B}{G} \\
		\hline
		\hline
		$sp_0$ 
			& \transition{si_0}{0}{G} 
			& \transition{si_1}{1}{G}
			&                         \\
		\hline
		$sp_1$ 
			& \transition{si_1}{0}{G} 
			& \transition{si_2}{1}{G}
			& \transition{f   }{B}{I} \\
		\hline
		$sp_2$ 
			& \transition{si_2}{0}{G} 
			& \transition{si_0}{1}{G}
			& \transition{f   }{B}{I} \\
		\hline
		\hline
		$si_0$ 
			& \transition{sp_0}{0}{G} 
			& \transition{sp_2}{1}{G}
			&                         \\
		\hline
		$si_1$ 
			& \transition{sp_1}{0}{G} 
			& \transition{sp_0}{1}{G}
			& \transition{f   }{B}{I} \\
		\hline
		$si_2$ 
			& \transition{sp_2}{0}{G} 
			& \transition{sp_1}{1}{G}
			& \transition{f   }{B}{I} \\
		\hline
	\end{tabular}
\end{center}




	\subsection{Coder pour tester}

		Sur le site de téléchargement de ce document se trouvent
\verb+not-divisible-by-2.txt+ et \verb+not-divisible-by-3.txt+
dans le sous-dossier \verb+turing/not-divisible-by-2-or-3+
qui contiennent des codes utilisables sur le site \url{https://turingmachinesimulator.com}.



\end{document}
