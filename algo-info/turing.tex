 % !TEX encoding = UTF-8 Unicode
\documentclass[a4paper, 12pt]{scrartcl}

\usepackage[utf8]{inputenc}
\usepackage[T1]{fontenc}

\usepackage[top=2.5cm, bottom=2.5cm, left=1.95cm, right=1.95cm]{geometry}

\usepackage[french]{babel}
\usepackage{listings}

\usepackage[hidelinks]{hyperref}

\usepackage[fr, apmep]{lyxam}
\usepackage{lymath}
\usepackage{fancyvrb}
\usepackage{bera}
\usepackage{enumitem}
\usepackage{multicol}

\usepackage{lastpage}
\usepackage{xcolor}
\usepackage{graphicx}
\usepackage{setspace}

\usepackage{tikz}
\usetikzlibrary{arrows,positioning, calc} 

\usepackage{fancyvrb}


\usepackage[
    type={CC},
    modifier={by-nc-sa},
	version={4.0},
]{doclicense}


\newcommand\boxit[1]{\fbox{\makebox[.85em]{#1}\vphantom{$pX^M$}}}
\newcommand\fboxit[1]{\fcolorbox{black}{yellow}{\makebox[.85em]{#1}\vphantom{$pX^M$}}}
\newcommand\nboxit[1]{\fcolorbox{black}{lightgray}{\makebox[.85em]{#1}\vphantom{$pX^M$}}}
\newcommand\wboxit[1]{\fcolorbox{black}{red!50}{\makebox[.85em]{#1}\vphantom{$pX^M$}}}
\newcommand\noboxit[1]{\fcolorbox{white}{white}{\makebox[.85em]{#1}\vphantom{$pX^M$}}}

\newcommand\emptybox{\boxit{\phantom{A}}}
\newcommand\nemptybox{\nboxit{\phantom{A}}}
\newcommand\wemptybox{\wboxit{\phantom{A}}}

\newcommand\head{\noboxit{$\uparrow$}}
\newcommand\deah{\noboxit{$\downarrow$}}


\newcommand\boxedU{\boxit{\bfseries ?}}
\newcommand\boxedB{\boxit{\bfseries K}}
\newcommand\boxedW{\boxit{\bfseries W}}

\newcommand\fboxedB{\fboxit{\bfseries K}}
\newcommand\fboxedW{\fboxit{\bfseries W}}

\newcommand\nboxedB{\nboxit{\bfseries K}}
\newcommand\nboxedW{\nboxit{\bfseries W}}

\newcommand\wboxedB{\wboxit{\bfseries K}}
\newcommand\wboxedW{\wboxit{\bfseries W}}



\newcommand\twocoord[2]{{\scriptsize\begin{matrix}#1\\#2\end{matrix}}}
\newcommand\threecoord[3]{{\scriptsize\begin{matrix}#1\\#2\\#3\end{matrix}}}


\newcommand\transition[3]{%
	$\left(\, #1 \,, #2 \,, #3 \,\right)^{\vphantom{4}}$
}

\newcommand\myquote[1]{\emph{\og #1 \fg}}


\newcommand\boolope[1]{\text{\bfseries#1}}


\begin{document}

\title{BROUILLON - Quelques machines de Turing déterministes}
\author{Christophe BAL}
\date{7 Février -- 20 Mars 2020}

\maketitle

\begin{center}
	\itshape
	Document, avec son source \LaTeX, disponible sur la page
	
	\url{https://github.com/bc-writing/drafts}.
\end{center}


\bigskip


\begin{center}
	\hrule\vspace{.3em}
	{
		\fontsize{1.35em}{1em}\selectfont
		\textbf{Mentions \og légales \fg}
	}
			
	\vspace{0.45em}
	\doclicenseThis
	\hrule
\end{center}


\bigskip
\setcounter{tocdepth}{2}
\tableofcontents



\newpage
\section{Écriture binaire des naturels pairs}

	\subsection{La méthode}

	Commençons par un exemple simple qui va nous permettre de fixer les notations que nous allons utiliser.
Voyons comment repérer un entier naturel pair à partir de son écriture binaire.
La réponse est évidente : un naturel est pair si et seulement si son écriture binaire se finit par un zéro.
Il suffit donc de parcourir cette écriture binaire et d'analyser le chiffre le plus à droite. Ceci se schématise comme suit où \head{} indique la tête de lecture.

\begin{multicols}{2}

%

\emptybox\emptybox%
	\boxit{1}\boxit{0}\boxit{0}\boxit{1}\boxit{1}%
\emptybox\emptybox

\phantom{%
	\emptybox\emptybox}%
	\head


\medskip %

\emptybox\emptybox%
	\boxit{1}\boxit{0}\boxit{0}\boxit{1}\boxit{1}%
\emptybox\emptybox

\phantom{%
	\emptybox\emptybox
	\emptybox}%
	\head


\medskip %

\emptybox\emptybox%
	\boxit{1}\boxit{0}\boxit{0}\boxit{1}\boxit{1}%
\emptybox\emptybox

\phantom{%
	\emptybox\emptybox
	\emptybox\emptybox}%
	\head


\vfill\null
\columnbreak

\medskip %

\emptybox\emptybox%
	\boxit{1}\boxit{0}\boxit{0}\boxit{1}\boxit{1}%
\emptybox\emptybox

\phantom{%
	\emptybox\emptybox
	\emptybox\emptybox\emptybox}%
	\head


\medskip %

\emptybox\emptybox%
	\boxit{1}\boxit{0}\boxit{0}\boxit{1}\boxit{1}%
\emptybox\emptybox

\phantom{%
	\emptybox\emptybox
	\emptybox\emptybox\emptybox\emptybox}%
	\head


\medskip %

\emptybox\emptybox%
	\boxit{1}\boxit{0}\boxit{0}\boxit{1}\boxit{1}%
\emptybox\emptybox

\phantom{%
	\emptybox\emptybox
	\emptybox\emptybox\emptybox\emptybox\emptybox}%
	\head

\vfill\null
\end{multicols}

\vspace{-1em}

Pourquoi s'arrêter à la première case vide ? L'idée va être de garder en mémoire la valeur de la dernière case visitée.
Dans notre exemple, lors du dernier mouvement, on sait que la case précédente était un $1$ et donc que l'écriture binaire n'est pas celle d'un entier naturel pair.



\subsection{Une table des transitions}

	Une méthode simple et généraliste consiste à modifier les tables des transitions précédentes mécaniquement en respectant les deux règles suivantes.
\begin{enumerate}
	\item Toutes les cellules qui contiennent l'état final sont vidées.

	\item Toutes les cases vides, sauf celle correspondant à la case vide et l'état initial, sont remplies pour indiquer un passage à l'état final
		  \footnote{
			Ne pas oublier de traiter les états bloquants non indiqués sur la table initiale !
		  }.
	      \textbf{La cellule non traitée l'est afin d'éviter d'accepter le mot vide.} 
\end{enumerate}



Ceci nous donne la table des transitions ci-après pour les naturels impairs.
\begin{center}
	\begin{tabular}{|c||c|c|c|}
		\hline
		$\delta$ 
			& $0$ 
			& $1$
			& $B$ \\
		\hline
		\hline
		$q_0$
			& \transition{\ell_0}{0}{D}
			& \transition{\ell_1}{1}{D}
			&                           \\
		\hline
		\hline
		$\ell_0$
			& \transition{\ell_0}{0}{D}
			& \transition{\ell_1}{1}{D}
			&                           \\
		\hline
		$\ell_1$
			& \transition{\ell_0}{0}{D}
			& \transition{\ell_1}{1}{D}
			& \transition{f     }{B}{I} \\
		\hline
	\end{tabular}
\end{center}


Pour les non multiples de $3$, on obtient la table des transitions suivante.
\begin{center}
	\begin{tabular}{|c||c|c|c|}
		\hline
		$\delta$ 
			& $0$ 
			& $1$
			& $B$ \\
		\hline
		\hline
		$q_0$ 
			& \transition{d}{0}{D} 
			& \transition{d}{1}{D}
			&                      \\
		\hline
		$d$ 
			& \transition{d}{0}{D} 
			& \transition{d}{1}{D}
			& \transition{sp_0}{B}{G} \\
		\hline
		\hline
		$sp_0$ 
			& \transition{si_0}{0}{G} 
			& \transition{si_1}{1}{G}
			&                         \\
		\hline
		$sp_1$ 
			& \transition{si_1}{0}{G} 
			& \transition{si_2}{1}{G}
			& \transition{f   }{B}{I} \\
		\hline
		$sp_2$ 
			& \transition{si_2}{0}{G} 
			& \transition{si_0}{1}{G}
			& \transition{f   }{B}{I} \\
		\hline
		\hline
		$si_0$ 
			& \transition{sp_0}{0}{G} 
			& \transition{sp_2}{1}{G}
			&                         \\
		\hline
		$si_1$ 
			& \transition{sp_1}{0}{G} 
			& \transition{sp_0}{1}{G}
			& \transition{f   }{B}{I} \\
		\hline
		$si_2$ 
			& \transition{sp_2}{0}{G} 
			& \transition{sp_1}{1}{G}
			& \transition{f   }{B}{I} \\
		\hline
	\end{tabular}
\end{center}




\subsection{Coder pour tester}

	Sur le site de téléchargement de ce document se trouvent
\verb+not-divisible-by-2.txt+ et \verb+not-divisible-by-3.txt+
dans le sous-dossier \verb+turing/not-divisible-by-2-or-3+
qui contiennent des codes utilisables sur le site \url{https://turingmachinesimulator.com}.



\newpage
\section{Écriture binaire des multiples de 3} \label{divisibility-by-3}

	\subsection{La méthode}

	Commençons par un exemple simple qui va nous permettre de fixer les notations que nous allons utiliser.
Voyons comment repérer un entier naturel pair à partir de son écriture binaire.
La réponse est évidente : un naturel est pair si et seulement si son écriture binaire se finit par un zéro.
Il suffit donc de parcourir cette écriture binaire et d'analyser le chiffre le plus à droite. Ceci se schématise comme suit où \head{} indique la tête de lecture.

\begin{multicols}{2}

%

\emptybox\emptybox%
	\boxit{1}\boxit{0}\boxit{0}\boxit{1}\boxit{1}%
\emptybox\emptybox

\phantom{%
	\emptybox\emptybox}%
	\head


\medskip %

\emptybox\emptybox%
	\boxit{1}\boxit{0}\boxit{0}\boxit{1}\boxit{1}%
\emptybox\emptybox

\phantom{%
	\emptybox\emptybox
	\emptybox}%
	\head


\medskip %

\emptybox\emptybox%
	\boxit{1}\boxit{0}\boxit{0}\boxit{1}\boxit{1}%
\emptybox\emptybox

\phantom{%
	\emptybox\emptybox
	\emptybox\emptybox}%
	\head


\vfill\null
\columnbreak

\medskip %

\emptybox\emptybox%
	\boxit{1}\boxit{0}\boxit{0}\boxit{1}\boxit{1}%
\emptybox\emptybox

\phantom{%
	\emptybox\emptybox
	\emptybox\emptybox\emptybox}%
	\head


\medskip %

\emptybox\emptybox%
	\boxit{1}\boxit{0}\boxit{0}\boxit{1}\boxit{1}%
\emptybox\emptybox

\phantom{%
	\emptybox\emptybox
	\emptybox\emptybox\emptybox\emptybox}%
	\head


\medskip %

\emptybox\emptybox%
	\boxit{1}\boxit{0}\boxit{0}\boxit{1}\boxit{1}%
\emptybox\emptybox

\phantom{%
	\emptybox\emptybox
	\emptybox\emptybox\emptybox\emptybox\emptybox}%
	\head

\vfill\null
\end{multicols}

\vspace{-1em}

Pourquoi s'arrêter à la première case vide ? L'idée va être de garder en mémoire la valeur de la dernière case visitée.
Dans notre exemple, lors du dernier mouvement, on sait que la case précédente était un $1$ et donc que l'écriture binaire n'est pas celle d'un entier naturel pair.



\subsection{Une table des transitions}

	Une méthode simple et généraliste consiste à modifier les tables des transitions précédentes mécaniquement en respectant les deux règles suivantes.
\begin{enumerate}
	\item Toutes les cellules qui contiennent l'état final sont vidées.

	\item Toutes les cases vides, sauf celle correspondant à la case vide et l'état initial, sont remplies pour indiquer un passage à l'état final
		  \footnote{
			Ne pas oublier de traiter les états bloquants non indiqués sur la table initiale !
		  }.
	      \textbf{La cellule non traitée l'est afin d'éviter d'accepter le mot vide.} 
\end{enumerate}



Ceci nous donne la table des transitions ci-après pour les naturels impairs.
\begin{center}
	\begin{tabular}{|c||c|c|c|}
		\hline
		$\delta$ 
			& $0$ 
			& $1$
			& $B$ \\
		\hline
		\hline
		$q_0$
			& \transition{\ell_0}{0}{D}
			& \transition{\ell_1}{1}{D}
			&                           \\
		\hline
		\hline
		$\ell_0$
			& \transition{\ell_0}{0}{D}
			& \transition{\ell_1}{1}{D}
			&                           \\
		\hline
		$\ell_1$
			& \transition{\ell_0}{0}{D}
			& \transition{\ell_1}{1}{D}
			& \transition{f     }{B}{I} \\
		\hline
	\end{tabular}
\end{center}


Pour les non multiples de $3$, on obtient la table des transitions suivante.
\begin{center}
	\begin{tabular}{|c||c|c|c|}
		\hline
		$\delta$ 
			& $0$ 
			& $1$
			& $B$ \\
		\hline
		\hline
		$q_0$ 
			& \transition{d}{0}{D} 
			& \transition{d}{1}{D}
			&                      \\
		\hline
		$d$ 
			& \transition{d}{0}{D} 
			& \transition{d}{1}{D}
			& \transition{sp_0}{B}{G} \\
		\hline
		\hline
		$sp_0$ 
			& \transition{si_0}{0}{G} 
			& \transition{si_1}{1}{G}
			&                         \\
		\hline
		$sp_1$ 
			& \transition{si_1}{0}{G} 
			& \transition{si_2}{1}{G}
			& \transition{f   }{B}{I} \\
		\hline
		$sp_2$ 
			& \transition{si_2}{0}{G} 
			& \transition{si_0}{1}{G}
			& \transition{f   }{B}{I} \\
		\hline
		\hline
		$si_0$ 
			& \transition{sp_0}{0}{G} 
			& \transition{sp_2}{1}{G}
			&                         \\
		\hline
		$si_1$ 
			& \transition{sp_1}{0}{G} 
			& \transition{sp_0}{1}{G}
			& \transition{f   }{B}{I} \\
		\hline
		$si_2$ 
			& \transition{sp_2}{0}{G} 
			& \transition{sp_1}{1}{G}
			& \transition{f   }{B}{I} \\
		\hline
	\end{tabular}
\end{center}




\subsection{Coder pour tester}

	Sur le site de téléchargement de ce document se trouvent
\verb+not-divisible-by-2.txt+ et \verb+not-divisible-by-3.txt+
dans le sous-dossier \verb+turing/not-divisible-by-2-or-3+
qui contiennent des codes utilisables sur le site \url{https://turingmachinesimulator.com}.


\subsection{Généralisations}

	Le raisonnement précédent s'est appuyé sur $4 = 3 + 1$ soit $3 = 2^2 - 1$. Il est en fait facile de généraliser ce qui a été fait pour repérér les nombres divisibles par $2^k - 1$ où $k \in \NNs$.
Ceci étant dit, les tables des transitions vont croître de façon exponentielle !


\medskip


On peut en fait construire un automate d'état déterministe fini qui reconnait les multiples de $3$, et plus généralement de $d \in \NN - \setgene{0 ; 1}$, à partir de l'écriture d'un nombre dans une base quelconque $b \in \NN - \setgene{0 ; 1}$.
Ceci prouve que l'on peut construire des expressions rationnelles pour savoir si une écriture correspond à un multiple.



\newpage
\section{Écriture binaire des naturels impairs ou de ceux non multiples de 3}

	\subsection{Les tables des transitions}

	Une méthode simple et généraliste consiste à modifier les tables des transitions précédentes mécaniquement en respectant les deux règles suivantes.
\begin{enumerate}
	\item Toutes les cellules qui contiennent l'état final sont vidées.

	\item Toutes les cases vides, sauf celle correspondant à la case vide et l'état initial, sont remplies pour indiquer un passage à l'état final
		  \footnote{
			Ne pas oublier de traiter les états bloquants non indiqués sur la table initiale !
		  }.
	      \textbf{La cellule non traitée l'est afin d'éviter d'accepter le mot vide.} 
\end{enumerate}



Ceci nous donne la table des transitions ci-après pour les naturels impairs.
\begin{center}
	\begin{tabular}{|c||c|c|c|}
		\hline
		$\delta$ 
			& $0$ 
			& $1$
			& $B$ \\
		\hline
		\hline
		$q_0$
			& \transition{\ell_0}{0}{D}
			& \transition{\ell_1}{1}{D}
			&                           \\
		\hline
		\hline
		$\ell_0$
			& \transition{\ell_0}{0}{D}
			& \transition{\ell_1}{1}{D}
			&                           \\
		\hline
		$\ell_1$
			& \transition{\ell_0}{0}{D}
			& \transition{\ell_1}{1}{D}
			& \transition{f     }{B}{I} \\
		\hline
	\end{tabular}
\end{center}


Pour les non multiples de $3$, on obtient la table des transitions suivante.
\begin{center}
	\begin{tabular}{|c||c|c|c|}
		\hline
		$\delta$ 
			& $0$ 
			& $1$
			& $B$ \\
		\hline
		\hline
		$q_0$ 
			& \transition{d}{0}{D} 
			& \transition{d}{1}{D}
			&                      \\
		\hline
		$d$ 
			& \transition{d}{0}{D} 
			& \transition{d}{1}{D}
			& \transition{sp_0}{B}{G} \\
		\hline
		\hline
		$sp_0$ 
			& \transition{si_0}{0}{G} 
			& \transition{si_1}{1}{G}
			&                         \\
		\hline
		$sp_1$ 
			& \transition{si_1}{0}{G} 
			& \transition{si_2}{1}{G}
			& \transition{f   }{B}{I} \\
		\hline
		$sp_2$ 
			& \transition{si_2}{0}{G} 
			& \transition{si_0}{1}{G}
			& \transition{f   }{B}{I} \\
		\hline
		\hline
		$si_0$ 
			& \transition{sp_0}{0}{G} 
			& \transition{sp_2}{1}{G}
			&                         \\
		\hline
		$si_1$ 
			& \transition{sp_1}{0}{G} 
			& \transition{sp_0}{1}{G}
			& \transition{f   }{B}{I} \\
		\hline
		$si_2$ 
			& \transition{sp_2}{0}{G} 
			& \transition{sp_1}{1}{G}
			& \transition{f   }{B}{I} \\
		\hline
	\end{tabular}
\end{center}




\subsection{Coder pour tester}

	Sur le site de téléchargement de ce document se trouvent
\verb+not-divisible-by-2.txt+ et \verb+not-divisible-by-3.txt+
dans le sous-dossier \verb+turing/not-divisible-by-2-or-3+
qui contiennent des codes utilisables sur le site \url{https://turingmachinesimulator.com}.



\newpage
\section{Avec deux bandes}

	Dans cette section, \textbf{nous cherchons à proposer des méthodes généralistes automatiques mais non nécessairement optimales}
	\emph{(voir à ce sujet la section \ref{better-2-and-3})}.

	\subsection{Écriture binaire d'un multiple de 2 et 3}

	\subsubsection{Les tables des transitions}

	Une méthode simple et généraliste consiste à modifier les tables des transitions précédentes mécaniquement en respectant les deux règles suivantes.
\begin{enumerate}
	\item Toutes les cellules qui contiennent l'état final sont vidées.

	\item Toutes les cases vides, sauf celle correspondant à la case vide et l'état initial, sont remplies pour indiquer un passage à l'état final
		  \footnote{
			Ne pas oublier de traiter les états bloquants non indiqués sur la table initiale !
		  }.
	      \textbf{La cellule non traitée l'est afin d'éviter d'accepter le mot vide.} 
\end{enumerate}



Ceci nous donne la table des transitions ci-après pour les naturels impairs.
\begin{center}
	\begin{tabular}{|c||c|c|c|}
		\hline
		$\delta$ 
			& $0$ 
			& $1$
			& $B$ \\
		\hline
		\hline
		$q_0$
			& \transition{\ell_0}{0}{D}
			& \transition{\ell_1}{1}{D}
			&                           \\
		\hline
		\hline
		$\ell_0$
			& \transition{\ell_0}{0}{D}
			& \transition{\ell_1}{1}{D}
			&                           \\
		\hline
		$\ell_1$
			& \transition{\ell_0}{0}{D}
			& \transition{\ell_1}{1}{D}
			& \transition{f     }{B}{I} \\
		\hline
	\end{tabular}
\end{center}


Pour les non multiples de $3$, on obtient la table des transitions suivante.
\begin{center}
	\begin{tabular}{|c||c|c|c|}
		\hline
		$\delta$ 
			& $0$ 
			& $1$
			& $B$ \\
		\hline
		\hline
		$q_0$ 
			& \transition{d}{0}{D} 
			& \transition{d}{1}{D}
			&                      \\
		\hline
		$d$ 
			& \transition{d}{0}{D} 
			& \transition{d}{1}{D}
			& \transition{sp_0}{B}{G} \\
		\hline
		\hline
		$sp_0$ 
			& \transition{si_0}{0}{G} 
			& \transition{si_1}{1}{G}
			&                         \\
		\hline
		$sp_1$ 
			& \transition{si_1}{0}{G} 
			& \transition{si_2}{1}{G}
			& \transition{f   }{B}{I} \\
		\hline
		$sp_2$ 
			& \transition{si_2}{0}{G} 
			& \transition{si_0}{1}{G}
			& \transition{f   }{B}{I} \\
		\hline
		\hline
		$si_0$ 
			& \transition{sp_0}{0}{G} 
			& \transition{sp_2}{1}{G}
			&                         \\
		\hline
		$si_1$ 
			& \transition{sp_1}{0}{G} 
			& \transition{sp_0}{1}{G}
			& \transition{f   }{B}{I} \\
		\hline
		$si_2$ 
			& \transition{sp_2}{0}{G} 
			& \transition{sp_1}{1}{G}
			& \transition{f   }{B}{I} \\
		\hline
	\end{tabular}
\end{center}




\subsubsection{Une table efficace} \label{better-2-and-3}

	Si l'on sort des méthodes généralistes et automatisables, on peut faire une table plus simple.
En effet, pour tester si l'on a un multiple de $3$, nous partons de la droite.
Or nous savons aussi que le dernier chiffre lu à droite nous permet de savoir si l'on a ou non un nombre pair.
Il suffit donc lors du déplacement à droite de garder la trace de ce dernier chiffre, puis de continuer le travail si l'on a bien un zéro final.
Ceci nous donne ci-après une table des transitions plus courte que les précédentes.

\begin{center}
	\begin{tabular}{|c||c|c|c|}
% MUTIPLE OF 3
		\hline
		$\delta$ 
			& $0$ 
			& $1$
			& $B$ \\
		\hline
		\hline
		$q_0$ 
			& \transition{\ell_0}{0}{D} 
			& \transition{\ell_1}{1}{D}
			&                           \\
		\hline
		$\ell_0$
			& \transition{\ell_0}{0}{D}
			& \transition{\ell_1}{1}{D}
			& \transition{sp_0  }{B}{G} \\
		\hline
		$\ell_1$
			& \transition{\ell_0}{0}{D}
			& \transition{\ell_1}{1}{D}
			&                           \\
		\hline
		\hline
		$sp_0$ 
			& \transition{si_0}{0}{G} 
			& \transition{si_1}{1}{G}
			& \transition{f   }{B}{I} \\
		\hline
		$sp_1$ 
			& \transition{si_1}{0}{G} 
			& \transition{si_2}{1}{G}
			&                         \\
		\hline
		$sp_2$ 
			& \transition{si_2}{0}{G} 
			& \transition{si_0}{1}{G}
			&                         \\
		\hline
		\hline
		$si_0$ 
			& \transition{sp_0}{0}{G} 
			& \transition{sp_2}{1}{G}
			& \transition{f   }{B}{I} \\
		\hline
		$si_1$ 
			& \transition{sp_1}{0}{G} 
			& \transition{sp_0}{1}{G}
			&                         \\
		\hline
		$si_2$ 
			& \transition{sp_2}{0}{G} 
			& \transition{sp_1}{1}{G}
			&                         \\
		\hline
	\end{tabular}
\end{center}


\subsubsection{Coder pour tester}

	Sur le site de téléchargement de ce document se trouvent
\verb+not-divisible-by-2.txt+ et \verb+not-divisible-by-3.txt+
dans le sous-dossier \verb+turing/not-divisible-by-2-or-3+
qui contiennent des codes utilisables sur le site \url{https://turingmachinesimulator.com}.



\newpage
\subsection{Écriture binaire d'un multiple de 2 ou\,/\,et de 3}

	\subsubsection{Des tables des transitions} \label{2-or-3}

	Une méthode simple et généraliste consiste à modifier les tables des transitions précédentes mécaniquement en respectant les deux règles suivantes.
\begin{enumerate}
	\item Toutes les cellules qui contiennent l'état final sont vidées.

	\item Toutes les cases vides, sauf celle correspondant à la case vide et l'état initial, sont remplies pour indiquer un passage à l'état final
		  \footnote{
			Ne pas oublier de traiter les états bloquants non indiqués sur la table initiale !
		  }.
	      \textbf{La cellule non traitée l'est afin d'éviter d'accepter le mot vide.} 
\end{enumerate}



Ceci nous donne la table des transitions ci-après pour les naturels impairs.
\begin{center}
	\begin{tabular}{|c||c|c|c|}
		\hline
		$\delta$ 
			& $0$ 
			& $1$
			& $B$ \\
		\hline
		\hline
		$q_0$
			& \transition{\ell_0}{0}{D}
			& \transition{\ell_1}{1}{D}
			&                           \\
		\hline
		\hline
		$\ell_0$
			& \transition{\ell_0}{0}{D}
			& \transition{\ell_1}{1}{D}
			&                           \\
		\hline
		$\ell_1$
			& \transition{\ell_0}{0}{D}
			& \transition{\ell_1}{1}{D}
			& \transition{f     }{B}{I} \\
		\hline
	\end{tabular}
\end{center}


Pour les non multiples de $3$, on obtient la table des transitions suivante.
\begin{center}
	\begin{tabular}{|c||c|c|c|}
		\hline
		$\delta$ 
			& $0$ 
			& $1$
			& $B$ \\
		\hline
		\hline
		$q_0$ 
			& \transition{d}{0}{D} 
			& \transition{d}{1}{D}
			&                      \\
		\hline
		$d$ 
			& \transition{d}{0}{D} 
			& \transition{d}{1}{D}
			& \transition{sp_0}{B}{G} \\
		\hline
		\hline
		$sp_0$ 
			& \transition{si_0}{0}{G} 
			& \transition{si_1}{1}{G}
			&                         \\
		\hline
		$sp_1$ 
			& \transition{si_1}{0}{G} 
			& \transition{si_2}{1}{G}
			& \transition{f   }{B}{I} \\
		\hline
		$sp_2$ 
			& \transition{si_2}{0}{G} 
			& \transition{si_0}{1}{G}
			& \transition{f   }{B}{I} \\
		\hline
		\hline
		$si_0$ 
			& \transition{sp_0}{0}{G} 
			& \transition{sp_2}{1}{G}
			&                         \\
		\hline
		$si_1$ 
			& \transition{sp_1}{0}{G} 
			& \transition{sp_0}{1}{G}
			& \transition{f   }{B}{I} \\
		\hline
		$si_2$ 
			& \transition{sp_2}{0}{G} 
			& \transition{sp_1}{1}{G}
			& \transition{f   }{B}{I} \\
		\hline
	\end{tabular}
\end{center}




\subsubsection{Coder pour tester}

	Sur le site de téléchargement de ce document se trouvent
\verb+not-divisible-by-2.txt+ et \verb+not-divisible-by-3.txt+
dans le sous-dossier \verb+turing/not-divisible-by-2-or-3+
qui contiennent des codes utilisables sur le site \url{https://turingmachinesimulator.com}.




\newpage
\subsection{Écriture binaire d'un multiple soit de 2, soit de 3 mais pas des deux en même temps}

	\subsubsection{Une table des transitions}

	Une méthode simple et généraliste consiste à modifier les tables des transitions précédentes mécaniquement en respectant les deux règles suivantes.
\begin{enumerate}
	\item Toutes les cellules qui contiennent l'état final sont vidées.

	\item Toutes les cases vides, sauf celle correspondant à la case vide et l'état initial, sont remplies pour indiquer un passage à l'état final
		  \footnote{
			Ne pas oublier de traiter les états bloquants non indiqués sur la table initiale !
		  }.
	      \textbf{La cellule non traitée l'est afin d'éviter d'accepter le mot vide.} 
\end{enumerate}



Ceci nous donne la table des transitions ci-après pour les naturels impairs.
\begin{center}
	\begin{tabular}{|c||c|c|c|}
		\hline
		$\delta$ 
			& $0$ 
			& $1$
			& $B$ \\
		\hline
		\hline
		$q_0$
			& \transition{\ell_0}{0}{D}
			& \transition{\ell_1}{1}{D}
			&                           \\
		\hline
		\hline
		$\ell_0$
			& \transition{\ell_0}{0}{D}
			& \transition{\ell_1}{1}{D}
			&                           \\
		\hline
		$\ell_1$
			& \transition{\ell_0}{0}{D}
			& \transition{\ell_1}{1}{D}
			& \transition{f     }{B}{I} \\
		\hline
	\end{tabular}
\end{center}


Pour les non multiples de $3$, on obtient la table des transitions suivante.
\begin{center}
	\begin{tabular}{|c||c|c|c|}
		\hline
		$\delta$ 
			& $0$ 
			& $1$
			& $B$ \\
		\hline
		\hline
		$q_0$ 
			& \transition{d}{0}{D} 
			& \transition{d}{1}{D}
			&                      \\
		\hline
		$d$ 
			& \transition{d}{0}{D} 
			& \transition{d}{1}{D}
			& \transition{sp_0}{B}{G} \\
		\hline
		\hline
		$sp_0$ 
			& \transition{si_0}{0}{G} 
			& \transition{si_1}{1}{G}
			&                         \\
		\hline
		$sp_1$ 
			& \transition{si_1}{0}{G} 
			& \transition{si_2}{1}{G}
			& \transition{f   }{B}{I} \\
		\hline
		$sp_2$ 
			& \transition{si_2}{0}{G} 
			& \transition{si_0}{1}{G}
			& \transition{f   }{B}{I} \\
		\hline
		\hline
		$si_0$ 
			& \transition{sp_0}{0}{G} 
			& \transition{sp_2}{1}{G}
			&                         \\
		\hline
		$si_1$ 
			& \transition{sp_1}{0}{G} 
			& \transition{sp_0}{1}{G}
			& \transition{f   }{B}{I} \\
		\hline
		$si_2$ 
			& \transition{sp_2}{0}{G} 
			& \transition{sp_1}{1}{G}
			& \transition{f   }{B}{I} \\
		\hline
	\end{tabular}
\end{center}




\subsubsection{Coder pour tester}

	Sur le site de téléchargement de ce document se trouvent
\verb+not-divisible-by-2.txt+ et \verb+not-divisible-by-3.txt+
dans le sous-dossier \verb+turing/not-divisible-by-2-or-3+
qui contiennent des codes utilisables sur le site \url{https://turingmachinesimulator.com}.




\newpage
\section{Détecter les palindromes avec une seule bande}

	\subsection{Un exemple avec le mot abbca}

	Voici les grandes étapes présentées sur deux colonnes avec une 1\iere{} phase de duplication du mot puis une 2\ieme{} testant la présence ou non d'un palindrome.


\begin{multicols}{2}

% GESTION DE LA LETTRE LA PLUS À DROITE

\emptybox\emptybox%
	\boxit{a}\boxit{b}\boxit{b}\boxit{c}\boxit{a}%
\emptybox\emptybox

\phantom{\emptybox\emptybox}%
	\head


\medskip % RECHERCHE DE LA DERNIÈRE LETTRE
\emptybox\emptybox%
	\fboxit{a}\boxit{b}\boxit{b}\boxit{c}\boxit{a}%
\emptybox\emptybox

\phantom{\emptybox\emptybox%
	\emptybox\emptybox\emptybox\emptybox}%
	\head


\medskip % BONNE LETTRE
\emptybox\emptybox%
	\fboxit{a}\boxit{b}\boxit{b}\boxit{c}\fboxit{a}%
\emptybox\emptybox

\phantom{\emptybox\emptybox%
	\emptybox\emptybox\emptybox\emptybox}%
	\head


\medskip % EFFACEMENT DES LETTRES
\emptybox\emptybox%
	\emptybox\boxit{b}\boxit{b}\boxit{c}\emptybox%
\emptybox\emptybox

\phantom{\emptybox\emptybox%
	\emptybox}%
	\head

\vfill\null
\columnbreak

\medskip % GESTION DE LA NOUVELLE LETTRE LA PLUS À DROITE
\emptybox\emptybox%
	\emptybox\fboxit{b}\boxit{b}\boxit{c}\emptybox%
\emptybox\emptybox

\phantom{\emptybox\emptybox%
	\emptybox}%
	\head


\medskip % RECHERCHE DE LA DERNIÈRE LETTRE
\emptybox\emptybox%
	\emptybox\fboxit{b}\boxit{b}\boxit{c}\emptybox%
\emptybox\emptybox

\phantom{\emptybox\emptybox%
	\emptybox\emptybox\emptybox}%
	\head


\medskip % MAUVAISE LETTRE
\emptybox\emptybox%
	\emptybox\fboxit{b}\boxit{b}\wboxit{c}\emptybox%
\emptybox\emptybox

\phantom{\emptybox\emptybox%
	\emptybox\emptybox\emptybox}%
	\head

\end{multicols}




\subsection{Un exemple avec le mot abbcbba}

	On fait comme précédemment en devant tout parcourir !
Voici les grandes étapes.

\begin{multicols}{2}

% ????

\phantom{\emptybox\emptybox}%
	\deah

\emptybox\emptybox%
	\boxit{a}\boxit{b}\boxit{b}\boxit{c}\boxit{b}\boxit{b}\boxit{a}%
\emptybox\emptybox

\emptybox\emptybox%
	\emptybox\emptybox\emptybox\emptybox\emptybox\emptybox\emptybox%
\emptybox\emptybox

\phantom{\emptybox\emptybox}%
	\head


\medskip % ????

\phantom{\emptybox\emptybox\emptybox\emptybox\emptybox\emptybox\emptybox\emptybox\emptybox}%
	\deah

\emptybox\emptybox%
	\boxit{a}\boxit{b}\boxit{b}\boxit{c}\boxit{b}\boxit{b}\boxit{a}%
\emptybox\emptybox

\emptybox\emptybox%
	\nboxit{a}\nboxit{b}\nboxit{b}\nboxit{c}\nboxit{b}\nboxit{b}\nboxit{a}%
\emptybox\emptybox

\phantom{\emptybox\emptybox\emptybox\emptybox\emptybox\emptybox\emptybox\emptybox\emptybox}%
	\head


\vfill\null
\columnbreak % ????

\phantom{\emptybox}%
	\deah

\emptybox\emptybox%
	\boxit{a}\boxit{b}\boxit{b}\boxit{c}\boxit{b}\boxit{b}\boxit{a}%
\emptybox\emptybox

\emptybox\emptybox%
	\nboxit{a}\nboxit{b}\nboxit{b}\nboxit{c}\nboxit{b}\nboxit{b}\nboxit{a}%
\emptybox\emptybox

\phantom{\emptybox\emptybox\emptybox\emptybox\emptybox\emptybox\emptybox\emptybox\emptybox}%
	\head


\medskip % ????

\phantom{\emptybox\emptybox\emptybox\emptybox\emptybox\emptybox\emptybox\emptybox\emptybox}%
	\deah

\emptybox\emptybox%
	\fboxit{a}\fboxit{b}\fboxit{b}\fboxit{c}\fboxit{b}\fboxit{b}\fboxit{a}%
\emptybox\emptybox

\emptybox\emptybox%
	\fboxit{a}\fboxit{b}\fboxit{b}\fboxit{c}\fboxit{b}\fboxit{b}\fboxit{a}%
\emptybox\emptybox

\phantom{\emptybox}%
	\head

\vfill\null
\end{multicols}



\subsection{Une table des transitions} \label{table-palindrome-1-tape}

	Une méthode simple et généraliste consiste à modifier les tables des transitions précédentes mécaniquement en respectant les deux règles suivantes.
\begin{enumerate}
	\item Toutes les cellules qui contiennent l'état final sont vidées.

	\item Toutes les cases vides, sauf celle correspondant à la case vide et l'état initial, sont remplies pour indiquer un passage à l'état final
		  \footnote{
			Ne pas oublier de traiter les états bloquants non indiqués sur la table initiale !
		  }.
	      \textbf{La cellule non traitée l'est afin d'éviter d'accepter le mot vide.} 
\end{enumerate}



Ceci nous donne la table des transitions ci-après pour les naturels impairs.
\begin{center}
	\begin{tabular}{|c||c|c|c|}
		\hline
		$\delta$ 
			& $0$ 
			& $1$
			& $B$ \\
		\hline
		\hline
		$q_0$
			& \transition{\ell_0}{0}{D}
			& \transition{\ell_1}{1}{D}
			&                           \\
		\hline
		\hline
		$\ell_0$
			& \transition{\ell_0}{0}{D}
			& \transition{\ell_1}{1}{D}
			&                           \\
		\hline
		$\ell_1$
			& \transition{\ell_0}{0}{D}
			& \transition{\ell_1}{1}{D}
			& \transition{f     }{B}{I} \\
		\hline
	\end{tabular}
\end{center}


Pour les non multiples de $3$, on obtient la table des transitions suivante.
\begin{center}
	\begin{tabular}{|c||c|c|c|}
		\hline
		$\delta$ 
			& $0$ 
			& $1$
			& $B$ \\
		\hline
		\hline
		$q_0$ 
			& \transition{d}{0}{D} 
			& \transition{d}{1}{D}
			&                      \\
		\hline
		$d$ 
			& \transition{d}{0}{D} 
			& \transition{d}{1}{D}
			& \transition{sp_0}{B}{G} \\
		\hline
		\hline
		$sp_0$ 
			& \transition{si_0}{0}{G} 
			& \transition{si_1}{1}{G}
			&                         \\
		\hline
		$sp_1$ 
			& \transition{si_1}{0}{G} 
			& \transition{si_2}{1}{G}
			& \transition{f   }{B}{I} \\
		\hline
		$sp_2$ 
			& \transition{si_2}{0}{G} 
			& \transition{si_0}{1}{G}
			& \transition{f   }{B}{I} \\
		\hline
		\hline
		$si_0$ 
			& \transition{sp_0}{0}{G} 
			& \transition{sp_2}{1}{G}
			&                         \\
		\hline
		$si_1$ 
			& \transition{sp_1}{0}{G} 
			& \transition{sp_0}{1}{G}
			& \transition{f   }{B}{I} \\
		\hline
		$si_2$ 
			& \transition{sp_2}{0}{G} 
			& \transition{sp_1}{1}{G}
			& \transition{f   }{B}{I} \\
		\hline
	\end{tabular}
\end{center}




\subsection{Coder pour tester}

	Sur le site de téléchargement de ce document se trouvent
\verb+not-divisible-by-2.txt+ et \verb+not-divisible-by-3.txt+
dans le sous-dossier \verb+turing/not-divisible-by-2-or-3+
qui contiennent des codes utilisables sur le site \url{https://turingmachinesimulator.com}.




\newpage
\section{Détecter les palindromes avec deux bandes}

	\subsection{Un exemple avec le mot abbca}

	Voici les grandes étapes présentées sur deux colonnes avec une 1\iere{} phase de duplication du mot puis une 2\ieme{} testant la présence ou non d'un palindrome.


\begin{multicols}{2}

% GESTION DE LA LETTRE LA PLUS À DROITE

\emptybox\emptybox%
	\boxit{a}\boxit{b}\boxit{b}\boxit{c}\boxit{a}%
\emptybox\emptybox

\phantom{\emptybox\emptybox}%
	\head


\medskip % RECHERCHE DE LA DERNIÈRE LETTRE
\emptybox\emptybox%
	\fboxit{a}\boxit{b}\boxit{b}\boxit{c}\boxit{a}%
\emptybox\emptybox

\phantom{\emptybox\emptybox%
	\emptybox\emptybox\emptybox\emptybox}%
	\head


\medskip % BONNE LETTRE
\emptybox\emptybox%
	\fboxit{a}\boxit{b}\boxit{b}\boxit{c}\fboxit{a}%
\emptybox\emptybox

\phantom{\emptybox\emptybox%
	\emptybox\emptybox\emptybox\emptybox}%
	\head


\medskip % EFFACEMENT DES LETTRES
\emptybox\emptybox%
	\emptybox\boxit{b}\boxit{b}\boxit{c}\emptybox%
\emptybox\emptybox

\phantom{\emptybox\emptybox%
	\emptybox}%
	\head

\vfill\null
\columnbreak

\medskip % GESTION DE LA NOUVELLE LETTRE LA PLUS À DROITE
\emptybox\emptybox%
	\emptybox\fboxit{b}\boxit{b}\boxit{c}\emptybox%
\emptybox\emptybox

\phantom{\emptybox\emptybox%
	\emptybox}%
	\head


\medskip % RECHERCHE DE LA DERNIÈRE LETTRE
\emptybox\emptybox%
	\emptybox\fboxit{b}\boxit{b}\boxit{c}\emptybox%
\emptybox\emptybox

\phantom{\emptybox\emptybox%
	\emptybox\emptybox\emptybox}%
	\head


\medskip % MAUVAISE LETTRE
\emptybox\emptybox%
	\emptybox\fboxit{b}\boxit{b}\wboxit{c}\emptybox%
\emptybox\emptybox

\phantom{\emptybox\emptybox%
	\emptybox\emptybox\emptybox}%
	\head

\end{multicols}




\subsection{Un exemple avec le mot abbcbba}

	On fait comme précédemment en devant tout parcourir !
Voici les grandes étapes.

\begin{multicols}{2}

% ????

\phantom{\emptybox\emptybox}%
	\deah

\emptybox\emptybox%
	\boxit{a}\boxit{b}\boxit{b}\boxit{c}\boxit{b}\boxit{b}\boxit{a}%
\emptybox\emptybox

\emptybox\emptybox%
	\emptybox\emptybox\emptybox\emptybox\emptybox\emptybox\emptybox%
\emptybox\emptybox

\phantom{\emptybox\emptybox}%
	\head


\medskip % ????

\phantom{\emptybox\emptybox\emptybox\emptybox\emptybox\emptybox\emptybox\emptybox\emptybox}%
	\deah

\emptybox\emptybox%
	\boxit{a}\boxit{b}\boxit{b}\boxit{c}\boxit{b}\boxit{b}\boxit{a}%
\emptybox\emptybox

\emptybox\emptybox%
	\nboxit{a}\nboxit{b}\nboxit{b}\nboxit{c}\nboxit{b}\nboxit{b}\nboxit{a}%
\emptybox\emptybox

\phantom{\emptybox\emptybox\emptybox\emptybox\emptybox\emptybox\emptybox\emptybox\emptybox}%
	\head


\vfill\null
\columnbreak % ????

\phantom{\emptybox}%
	\deah

\emptybox\emptybox%
	\boxit{a}\boxit{b}\boxit{b}\boxit{c}\boxit{b}\boxit{b}\boxit{a}%
\emptybox\emptybox

\emptybox\emptybox%
	\nboxit{a}\nboxit{b}\nboxit{b}\nboxit{c}\nboxit{b}\nboxit{b}\nboxit{a}%
\emptybox\emptybox

\phantom{\emptybox\emptybox\emptybox\emptybox\emptybox\emptybox\emptybox\emptybox\emptybox}%
	\head


\medskip % ????

\phantom{\emptybox\emptybox\emptybox\emptybox\emptybox\emptybox\emptybox\emptybox\emptybox}%
	\deah

\emptybox\emptybox%
	\fboxit{a}\fboxit{b}\fboxit{b}\fboxit{c}\fboxit{b}\fboxit{b}\fboxit{a}%
\emptybox\emptybox

\emptybox\emptybox%
	\fboxit{a}\fboxit{b}\fboxit{b}\fboxit{c}\fboxit{b}\fboxit{b}\fboxit{a}%
\emptybox\emptybox

\phantom{\emptybox}%
	\head

\vfill\null
\end{multicols}



\subsection{La table des transitions} \label{duplicate-table}

	Une méthode simple et généraliste consiste à modifier les tables des transitions précédentes mécaniquement en respectant les deux règles suivantes.
\begin{enumerate}
	\item Toutes les cellules qui contiennent l'état final sont vidées.

	\item Toutes les cases vides, sauf celle correspondant à la case vide et l'état initial, sont remplies pour indiquer un passage à l'état final
		  \footnote{
			Ne pas oublier de traiter les états bloquants non indiqués sur la table initiale !
		  }.
	      \textbf{La cellule non traitée l'est afin d'éviter d'accepter le mot vide.} 
\end{enumerate}



Ceci nous donne la table des transitions ci-après pour les naturels impairs.
\begin{center}
	\begin{tabular}{|c||c|c|c|}
		\hline
		$\delta$ 
			& $0$ 
			& $1$
			& $B$ \\
		\hline
		\hline
		$q_0$
			& \transition{\ell_0}{0}{D}
			& \transition{\ell_1}{1}{D}
			&                           \\
		\hline
		\hline
		$\ell_0$
			& \transition{\ell_0}{0}{D}
			& \transition{\ell_1}{1}{D}
			&                           \\
		\hline
		$\ell_1$
			& \transition{\ell_0}{0}{D}
			& \transition{\ell_1}{1}{D}
			& \transition{f     }{B}{I} \\
		\hline
	\end{tabular}
\end{center}


Pour les non multiples de $3$, on obtient la table des transitions suivante.
\begin{center}
	\begin{tabular}{|c||c|c|c|}
		\hline
		$\delta$ 
			& $0$ 
			& $1$
			& $B$ \\
		\hline
		\hline
		$q_0$ 
			& \transition{d}{0}{D} 
			& \transition{d}{1}{D}
			&                      \\
		\hline
		$d$ 
			& \transition{d}{0}{D} 
			& \transition{d}{1}{D}
			& \transition{sp_0}{B}{G} \\
		\hline
		\hline
		$sp_0$ 
			& \transition{si_0}{0}{G} 
			& \transition{si_1}{1}{G}
			&                         \\
		\hline
		$sp_1$ 
			& \transition{si_1}{0}{G} 
			& \transition{si_2}{1}{G}
			& \transition{f   }{B}{I} \\
		\hline
		$sp_2$ 
			& \transition{si_2}{0}{G} 
			& \transition{si_0}{1}{G}
			& \transition{f   }{B}{I} \\
		\hline
		\hline
		$si_0$ 
			& \transition{sp_0}{0}{G} 
			& \transition{sp_2}{1}{G}
			&                         \\
		\hline
		$si_1$ 
			& \transition{sp_1}{0}{G} 
			& \transition{sp_0}{1}{G}
			& \transition{f   }{B}{I} \\
		\hline
		$si_2$ 
			& \transition{sp_2}{0}{G} 
			& \transition{sp_1}{1}{G}
			& \transition{f   }{B}{I} \\
		\hline
	\end{tabular}
\end{center}




\subsection{Coder pour tester}

	Sur le site de téléchargement de ce document se trouvent
\verb+not-divisible-by-2.txt+ et \verb+not-divisible-by-3.txt+
dans le sous-dossier \verb+turing/not-divisible-by-2-or-3+
qui contiennent des codes utilisables sur le site \url{https://turingmachinesimulator.com}.




\newpage
\section{Deux fois plus de b que de a}

	\subsection{Un exemple avec le mot babbbabab}

	Commençons par un exemple simple qui va nous permettre de fixer les notations que nous allons utiliser.
Voyons comment repérer un entier naturel pair à partir de son écriture binaire.
La réponse est évidente : un naturel est pair si et seulement si son écriture binaire se finit par un zéro.
Il suffit donc de parcourir cette écriture binaire et d'analyser le chiffre le plus à droite. Ceci se schématise comme suit où \head{} indique la tête de lecture.

\begin{multicols}{2}

%

\emptybox\emptybox%
	\boxit{1}\boxit{0}\boxit{0}\boxit{1}\boxit{1}%
\emptybox\emptybox

\phantom{%
	\emptybox\emptybox}%
	\head


\medskip %

\emptybox\emptybox%
	\boxit{1}\boxit{0}\boxit{0}\boxit{1}\boxit{1}%
\emptybox\emptybox

\phantom{%
	\emptybox\emptybox
	\emptybox}%
	\head


\medskip %

\emptybox\emptybox%
	\boxit{1}\boxit{0}\boxit{0}\boxit{1}\boxit{1}%
\emptybox\emptybox

\phantom{%
	\emptybox\emptybox
	\emptybox\emptybox}%
	\head


\vfill\null
\columnbreak

\medskip %

\emptybox\emptybox%
	\boxit{1}\boxit{0}\boxit{0}\boxit{1}\boxit{1}%
\emptybox\emptybox

\phantom{%
	\emptybox\emptybox
	\emptybox\emptybox\emptybox}%
	\head


\medskip %

\emptybox\emptybox%
	\boxit{1}\boxit{0}\boxit{0}\boxit{1}\boxit{1}%
\emptybox\emptybox

\phantom{%
	\emptybox\emptybox
	\emptybox\emptybox\emptybox\emptybox}%
	\head


\medskip %

\emptybox\emptybox%
	\boxit{1}\boxit{0}\boxit{0}\boxit{1}\boxit{1}%
\emptybox\emptybox

\phantom{%
	\emptybox\emptybox
	\emptybox\emptybox\emptybox\emptybox\emptybox}%
	\head

\vfill\null
\end{multicols}

\vspace{-1em}

Pourquoi s'arrêter à la première case vide ? L'idée va être de garder en mémoire la valeur de la dernière case visitée.
Dans notre exemple, lors du dernier mouvement, on sait que la case précédente était un $1$ et donc que l'écriture binaire n'est pas celle d'un entier naturel pair.



\subsection{Une table des transitions}

	Une méthode simple et généraliste consiste à modifier les tables des transitions précédentes mécaniquement en respectant les deux règles suivantes.
\begin{enumerate}
	\item Toutes les cellules qui contiennent l'état final sont vidées.

	\item Toutes les cases vides, sauf celle correspondant à la case vide et l'état initial, sont remplies pour indiquer un passage à l'état final
		  \footnote{
			Ne pas oublier de traiter les états bloquants non indiqués sur la table initiale !
		  }.
	      \textbf{La cellule non traitée l'est afin d'éviter d'accepter le mot vide.} 
\end{enumerate}



Ceci nous donne la table des transitions ci-après pour les naturels impairs.
\begin{center}
	\begin{tabular}{|c||c|c|c|}
		\hline
		$\delta$ 
			& $0$ 
			& $1$
			& $B$ \\
		\hline
		\hline
		$q_0$
			& \transition{\ell_0}{0}{D}
			& \transition{\ell_1}{1}{D}
			&                           \\
		\hline
		\hline
		$\ell_0$
			& \transition{\ell_0}{0}{D}
			& \transition{\ell_1}{1}{D}
			&                           \\
		\hline
		$\ell_1$
			& \transition{\ell_0}{0}{D}
			& \transition{\ell_1}{1}{D}
			& \transition{f     }{B}{I} \\
		\hline
	\end{tabular}
\end{center}


Pour les non multiples de $3$, on obtient la table des transitions suivante.
\begin{center}
	\begin{tabular}{|c||c|c|c|}
		\hline
		$\delta$ 
			& $0$ 
			& $1$
			& $B$ \\
		\hline
		\hline
		$q_0$ 
			& \transition{d}{0}{D} 
			& \transition{d}{1}{D}
			&                      \\
		\hline
		$d$ 
			& \transition{d}{0}{D} 
			& \transition{d}{1}{D}
			& \transition{sp_0}{B}{G} \\
		\hline
		\hline
		$sp_0$ 
			& \transition{si_0}{0}{G} 
			& \transition{si_1}{1}{G}
			&                         \\
		\hline
		$sp_1$ 
			& \transition{si_1}{0}{G} 
			& \transition{si_2}{1}{G}
			& \transition{f   }{B}{I} \\
		\hline
		$sp_2$ 
			& \transition{si_2}{0}{G} 
			& \transition{si_0}{1}{G}
			& \transition{f   }{B}{I} \\
		\hline
		\hline
		$si_0$ 
			& \transition{sp_0}{0}{G} 
			& \transition{sp_2}{1}{G}
			&                         \\
		\hline
		$si_1$ 
			& \transition{sp_1}{0}{G} 
			& \transition{sp_0}{1}{G}
			& \transition{f   }{B}{I} \\
		\hline
		$si_2$ 
			& \transition{sp_2}{0}{G} 
			& \transition{sp_1}{1}{G}
			& \transition{f   }{B}{I} \\
		\hline
	\end{tabular}
\end{center}




\subsection{Avec une seule bande ?}

	Il a été facile de résoudre la problème avec trois bandes. Essayons de voir si l'on peut se limiter à une seule bande
\footnote{
	Théoriquement on sait que toute machine à $n$ bandes peut être traduite en une machine à une seule bande.
	Malheureusement le procédé ne produit pas forcément des machines qu'un humain aurait conçu tout seul donc nous allons laisser de côté ce procédé.
}
en n'utilisant pas la lettre supplémentaire X
\footnote{
	Là est le mini-défi.
}.
Voici les grandes lignes de la méthode.

\begin{multicols}{2}
%
\emptybox\emptybox%
	\wboxit{\text{b}}\fboxit{\text{a}}\wboxit{\text{b}}\wboxit{\text{b}}\wboxit{\text{b}}\fboxit{\text{a}}\wboxit{\text{b}}\fboxit{\text{a}}\wboxit{\text{b}}%
\emptybox\emptybox

\phantom{\emptybox\emptybox}%
	\head


\medskip %


\emptybox\emptybox%
	\fboxit{\text{a}}\fboxit{\text{a}}\fboxit{\text{a}}\wboxit{\text{b}}\wboxit{\text{b}}\wboxit{\text{b}}\wboxit{\text{b}}\wboxit{\text{b}}\wboxit{\text{b}}%
\emptybox\emptybox

\phantom{\emptybox\emptybox\emptybox\emptybox\emptybox\emptybox\emptybox\emptybox\emptybox\emptybox\emptybox}%
	\head


\medskip %


\emptybox\emptybox%
	\fboxit{\text{a}}\fboxit{\text{a}}\fboxit{\text{a}}\wboxit{\text{b}}\wboxit{\text{b}}\wboxit{\text{b}}\wboxit{\text{b}}\emptybox\emptybox%
\emptybox\emptybox

\phantom{\emptybox}%
	\head


\medskip %


\emptybox\emptybox%
	\emptybox\fboxit{\text{a}}\fboxit{\text{a}}\wboxit{\text{b}}\wboxit{\text{b}}\wboxit{\text{b}}\wboxit{\text{b}}\emptybox\emptybox%
\emptybox\emptybox

\phantom{\emptybox\emptybox\emptybox\emptybox\emptybox\emptybox\emptybox\emptybox\emptybox}%
	\head


\medskip %


\emptybox\emptybox%
	\emptybox\fboxit{\text{a}}\fboxit{\text{a}}\wboxit{\text{b}}\wboxit{\text{b}}\emptybox\emptybox\emptybox\emptybox%
\emptybox\emptybox

\phantom{\emptybox\emptybox}%
	\head


\medskip %


\emptybox\emptybox%
	\emptybox\emptybox\fboxit{\text{a}}\wboxit{\text{b}}\wboxit{\text{b}}\emptybox\emptybox\emptybox\emptybox%
\emptybox\emptybox

\phantom{\emptybox\emptybox\emptybox\emptybox\emptybox\emptybox\emptybox}%
	\head


\medskip %


\emptybox\emptybox%
	\emptybox\emptybox\fboxit{\text{a}}\emptybox\emptybox\emptybox\emptybox\emptybox%
\emptybox\emptybox

\phantom{\emptybox\emptybox\emptybox}%
	\head


\medskip %


\emptybox\emptybox%
	\emptybox\emptybox\emptybox\emptybox\emptybox\emptybox\emptybox\emptybox%
\emptybox\emptybox

\phantom{\emptybox\emptybox\emptybox\emptybox\emptybox}%
	\head

\end{multicols}


On aboutit alors à la table des transitions suivante où le plus délicat est la réorganisation des a et des b
\emph{(on effectue un tri à bulles)}.
L'état $g$ et ceux de type $s$ sont chargés de cette réorganisation, tandis que les états de type $e$ s'occupent de la procédure d'effacement.


\begin{center}
	\emph{\small Phase 1 : réorganisation \emph{(tri à bulles)}.}
	
	\smallskip
	\begin{tabular}{|c||c|c|c|}
% MUTIPLE OF 3
		\hline
		$\delta$ 
			& a 
			& b
			& $B$ \\
		\hline
		\hline
		$q_0$ 
			& \transition{s_a}{\text{a}}{D} 
			& \transition{s_b}{\text{b}}{D}
			&                        \\
		\hline
		$s_a$ 
			& \transition{s_a}{\text{a}}{D} 
			& \transition{s_b}{\text{b}}{D}
			& \transition{e_b}{B       }{G} \\
		\hline
		$s_b$ 
			& \transition{s^{\,\prime}_b}{\text{b}}{G}
			& \transition{s_b           }{\text{b}}{D}
			& \transition{e_b           }{B       }{G} \\
		\hline
		$s^{\,\prime}_b$ 
			&
			& \transition{s^{\,\prime\prime}_b}{\text{a}}{D}
			&                                         \\
		\hline
		$s^{\,\prime\prime}_b$ 
			& \transition{s^{\,\prime}_b       }{\text{b}}{G}
			& \transition{s^{\,\prime\prime}_b }{\text{b}}{D}
			& \transition{g                    }{\text{b}}{G} \\
		\hline
		$g$ 
			& \transition{g  }{\text{a}}{G}
			& \transition{g  }{\text{b}}{G}
			& \transition{q_0}{B       }{D} \\
		\hline
	\end{tabular}
\end{center}


\begin{center}

	\emph{\small Phase 2 : effacement.}
	
	\smallskip
	\begin{tabular}{|c||c|c|c|}
% MUTIPLE OF 3
		\hline
		$\delta$ 
			& a 
			& b
			& $B$ \\
		\hline
		\hline
		$e_b$ 
			&
			& \transition{e^{\,\prime}_b}{B}{G} 
			& \transition{f             }{B}{I} \\
		\hline
		$e^{\,\prime}_b$
			&
			& \transition{g_a}{B}{G} 
			&                        \\
		\hline
		$g_a$
			& \transition{g_a}{\text{a}}{G}
			& \transition{g_a}{\text{b}}{G}
			& \transition{e_a}{B       }{D} \\
		\hline
		$e_a$
			& \transition{d_b}{B}{D}
			& 
			&                        \\
		\hline
		$d_b$
			& \transition{d_b}{\text{a}}{D}
			& \transition{d_b}{\text{b}}{D}
			& \transition{e_b}{B       }{G} \\
		\hline
	\end{tabular}
\end{center}


\subsection{Coder pour tester}

	Sur le site de téléchargement de ce document se trouvent
\verb+not-divisible-by-2.txt+ et \verb+not-divisible-by-3.txt+
dans le sous-dossier \verb+turing/not-divisible-by-2-or-3+
qui contiennent des codes utilisables sur le site \url{https://turingmachinesimulator.com}.



\end{document}
