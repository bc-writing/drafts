Commençons par un exemple simple qui va nous permettre de fixer les notations que nous allons utiliser.
Voyons comment repérer un entier naturel pair à partir de son écriture binaire.
La réponse est évidente : un naturel est pair si et seulement si son écriture binaire se finit par un zéro.
Il suffit donc de parcourir cette écriture binaire et d'analyser le chiffre le plus à droite. Ceci se schématise comme suit où \head{} indique la tête de lecture.

\begin{multicols}{2}

%

\emptybox\emptybox%
	\boxit{1}\boxit{0}\boxit{0}\boxit{1}\boxit{1}%
\emptybox\emptybox

\phantom{%
	\emptybox\emptybox}%
	\head


\medskip %

\emptybox\emptybox%
	\boxit{1}\boxit{0}\boxit{0}\boxit{1}\boxit{1}%
\emptybox\emptybox

\phantom{%
	\emptybox\emptybox
	\emptybox}%
	\head


\medskip %

\emptybox\emptybox%
	\boxit{1}\boxit{0}\boxit{0}\boxit{1}\boxit{1}%
\emptybox\emptybox

\phantom{%
	\emptybox\emptybox
	\emptybox\emptybox}%
	\head


\vfill\null
\columnbreak

\medskip %

\emptybox\emptybox%
	\boxit{1}\boxit{0}\boxit{0}\boxit{1}\boxit{1}%
\emptybox\emptybox

\phantom{%
	\emptybox\emptybox
	\emptybox\emptybox\emptybox}%
	\head


\medskip %

\emptybox\emptybox%
	\boxit{1}\boxit{0}\boxit{0}\boxit{1}\boxit{1}%
\emptybox\emptybox

\phantom{%
	\emptybox\emptybox
	\emptybox\emptybox\emptybox\emptybox}%
	\head


\medskip %

\emptybox\emptybox%
	\boxit{1}\boxit{0}\boxit{0}\boxit{1}\boxit{1}%
\emptybox\emptybox

\phantom{%
	\emptybox\emptybox
	\emptybox\emptybox\emptybox\emptybox\emptybox}%
	\head

\vfill\null
\end{multicols}

\vspace{-1em}

Pourquoi s'arrêter à la première case vide ? L'idée va être de garder en mémoire la valeur de la dernière case visitée.
Dans notre exemple, lors du dernier mouvement, on sait que la case précédente était un $1$ et donc que l'écriture binaire n'est pas celle d'un entier naturel pair.
