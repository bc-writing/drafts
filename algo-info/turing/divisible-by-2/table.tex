Pour formaliser les étapes à suivre, on va écrire une table des transitions.
Nous travaillons avec des mots sur l'alphabet $\setgene{0 ; 1}$ et nous allons utiliser les états de transition suivants.

\begin{itemize}[label = \small\textbullet]
	\item $q_0$ est l'état initial avant tout déplacement.

	\item $\ell_0$ et $\ell_1$ sont les états indiquant que la dernière
	\footnote{
		$\ell$ pour \og{} last \fg{}.
	}
	case visitée contenait un $0$ ou un $1$ respectivement.

	\item L'état $f$ sera notre état final indiquant qu'un mot est bien l'écriture binaire d'un naturel pair.
\end{itemize}

Voici la table des transitions de notre machine de Turing où $B$ indique une case vide, tandis que $D$ et $G$ demandent de déplacer la tête de lecture \head{} vers la droite et la gauche respectivement, tandis que $I$ indique de ne pas faire bouger cette tête.
\begin{center}
	\begin{tabular}{|c||c|c|c|}
		\hline
		$\delta$ 
			& $0$ 
			& $1$
			& $B$ \\
		\hline
		\hline
		$q_0$
			& \transition{\ell_0}{0}{D}
			& \transition{\ell_1}{1}{D}
			& \\
		\hline
		\hline
		$\ell_0$
			& \transition{\ell_0}{0}{D}
			& \transition{\ell_1}{1}{D}
			& \transition{f     }{B}{I}      \\
		\hline
		$\ell_1$
			& \transition{\ell_0}{0}{D}
			& \transition{\ell_1}{1}{D}
			& \\
		\hline
	\end{tabular}
\end{center}

