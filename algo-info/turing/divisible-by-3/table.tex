Les explications de la section précédente permettent de comprendre la table des transitions suivantes.
\begin{center}
	\begin{tabular}{|c||c|c|c|}
		\hline
		$\delta$ 
			& $0$ 
			& $1$
			& $B$ \\
		\hline
		\hline
		$q_0$ 
			& $(d , 0, D)$ 
			& $(d , 1, D)$
			&  \\
		\hline
		$d$ 
			& $(d , 0, D)$ 
			& $(d , 1, D)$
			& $(sp_0, B, G)$ \\
		\hline
		\hline
		$sp_0$ 
			& $(si_0 , 0, G)$ 
			& $(si_1 , 1, G)$
			& $(f    , B, I)$ \\
		\hline
		$sp_1$ 
			& $(si_1 , 0, G)$ 
			& $(si_2 , 1, G)$
			&                 \\
		\hline
		$sp_2$ 
			& $(si_2 , 0, G)$ 
			& $(si_0 , 1, G)$
			&                 \\
		\hline
		\hline
		$si_0$ 
			& $(sp_0 , 0, G)$ 
			& $(sp_2 , 1, G)$
			& $(f    , B, I)$ \\
		\hline
		$si_1$ 
			& $(sp_1 , 0, G)$ 
			& $(sp_0 , 1, G)$
			&                 \\
		\hline
		$si_2$ 
			& $(sp_2 , 0, G)$ 
			& $(sp_1 , 1, G)$
			&                 \\
		\hline
	\end{tabular}
\end{center}