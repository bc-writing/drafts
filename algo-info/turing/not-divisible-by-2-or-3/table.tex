Une méthode simple et généraliste consiste à modifier les tables des transitions précédentes mécaniquement en respectant les deux règles suivantes.
\begin{enumerate}
	\item Toutes les cellules qui contiennent l'état final sont vidées.

	\item Toutes les cases vides, sauf celle correspondant à la case vide et l'état initial, sont remplies pour indiquer un passage à l'état final
		  \footnote{
			Ne pas oublier de traiter les états bloquants non indiqués sur la table initiale !
		  }.
	      \textbf{La cellule non traitée l'est afin d'éviter d'accepter le mot vide.} 
\end{enumerate}



Ceci nous donne la table des transitions ci-après pour les naturels impairs.
\begin{center}
	\begin{tabular}{|c||c|c|c|}
		\hline
		$\delta$ 
			& $0$ 
			& $1$
			& $B$ \\
		\hline
		\hline
		$q_0$
			& \transition{\ell_0}{0}{D}
			& \transition{\ell_1}{1}{D}
			&                           \\
		\hline
		\hline
		$\ell_0$
			& \transition{\ell_0}{0}{D}
			& \transition{\ell_1}{1}{D}
			&                           \\
		\hline
		$\ell_1$
			& \transition{\ell_0}{0}{D}
			& \transition{\ell_1}{1}{D}
			& \transition{f     }{B}{I} \\
		\hline
	\end{tabular}
\end{center}


Pour les non multiples de $3$, on obtient la table des transitions suivante.
\begin{center}
	\begin{tabular}{|c||c|c|c|}
		\hline
		$\delta$ 
			& $0$ 
			& $1$
			& $B$ \\
		\hline
		\hline
		$q_0$ 
			& \transition{d}{0}{D} 
			& \transition{d}{1}{D}
			&                      \\
		\hline
		$d$ 
			& \transition{d}{0}{D} 
			& \transition{d}{1}{D}
			& \transition{sp_0}{B}{G} \\
		\hline
		\hline
		$sp_0$ 
			& \transition{si_0}{0}{G} 
			& \transition{si_1}{1}{G}
			&                         \\
		\hline
		$sp_1$ 
			& \transition{si_1}{0}{G} 
			& \transition{si_2}{1}{G}
			& \transition{f   }{B}{I} \\
		\hline
		$sp_2$ 
			& \transition{si_2}{0}{G} 
			& \transition{si_0}{1}{G}
			& \transition{f   }{B}{I} \\
		\hline
		\hline
		$si_0$ 
			& \transition{sp_0}{0}{G} 
			& \transition{sp_2}{1}{G}
			&                         \\
		\hline
		$si_1$ 
			& \transition{sp_1}{0}{G} 
			& \transition{sp_0}{1}{G}
			& \transition{f   }{B}{I} \\
		\hline
		$si_2$ 
			& \transition{sp_2}{0}{G} 
			& \transition{sp_1}{1}{G}
			& \transition{f   }{B}{I} \\
		\hline
	\end{tabular}
\end{center}

