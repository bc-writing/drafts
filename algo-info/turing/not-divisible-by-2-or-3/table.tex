Une méthode simple et généraliste consiste à modifier les tables des transitions précédentes mécaniquement en respectant les deux règles suivantes.
\begin{enumerate}
	\item Toutes les cellules qui contiennent l'état final sont vidées.

	\item Toutes les cases vides, sauf celle correspondant à la case vide et l'état initial, sont remplies pour indiquer un passage à l'état final. 
	      \textbf{La cellule non traitée l'est afin d'éviter d'accepter le mot vide.} 
\end{enumerate}



Ceci nous donne la table des transitions ci-après pour les naturels impairs.
\begin{center}
	\begin{tabular}{|c||c|c|c|}
		\hline
		$\delta$ 
			& $0$ 
			& $1$
			& $B$ \\
		\hline
		\hline
		$q_0$
			& $(\ell_0 , 0 , D)$
			& $(\ell_1 , 1 , D)$
			&                     \\
		\hline
		\hline
		$\ell_0$
			& $(\ell_0 , 0 , D)$
			& $(\ell_1 , 1 , D)$
			&                     \\
		\hline
		$\ell_1$
			& $(\ell_0 , 0 , D)$
			& $(\ell_1 , 1 , D)$
			& $(f      , B , I)$ \\
		\hline
	\end{tabular}
\end{center}


Pour les non multiples de $3$, on obtient la table des transitions suivante.
\begin{center}
	\begin{tabular}{|c||c|c|c|}
		\hline
		$\delta$ 
			& $0$ 
			& $1$
			& $B$ \\
		\hline
		\hline
		$q_0$ 
			& $(d , 0, D)$ 
			& $(d , 1, D)$
			&  \\
		\hline
		$d$ 
			& $(d , 0, D)$ 
			& $(d , 1, D)$
			& $(sp_0, B, G)$ \\
		\hline
		\hline
		$sp_0$ 
			& $(si_0 , 0, G)$ 
			& $(si_1 , 1, G)$
			&                 \\
		\hline
		$sp_1$ 
			& $(si_1 , 0, G)$ 
			& $(si_2 , 1, G)$
			& $(f    , B, I)$ \\
		\hline
		$sp_2$ 
			& $(si_2 , 0, G)$ 
			& $(si_0 , 1, G)$
			& $(f    , B, I)$ \\
		\hline
		\hline
		$si_0$ 
			& $(sp_0 , 0, G)$ 
			& $(sp_2 , 1, G)$
			&                 \\
		\hline
		$si_1$ 
			& $(sp_1 , 0, G)$ 
			& $(sp_0 , 1, G)$
			& $(f    , B, I)$ \\
		\hline
		$si_2$ 
			& $(sp_2 , 0, G)$ 
			& $(sp_1 , 1, G)$
			& $(f    , B, I)$ \\
		\hline
	\end{tabular}
\end{center}

