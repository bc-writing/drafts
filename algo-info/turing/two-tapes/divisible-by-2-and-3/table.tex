Une 1\iere{} idée consiste à utiliser la machine qui repère les multiples de $3$ et si elle finit de passer à l'analyse avec la machine repérant les pairs
\footnote{
	On commence par les multiples de $3$ car notre machine en cas de succès positionne la tête de lecture juste à gauche du premier chiffre. Ceci va nous simplifier un peu la tâche.
}.
Cette démarche est valable car les machines que nous avons présentées ne modifient pas la donnée en entrée
\footnote{
	Ceci montre aussi comment composer séquentiellement des machines.
}.
Ceci nous conduit à la table des transitions ci-après où le nouvel état $m$ indique qu'un multiple de $3$ a été repéré.

\begin{center}
	\begin{tabular}{|c||c|c|c|}
% MUTIPLE OF 3
		\hline
		$\delta$ 
			& $0$ 
			& $1$
			& $B$ \\
		\hline
		\hline
		$q_0$ 
			& $(d , 0, D)$ 
			& $(d , 1, D)$
			&  \\
		\hline
		$d$ 
			& $(d , 0, D)$ 
			& $(d , 1, D)$
			& $(sp_0, B, G)$ \\
		\hline
		\hline
		$sp_0$ 
			& $(si_0 , 0, G)$ 
			& $(si_1 , 1, G)$
			& $(m    , B, I)$ \\
		\hline
		$sp_1$ 
			& $(si_1 , 0, G)$ 
			& $(si_2 , 1, G)$
			&                 \\
		\hline
		$sp_2$ 
			& $(si_2 , 0, G)$ 
			& $(si_0 , 1, G)$
			&                 \\
		\hline
		\hline
		$si_0$ 
			& $(sp_0 , 0, G)$ 
			& $(sp_2 , 1, G)$
			& $(m    , B, I)$ \\
		\hline
		$si_1$ 
			& $(sp_1 , 0, G)$ 
			& $(sp_0 , 1, G)$
			&                 \\
		\hline
		$si_2$ 
			& $(sp_2 , 0, G)$ 
			& $(sp_1 , 1, G)$
			&                 \\
		\hline
% PARITY
		\hline
		$m$
			& $(\ell_0 , 0 , D)$
			& $(\ell_1 , 1 , D)$
			& $(m      , B , D)$ \\
		\hline
		\hline
		$\ell_0$
			& $(\ell_0 , 0 , D)$
			& $(\ell_1 , 1 , D)$
			& $(f      , B , I)$ \\
		\hline
		$\ell_1$
			& $(\ell_0 , 0 , D)$
			& $(\ell_1 , 1 , D)$
			&                    \\
		\hline
	\end{tabular}
\end{center}


Mais que faire si nous avons des machines qui agissent sur la donnée en entrée ?
Une réponse simple est de travailler avec deux bandes au lieu d'une seule en commençant toujours par dupliquer la donnée sur la 2\ieme{} bande.
Voici ce que cela peut donner schématiquement \emph{(nous mettons la 1\iere{} bande au-dessus)}.

\begin{multicols}{2}

% ????

\phantom{\emptybox\emptybox}%
	\deah

\emptybox\emptybox%
	\boxit{1}\boxit{1}\boxit{0}%
\emptybox\emptybox

\emptybox\emptybox%
	\emptybox\emptybox\emptybox%
\emptybox\emptybox

\phantom{\emptybox\emptybox}%
	\head


\medskip % ????

\phantom{\emptybox\emptybox\emptybox}%
	\deah

\emptybox\emptybox%
	\boxit{1}\boxit{1}\boxit{0}%
\emptybox\emptybox

\emptybox\emptybox%
	\nboxit{1}\emptybox\emptybox%
\emptybox\emptybox

\phantom{\emptybox\emptybox\emptybox}%
	\head


\medskip % ????

\phantom{\emptybox\emptybox\emptybox\emptybox}%
	\deah

\emptybox\emptybox%
	\boxit{1}\boxit{1}\boxit{0}%
\emptybox\emptybox

\emptybox\emptybox%
	\nboxit{1}\nboxit{1}\emptybox%
\emptybox\emptybox

\phantom{\emptybox\emptybox\emptybox\emptybox}%
	\head


\medskip % ????

\phantom{\emptybox\emptybox\emptybox\emptybox\emptybox}%
	\deah

\emptybox\emptybox%
	\boxit{1}\boxit{1}\boxit{0}%
\emptybox\emptybox

\emptybox\emptybox%
	\nboxit{1}\nboxit{1}\nboxit{0}%
\emptybox\emptybox

\phantom{\emptybox\emptybox\emptybox\emptybox\emptybox}%
	\head

\end{multicols}


Une fois ceci fait, on repositionne les deux têtes à gauche au début des données.
Il suffit alors de faire agir une sous-machine sur une bande puis si elle finit de passer la main à la seconde sous-machine sur l'autre bande.
Ceci nous amène à la table des transitions suivantes où $\twocoord{1}{B}$ indique la lecture d'une case du haut avec un $1$ et d'une case du bas vide, et on utilise la même convention pour les déplacements.
Les nouveaux états utilisés sont
$c$ pour \myquote{copier}, 
$g$ pour \myquote{retourner à gauche},
$t_m$ pour \myquote{tester un multiple de $3$} et
$t_p$ pour \myquote{tester un pair}.
Nous avons éclaté la table en plusieurs parties car le procédé choisi, bien qu'automatisable, fait exploser le nombre de cas à traiter.
De plus pour alléger la présentation, nous utilisons $\twocoord{1}{\bullet}$ pour indiquer un $1$ au-dessus de n'importe quelle case
\emph{(lorsque ce raccourci est utilisé pour une règle d'écriture, ceci indique que le caractère correspondant à $\bullet$ est inchangé)}.

\begin{center}
	\emph{\small Phase 1 : copie de l'entrée.}
	
	\smallskip
	\renewcommand{\arraystretch}{1.25}
	\begin{tabular}{|c||c|c|c|c|c|}
		\hline
		$\delta$ 
			& $\twocoord{0}{B}$ 
			& $\twocoord{1}{B}$ 
			& $\twocoord{B}{B}$ 
			& $\twocoord{0}{0}$ 
			& $\twocoord{1}{1}$ \\
		\hline
		\hline
		$q_0$ 
			& $\left( c, \twocoord{0}{0}, \twocoord{D}{D} \right)$ 
			& $\left( c, \twocoord{1}{1}, \twocoord{D}{D} \right)$
			&                   
			&                   
			&                                                      \\
		\hline
		$c$ 
			& $\left( c, \twocoord{0}{0}, \twocoord{D}{D} \right)$ 
			& $\left( c, \twocoord{1}{1}, \twocoord{D}{D} \right)$
			& $\left( g, \twocoord{B}{B}, \twocoord{G}{G} \right)$
			&                   
			&                                                      \\
		\hline
		$g$ 
			&                     
			&                   
			& $\left( t_m, \twocoord{B}{B}, \twocoord{D}{D} \right)$
			& $\left( g  , \twocoord{0}{0}, \twocoord{G}{G} \right)$ 
			& $\left( g  , \twocoord{1}{1}, \twocoord{G}{G} \right)$ \\
		\hline
	\end{tabular}
	\renewcommand{\arraystretch}{1}
\end{center}



\begin{center}
	\emph{\small Phase 2 : a-t-on un multiple de $3$ via la bande du haut ?}
	
	\smallskip
	\renewcommand{\arraystretch}{1.25}
	\begin{tabular}{|c||c|c|c|}
		\hline
		$\delta$ 
			& $\twocoord{0}{\bullet}$ 
			& $\twocoord{1}{\bullet}$ 
			& $\twocoord{B}{\bullet}$  \\
		\hline
		\hline
		$t_m$ 
			& $\left( d , \twocoord{0}{\bullet}, \twocoord{D}{I} \right)$ 
			& $\left( d , \twocoord{1}{\bullet}, \twocoord{D}{I} \right)$
			&  \\
		\hline
		$d$ 
			& $\left( d   , \twocoord{0}{\bullet}, \twocoord{D}{I} \right)$ 
			& $\left( d   , \twocoord{1}{\bullet}, \twocoord{D}{I} \right)$
			& $\left( sp_0, \twocoord{B}{\bullet}, \twocoord{G}{I} \right)$ \\
		\hline
		\hline
		$sp_0$ 
			& $\left( si_0 , \twocoord{0}{\bullet}, \twocoord{G}{I} \right)$ 
			& $\left( si_1 , \twocoord{1}{\bullet}, \twocoord{G}{I} \right)$
			& $\left( t_p  , \twocoord{B}{\bullet}, \twocoord{I}{I} \right)$ \\
		\hline
		$sp_1$ 
			& $\left( si_1 , \twocoord{0}{\bullet}, \twocoord{G}{I} \right)$ 
			& $\left( si_2 , \twocoord{1}{\bullet}, \twocoord{G}{I} \right)$
			&                 \\
		\hline
		$sp_2$ 
			& $\left( si_2 , \twocoord{0}{\bullet}, \twocoord{G}{I} \right)$ 
			& $\left( si_0 , \twocoord{1}{\bullet}, \twocoord{G}{I} \right)$
			&                 \\
		\hline
		\hline
		$si_0$ 
			& $\left( sp_0 , \twocoord{0}{\bullet}, \twocoord{G}{I} \right)$ 
			& $\left( sp_2 , \twocoord{1}{\bullet}, \twocoord{G}{I} \right)$
			& $\left( t_p  , \twocoord{B}{\bullet}, \twocoord{I}{I} \right)$ \\
		\hline
		$si_1$ 
			& $\left( sp_1 , \twocoord{0}{\bullet}, \twocoord{G}{I} \right)$ 
			& $\left( sp_0 , \twocoord{1}{\bullet}, \twocoord{G}{I} \right)$
			&                 \\
		\hline
		$si_2$ 
			& $\left( sp_2 , \twocoord{0}{\bullet}, \twocoord{G}{I} \right)$ 
			& $\left( sp_1 , \twocoord{1}{\bullet}, \twocoord{G}{I} \right)$
			&                 \\
		\hline
	\end{tabular}
	\renewcommand{\arraystretch}{1}
\end{center}



\newpage


\begin{center}
	\emph{\small Phase 3 : a-t-on un multiple de $3$ qui est aussi pair via la bande du bas ?}
	
	\smallskip
	\renewcommand{\arraystretch}{1.25}
	\begin{tabular}{|c||c|c|c|}
		\hline
		$\delta$ 
			& $\twocoord{\bullet}{0}$ 
			& $\twocoord{\bullet}{1}$ 
			& $\twocoord{\bullet}{B}$  \\
		\hline
		\hline
		$t_p$
			& $\left( \ell_0 , \twocoord{\bullet}{0} , \twocoord{I}{D} \right)$
			& $\left( \ell_1 , \twocoord{\bullet}{1} , \twocoord{I}{D} \right)$
			&                                                                   \\
		\hline
		\hline
		$\ell_0$
			& $\left( \ell_0 , \twocoord{\bullet}{0} , \twocoord{I}{D} \right)$
			& $\left( \ell_1 , \twocoord{\bullet}{1} , \twocoord{I}{D} \right)$
			& $\left( f      , \twocoord{\bullet}{B} , \twocoord{I}{I} \right)$ \\
		\hline
		$\ell_1$
			& $\left( \ell_0 , \twocoord{\bullet}{0} , \twocoord{I}{D} \right)$
			& $\left( \ell_1 , \twocoord{\bullet}{1} , \twocoord{I}{D} \right)$
			&                                                                   \\
		\hline
	\end{tabular}
	\renewcommand{\arraystretch}{1}
\end{center}