On cherche ici à faire une table pour un ou exclusif.
En utilisant l'identité booléenne $A \boolope{ OUEX } B \eqid* (\boolope{NON } A \boolope{ ET } B) \boolope{ OU } (A \boolope{ ET NON } B)$, nous pouvons appliquer ce que nous avons utilisé précédemment pour traduire les opérateurs logiques $\boolope{ET}$, $\boolope{OU}$ et $\boolope{NON}$.
Bien que théoriquement correct, ce raisonnement automatique va nous conduire à une trop \myquote{grosse} table des transitions.


\medskip


Pour obtenir une table de taille raisonnable, nous allons ajouter une nouvelle bande.
Expliquons comment faire automatiquement via la table \myquote{optimisée} de la section \ref{2-or-3} \emph{(la méthode présentée est généralisable aux tables à plusieurs bandes)}.
\begin{enumerate}
	\item La bande supplémentaire va juste nous servir à \myquote{compter} les cas gagnants.

	\item Nous allons court-circuiter le 1\ier{} cas final pour aller à l'état $sp_0$  tout en gardant la trace du succès du 1\ier{} test.

	\item Il reste à gérer les états bloquants de la 2\ieme{} machine lorsque la 1\iere{} a fonctionné.
	      Dans l'optique d'utilisation séquentielle de machines, nous remettons à zéro la bande supplémentaire.
\end{enumerate}


\medskip

Nous utilisons $\twocoord{0}{\bullet}$ pour simplifier la table. Dans cette écriture, {\scriptsize$\bullet$} indique un caractère quelconque. Dans une transition, il indique le caractère initial qui est donc inchangé.


\begin{center}
	\emph{\small Phase 1 : a-t-on un pair via la bande du bas ?}
	
	\smallskip
	\begin{tabular}{|c||c|c|c|}
% MUTIPLE OF 3
		\hline
		$\delta$ 
			& $\twocoord{0}{B}$ 
			& $\twocoord{1}{B}$
			& $\twocoord{B}{B}$ \\
		\hline
		\hline
		$q_0$ 
			& \transition{\ell_0}{\twocoord{0}{B}}{\twocoord{D}{I}} 
			& \transition{\ell_1}{\twocoord{1}{B}}{\twocoord{D}{I}}
			&                                                       \\
		\hline
		$\ell_0$
			& \transition{\ell_0 }{\twocoord{0}{B}}{\twocoord{D}{I}} 
			& \transition{\ell_1 }{\twocoord{1}{B}}{\twocoord{D}{I}}
			& \transition{sp_0   }{\twocoord{B}{1}}{\twocoord{G}{I}} \\
		\hline
		$\ell_1$
			& \transition{\ell_0}{\twocoord{0}{B}}{\twocoord{D}{I}} 
			& \transition{\ell_1}{\twocoord{1}{B}}{\twocoord{D}{I}}
			& \transition{sp_0  }{\twocoord{B}{B}}{\twocoord{G}{I}} \\
		\hline
	\end{tabular}
\end{center}



\begin{center}
	\emph{\small Phase 2 : a-t-on un multiple de $3$ via la bande du bas ?}
	
	\smallskip
	\begin{tabular}{|c||c|c|c|c|}
% MUTIPLE OF 3
		\hline
		$\delta$ 
			& $\twocoord{0}{\bullet}$ 
			& $\twocoord{1}{\bullet}$
			& $\twocoord{B}{B}$       
			& $\twocoord{B}{1}$ \\
		\hline
		\hline
		$sp_0$ 
			& \transition{si_0}{\twocoord{0}{\bullet}}{\twocoord{G}{I}} 
			& \transition{si_1}{\twocoord{1}{\bullet}}{\twocoord{G}{I}}
			& \transition{f   }{\twocoord{B}{B}      }{\twocoord{I}{I}}
			&                                                           \\
		\hline
		$sp_1$ 
			& \transition{si_1}{\twocoord{0}{\bullet}}{\twocoord{G}{I}} 
			& \transition{si_2}{\twocoord{1}{\bullet}}{\twocoord{G}{I}}
			&
			& \transition{f   }{\twocoord{B}{B}      }{\twocoord{I}{I}} \\
		\hline
		$sp_2$ 
			& \transition{si_2}{\twocoord{0}{\bullet}}{\twocoord{G}{I}} 
			& \transition{si_0}{\twocoord{1}{\bullet}}{\twocoord{G}{I}}
			&
			& \transition{f   }{\twocoord{B}{B}}{\twocoord{I}{I}}       \\
		\hline
		\hline
		$si_0$ 
			& \transition{sp_0}{\twocoord{0}{\bullet}}{\twocoord{G}{I}} 
			& \transition{sp_2}{\twocoord{1}{\bullet}}{\twocoord{G}{I}}
			& \transition{f   }{\twocoord{B}{B}      }{\twocoord{I}{I}}
			&                                                           \\
		\hline
		$si_1$ 
			& \transition{sp_1}{\twocoord{0}{\bullet}}{\twocoord{G}{I}} 
			& \transition{sp_0}{\twocoord{1}{\bullet}}{\twocoord{G}{I}}
			&
			& \transition{f   }{\twocoord{B}{B}      }{\twocoord{I}{I}} \\
		\hline
		$si_2$ 
			& \transition{sp_2}{\twocoord{0}{\bullet}}{\twocoord{G}{I}} 
			& \transition{sp_1}{\twocoord{1}{\bullet}}{\twocoord{G}{I}}
			&
			& \transition{f   }{\twocoord{B}{B}      }{\twocoord{I}{I}} \\
		\hline
	\end{tabular}
\end{center}

