Il a été facile de résoudre la problème avec trois bandes. Essayons de voir si l'on peut se limiter à une seule bande
\footnote{
	Théoriquement on sait que toute machine à $n$ bandes peut être traduite en une machine à une seule bande.
	Malheureusement le procédé ne produit pas forcément des machines qu'un humain aurait conçu tout seul donc nous allons laisser de côté ce procédé.
}
en n'utilisant pas la lettre supplémentaire X
\footnote{
	Là est le mini-défi.
}.
Voici les grandes lignes de la méthode.

\begin{multicols}{2}
%
\emptybox\emptybox%
	\wboxit{\text{b}}\fboxit{\text{a}}\wboxit{\text{b}}\wboxit{\text{b}}\wboxit{\text{b}}\fboxit{\text{a}}\wboxit{\text{b}}\fboxit{\text{a}}\wboxit{\text{b}}%
\emptybox\emptybox

\phantom{\emptybox\emptybox}%
	\head


\medskip %


\emptybox\emptybox%
	\fboxit{\text{a}}\fboxit{\text{a}}\fboxit{\text{a}}\wboxit{\text{b}}\wboxit{\text{b}}\wboxit{\text{b}}\wboxit{\text{b}}\wboxit{\text{b}}\wboxit{\text{b}}%
\emptybox\emptybox

\phantom{\emptybox\emptybox\emptybox\emptybox\emptybox\emptybox\emptybox\emptybox\emptybox\emptybox\emptybox}%
	\head


\medskip %


\emptybox\emptybox%
	\fboxit{\text{a}}\fboxit{\text{a}}\fboxit{\text{a}}\wboxit{\text{b}}\wboxit{\text{b}}\wboxit{\text{b}}\wboxit{\text{b}}\emptybox\emptybox%
\emptybox\emptybox

\phantom{\emptybox}%
	\head


\medskip %


\emptybox\emptybox%
	\emptybox\fboxit{\text{a}}\fboxit{\text{a}}\wboxit{\text{b}}\wboxit{\text{b}}\wboxit{\text{b}}\wboxit{\text{b}}\emptybox\emptybox%
\emptybox\emptybox

\phantom{\emptybox\emptybox\emptybox\emptybox\emptybox\emptybox\emptybox\emptybox\emptybox}%
	\head


\medskip %


\emptybox\emptybox%
	\emptybox\fboxit{\text{a}}\fboxit{\text{a}}\wboxit{\text{b}}\wboxit{\text{b}}\emptybox\emptybox\emptybox\emptybox%
\emptybox\emptybox

\phantom{\emptybox\emptybox}%
	\head


\medskip %


\emptybox\emptybox%
	\emptybox\emptybox\fboxit{\text{a}}\wboxit{\text{b}}\wboxit{\text{b}}\emptybox\emptybox\emptybox\emptybox%
\emptybox\emptybox

\phantom{\emptybox\emptybox\emptybox\emptybox\emptybox\emptybox\emptybox}%
	\head


\medskip %


\emptybox\emptybox%
	\emptybox\emptybox\fboxit{\text{a}}\emptybox\emptybox\emptybox\emptybox\emptybox%
\emptybox\emptybox

\phantom{\emptybox\emptybox\emptybox}%
	\head


\medskip %


\emptybox\emptybox%
	\emptybox\emptybox\emptybox\emptybox\emptybox\emptybox\emptybox\emptybox%
\emptybox\emptybox

\phantom{\emptybox\emptybox\emptybox\emptybox\emptybox}%
	\head

\end{multicols}


On aboutit alors à la table des transitions suivante où le plus délicat est la réorganisation des a et des b
\emph{(on effectue un tri à bulles)}.
L'état $g$ et ceux de type $s$ sont chargés de cette réorganisation, tandis que les états de type $e$ s'occupent de la procédure d'effacement.


\begin{center}
	\begin{tabular}{|c||c|c|c|}
% MUTIPLE OF 3
		\hline
		$\delta$ 
			& a 
			& b
			& $B$ \\
		\hline
		\hline
		$q_0$ 
			& \transition{s_a}{\text{a}}{D} 
			& \transition{s_b}{\text{b}}{D}
			&                        \\
		\hline
		$s_a$ 
			& \transition{s_a}{\text{a}}{D} 
			& \transition{s_b}{\text{b}}{D}
			& \transition{e_b}{B       }{G} \\
		\hline
		$s_b$ 
			& \transition{s^{\,\prime}_b}{\text{b}}{G}
			& \transition{s_b           }{\text{b}}{D}
			& \transition{e_b           }{B       }{G} \\
		\hline
		$s^{\,\prime}_b$ 
			&
			& \transition{s^{\,\prime\prime}_b}{\text{a}}{D}
			&                                         \\
		\hline
		$s^{\,\prime\prime}_b$ 
			& \transition{s^{\,\prime}_b       }{\text{b}}{G}
			& \transition{s^{\,\prime\prime}_b }{\text{b}}{D}
			& \transition{g                    }{\text{b}}{G} \\
		\hline
		$g$ 
			& \transition{g  }{\text{a}}{G}
			& \transition{g  }{\text{b}}{G}
			& \transition{q_0}{B       }{D} \\
		\hline
		\hline
		$e_b$ 
			&
			& \transition{e^{\,\prime}_b}{B}{G} 
			& \transition{f             }{B}{I} \\
		\hline
		$e^{\,\prime}_b$
			&
			& \transition{g_a}{B}{G} 
			&                        \\
		\hline
		$g_a$
			& \transition{g_a}{\text{a}}{G}
			& \transition{g_a}{\text{b}}{G}
			& \transition{e_a}{B       }{D} \\
		\hline
		$e_a$
			& \transition{d_b}{B}{D}
			& 
			&                        \\
		\hline
		$d_b$
			& \transition{d_b}{\text{a}}{D}
			& \transition{d_b}{\text{b}}{D}
			& \transition{e_b}{B       }{G} \\
		\hline
	\end{tabular}
\end{center}