Formalisons les étapes utilisées dans les exemples précédents pour des mots sur l'alphabet $\setgene{\text{a} ; \text{b} ; \text{c}}$ \emph{(la méthode est généralisable sans souci)}. Nous avons besoin des états suivants.

\begin{itemize}[label = \small\textbullet]
	\item $q_0$ est l'état initial mais aussi celui pour chaque début de nouvelle recherche de lettres identiques en début et en fin de mot.

	\item Il nous faut des états pour rechercher une lettre identique à droite, ce sera les états $d_\text{a}$ , $d_\text{b}$ et $d_\text{c}$ .
	
	\item Les états précédents vont en fait nous permettre d'arriver sur la case vide à droite de la dernière lettre. Nous ajoutons donc les états $v_\text{a}$ , $v_\text{b}$ et $v_\text{c}$ pour valider ou non la dernière lettre.
	
	\item Il nous faut aussi un état $g$ pour retourner vers la gauche recommencer l'analyse avec la lettre suivante ainsi qu'un état $e$ pour effacer la 1\iere{} lettre qui avait été repérée.
	
	\item Enfin l'état $f$ sera notre état final indiquant qu'un mot est bien un palindrome.
\end{itemize}

Voici finalement la table des transitions de notre machine de Turing pour repérer les palindromes sur l'alphabet $\setgene{\text{a} ; \text{b} ; \text{c}}$. Notez bien les transitions $(q_0 , B)$ et $(e , B)$ importantes pour arriver à l'état final.
\begin{center}
	\begin{tabular}{|c||c|c|c|c|}
		\hline
		$\delta$ 
			& a 
			& b 
			& c 
			& $B$ \\
		\hline
		\hline
		$q_0$
			& $(d_\text{a} , \text{a} , D)$
			& $(d_\text{b} , \text{b} , D)$
			& $(d_\text{c} , \text{c} , D)$
			& $(f          , B        , D)$ \\
		\hline
		\hline
		$d_a$
			& $(d_\text{a} , \text{a} , D)$
			& $(d_\text{a} , \text{b} , D)$
			& $(d_\text{a} , \text{c} , D)$
			& $(v_\text{a} , B        , G)$ \\
		\hline
		$d_b$
			& $(d_\text{b} , \text{a} , D)$
			& $(d_\text{b} , \text{b} , D)$
			& $(d_\text{b} , \text{c} , D)$
			& $(v_\text{b} , B        , G)$ \\
		\hline
		$d_c$
			& $(d_\text{c} , \text{a} , D)$
			& $(d_\text{c} , \text{b} , D)$
			& $(d_\text{c} , \text{c} , D)$
			& $(v_\text{c} , B        , G)$ \\
		\hline
		\hline
		$v_a$
			& $(g , B , G)$
			& 
			& 
			&  \\
		\hline
		$v_b$
			& 
			& $(g , B , G)$
			& 
			&  \\
		\hline
		$v_c$
			& 
			& 
			& $(g , B , G)$
			&  \\
		\hline
		\hline
		$g$
			& $(g , \text{a} , G)$
			& $(g , \text{b} , G)$
			& $(g , \text{c} , G)$
			& $(e , B        , D)$ \\
		\hline
		$e$
			& $(q_0 , B , D)$
			& $(q_0 , B , D)$
			& $(q_0 , B , D)$
			& $(q_0 , B , D)$ \\
		\hline
	\end{tabular}
\end{center}

