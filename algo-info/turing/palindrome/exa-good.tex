On reprend la même méthodologie que dans la section précédente en la complétant sur la fin. Voici les étapes principales.

\begin{multicols}{2}

\emptybox\emptybox%
	\boxit{a}\boxit{b}\boxit{b}\boxit{c}\boxit{b}\boxit{b}\boxit{a}%
\emptybox\emptybox

\phantom{%
	\emptybox\emptybox}%
	\head
	

\medskip % 1ER EFFACEEMENT

\emptybox\emptybox%
	\fboxit{a}\boxit{b}\boxit{b}\boxit{c}\boxit{b}\boxit{b}\fboxit{a}%
\emptybox\emptybox

\phantom{%
	\emptybox\emptybox%
	\emptybox\emptybox\emptybox\emptybox\emptybox\emptybox}%
	\head
	

\vfill\null
\columnbreak


\emptybox\emptybox%
	\emptybox\boxit{b}\boxit{b}\boxit{c}\boxit{b}\boxit{b}\emptybox%
\emptybox\emptybox

\phantom{%
	\emptybox\emptybox\emptybox}%
	\head
	

\medskip % 2IÈME EFFACEEMENT

\emptybox\emptybox%
	\emptybox\fboxit{b}\boxit{b}\boxit{c}\boxit{b}\fboxit{b}\emptybox%
\emptybox\emptybox

\phantom{%
	\emptybox\emptybox%
	\emptybox\emptybox\emptybox\emptybox\emptybox}%
	\head

\vfill\null
\end{multicols}


\begin{multicols}{2}

\emptybox\emptybox%
	\emptybox\emptybox\boxit{b}\boxit{c}\boxit{b}\emptybox\emptybox%
\emptybox\emptybox

\phantom{%
	\emptybox\emptybox\emptybox\emptybox}%
	\head


\medskip % 3IÈME EFFACEEMENT

\emptybox\emptybox%
	\emptybox\emptybox\fboxit{b}\boxit{c}\fboxit{b}\emptybox\emptybox%
\emptybox\emptybox

\phantom{%
	\emptybox\emptybox%
	\emptybox\emptybox\emptybox\emptybox}%
	\head
	

\medskip

\emptybox\emptybox%
	\emptybox\emptybox\emptybox\boxit{c}\emptybox\emptybox\emptybox%
\emptybox\emptybox

\phantom{%
	\emptybox\emptybox\emptybox\emptybox\emptybox}%
	\head
	

\vfill\null
\columnbreak


% DERNIER EFFACEEMENT

\emptybox\emptybox%
	\emptybox\emptybox\emptybox\fboxit{c}\emptybox\emptybox\emptybox%
\emptybox\emptybox

\phantom{%
	\emptybox\emptybox%
	\emptybox\emptybox\emptybox}%
	\head
	

\medskip

\emptybox\emptybox%
	\emptybox\emptybox\emptybox\emptybox\emptybox\emptybox\emptybox%
\emptybox\emptybox

\phantom{%
	\emptybox\emptybox\emptybox\emptybox\emptybox\emptybox}%
	\head

\vfill\null
\end{multicols}


\vspace{-1em}

Quelle est la nouveauté ici ? Il faut juste ajouter après chaque nouvelle phase suite à un effacement la règle suivante : si la tête de lecture est sur une case vide alors on est dans un état final qui marque la validation d'un mot palindromique.