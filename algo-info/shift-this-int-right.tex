\documentclass[12pt]{amsart}
\usepackage[T1]{fontenc}
\usepackage[utf8]{inputenc}

\usepackage[top=1.95cm, bottom=1.95cm, left=2.35cm, right=2.35cm]{geometry}

\usepackage{hyperref}
\usepackage{enumitem}
\usepackage{tcolorbox}
\usepackage{multicol}
\usepackage{fancyvrb}
\usepackage{xstring}
\usepackage{amsmath}
\usepackage[french]{babel}
\usepackage[
    type={CC},
    modifier={by-nc-sa},
	version={4.0},
]{doclicense}
\usepackage{textcomp}
\usepackage{xcolor}
\usepackage{tcolorbox}
\usepackage{pgffor}
      
\usepackage{lymath}
\usepackage{circledsteps}
    
\newtheorem{fact}{Fait}%[section]


\newtheorem{definition}{Définition}%[section]
\newtheorem{theorem}{Théorème}%[section]
\newtheorem{example}{Exemple}%[section]
\newtheorem{exercise}{Exercice}%[section]
\newtheorem{remark}{Remarque}%[section]
\newtheorem*{notations}{Notations}

\setlength\parindent{0pt}




\DeclareMathMacro{\ediv}{\operatorname{\div}}


\newcommand\centerit[1]{%%
	\smallskip
	
	\begin{center}
		#1
	\end{center}
}


\newcommand\sign[1]{\fbox{\scriptsize $#1$}}
\newcommand\phantomsign{\phantom{\sign{-}}}


% Source pour le \@tfor :
% 	* https://tex.stackexchange.com/a/253205/6880
\makeatletter
	\newcommand\equality{\@ifstar{\@equality@star@}{\@equality@no@star@}}

	\newcommand\@equality@star@[1]{
		\smallskip
	
		$=$ {} \qquad  $\leftarrow$ #1
	
		\smallskip
	}

	\newcommand\@equality@no@star@{
		\smallskip
	
		$=$

		\smallskip
	}

% Nothing special for the cell...	
	\newcommand\elasticbox[1]{%
		\fcolorbox{black}{white}{\normalfont #1}%
	}

	\newcommand\fixedboxcolored[3]{%
		\fcolorbox{#1}{#2}{\makebox[1em]{\normalfont\vphantom{?}#3}}%
	}

	\newcommand\fixedbox[1]{%
		\fixedboxcolored{black}{white}{\makebox[1em]{\normalfont\vphantom{?}#1}}%
	}

% ... will move
	\newcommand\elasticboxwill[1]{%
		\fcolorbox{black}{red!15}{\normalfont #1}%
	}

	\newcommand\fixedboxwill[1]{%
		\fixedboxcolored{black}{red!15}{\normalfont #1}%
	}

% ... has moved
	\newcommand\elasticboxhas[1]{%
		\fcolorbox{black}{black!10}{\normalfont #1}%
	}
	
	\newcommand\fixedboxhas[1]{%
		\fixedboxcolored{black}{black!10}{\normalfont #1}%
	}

% Let's go !
	\newcommand\onefortwocomplemen{\fixedboxcolored{black}{green!15}{\bfseries 1}}


\newcommand\zerofortwocomplemen{\fixedboxcolored{black}{green!15}{\bfseries 0}}


	\newcommand\twocomplement{\fixedboxwill{$\bullet$}}

	\newcommand\@empty@content{\phantom{?}}
	\newcommand\emptycell{\fixedbox{\@empty@content}}
	\newcommand\one{\fixedbox{\bfseries 1}}
	\newcommand\zero{\fixedbox{\bfseries 0}}
	
	\newcommand\@ellipsis@{\,\,\vphantom{?}$\cdots$\,\,}
	\newcommand\myellipsis{\elasticbox{\@ellipsis@}}

% ... will move
	\newcommand\emptycellwill{\fixedboxwill{\@empty@content}}
	\newcommand\onewill{\fixedboxwill{\bfseries 1}}
	\newcommand\zerowill{\fixedboxwill{\bfseries 0}}
	\newcommand\myellipsiswill{\elasticboxwill{\@ellipsis@}}
	
% ... has moved
	\newcommand\emptycellhas{\fixedboxhas{\@empty@content}}
	\newcommand\onehas{\fixedboxhas{\bfseries 1}}
	\newcommand\zerohas{\fixedboxhas{\bfseries 0}}
	\newcommand\myellipsishas{\elasticboxhas{\@ellipsis@}}


% 0 : 0 normal
% 1 : 1 normal
% . : vide normal
% - : ellipsis
%
% Z : 0 qui va bouger
% U : 1 qui va bouger
% v : vide qui va bouger
% = : ellipsis qui va bouger
%
% z : 0 qui vient de bouger
% u : 1 qui vient de bouger
% a : vide qui vient de bouger
% + : ellipsis
%
%
% * : 1 pour complément à 2 + 1
% o : complément à 2 + 1
%
% < Ouvre un cadre de surlignement de plusieurs cellules
% < Ferme un cadre de surlignement de plusieurs cellules
	\newcommand\binary[1]{%
		\@tfor\next:=#1\do{%
			\if\next 0\zero{}\else%
		    \if\next 1\one{}\else%
			\if\next .\emptycell{}\else%
			\if\next -\myellipsis{}\else%
			%
			\if\next Z\zerowill{}\else%
		    \if\next U\onewill{}\else%
		    \if\next v\emptycellwill{}\else%
			\if\next =\myellipsiswill{}\else%
		    %
			\if\next z\zerohas{}\else%
		    \if\next u\onehas{}\else%
		    \if\next a\emptycellhas{}\else%
			\if\next +\myellipsishas{}\else%
			%
			\if\next *\onefortwocomplemen{}\else%
			\if\next c\zerofortwocomplemen{}\else%
			\if\next o\twocomplement{}\else%
		    %
\quad {\bfseries [[ illegal character : \, {\normalfont \next{}} \, ]]} \quad % BUG
			\fi%  :
			\fi%  a
			\fi%  u
			\fi%  p
			%
			\fi%  =
			\fi%  v
			\fi%  n
			\fi%  b
			%
			\fi%  -
			\fi%  .
			\fi%  N
			\fi%  B
			%
			\fi%  *
			\fi%  c
			\fi%  o
		}%
	}
	

	\newcommand\autobox[1]{%
      \foreach \i in {1, ..., #1} {%
        \fixedbox{\i}%
      }
	}
	

	\newcommand\binaryplus{\@ifstar{\@binaryplus@star@}{\@binaryplus@no@star@}}

	\newcommand\@binaryplus@no@star@[2]{
		\binary{#1}
		\setbox0=\hbox{#2\unskip}\ifdim\wd0=0pt \else {\small \ $\leftarrow$ #2}\fi
	}

	\newcommand\@binaryplus@star@[2]{
		\@binaryplus@no@star@{#1}{#2}
		
		\StrLen{#1}[\@length@]
		\noindent  
		\autobox{\@length@}
	}

	\newcommand\listbox[1]{%
		\@tfor\next:=#1\do{\fixedbox{\next}}%
	}
\makeatother


 
\begin{document}

\title{BROUILLON - CANDIDAT - Quels décalages entre les entiers signés et les non signés ?}
\author{Christophe BAL}
\date{17 Mars 2019 - 11 Janvier 2020}

\maketitle

\begin{center}
	\itshape
	Document, avec son source \LaTeX, disponible sur la page
	
	\url{https://github.com/bc-writing/drafts}.
\end{center}


\bigskip


\begin{center}
	\hrule\vspace{.3em}
	{
		\fontsize{1.35em}{1em}\selectfont
		\textbf{Mentions \og légales \fg}
	}
			
	\vspace{0.45em}
	\doclicenseThis
	\hrule
\end{center}

	
\bigskip
\setcounter{tocdepth}{2}
\tableofcontents



% ----------------- %


\section{Entiers non signés}

Dans ce document, nous allons travailler uniquement avec des entiers non signés, c'est à dire des naturels, dont l'écriture en base 2 tient sur 8 bits même si les raisonnements tenus se généralisent à 16, 32 ou 64 bits. Ce choix tient juste à des contraintes de lisibilité du document.

\medskip

Voici des exemples de représentations de type little-endian : on part à gauche de la puissance la plus haute, pour nous ce sera toujours $2^7$, puis on va vers la plus basse \emph{(c'est le même principe que l'écriture des naturels en base $10$)}.

\medskip

\binaryplus{10000101}{$133 = 128 + 4 + 1 = 2^7 + 2^2 + 2^0$}

\medskip

\binaryplus{00100110}{$38 = 32 + 4 + 2 = 2^5 + 2^2 + 2^1$}

\medskip

Avec 8 bits on peut représenter les naturels de $0$ à $1 + 2 + 2^2 + \cdots + 2^7 = 2^8 - 1 = 255$.


% ----------------- %


\section{Entiers signés}

\subsection{Comment ça marche ?}

Voici des exemples de représentations toujours sur 8 bits et de type little-endian où la case rouge indiquera toujours un bit de signe.

\medskip

\binaryplus{Z1111000}{$120 = 64 + 32 + 16 + 8 = \fbox{$0 \times \left( -2^7 \right)$} + 2^6 + 2^5 + 2^4 + 2^3$}

\medskip

\binaryplus{U0001000}{$- 120 = -128 + 8 = \fbox{$1 \times \left( -2^7 \right)$} + 2^3$}

\medskip

\binaryplus{U0000000}{$- 128 = \fbox{$1 \times \left( -2^7 \right)$} + 0$}

\medskip

Avec 8 bits on peut représenter les naturels de $-2^7 = -128$ à $1 + 2 + 2^2 + \cdots + 2^6 = 2^7 - 1 = 127$.

\medskip

Ce choix permet d'effectuer des additions relatives toujours sous forme d'addition bit à bit, et donc par conséquent de faire aussi des soustractions via des additions bit à bit. Voici quelques exemples sans dépassement de capacité.
\begin{itemize}[label=\small\textbullet]
	\medskip\item Addition de deux positifs.

		  \smallskip
		  
		  \noindent \phantom{$+$} \binaryplus{Z1010100}{$84 = 64 + 16 + 4$}
	
		  \noindent          $+$  \binaryplus{Z0010001}{$17 = 16 + 1$}
	
		  \noindent          $=$  \binaryplus{Z1100101}{$101 = 64 + 32 + 4 + 1$}


	\medskip\item Addition de deux négatifs \emph{(la retenue finale hors capacité est ignorée)}.

		  \smallskip
		  
		  \noindent \phantom{$+$} \binaryplus{U0101100}{$-84 = \fbox{$-128$} + 44 = \fbox{$-128$} + 32 + 8 + 4$}
	
		  \noindent          $+$  \binaryplus{U1101111}{$-17 = \fbox{$-128$} + 111 = \fbox{$-128$} + 64 + 32 + 8 + 4 + 2 + 1$}
	
		  \noindent          $=$  \binaryplus{U0011011}{$-101 = \fbox{$-128$} + 27 = \fbox{$-128$} + 16 + 8 + 2 + 1$}


	\medskip\item Addition de deux entiers de signes différents.

		  \smallskip
		  
		  \noindent \phantom{$+$} \binaryplus{Z1010100}{$84 = 64 + 16 + 4$}
	
		  \noindent          $+$  \binaryplus{U0011011}{$-101 = \fbox{$-128$} + 27 = 16 + 8 + 2 + 1$}
	
		  \noindent          $=$  \binaryplus{U1101111}{$-17 = \fbox{$-128$} + 111 = -128 + 64 + 32 + 8 + 4 + 2 + 1$}


	\medskip\item Autre addition de deux entiers de signes différents \emph{(la retenue finale hors capacité est ignorée)}.

		  \smallskip
		  
		  \noindent \phantom{$+$} \binaryplus{Z1010100}{$84 = 64 + 16 + 4$}
	
		  \noindent          $+$  \binaryplus{U1000011}{$-61 = \fbox{$-128$} + 67 = 64 + 2 + 1$}
	
		  \noindent          $=$  \binaryplus{Z0010111}{$23 = 16 + 4 + 2 + 1$}
\end{itemize}



\subsection{Qu'est-ce qui motive ces choix ?}

Pour comprendre comment découvrir ce type de procédé, nous allons raisonner en base $10$ et imaginer que nous voulions calculer $189 - 32 = 157$ en utilisant uniquement des additions. Traditionnellement on fait comme suit.
\begin{center}
\begin{tabular}{cccc}
	    & 1 & 8 & 9 \\
	$-$ & 0 & 3 & 2 \\
	\hline
    $=$ & 1 & 5 & 7 \\
\end{tabular}
\end{center}
Maintenant ajoutons $1000$ à $189 - 32$. Nous obtenons :
\begin{center}
\begin{tabular}{ccccc}
	    & 0 & 1 & 8 & 9 \\
	$-$ & 0 & 0 & 3 & 2 \\
	$+$ & 1 & 0 & 0 & 0 \\
	\hline
    $=$ & 1 & 1 & 5 & 7 \\
\end{tabular}
\end{center}
Comme le résultat initial tenait sur les trois chiffres de droite, l'ajout de $1000$ donne un nouveau résultat dont nous savons que les trois chiffres de droite sont ceux de $189 - 32$.
D'autre part, $1000 + 189 - 32 = 189 + 968$ est une simple addition.

\smallskip

Notons aussi que $1000 - 32 = 968 = 967 + 1$ où $9$ , $6$ et $7$ sont les compléments à $9$ de $0$ , $3$ et $2$ respectivement.
L'opération $1000 - 32$ se fait donc rapidement. De plus, la soustraction $1157 - 1000$ revient à ignorer le $1$ tout à droite. En résumé, nous avons fait comme suit où $\bullet$ indique une ligne où a été fait le calcul d'un complément à $9$ plus $1$ :

\begin{center}
\begin{tabular}{ccccc}
	    & 0 & 1 & 8 & 9 \\
	$-$ & 0 & 0 & 3 & 2 \\
	\hline
	\hline
	    & 0 & 1 & 8 & 9 \\
	$+$ & $\bullet$ & 9 & 6 & 8 \\
	\hline
    $=$ & 1 & 1 & 5 & 7 \\
	\hline
	\hline
    $+$ &  & 1 & 5 & 7 \\
\end{tabular}
\end{center}

\smallskip

Très bien mais que se passe-t-il avec $32 - 189 = -157$ qui est négatif ? Testons pour voir.
\begin{center}
\begin{tabular}{ccccc}
	    & 0 & 0 & 3 & 2 \\
	$-$ & 0 & 1 & 8 & 9 \\
	\hline
	\hline
	    & 0 & 0 & 3 & 2 \\
	$+$ & $\bullet$ & 8 & 1 & 1 \\
	\hline
    $=$ & 0 & 8 & 4 & 3 \\
\end{tabular}
\end{center}
Oups ! Nous ne voyons pas directement $157$. C'est normal puisque $843 - 1000 = -157$ n'a pas été effectué.
Dans un tel cas, on note que le chiffre à gauche étant $0$, on effectue un complément à $9$ plus $1$ tout en ajoutant un signe moins afin d'obtenir le résultat final.
En résumé, nous procédons comme suit. 
\begin{center}
\begin{tabular}{ccccc}
	    & 0 & 0 & 3 & 2 \\
	$-$ & 0 & 1 & 8 & 9 \\
	\hline
	\hline
	    & 0 & 0 & 3 & 2 \\
	$+$ & $\bullet$ & 8 & 1 & 1 \\
	\hline
    $=$ & 0 & 8 & 4 & 3 \\
	\hline
	\hline
	$-$ & $\bullet$ & 1 & 5 & 7 \\
\end{tabular}
\end{center}


\smallskip

Ne nous emballons pas trop vite car il reste un cas problématique que nous n'avons pas abordé.
Examinons ce qu'il se passe avec $-32 - 189 = - 221$ ?
\begin{center}
\begin{tabular}{ccccc}
	$-$ & 0 & 0 & 3 & 2 \\
	$-$ & 0 & 1 & 8 & 9 \\
	\hline
	\hline
	    & $\bullet$ & 9 & 6 & 8 \\
	$+$ & $\bullet$ & 8 & 1 & 1 \\
	\hline
    $=$ & 1 & 7 & 7 & 9 \\
\end{tabular}
\end{center}

Nous devons affiner notre règle car sinon ici nous aurions un résultat positif !
En fait, il est facile de voir qu'il suffit de comparer le nombre de compléments à $9$ plus $1$ effectués, c'est à dire le nombre de $1000$ ajoutés, au chiffre tout à gauche qui sera $0$ ou $1$.
Ici nous avons deux compléments et un chiffre $1$ ce qui nous demande d'appliquer $2 - 1 = 1$ complément au résultat final tout en mettant un signe moins.
En résumé, nous avons :
\begin{center}
\begin{tabular}{ccccc}
	$-$ & 0 & 0 & 3 & 2 \\
	$-$ & 0 & 1 & 8 & 9 \\
	\hline
	\hline
	    & $\bullet$ & 9 & 6 & 8 \\
	$+$ & $\bullet$ & 8 & 1 & 1 \\
	\hline
    $=$ & 1 & 7 & 7 & 9 \\
	\hline
	\hline
    $-$ & $\bullet$ & 2 & 2 & 1 \\
\end{tabular}
\end{center}


\medskip


Nous trouvons bien $- 32 - 189 = - 221$. Que c'est beau !
Les deux cas suivants achèvent notre exploration expérimentale.
\begin{multicols}{2}
\begin{center}
\begin{tabular}{ccccc}
	    & 0 & 0 & 3 & 2 \\
	$+$ & 0 & 1 & 8 & 9 \\
	\hline
	$=$ & 0 & 2 & 2 & 1 \\
	\hline
	\hline
	$+$ &   & 2 & 2 & 1 \\
\end{tabular}
\end{center}

\null\vfill

\columnbreak

\begin{center}
\begin{tabular}{ccccc}
	$-$ & 0 & 0 & 3 & 2 \\
	$+$ & 0 & 1 & 8 & 9 \\
	\hline
	\hline
	    & $\bullet$ & 9 & 6 & 8 \\
	$+$ & 0 & 1 & 8 & 9 \\
	\hline
	$=$ & 1 & 1 & 5 & 7 \\
	\hline
	\hline
	$+$ &   & 1 & 5 & 7 \\
\end{tabular}
\end{center}
\end{multicols}


\begin{exercise}
	Dans la méthode ci-dessous, il existe des cas problématiques. Lesquels ?
\end{exercise}


\begin{exercise}
	En laissant les cas problématiques de côté, démontrer le caractère général de la méthode que nous avons juste exposée via quelques exemples.
\end{exercise}




\subsection{Pourquoi ça marche ?}

Revenons à nos entiers signés écrits sur $8$ bits. Reprenons le calcul de $84 - 101 = -17$ en nous inspirant de ce qui a été vu dans la section précédente avec la base $10$. Ici $\bullet$ indique une ligne où a été fait le calcul d'un complément à $1$ plus $1$.

\medskip

\begin{center}
\begin{tabular}{ll}
	    & \!\!\binary{Z1010100}  	\\
	$-$ & \!\!\binary{Z1100101} 	\\[.8ex]
	\hline
	\hline 							\\[-2ex]
	    & \!\!\binary{Z1010100} 	\\
	$+$ & \!\!\binary{o0011011} 	\\[.8ex]
	\hline \\[-2ex]
	$=$ & \!\!\binary{Z1101111} 	\\[.8ex]
	\hline
	\hline 							\\[-2ex]
	$-$ & \!\!\binary{o0010001} 	\\
\end{tabular}
\end{center}

\medskip

Tout s'éclaire ! Reprenons le cas de $84 - 61 = 23$ où l'on avait une retenue à ignorer.

\medskip

\begin{center}
\begin{tabular}{ll}
	    & \!\!\binary{Z1010100}  	\\
	$-$ & \!\!\binary{Z0111101} 	\\[.8ex]
	\hline
	\hline 							\\[-2ex]
	    & \!\!\binary{Z1010100} 	\\
	$+$ & \!\!\binary{o1000011} 	\\[.8ex]
	\hline \\[-2ex]
	$=$ & \!\!\binary{U0010111} 	\\[.8ex]
	\hline
	\hline 							\\[-2ex]
	$+$ & \!\!\binary{Z0010111} 	\\
\end{tabular}
\end{center}

\medskip

Avec le nouvel éclairage, nous notons qu'en fait les bits rouges ne servent qu'à indiquer s'il faut ou non effectuer un complément à $1$ plus $1$ en changeant de signe.
Avec la façon de stocker les entiers signés, nous avons alors les correspondances suivantes où les entiers additionnés $a$ et $b$ sont dans $\intervalC{-128}{127}$ et leur somme aussi, ce qui revient à ne considérer que les cas de non dépassement de capacité.
Ci-après, \binary{*} et les cases sur fond gris indiquent l'utilisation d'un complément à $1$ plus $1$ que nous avions noté \binary{o} ci-dessus, et le symbole $\sim$ est pour le calcul de la somme sans le bit de signe qui lui entre en jeu dans la ligne avec le signe $=$ .
\begin{multicols}{4}
    \begin{center}
	\begin{tabular}{ll}
	    & \!\!\binary{Z-}  		\\
	$-$ & \!\!\binary{Z-} 		\\[.8ex]
	\hline
	\hline 						\\[-2ex]
	    & \!\!\binary{Z-} 		\\
	$+$ & \!\!\binary{*+} 		\\[.8ex]
	\hline \\[-2ex]
	$\sim$ & \!\!\binary{Z-} 	\\[.8ex]
	\hline
	\hline 						\\[-2ex]
	$=$ & \!\!\binary{U-} 		\\
	\end{tabular}
	
	\medskip\itshape\footnotesize
	
	Soustraction 1
	
	de deux naturels
	\end{center}


	\null\vfill
	\columnbreak
	
	
	\begin{center}
	\begin{tabular}{ll}
	    & \!\!\binary{Z-}  		\\
	$-$ & \!\!\binary{Z-} 		\\[.8ex]
	\hline
	\hline 						\\[-2ex]
	    & \!\!\binary{Z-} 		\\
	$+$ & \!\!\binary{*+} 		\\[.8ex]
	\hline \\[-2ex]
	$\sim$ & \!\!\binary{U-} 	\\[.8ex]
	\hline
	\hline 						\\[-2ex]
	$=$ & \!\!\binary{Z-} 		\\
	\end{tabular}
	
	\medskip\itshape\footnotesize
	
	Soustraction 2
	
	de deux naturels
	\end{center}


	\null\vfill
	\columnbreak
	
	
	\begin{center}
	\begin{tabular}{ll}
	$-$ & \!\!\binary{Z-}  		\\
	$-$ & \!\!\binary{Z-} 		\\[.8ex]
	\hline
	\hline 						\\[-2ex]
	    & \!\!\binary{*+} 		\\
	$+$ & \!\!\binary{*+} 		\\[.8ex]
	\hline \\[-2ex]
	$\sim$ & \!\!\binary{Z-} 	\\[.8ex]
	\hline
	\hline 						\\[-2ex]
	$=$ & \!\!\binary{Z-} 		\\
	\end{tabular}
	
	\medskip\itshape\footnotesize
	
	Addition 1 de
	
	deux relatifs négatifs
	\end{center}


	\null\vfill
	\columnbreak
	
	
	\begin{center}
	\begin{tabular}{ll}
	$-$ & \!\!\binary{Z-}  		\\
	$-$ & \!\!\binary{Z-} 		\\[.8ex]
	\hline
	\hline 						\\[-2ex]
	    & \!\!\binary{*+} 		\\
	$+$ & \!\!\binary{*+} 		\\[.8ex]
	\hline \\[-2ex]
	$\sim$ & \!\!\binary{U-} 	\\[.8ex]
	\hline
	\hline 						\\[-2ex]
	$=$ & \!\!\binary{U-} 		\\
	\end{tabular}
	
	\medskip\itshape\footnotesize
	
	Addition 2 de
	
	deux relatifs négatifs
	\end{center}


	\null\vfill
\end{multicols}

\vspace{-1.5em}

Les soustractions et l'addition 2 ne posent aucun souci.
Par contre l'addition 1 est problématique avec son résultat positif. 
En fait cette addition contredit notre hypothèse de non dépassement de capacité. Nous allons voir pourquoi.

\medskip

Le cas de l'addition 1 correspond à $(a ; b) \in \intervalC{-128}{-1}^2$ tel que $(128 + a) + (128 + b) \in \intervalC{0}{127}$ soit $0 \leq 256 + a + b \leq 127$ \emph{i.e.} $-256 \leq a + b \leq -129$ ce qui correspond à un dépassement de capacité comme annoncé.
 
\medskip

Ceci achève de démontrer la validité des procédés d'addition et de soustraction d'entiers signés dans les cas de non dépassement de capacité.
Notez que le cas évident d'une addition de deux naturels a été omis, et aussi que $-a + b = b - a = b + (-a)$ et le fait qu'un changement de signe n'est autre qu'un complément à $1$ plus $1$ permettent de compléter les cas non indiqués ci-dessus.

\begin{exercise}
	Étudiez plus généralement le cas d'une base $b \in \NN_{\geq 3}$ quelconque.
\end{exercise}




% ----------------- %


\section{Division par $2$ et décalage à droite}

\subsection{Entiers non signés}

Lorsque l'on calcule $(n\, \ediv{} \, 2)$ le quotient entier par défaut d'un naturel non signé $n$ divisé par $2$, il suffit de retirer son bit de poids faible. Voici un exemple.


\medskip

\binaryplus{11010111}{$n_0 = 128 + 64 + 16 + 4 + 2 + 1 = 215$}

\medskip

\binaryplus{z1101011}{$n_1 = n_0 \, \ediv{} \, 2 = 107$}

\medskip

\binaryplus{zz110101}{$n_2 = n_1 \, \ediv{} \, 2 = 53$}

\medskip

\binaryplus{zzz11010}{$n_3 = n_2 \, \ediv{} \, 2 = 26$}

\medskip


Chaque étape correspond à un décalage vers la droite tout en complétant par un zéro à gauche.


\subsection{Entiers signés}

Examinons ce qu'il se passe lorsque l'on calcule  $(n\, \ediv{} \, 2)$ le quotient entier par défaut d'un naturel signé $n \,\text{<}\, 0$ divisé par $2$.


\medskip

\binaryplus{U1010111}{$n_0 = -128 + 64 + 16 + 4 + 2 + 1 = -41$}

\medskip

\binaryplus{uU101011}{$n_1 = n_0 \, \ediv{} \, 2 = -21 = -128 + 64 + 32 + 8 + 2 + 1$}

\medskip


On en déduit que quelque soit le signe de l'entier signé, le quotient entier par défaut correspond à un décalage vers la droite tout en conservant le bit de signe tout à gauche.
Notez bien que l'on décale tous les bits y compris celui du signe. Démontrons nos affirmations.

\medskip

Le cas $n \in \intervalC{0}{127}$ étant évident, considérons $n \in \intervalC{-128}{-1}$ .
La représentation de $n$ s'obtient alors via
$\displaystyle n = -2^7 + \sum_{0 \leq k \leq 6} b_k \, 2^k$ avec chaque $b_k$ dans $\geneset{0 ; 1}$ .
Nous avons alors :

\medskip

$\displaystyle n\, \ediv{} \, 2 
	= -2^6 + \sum_{1 \leq k \leq 6} b_k \, 2^{k-1}$

\smallskip

$\displaystyle n\, \ediv{} \, 2 
	= -2\times2^6 + 2^6 + \sum_{0 \leq k \leq 5} b_{k-1} \, 2^k$

\smallskip

$\displaystyle n\, \ediv{} \, 2 
	= -2^7 + 2^6 + \sum_{0 \leq k \leq 5} b_{k-1} \, 2^k$
	
\medskip

La dernière identité correspond bien au décalage avec conservation de signe qui a été annoncé plus haut.



\subsection{Synthèse}

Nous constatons que nous avons deux types de décalage à droite.
Dans le cas des entiers non signés, le décalage vers la droite est noté \texttt{>\,\!>\,\!>} et il est appelé \emph{\og décalage logique à droite \fg}.
Dans le cas des entiers signés, le décalage vers la droite est noté \texttt{>\,\!>} et il est appelé \emph{\og décalage arithmétique à droite \fg}.


% ----------------- %


\section{Multiplication par $2$ et décalage à gauche}

\subsection{Entiers non signés}

Lorsque l'on calcule $2n$ le double d'un naturel non signé $n$, il suffit de retirer son bit de poids fort tout en ajoutant un zéro comme nouveau bit de poids faible. Voici un exemple sans dépassement de capacité.


\medskip

\binaryplus{00010111}{$n_0 = 16 + 4 + 2 + 1 = 23$}

\medskip

\binaryplus{0010111z}{$n_1 = 2 n_0 = 46$}

\medskip

\binaryplus{010111zz}{$n_2 = 2 n_1 = 92$}

\medskip

\binaryplus{10111zzz}{$n_3 = 2 n_2 = 184$}

\medskip


Chaque étape correspond à un décalage vers la gauche tout en complétant par un zéro à droite.


\subsection{Entiers signés}
Examinons ce qu'il se passe lorsque l'on calcule  $2 n$ si $n < 0$ est un naturel signé.
Voici un exemple sans dépassement de capacité.


\medskip

\binaryplus{U1010111}{$n_0 = -128 + 64 + 16 + 4 + 2 + 1 = -41$}

\medskip

\binaryplus{U010111z}{$n_1 = 2n_0 = -82 = -128 + 32 + 8 + 4 + 2$}

\medskip


On effectue un décalage à gauche de tous les bits, y compris celui du signe, comme avec les entiers non signés.
Mais est-ce toujours vrai ? Que se passe-t-il si le 2\ieme{} bit à gauche est nul ?
En fait, sous l'hypothèse de non dépassement de capacité, tout fonctionne sans souci comme nous allons l'expliquer proprement tout de suite.

\medskip

Le cas $n \in \intervalC{0}{127}$ étant évident, considérons $n \in \intervalC{-128}{-1}$ .
La représentation de $n$ s'obtient alors via
$\displaystyle n = -2^7 + \sum_{0 \leq k \leq 6} b_k \, 2^k$ avec chaque $b_k$ dans $\geneset{0 ; 1}$ .
Nous avons alors :

\medskip

$\displaystyle 2 n 
	= -2^8 + \sum_{0 \leq k \leq 6} b_k \, 2^{k+1}$

\smallskip

$\displaystyle 2 n 
	= -2^8 + b_6 \, 2^7 + \sum_{1 \leq k \leq 6} b_{k-1} \, 2^k$

\smallskip

$\displaystyle 2 n 
	= (b_6 - 2) \, 2^7 + \sum_{1 \leq k \leq 6} b_{k-1} \, 2^k$

\medskip

Si $b_6 = 1$ , nous avons bien le décalage annoncé. Supposons donc que $b_6 = 0$ . Dans ce cas, nous obtenons :

\medskip

$\displaystyle S = \sum_{0 \leq k \leq 6} b_k \, 2^k$

\smallskip

$\displaystyle S = \sum_{0 \leq k \leq 5} b_k \, 2^k$

\smallskip

$\displaystyle S \leq \sum_{0 \leq k \leq 5} 2^k$

\smallskip

$\displaystyle S \leq 2^6 - 1$

\medskip

Nous en déduisons que $n = -2^7 + S \leq -2^7 + 2^6 - 1 = -2^6 - 1$ , puis ensuite $2n \leq -2^7 - 2 \, \text{<} \, -2^7$ ce qui est bien un cas de dépassement de capacité.



\subsection{Synthèse}
Dans le cas des entiers signés ou non, nous utilisons le même décalage vers la gauche qui est noté \texttt{<\,\!<} .
Pour préciser sur quel type d'entiers on effectue un décalage, on parlera de \emph{\og décalage logique à gauche \fg} pour les entiers non signés,
et de \emph{\og décalage arithmétique à gauche \fg} pour les entiers signés.


\end{document}
