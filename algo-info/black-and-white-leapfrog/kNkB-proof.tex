Les démonstrations ci-dessous sont sans astuce particulière et faites en raisonnant sur la ligne de jeu telle que nous la connaissons. Autrement dit, nous n'allons pas travailler avec une représentation ad-hoc du jeu.


\begin{remark}
	Dans la suite, pour ne pas alourdir le document, nous indiquerons d'un seul coup plusieurs mouvements successifs.
	Par exemple, voici une méthode possible pour résoudre la configuration \config{3N.3B} qui va au passage nous éclairer pour le cas général.
	\begin{mvts}
		\medskip
		\item  \gamelineplus{NNN.bBB}{Un seul blanc peut bouger.}

		\medskip
		\item  \gamelineplus{Nnnp.BB}{Deux noirs peuvent bouger.}

		\medskip
		\item  \gamelineplus{N.ububb}{Trois blancs peuvent bouger.}

		\medskip
		\item  \gamelineplus{npnpnp.}{Trois noirs peuvent bouger.}

		\medskip
		\item  \gamelineplus{.bububu}{Par symétrie, on sait que l'on va gagner. Voir le fait \ref{symmetry-no-color}.}
	\end{mvts}

	Notons au passage la configuration particulière de l'étape \step{4} car ceci va être important pour notre preuve du cas général. 
\end{remark}



\begin{fact} \label{kNkB-reduction}
	Soit $k \in \NNs$. Suivant la parité de $k$ nous avons :
	\begin{enumerate}
		\item Si $k = 2r$ est pair, nous pouvons passer de \config{kN.kB} à \config{kNB.} .

		\item Si $k = 2r+1$ est impair, nous pouvons passer de \config{kN.kB} à \config{.kNB} .
	\end{enumerate}
\end{fact}


\begin{proof}
	Ceci découle du résultat plus général suivant en prenant $m = 0$.
\end{proof}



\begin{fact}
	Soit $(k ; m) \in \NN^2$. Suivant la parité de $k$ nous avons :
	\begin{enumerate}
		\item Si $k = 2r$ est pair, nous pouvons passer de \config{kNmNB.kB} à \config{(k+m)NB.} .

		\item Si $k = 2r+1$ est impair, nous pouvons passer de \config{kNmNB.kB} à \config{.(k+m)NB} .
	\end{enumerate}
\end{fact}


\begin{proof}
	Pour $j \in \NN$, notons $\probaset*{P}{j}$ la propriété \emph{\og $\forall m \in \NN$, $\forall k \in \NN$ tel que $0 \leq k \leq j$ , soit (1) est vrai, soit (2) est vrai \fg}.
	Prouvons $\probaset*{P}{j}$ par récurrence sur $j \in \NN$.
	
	\begin{itemize}[label=\small\textbullet]
		\item \textbf{Cas de base :} pour la suite nous devons à la fois prouver $\probaset*{P}{0}$ et $\probaset*{P}{1}$ .
		
		\noindent
		$\probaset*{P}{0}$ est clairement vérifiée puisque $\probaset*{P}{0}$ est \emph{\og $\forall m \in \NN$, nous pouvons passer de \config{mNB.} à \config{(0+m)NB.} \fg} .
		
		\noindent
		Passons à $\probaset*{P}{1}$ . Nous devons valider la proposition \emph{\og $\forall m \in \NN$, nous pouvons passer de \config{1NmNB.1B} à \config{.(1+m)NB} \fg} .
		Les cas $m = 0$ et $m = 1$ sont faciles à traiter, tandis que pour $m \geq 2$ , il suffit de faire les mouvements suivants.
	\begin{mvts}
		\medskip
		\item \gamelineplus{nnB-nB.B}{Du type \config{1NmNB.1B} .}

		\medskip
		\item \gamelineplus{.uB-uBuB}{Du type \config{.(1+m)NB} .}
	\end{mvts}


		\item \textbf{Hérédité :} supposons que $\probaset*{P}{j}$ avec en plus $j \geq 1$. On peut faire ceci car si $j = 0$ alors $j + 1 = 1$ donne un cas qui a déjà été traité. Nous devons alors déduire $\probaset*{P}{j+1}$ de $\probaset*{P}{j}$.
		
		\medskip
		
		\noindent
		Si $m = 0$ alors nous faisons les mouvements suivants.
		\begin{mvts}
			\medskip
			\item \gamelineplus{-NNn.BBB-}{Du type \config{(j+1)N0NB.(j+1)B} .}

			\medskip
			\item \gameline{-NN.ubbB-}

			\medskip
			\item \gamelineplus{-NNpNp.B-}{Du type \config{(j-1)N2NB.(j-1)B} .}
		\end{mvts}
			
		\noindent
		Comme $j - 1 \geq 0$ et $j + 1 \geq 2$ sont de même parité, la validité de $\probaset*{P}{j}$ permet, si $j+1 = 2r+1$ est impair, d'arriver à \config{.(j-1+2)NB} c'est à dire à \config{.(j+1)NB} , et sinon d'arriver à \config{(j+1)NB.} .
			
		\medskip
			
		\noindent
		Si $m \geq 1$ alors nous faisons les mouvements suivants avec un abus de notations évidents pour les \, \gameline{NB} \, centraux si $m = 1$ mais ceci n'invalide pas le raisonnement fait.
		\begin{mvts}
			\medskip
			\item \gameline{-NNnnB-nB.BBB-}

			\medskip
			\item \gameline{-NN.ub-ububbB-}

			\medskip
			\item \gameline{-NNpNp-NpNp.B-}
		\end{mvts}
			
		\noindent
		La dernière configuration étant du type \config{(j-1)N(m+2)NB.(j-1)B} , nous pouvons de nouveau arriver à \config{.(j+1+m)NB} ou \config{(j+1+m)NB.} suivant la parité de $(j+1)$.
	\end{itemize}
\end{proof}


 
\begin{fact} \label{kNkB-resoluble}
	$\forall k \in \NNs$, la configuration \config{kN.kB} est résoluble.
\end{fact}


\begin{proof}
	Distinguons deux cas avec des abus de notations évidents qui sont juste là pour faciliter la compréhension.

	\begin{itemize}[label=\small\textbullet]
		\item  Supposons que $k = 2r$ soit pair. Nous pouvons alors faire les mouvements suivants d'après le fait \ref{kNkB-reduction}.
		\begin{mvts}
			\medskip
			\item \gameline{-NNN.BBB-}

			\medskip
			\item \gameline{nBnB-nB.}

			\medskip
			\item \gameline{.Bu-BuBu}
		\end{mvts}
		
		\noindent
		Comme les configurations des étapes \step{2} et \step{3} sont symétriques, en reprenant les mouvements à rebours de \step{2} à \step{1} et en échangeant juste les sens de parcours, et non les couleurs, nous arrivons finalement à la configuration symétrique de  \, \gameline{-NN.BB-} \, qui est \, \gameline{-BB.NN-} \, ce qui nous fait gagner \emph{(voir le fait \ref{symmetry-no-color})}.


		\item Supposons que $k = 2r + 1$ soit impair. Nous pouvons alors faire les mouvements suivants de nouveau en faisant appel au fait \ref{kNkB-reduction}.
		\begin{mvts}
			\medskip
			\item \gameline{-NNN.BBB-}

			\medskip
			\item \gameline{.NbNb-Nb}

			\medskip
			\item \gameline{pN-pNpN.}
		\end{mvts}
		
		\noindent
		Nous pouvons conclure comme précédemment avec un argument de symétrie.
	\end{itemize}
\end{proof}



\begin{remark}
	Si vous reprenez les preuves de cette section, vous noterez que nous avons de nouveau appliquer une tactique gloutonne en commençant par faire bouger un mouton noir. \textbf{Nous avons une preuve algorithmique constructive !}
\end{remark}

