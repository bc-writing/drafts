Imaginons que nous soyons de part et d'autre de la ligne de jeux.
Pour vous les règles de déplacement et les couleurs sont inversés par rapport à moi.
Si je résous une configuration, vous pourrez aussi la faire avec la votre un peu particulière.
Ceci nous donne l'idée de chercher des principes de symétrie. Les faits suivants en proposent deux.



\begin{fact} \label{symmetry-color}
	La configuration \config{kN.pB} est résoluble si et seulement si la configuration \config{pN.kB} l'est aussi. 
	Par exemple, nous avons :
	
	\centerit{%
		\gameline{NNNN.BB} est résoluble.
		\, $\Longleftrightarrow$ \, 
		\gameline{NN.BBBB} est résoluble.
	}
	
	\medskip
	
	Plus généralement, considérons deux configurations $\probaset*{C}{1}$ et $\probaset*{C}{2}$ , et pour chaque $k \in \geneset{1 ; 2}$ notons $\probaset*{R}{k}$ la configuration obtenue en faisant un demi-tour et en échangeant les couleurs.
	Alors il existe des mouvements permettant de passer de $\probaset*{C}{1}$ à $\probaset*{C}{2}$ si et seulement si il en existe pour aller de $\probaset*{R}{1}$ à $\probaset*{R}{2}$ \emph{(attention à l'ordre des indices)}.  
\end{fact}


\begin{proof}
	Donnons un exemple d'application des transformations.
	\begin{enumerate}
		\item On applique un demi-tour à la ligne de jeu :
		\centerit{%
			\gameline{nnnn.pp}
			\, devient \, 
			\gameline{pp.nnnn} .
		}
		
		\noindent
		Tout mouvement ou saut fait dans un sens sur une ligne de jeu sera fait dans l'autre sens sur l'autre.

		\item Après le demi-tour, on échange les couleurs pour revenir aux règles classiques du jeu :
		\centerit{%
			\gameline{pp.nnnn}
			\, devient \,
			\gameline{uu.bbbb} .
		}
	\end{enumerate}
	
	Revenons au cas général.
	Si l'on peut passer de $\probaset*{C}{1}$ à $\probaset*{C}{2}$ alors il suffit de reprendre la même séquence de mouvements en échangeant les couleurs et les sens de parcours pour aller de $\probaset*{R}{1}$ à $\probaset*{R}{2}$ .
	Ceci s'applique en particulier à la résolution d'un jeu telle que nous l'avons indiquée. 
	Ceci est un petit truc tout bête qui va nous rendre un énorme service très bientôt.
\end{proof}


Redonnons les mouvements proposés pour résoudre la configuration \config{2N.2B}.
\begin{mvts}
	\medskip
	\item  \gameline{NN.bB}
	
	\medskip
	\item  \gameline{Nnp.B}
	
	\medskip
	\item  \gameline{n.BuB}
	
	\medskip
	\item  \gameline{.ubNB}
	
	\medskip
	\item  \gameline{pN.Nb}
	
	\medskip
	\item  \gameline{BNpn.}
	
	\medskip
	\item  \gameline{BnB.u}
	
	\medskip
	\item  \gameline{B.buN}
	
	\medskip
	\item  \gameline{Bp.NN}
\end{mvts}


Constatez-vous quelque chose ? Si vous regardez de part et d'autre de l'étape \step{5}, nous avons une autre forme de symétrie. Mettons-là en valeur.
\begin{multicols}{2}
	\medskip \step{1} 
	\gameline{nn.pp} \,\,\,\,\rotatebox[origin=c]{270}{$\Rsh$}
	
	\medskip \step{2} 
	\gameline{nnb.p} \quad $\shortdownarrow$
	
	\medskip \step{3} 
	\gameline{n.pup} \quad $\shortdownarrow$
	
	\medskip \step{4} 
	\gameline{.upup} \quad $\shortdownarrow$

	% ----------- %
	
	\medskip \hfill{} $\Rsh$ \quad \step{9}
	\gameline{pp.nn}

	\medskip \hfill{} $\shortuparrow$ \quad \step{8}
	\gameline{p.bnn}

	\medskip \hfill{} $\shortuparrow$ \quad \step{7}
	\gameline{pup.n}

	\medskip \hfill{} $\shortuparrow$ \quad \step{6}
	\gameline{pupu.}
\end{multicols}
\vspace{-1em}
\begin{center}
	\medskip \rotatebox[origin=c]{180}{$\Lsh$} \, $\shortrightarrow$ \quad \step{5}
	\gameline{BN.NB} \quad $\shortrightarrow$ \, \rotatebox[origin=c]{270}{$\Lsh$}
\end{center}

Ceci motive le fait général suivant.


\begin{fact} \label{symmetry-no-color}
	Considérons deux configurations $\probaset*{C}{1}$ et $\probaset*{C}{2}$ , et pour chaque $k \in \geneset{1 ; 2}$ notons $\probaset*{R}{k}$ la configuration obtenue en faisant juste un demi-tour sans échanger les couleurs.
	Alors il existe des mouvements permettant de passer de $\probaset*{C}{1}$ à $\probaset*{C}{2}$ si et seulement si il en existe pour aller de $\probaset*{R}{2}$ à $\probaset*{R}{1}$ \emph{(attention à l'ordre des indices)}.
\end{fact}


\begin{proof}
	Il suffit de raisonner sur deux étapes successives en considérant les différents mouvements possibles.
\end{proof}
