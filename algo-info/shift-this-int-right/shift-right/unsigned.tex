Examinons ce qu'il se passe lorsque l'on calcule  $(n\, \ediv{} \, 2)$ le quotient entier par défaut d'un naturel signé $n \,\text{<}\, 0$ divisé par $2$.


\medskip

\binaryplus{U1010111}{$n_0 = -128 + 64 + 16 + 4 + 2 + 1 = -41$}

\medskip

\binaryplus{uU101011}{$n_1 = n_0 \, \ediv{} \, 2 = -21 = -128 + 64 + 32 + 8 + 2 + 1$}

\medskip


On en déduit que quelque soit le signe de l'entier signé, le quotient entier par défaut correspond à un décalage vers la droite tout en conservant le bit de signe tout à gauche.
Notez bien que l'on décale tous les bits y compris celui du signe. Démontrons nos affirmations.

\medskip

Le cas $n \in \intervalC{0}{127}$ étant évident, considérons $n \in \intervalC{-128}{-1}$ .
La représentation de $n$ s'obtient alors via
$\displaystyle n = -2^7 + \sum_{0 \leq k \leq 6} b_k \, 2^k$ avec chaque $b_k$ dans $\geneset{0 ; 1}$ .
Nous avons alors :

\medskip

$\displaystyle n\, \ediv{} \, 2 
	= -2^6 + \sum_{1 \leq k \leq 6} b_k \, 2^{k-1}$

\smallskip

$\displaystyle n\, \ediv{} \, 2 
	= -2\times2^6 + 2^6 + \sum_{0 \leq k \leq 5} b_{k-1} \, 2^k$

\smallskip

$\displaystyle n\, \ediv{} \, 2 
	= -2^7 + 2^6 + \sum_{0 \leq k \leq 5} b_{k-1} \, 2^k$
	
\medskip

La dernière identité correspond bien au décalage avec conservation de signe qui a été annoncé plus haut.
