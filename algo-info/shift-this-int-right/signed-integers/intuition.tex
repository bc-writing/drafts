Pour comprendre comment découvrir ce type de procédé, nous allons raisonner en base $10$ et imaginer que nous voulions calculer $189 - 32 = 157$ en utilisant uniquement des additions. Traditionnellement on fait comme suit.
\begin{center}
\begin{tabular}{cccc}
	    & 1 & 8 & 9 \\
	$-$ & 0 & 3 & 2 \\
	\hline
    $=$ & 1 & 5 & 7 \\
\end{tabular}
\end{center}
Maintenant ajoutons $1000$ à $189 - 32$. Nous obtenons :
\begin{center}
\begin{tabular}{ccccc}
	    & 0 & 1 & 8 & 9 \\
	$-$ & 0 & 0 & 3 & 2 \\
	$+$ & 1 & 0 & 0 & 0 \\
	\hline
    $=$ & 1 & 1 & 5 & 7 \\
\end{tabular}
\end{center}
Comme le résultat initial tenait sur les trois chiffres de droite, l'ajout de $1000$ donne un nouveau résultat dont nous savons que les trois chiffres de droite sont ceux de $189 - 32$.
D'autre part, $1000 + 189 - 32 = 189 + 968$ est une simple addition.

\smallskip

Notons aussi que $1000 - 32 = 999 - 032 + 1 = 967 + 1$ où $9$ , $6$ et $7$ sont les compléments à $9$ de $0$ , $3$ et $2$ respectivement.
L'opération $1000 - 32$ se fait donc rapidement. De plus, la soustraction $1157 - 1000$ revient à ignorer le $1$ tout à droite. En résumé, nous avons fait comme suit où $\bullet$ indique une ligne où a été fait le calcul d'un complément à $9$ plus $1$ :

\begin{center}
\begin{tabular}{ccccc}
	    & 0 & 1 & 8 & 9 \\
	$-$ & 0 & 0 & 3 & 2 \\
	\hline
	\hline
	    & 0 & 1 & 8 & 9 \\
	$+$ & $\bullet$ & 9 & 6 & 8 \\
	\hline
    $=$ & 1 & 1 & 5 & 7 \\
	\hline
	\hline
    $+$ &  & 1 & 5 & 7 \\
\end{tabular}
\end{center}

\smallskip

Très bien mais que se passe-t-il avec $32 - 189 = -157$ qui est négatif ? Testons pour voir.
\begin{center}
\begin{tabular}{ccccc}
	    & 0 & 0 & 3 & 2 \\
	$-$ & 0 & 1 & 8 & 9 \\
	\hline
	\hline
	    & 0 & 0 & 3 & 2 \\
	$+$ & $\bullet$ & 8 & 1 & 1 \\
	\hline
    $=$ & 0 & 8 & 4 & 3 \\
\end{tabular}
\end{center}
Oups ! Nous ne voyons pas directement $157$. C'est normal puisque $843 - 1000 = -157$ n'a pas été effectué.
Dans un tel cas, on note que le chiffre à gauche étant $0$, on effectue un complément à $9$ plus $1$ tout en ajoutant un signe moins afin d'obtenir le résultat final.
En résumé, nous procédons comme suit. 
\begin{center}
\begin{tabular}{ccccc}
	    & 0 & 0 & 3 & 2 \\
	$-$ & 0 & 1 & 8 & 9 \\
	\hline
	\hline
	    & 0 & 0 & 3 & 2 \\
	$+$ & $\bullet$ & 8 & 1 & 1 \\
	\hline
    $=$ & 0 & 8 & 4 & 3 \\
	\hline
	\hline
	$-$ & $\bullet$ & 1 & 5 & 7 \\
\end{tabular}
\end{center}


\smallskip

Ne nous emballons pas trop vite car il reste un cas problématique que nous n'avons pas abordé.
Examinons ce qu'il se passe avec $-32 - 189 = - 221$ ?
\begin{center}
\begin{tabular}{ccccc}
	$-$ & 0 & 0 & 3 & 2 \\
	$-$ & 0 & 1 & 8 & 9 \\
	\hline
	\hline
	    & $\bullet$ & 9 & 6 & 8 \\
	$+$ & $\bullet$ & 8 & 1 & 1 \\
	\hline
    $=$ & 1 & 7 & 7 & 9 \\
\end{tabular}
\end{center}

Nous devons affiner notre règle car sinon ici nous aurions un résultat positif !
En fait, il est facile de voir qu'il suffit de comparer le nombre de compléments à $9$ plus $1$ effectués, c'est à dire le nombre de $1000$ ajoutés, au chiffre tout à gauche qui sera $0$ ou $1$.
Ici nous avons deux compléments et un chiffre $1$ ce qui nous demande d'appliquer $2 - 1 = 1$ complément au résultat final tout en mettant un signe moins.
En résumé, nous avons :
\begin{center}
\begin{tabular}{ccccc}
	$-$ & 0 & 0 & 3 & 2 \\
	$-$ & 0 & 1 & 8 & 9 \\
	\hline
	\hline
	    & $\bullet$ & 9 & 6 & 8 \\
	$+$ & $\bullet$ & 8 & 1 & 1 \\
	\hline
    $=$ & 1 & 7 & 7 & 9 \\
	\hline
	\hline
    $-$ & $\bullet$ & 2 & 2 & 1 \\
\end{tabular}
\end{center}


\medskip


Nous trouvons bien $- 32 - 189 = - 221$. Que c'est beau !
Les deux cas suivants achèvent notre exploration expérimentale.
\begin{multicols}{2}
\begin{center}
\begin{tabular}{ccccc}
	    & 0 & 0 & 3 & 2 \\
	$+$ & 0 & 1 & 8 & 9 \\
	\hline
	$=$ & 0 & 2 & 2 & 1 \\
	\hline
	\hline
	$+$ &   & 2 & 2 & 1 \\
\end{tabular}
\end{center}

\null\vfill

\columnbreak

\begin{center}
\begin{tabular}{ccccc}
	$-$ & 0 & 0 & 3 & 2 \\
	$+$ & 0 & 1 & 8 & 9 \\
	\hline
	\hline
	    & $\bullet$ & 9 & 6 & 8 \\
	$+$ & 0 & 1 & 8 & 9 \\
	\hline
	$=$ & 1 & 1 & 5 & 7 \\
	\hline
	\hline
	$+$ &   & 1 & 5 & 7 \\
\end{tabular}
\end{center}
\end{multicols}


\begin{exercise}
	Dans la méthode ci-dessous, il existe des cas problématiques. Lesquels ?
\end{exercise}


\begin{exercise}
	En laissant les cas problématiques de côté, démontrer le caractère général de la méthode que nous avons juste exposée via quelques exemples.
\end{exercise}

