Revenons à nos entiers signés écrits sur $8$ bits. Reprenons le calcul de $84 - 101 = -17$ en nous inspirant de ce qui a été vu dans la section précédente avec la base $10$. Ici $\bullet$ indique une ligne où a été fait le calcul d'un complément à $1$ plus $1$.

\medskip

\begin{center}
\begin{tabular}{ll}
	    & \!\!\binary{Z1010100}  	\\
	$-$ & \!\!\binary{Z1100101} 	\\[.8ex]
	\hline
	\hline 							\\[-2ex]
	    & \!\!\binary{Z1010100} 	\\
	$+$ & \!\!\binary{o0011011} 	\\[.8ex]
	\hline \\[-2ex]
	$=$ & \!\!\binary{Z1101111} 	\\[.8ex]
	\hline
	\hline 							\\[-2ex]
	$-$ & \!\!\binary{o0010001} 	\\
\end{tabular}
\end{center}

\medskip

Tout s'éclaire ! Reprenons le cas de $84 - 61 = 23$ où l'on avait une retenue à ignorer.

\medskip

\begin{center}
\begin{tabular}{ll}
	    & \!\!\binary{Z1010100}  	\\
	$-$ & \!\!\binary{Z0111101} 	\\[.8ex]
	\hline
	\hline 							\\[-2ex]
	    & \!\!\binary{Z1010100} 	\\
	$+$ & \!\!\binary{o1000011} 	\\[.8ex]
	\hline \\[-2ex]
	$=$ & \!\!\binary{U0010111} 	\\[.8ex]
	\hline
	\hline 							\\[-2ex]
	$+$ & \!\!\binary{Z0010111} 	\\
\end{tabular}
\end{center}

\medskip

Avec le nouvel éclairage, nous notons qu'en fait les bits rouges ne servent qu'à indiquer s'il faut ou non effectuer un complément à $1$ plus $1$ en changeant de signe.
Avec la façon de stocker les entiers signés, nous avons alors les correspondances suivantes où les entiers additionnés $a$ et $b$ sont dans $\intervalC{-128}{127}$ et leur somme aussi, ce qui revient à ne considérer que les cas de non dépassement de capacité.
Ci-après, \binary{*} et les cases sur fond gris indiquent l'utilisation d'un complément à $1$ plus $1$ que nous avions noté \binary{o} ci-dessus, et le symbole $\sim$ est pour le calcul de la somme sans le bit de signe qui lui entre en jeu dans la ligne avec le signe $=$ .
\begin{multicols}{4}
    \begin{center}
	\begin{tabular}{ll}
	    & \!\!\binary{Z-}  		\\
	$-$ & \!\!\binary{Z-} 		\\[.8ex]
	\hline
	\hline 						\\[-2ex]
	    & \!\!\binary{Z-} 		\\
	$+$ & \!\!\binary{*+} 		\\[.8ex]
	\hline \\[-2ex]
	$\sim$ & \!\!\binary{Z-} 	\\[.8ex]
	\hline
	\hline 						\\[-2ex]
	$=$ & \!\!\binary{U-} 		\\
	\end{tabular}
	
	\medskip\itshape\footnotesize
	
	Soustraction 1
	
	de deux naturels
	\end{center}


	\null\vfill
	\columnbreak
	
	
	\begin{center}
	\begin{tabular}{ll}
	    & \!\!\binary{Z-}  		\\
	$-$ & \!\!\binary{Z-} 		\\[.8ex]
	\hline
	\hline 						\\[-2ex]
	    & \!\!\binary{Z-} 		\\
	$+$ & \!\!\binary{*+} 		\\[.8ex]
	\hline \\[-2ex]
	$\sim$ & \!\!\binary{U-} 	\\[.8ex]
	\hline
	\hline 						\\[-2ex]
	$=$ & \!\!\binary{Z-} 		\\
	\end{tabular}
	
	\medskip\itshape\footnotesize
	
	Soustraction 2
	
	de deux naturels
	\end{center}


	\null\vfill
	\columnbreak
	
	
	\begin{center}
	\begin{tabular}{ll}
	$-$ & \!\!\binary{Z-}  		\\
	$-$ & \!\!\binary{Z-} 		\\[.8ex]
	\hline
	\hline 						\\[-2ex]
	    & \!\!\binary{*+} 		\\
	$+$ & \!\!\binary{*+} 		\\[.8ex]
	\hline \\[-2ex]
	$\sim$ & \!\!\binary{Z-} 	\\[.8ex]
	\hline
	\hline 						\\[-2ex]
	$=$ & \!\!\binary{Z-} 		\\
	\end{tabular}
	
	\medskip\itshape\footnotesize
	
	Addition 1 de
	
	deux relatifs négatifs
	\end{center}


	\null\vfill
	\columnbreak
	
	
	\begin{center}
	\begin{tabular}{ll}
	$-$ & \!\!\binary{Z-}  		\\
	$-$ & \!\!\binary{Z-} 		\\[.8ex]
	\hline
	\hline 						\\[-2ex]
	    & \!\!\binary{*+} 		\\
	$+$ & \!\!\binary{*+} 		\\[.8ex]
	\hline \\[-2ex]
	$\sim$ & \!\!\binary{U-} 	\\[.8ex]
	\hline
	\hline 						\\[-2ex]
	$=$ & \!\!\binary{U-} 		\\
	\end{tabular}
	
	\medskip\itshape\footnotesize
	
	Addition 2 de
	
	deux relatifs négatifs
	\end{center}


	\null\vfill
\end{multicols}

\vspace{-1.5em}

Les soustractions et l'addition 2 ne posent aucun souci.
Par contre l'addition 1 est problématique avec son résultat positif. 
En fait cette addition contredit notre hypothèse de non dépassement de capacité. Nous allons voir pourquoi.

\medskip

Le cas de l'addition 1 correspond à $(a ; b) \in \intervalC{-128}{-1}^2$ tel que $(128 + a) + (128 + b) \in \intervalC{0}{127}$ soit $0 \leq 256 + a + b \leq 127$ \emph{i.e.} $-256 \leq a + b \leq -129$ ce qui correspond à un dépassement de capacité comme annoncé.
 
\medskip

Ceci achève de démontrer la validité des procédés d'addition et de soustraction d'entiers signés dans les cas de non dépassement de capacité.
Notez que le cas évident d'une addition de deux naturels a été omis, et aussi que $-a + b = b - a = b + (-a)$ et le fait qu'un changement de signe n'est autre qu'un complément à $1$ plus $1$ permettent de compléter les cas non indiqués ci-dessus.

\begin{exercise}
	Étudiez plus généralement le cas d'une base $b \in \NN_{\geq 3}$ quelconque.
\end{exercise}

