Dans la représentation ci-dessous, $(q_k)_{1 \leq k \leq n}$ est la suite des quotients et $(r_k)_{2 \leq k \leq n}$ celle des restes fournis par l'\algoeucl.

\showstepnovfill{L'\algoeucl{} au complet où $n \geq 1$ forcément.}{tikz/why/algo-euclide-all}


\medskip


Nous savons que $(q_k)_{1 \leq k \leq n} \subseteq \NNs$ avec $q_n \geq 2$, et que $(r_k)_{1 \leq k \leq n} \subseteq \NNs$ est strictement décroissante.
Comme $r_k = q_{k+1} r_{k+1} + r_{k+2}$ si $1 \leq k \leq n-2$, nous avons $r_k \geq r_{k+1} + r_{k+2} > 2 r_{k+2}$ d'où $r_k \geq 2 r_{k+2} + 1$ dès que $1 \leq k \leq n-2$.
Ceci nous donne les majorations suivantes.

\begin{itemize}[label=\small\textbullet]
	\item \textbf{Cas $n = 2p \geq 2$ est pair.}
	      Comme $r_{2p} = r_n \geq 1$, nous avons :
	      
	      \noindent
	      $b > r_2$
	      
	      \noindent
	      $\phantom{b} \geq 2 r_4 + 1$
	      
	      \noindent
	      $\phantom{b} \geq 4 r_6 + 2 + 1$
	      
	      \noindent
	      $\phantom{b} \dots$
	      
	      \noindent
	      $\phantom{b} \geq 2^{p-1} r_{2p} + \displaystyle \sum_{i=0}^{p-2} 2^i$
	      
	      \noindent
	      $\phantom{b} \geq 2^{p-1} + 2^{p-1} - 1$
	      
	      \noindent
	      $\phantom{b} = 2^p - 1$
	      
	      \noindent
	      Nous avons donc $2^p \leq b$ puis $\log(2^p) \leq \log b$ et $p \leq \frac{\log b}{\log 2}$ où $\log$ désigne le logarithme décimal.
	      Donc $n \leq \frac{2}{\log 2} \cdot \log b$ ici.


	\item \textbf{Cas $n = 2p+1$ est impair.}
	      Comme $r_{2p+1} = r_n \geq 1$, nous avons :
	      	       
	      \noindent
	      $b = r_1$
	      	       
	      \noindent
	      $\phantom{b} > 2 r_3 + 1$
	      	       
	      \noindent
	      $\phantom{b} \dots$
	      	       
	      \noindent
	      $\phantom{b} \geq 2^p r_{2p+1} + \displaystyle \sum_{i=0}^{p-1} 2^i$
	      	       
	      \noindent
	      $b > 2^p + 2^p - 1$
	      	       
	      \noindent
	      $\phantom{b} \geq 2^{p + 1} - 1$
	      
	      \noindent
	      Nous avons donc $2^{p + 1} \leq b$ puis $p + 1 \leq \frac{\log b}{\log 2}$.
	      Donc $n \leq \frac{2}{\log 2} \cdot \log b - 1$ ici.
\end{itemize}


Dans les deux cas, $n \leq \frac{2}{\log 2} \cdot \log b$.
Comme $\frac{2}{\log 2} \approx 6,65$, notant $d$ le nombre de chiffres décimaux de $b$, de sorte que $\log b < d$, nous avons l'estimation $n < 7d$.


\medskip


En résumé, l'\algoeucl{} appliqué à $(a ; b) \in \NNs \!\times \NNs$ avec $a \geq b$ demandera au maximum $7d - 1$ étapes où $d$ est le nombre de chiffres décimaux de $b$.


\begin{remark}
	Cette estimation, rapide à établir, montre que le nombre d'étapes augmente de façon logarithmique par rapport à $b$ le plus petit des entiers naturels $a$ et $b$ auxquels est appliqué l'\algoeucl.
\end{remark}

