\documentclass[12pt]{amsart}
\usepackage[T1]{fontenc}
\usepackage[utf8]{inputenc}

\usepackage[top=1.95cm, bottom=1.95cm, left=2.35cm, right=2.35cm]{geometry}

\usepackage{hyperref}
\usepackage{enumitem}
\usepackage{tcolorbox}
\usepackage{float}
\usepackage{cleveref}
\usepackage{multicol}
\usepackage{fancyvrb}
\usepackage{amsmath}
\usepackage[french]{babel}
\usepackage[
    type={CC},
    modifier={by-nc-sa},
	version={4.0},
]{doclicense}

\newcommand\floor[1]{\left\lfloor #1 \right\rfloor}

\newcommand\ourapplis{\setgeo*{A}{n \to k+n}}

\newcommand\npermu{\setgeo*{P}{n}}

\newcommand\badapplis[1]{\setgeo*{M}{#1}}

\usepackage{tnsmath}
\usepackage{tnsalgo}

% Source
%	* https://tex.stackexchange.com/a/51515/6880
\usepackage{minted}
\usemintedstyle{bw}
\renewcommand{\ttdefault}{pcr}

\newtheorem{fact}{Fait}[section]
\newtheorem{example}{Exemple}[section]
\newtheorem{remark}{Remarque}[section]
\newtheorem*{proof*}{Preuve}

\setlength\parindent{0pt}

\floatstyle{boxed} 
\restylefloat{figure}


\DeclareMathOperator{\taille}{\text{\normalfont\texttt{taille}}}

\newcommand\sqseq[2]{\fbox{$#1$}_{\,\,#2}}

\newcommand\myquote[1]{{\itshape \og #1 \fg}}


\newcommand\inputcode[2]{
	\bgroup
	\centering
	\begin{tcolorbox}[
%		width        = #1\linewidth,
%		title        = #2,
		title        = Code Python,
		fonttitle    = \bfseries\itshape\small,
		coltitle     = black,
		colbacktitle = black!10!white,
		colback      = white,
		breakable,
		center title]
		\small
		\inputminted{#1}{#2}
	\end{tcolorbox}
	\egroup
}



\begin{document}

\title{BROUILLON - Soustraire les puissances n\ieme{} de (n+1) naturels consécutifs}
\author{Christophe BAL}
\date{3 Octobre 2019 -- 22 Octobre 2020}

\maketitle

\begin{center}
	\itshape
	Document, avec son source \LaTeX, disponible sur la page
	
	\url{https://github.com/bc-writing/drafts}.
\end{center}


\bigskip


\begin{center}
	\hrule\vspace{.3em}
	{
		\fontsize{1.35em}{1em}\selectfont
		\textbf{Mentions \og légales \fg}
	}
			
	\vspace{0.45em}
	\doclicenseThis
	\hrule
\end{center}


\bigskip
\setcounter{tocdepth}{1}
\tableofcontents



\newpage
\section{Faire la différence avec des puissances}

Considérons l'algorithme suivant dont nous allons donner des cas d'application juste après.

\begin{algo}
	\caption{Version naturelle} \label{nat-algo}

	\In{$n \in \NNs$}
	\Out{?}
	
	\addalgoblank
	
	\Actions{
		Choisir $(n+1)$ naturels consécutifs : $k_1 < k_2 < \dots < k_{n+1}$.
		\\
		$L \Store [ k_1^n , k_2^n , \dots , k_{n+1}^n ]$
		\\
		\addalgoblank
		\While{$taille(L) \neq 1$}{
			$newL \Store \EmptyList$
			\\
			\ForRange*{i}{1}{taille(L) - 1}{
				Ajouter $(L[i + 1] - L[i])$ à droite de $newL$.
			}
			$L \Store newL$
		}
		\addalgoblank
		\Return{$L[1]$}
	}
\end{algo}


Pour $n = 2$ avec $k_1 = 3$ , $k_2 = 4$ et $k_3 = 5$, nous avons les valeurs suivantes de la liste $L$.

\begin{enumerate}
	\item $L = [3^2 , 4^2 , 5^2] = [9 , 16 , 25]$

	\item $L = [16 - 9 , 25 - 16] = [7, 9]$

	\item $L = [9 - 7] = [2]$
\end{enumerate}


L'algorithme renvoie donc $2$ ici mais que se passe-t-il si l'on choisit d'autres naturels consécutifs ? Avec $k_1 = 10$ , $k_2 = 11$ et $k_3 = 12$, nous obtenons :

\begin{enumerate}
	\item $L = [10^2 , 11^2 , 12^2] = [100 , 121 , 144]$

	\item $L = [121 - 100 , 144 - 121] = [21 , 23]$

	\item $L = [23 - 21] = [2]$
\end{enumerate}


L'algorithme renvoie de nouveau $2$. Que se passerait-il pour d'autres triplets de naturels consécutifs ? Pour se faire une bonne idée, il va falloir utiliser un programme. Ceci étant dit nous allons toute de suite faire l'hypothèse audacieuse que le choix des naturels consécutifs n'est pas important. Regardons alors ce que renvoie l'algorithme pour $n = 3$.

\begin{enumerate}
	\item $L = [0^3 , 1^3 , 2^3 , 3^3] = [0 , 1 , 8 , 27]$

	\item $L = [1 , 7 , 19]$

	\item $L = [6, 12]$

	\item $L = [6]$
\end{enumerate}


L'algorithme renvoie $6$ pour $n = 3$, et pour $n= 4$ ce qui suit nous donne que $24$ est renvoyé. Ceci nous fait alors penser à $n !$ et donc nous amène à conjecturer, un peu rapidement c'est vrai, que l'algorithme va toujours renvoyé $n !$ .

\begin{enumerate}
	\item $L = [0^4 , 1^4 , 2^4 , 3^4, 4^4] = [0 , 1 , 16 , 81 ,256]$

	\item $L = [1 , 15 , 65 , 175]$

	\item $L = [14, 50 , 110]$

	\item $L = [36 , 60]$

	\item $L = [24]$
\end{enumerate}


Il est temps de passer aux choses un plus sérieuses via une expérimentaion informatique bien plus poussée.





\newpage
\section{Expérimentations et conjecture}

Le code suivant
\footnote{
	Ce fichier \texttt{diff-n-cons-int-to-the-power-of-n/exploring-int-version.py} est disponible dans le sous-dossier sur le lieu de téléchargement de ce document.
}
permet de tester notre conjecture audacieuse : aucun test raté n'est révélé.

\medskip

\inputcode{python}{diff-n-cons-int-to-the-power-of-n/exploring-int-version.py}


% -------------------- %


Comme les valeurs des naturels consécutifs ne semblent pas importantes, on peut pousser l'expérimentation en travaillant avec $k_1 \in \RR$ , $k_2 = k_1 + 1$ , $k_3 = k_1 + 2$ , \dots
Ceci étant dit, il n'est pas toujours possible de travailler avec des réels en informatique : l'usage des flottants s'accompagnent de son lot d'arrondis.
Nous allons donc juste tester des valeurs rationnelles via le code suivant
\footnote{
	Ce fichier \texttt{diff-n-cons-int-to-the-power-of-n/exploring-frac-version.py} est disponible dans le sous-dossier sur le lieu de téléchargement de ce document.
}
qui ne révèle aucun test raté.

\medskip

\inputcode{python}{diff-n-cons-int-to-the-power-of-n/exploring-frac-version.py}


Il devient donc normal de penser que le phénomène est purement polynomial. Nous allons explorer ceci dans la section suivante.




\newpage
%\section{Une preuve polynomiale}
%
%Voici une légère modification de l'algorithme \ref{nat-algo} où l'on manipule des polynômes dans $\setpoly{\RR}{X}$.
Notons au passage que nous passons d'un algorithme a priori indéterministe, car on fait un choix de naturels consécutifs, à un autre complètement déterministe.

\begin{algo}
	\caption{Version polynomiale} \label{nat-2-poly-algo}

	\In{$n \in \NNs$}
	\Out{?}
	
	\addalgoblank
	
	\Actions{
		$L \Store [ X^n , (X+1)^n , \dots , (X+n)^n ]$
		\\
		\addalgoblank
		\While{$taille(L) \neq 1$}{
			$newL \Store \EmptyList$
			\\
			\ForRange*{i}{1}{taille(L) - 1}{
				\Append{$newL$}{$(L[i + 1] - L[i])$}
			}
			$L \Store newL$
		}
		\addalgoblank
		\Return{$L[1]$}
	}
\end{algo}


Il est aisée de traduire cet algorithme en Python. Le code suivant ne révèle aucun test raté.

\medskip

\inputcode{python}{diff-n-cons-int-to-the-power-of-n/exploring-poly-version.py}


\medskip


Il est clair que si l'on prouve que l'algorithme \ref{nat-2-poly-algo} renvoie $n!$, il en sera de même pour \ref{nat-algo}. Ceci découle directement de la validité de l'algorithme ci-dessous où $\setpoly{\RR_{n}}{X}$ désigne l'ensemble des polynômes réels de degré $n \in \NNs$.

\medskip

\begin{algo}
	\caption{Version polynomiale élargie} \label{gene-poly-algo}

	\In{$P \in \setpoly{\RR_{n}}{X}$ de coefficient dominant $a_n$}
	\Out{$n! \, a_n$}
	
	\addalgoblank
	
	\Actions{
		$L \Store \List{P(X) , P(X+1) , \dots , P(X+n)}$
		\\
		\addalgoblank
		\While{$taille(L) \neq 1$}{
			$newL \Store \EmptyList$
			\\
			\ForRange*{i}{1}{taille(L) - 1}{
				\Append{$newL$}{$(L[i + 1] - L[i])$}
			}
			$L \Store newL$
		}
		\addalgoblank
		\Return{$L[1]$}
	}
\end{algo}


\medskip


Pourquoi cet algorithme est-il valide ? Pour la suite, nous posons $P(x) = \dsum_{k=0}^{n} a_k X^k$.
\begin{enumerate}
	\item Il est immédiat que $\forall k \in \NNs$, $(X+1)^k - X^k = kX^{k-1} + R(X)$ où $\deg R < k-1$ avec la convention $\deg 0 = - \infty$.

% ----------------- %

	\item Le point précédent donne sans effort $P(X+1) - P(X) = n a_n X^{n-1} + S(X)$ où le polynôme $S$ vérifie $\deg S < n-1$. Tout est dit comme nous allons le voir.

% ----------------- %

	\item A la 1\iere{} itération de la boucle, la liste 
			  $\List{P(X) , P(X+1) , \dots , P(X+n)}$
	 	  est transformée en 
		      $\List{P(X+1) - P(X) , P(X+2) - P(X+1) , \dots , P(X+n) - P(X+n-1)}$
		  soit $\List{Q(X) , Q(X+1) , \dots , Q(X+n-1)}$
		  en posant $Q(X) = P(X+1) - P(X)$, un polynôme qui a pour coefficient dominant $n a_n$ et pour degré $\deg Q = n - 1$.

% ----------------- %

	\item Il est alors facile de faire une récurrence pour prouver que la boucle se finit en fournissant ce qui est annoncé.  
\end{enumerate}

Joli, efficace et éclairant sur le pourquoi du comment. Il se trouve que l'algorithme \ref{nat-algo} cache aussi une formule combinatoire très intéressante comme nous allons le voir dans les sections à venir.
L'exposé qui suit reprend la démarche proposée sur le site \href{https://mathlesstraveled.com}{The Math Less Traveled}
\footnote{
	Chercher les articles \emph{\og A combinatorial proof \fg} publiés courant septembre 2019.
} 

%
%
%
%\newpage
%\section{Une très jolie identité binomiale}
%
%De l'algorithme \ref{nat-algo}, nous allons juste garder les opérations sur les listes. Ceci nous donne l'algorithme suivant dont nous allons chercher à obtenir une formule explicite de la sortie $S_n$ en fonction de $a_0$, $a_1$, \dots{} , $a_n$.

\begin{algo}
	\algovoidcaption \label{list-to-binomial}

	\In{$[ a_0 , a_1 , \dots , a_n ]$ une liste de naturels}
	\Out{?}
	
	\addalgoblank
	
	\Actions{
		$L \Store [ a_0 , a_1 , \dots , a_n ]$
		\\
		\addalgoblank
		\While{$taille(L) \neq 1$}{
			$newL \Store \EmptyList$
			\\
			\ForRange*{i}{1}{taille(L) - 1}{
				\Append{$newL$}{$(L[i + 1] - L[i])$}
			}
			$L \Store newL$
		}
		\addalgoblank
		\Return{$L[1]$}
	}
\end{algo}


Pour $n = 0$ et $n = 1$, nous avons directement que $S_0 = a_0$ et $S_1 = - a_0 + a_1$ respectivement.

\medskip

Pour $n = 2$, nous avons les deux étapes suivantes.

\begin{enumerate}
	\item $[a_1 - a_0 , a_2 - a_1]$

	\item $[a_2 - a_1 - a_1 + a_0]$ d'où $S_2 = a_0 - 2 a_1 + a_2 $
\end{enumerate}

\medskip

Pour $n = 3$, nous avons les étapes suivantes où le triangle de Pascal pointe son nez.

\begin{enumerate}
	\item $[a_1 - a_0 , a_2 - a_1 , a_3 - a_2]$

	\item Le cas précédent donne alors $S_3 = (a_1 - a_0) - 2 (a_2 - a_1) + a_3 - a_2$ soit après simplification $S_3 = - a_0 + 3 a_1 - 3 a_2 + a_3$.
\end{enumerate}

\medskip

Plus généralement, nous allons supposer que $S_n = \dsum_{i=0}^{n} c_{i,n} a_k$ où les coefficients $c_{i,n}$ sont des entiers relatifs indépendants des valeurs des $a_k$.
En ajoutant un élément $a_{n+1}$ après la fin de la liste $[ a_0 , a_1 , \dots , a_n ]$, nous devons avoir :
\begin{flalign*}
	\dsum_{i=0}^{n+1} c_{i,n+1} a_k
		& = S_{n+1} 
		& \\[-1em]
		& = \dsum_{i=0}^{n} c_{i,n} (a_{i+1} - a_k)
		& \\
		& = \dsum_{i=1}^{n+1} c_{i-1,n} a_k - \dsum_{i=0}^{n} c_{i,n} a_k
		& \\
		& = - c_{0,n} a_0
		  + \dsum_{i=1}^{n} (c_{i-1,n} - c_{i,n}) a_k
		  + c_{n,n} a_n
\end{flalign*}

La suite $\left(  c_{i,n} \right)_{n \in \NN , i \in \ZintervalC{0}{n}}$ doit donc vérifier les conditions suivantes.

\begin{enumerate}
	\item $c_{0,0} = 1$
	
	\item $c_{0,1} = -1$ et $c_{1, 1} = 1$

	\item $c_{0,2} = 1$ , $c_{1, 2} = -2$ et $c_{2, 2} = 1$

	\item $\forall n \in \NN$, $c_{0,n+1} = - c_{0,n}$ et $c_{n+1,n+1} = c_{n,n}$.

	\item $\forall n \in \NN$, $\forall i \in \ZintervalC{1}{n}$, $c_{i,n+1} = c_{i-1,n} - c_{i,n}$.
\end{enumerate}


Il est aisé de voir que $c_{i,n} = (-1)^{i+n} \binom{n}{i}$ convient. Nous avons ainsi justifié que l'algorithme \ref{list-to-binomial} renvoie $\dsum_{i=0}^{n} (-1)^{i+n} \binom{n}{i} a_k$.
Ce résultat appliqué à la liste $[ k^n , (k+1)^n , \dots , (k+n)^n ]$ et la preuve de la section précédente nous donnent la très jolie formule suivante.
\[ \forall k \in \NN, \dsum_{i=0}^{n} (-1)^{i+n} \binom{n}{i} (k+i)^n = n! \]


La section finale suivante propose une preuve purement combinatoire de cette identité.


\begin{remark}
	Il est maintenant facile de prouver les identités suivantes.
	\[ \forall d \in \NNs, \forall k \in \NN, \dsum_{i=0}^{n} (-1)^{i+n} \binom{n}{i} (k+id)^n = d^n n! \]
\end{remark}


%
%
%\newpage
%\section{Une preuve combinatoire}
%
%Cf themathless travel

indiquer la source et rédiger tout efficacement !
%

\end{document}
