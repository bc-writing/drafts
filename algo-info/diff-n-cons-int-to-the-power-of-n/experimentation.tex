Le code suivant
\footnote{
	Ce fichier \texttt{diff-n-cons-int-to-the-power-of-n/exploring-int-version.py} est disponible dans le sous-dossier sur le lieu de téléchargement de ce document.
}
permet de tester notre conjecture audacieuse : aucun test raté n'est révélé.

\medskip

\inputcode{python}{diff-n-cons-int-to-the-power-of-n/exploring-int-version.py}


% -------------------- %


Comme les valeurs des naturels consécutifs ne semblent pas importantes, on peut pousser l'expérimentation en travaillant avec $k_1 \in \RR$ , $k_2 = k_1 + 1$ , $k_3 = k_1 + 2$ , \dots
Ceci étant dit, il n'est pas toujours possible de travailler avec des réels en informatique : l'usage des flottants s'accompagne de son lot d'arrondis.
Nous allons donc juste tester des valeurs rationnelles via le code suivant
\footnote{
	Ce fichier \texttt{diff-n-cons-int-to-the-power-of-n/exploring-frac-version.py} est disponible dans le sous-dossier sur le lieu de téléchargement de ce document.
}
qui ne révèle aucun test raté.

\medskip

\inputcode{python}{diff-n-cons-int-to-the-power-of-n/exploring-frac-version.py}


Il devient donc normal de penser que le phénomène est purement polynomial. Nous allons explorer ceci dans la section suivante.
