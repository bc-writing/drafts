De l'algorithme \ref{nat-algo}, nous allons juste garder les opérations sur les listes. Ceci nous donne l'algorithme suivant dont nous allons chercher à obtenir une formule explicite de la sortie $S_n$ en fonction de $a_0$, $a_1$, \dots{} , $a_n$.

\begin{algo}
	\algovoidcaption \label{list-to-binomial}

	\In{$[ a_0 , a_1 , \dots , a_n ]$ une liste de naturels}
	\Out{?}
	
	\addalgoblank
	
	\Actions{
		$L \Store [ a_0 , a_1 , \dots , a_n ]$
		\\
		\addalgoblank
		\While{$taille(L) \neq 1$}{
			$newL \Store \EmptyList$
			\\
			\ForRange*{i}{1}{taille(L) - 1}{
				\Append{$newL$}{$(L[i + 1] - L[i])$}
			}
			$L \Store newL$
		}
		\addalgoblank
		\Return{$L[1]$}
	}
\end{algo}


Pour $n = 0$ et $n = 1$, nous avons directement que $S_0 = a_0$ et $S_1 = - a_0 + a_1$ respectivement.

\medskip

Pour $n = 2$, nous avons les deux étapes suivantes.

\begin{enumerate}
	\item $[a_1 - a_0 , a_2 - a_1]$

	\item $[a_2 - a_1 - a_1 + a_0]$ d'où $S_2 = a_0 - 2 a_1 + a_2 $
\end{enumerate}

\medskip

Pour $n = 3$, nous avons les étapes suivantes où le triangle de Pascal pointe son nez.

\begin{enumerate}
	\item $[a_1 - a_0 , a_2 - a_1 , a_3 - a_2]$

	\item Le cas précédent donne alors $S_3 = (a_1 - a_0) - 2 (a_2 - a_1) + a_3 - a_2$ soit après simplification $S_3 = - a_0 + 3 a_1 - 3 a_2 + a_3$.
\end{enumerate}

\medskip

Plus généralement, nous allons supposer que $S_n = \dsum_{i=0}^{n} c_{i,n} a_k$ où les coefficients $c_{i,n}$ sont des entiers relatifs indépendants des valeurs des $a_k$.
En ajoutant un élément $a_{n+1}$ après la fin de la liste $[ a_0 , a_1 , \dots , a_n ]$, nous devons avoir :
\begin{flalign*}
	\dsum_{i=0}^{n+1} c_{i,n+1} a_k
		& = S_{n+1} 
		& \\[-1em]
		& = \dsum_{i=0}^{n} c_{i,n} (a_{i+1} - a_k)
		& \\
		& = \dsum_{i=1}^{n+1} c_{i-1,n} a_k - \dsum_{i=0}^{n} c_{i,n} a_k
		& \\
		& = - c_{0,n} a_0
		  + \dsum_{i=1}^{n} (c_{i-1,n} - c_{i,n}) a_k
		  + c_{n,n} a_n
\end{flalign*}

La suite $\left(  c_{i,n} \right)_{n \in \NN , i \in \ZintervalC{0}{n}}$ doit donc vérifier les conditions suivantes.

\begin{enumerate}
	\item $c_{0,0} = 1$
	
	\item $c_{0,1} = -1$ et $c_{1, 1} = 1$

	\item $c_{0,2} = 1$ , $c_{1, 2} = -2$ et $c_{2, 2} = 1$

	\item $\forall n \in \NN$, $c_{0,n+1} = - c_{0,n}$ et $c_{n+1,n+1} = c_{n,n}$.

	\item $\forall n \in \NN$, $\forall i \in \ZintervalC{1}{n}$, $c_{i,n+1} = c_{i-1,n} - c_{i,n}$.
\end{enumerate}


Il est aisé de voir que $c_{i,n} = (-1)^{i+n} \binom{n}{i}$ convient. Nous avons ainsi justifié que l'algorithme \ref{list-to-binomial} renvoie $\dsum_{i=0}^{n} (-1)^{i+n} \binom{n}{i} a_k$.
Ce résultat appliqué à la liste $[ k^n , (k+1)^n , \dots , (k+n)^n ]$ et la preuve de la section précédente nous donnent la très jolie formule suivante.
\[ \forall k \in \NN, \dsum_{i=0}^{n} (-1)^{i+n} \binom{n}{i} (k+i)^n = n! \]


La section finale suivante propose une preuve purement combinatoire de cette identité.


\begin{remark}
	Il est maintenant facile de prouver les identités suivantes que nous nommerons \myquote{identités du kid}
	\footnote{
		Donner sans explication, cette formule, qui contient $k+id$, est terrassante comme l'était en son temps Billy the Kid.
	}.
	\[ \forall d \in \NNs, \forall k \in \NN, \dsum_{i=0}^{n} (-1)^{i+n} \binom{n}{i} (k+id)^n = d^n n! \]
\end{remark}

