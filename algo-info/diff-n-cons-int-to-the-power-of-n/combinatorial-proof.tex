Notre dernière mission va consister à prouver la formule ci-après via de simples arguments de combinatoire.
\[ \forall k \in \NN, \dsum_{i=0}^{n} (-1)^{i+n} \binom{n}{i} (k+i)^n = n! \]

Commençons par noter que $n!$ est le nombre de permutations sur $[[n]] \eqdef \ZintervalC{1}{n}$ et $(k+i)^n$ celui des applications de $[[n]]$ sur $[[k+i]]$.
Nous noterons $\setgeo*{A}{n \to k+i}$ l'ensemble des applications de $[[n]]$ sur $[[k+i]]$.


\medskip


Comme $[[n]] \subseteq [[k+n]]$, nous allons nous concentrer sur $\ourapplis$. Dans ce contexte, pour compter les permutations de $[[n]]$, nous allons très classiquement étudier les éléments de $\ourapplis$ qui ne sont pas des permutations de $[[n]]$. 
Nous noterons $\npermu$  l'ensemble des permutations de $[[n]]$ sur $[[n]] \subset [[k+n]]$ .


\medskip


Soit donc $a \in \ourapplis$. Pour que $a$ ne soit pas une permutation sur $[[n]]$ il faut et il suffit que $\setproba*{I}{a}$ l'image de $a$ ne soit pas $[[n]]$.
Pour $i \in \ZintervalC{1}{n}$, notons alors $\badapplis{i}$ l'ensemble des applications $a$ telles que $\setproba*{I}{a}$ ne contienne pas l'élément $i$ de sorte que $\npermu = \badapplis{0} - \displaystyle\mathop{\cup}_{i=1}^{n} \badapplis{i}$.
Le principe d'inclusion-exclusion, le PIE pour les intimes, nous permet alors de faire les calculs suivants.
\begin{flalign*}
	n!
		& = \mathrm{card}\,\npermu 
		& \\
		& = \mathrm{card}\, \ourapplis
		  + \dsum_{i=1}^{n} (-1)^i \dsum_{1 \leq j_1 < \dots < j_i \leq n}
		    \mathrm{card} \left( \badapplis{j_1} \cap \dots \cap \badapplis{j_i} \right)
\end{flalign*}


Or $\mathrm{card} \left( \badapplis{j_1} \cap \dots \cap \badapplis{j_i} \right) = \binom{n}{i} (k + n - i)^n$ car nous avons $\binom{n}{i}$ choix possible pour $j_1$ , \dots{} , $j_i$ et il reste alors $(k + n - i)$ valeurs possibles pour les images des éléments de $[[n]]$. 
Il est clair que $\mathrm{card}\, \ourapplis = (-1)^0 \binom{n}{0} (k + n)^n$ de sorte que nous obtenons :
\[ n! = \dsum_{i=0}^{n} (-1)^i \binom{n}{i} (k + n - i)^n \]


Pour retomber sur la formule à démontrer, il suffit de faire le changement d'indices $j = n - i$ pour obtenir l'identité suivante, notre objectif, en utilisant $n-j \equiv j + n \modulo 2$ et $\binom{n}{i} = \binom{n}{n-i}$.
\[ n! = \dsum_{j=0}^{n} (-1)^{j+n} \binom{n}{j} (k + j)^n \text{ avec $k\in\NN$ quelconque} \]