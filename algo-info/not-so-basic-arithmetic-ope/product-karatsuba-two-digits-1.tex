\subsubsection{Avec le produit des sommes de Karatsuba}

Avec $n = 2$, nous avons une façon originale de faire le produit de deux entiers à deux chiffres. Considérons par exemple le produit $56 \cdot 74$. Voici ce que donne la méthode de Karatsuba dans ce cas.
	
\begin{enumerate}
	\item Ici $a = 5$ , $b = 6$ , $A = 7$ et $B = 4$ .

	\item $c_2 = 5 \cdot 7 = 35$ .

	\item $c_0 = 6 \cdot 4 = 24$ .

	\item $c_1 = (5 + 6) \, (7 + 4) - 35 - 24 = 121 - 35 - 24 = 62$ .

	\item $56 \cdot 74 = c_2 \cdot 10^2 + c_1 \cdot 10^1 + c_0 = 3500 + 620 + 24 = 4144$ .
\end{enumerate}
	
\medskip
	
Visuellement nous avons fait les calculs suivants où $c_0$ et $c_2$ sont des multiplications \og verticales \fg{} et $c_1$ s'obtient en soustrayant ces deux multiplications au produit \og d'additions horizontales \fg{}.
Cette méthode est efficace avec un crayon et du papier.
	
\begin{center}
	\medskip
	
	$\begin{NiceMatrix}[
		last-col,
		code-for-last-col = \RED{}
	]
		           & \phPLUS{} &            & \phPLUS{} & 5         & \PLUS{} & 6
        \\
                   &           &            &           &           &         &   
                   &&
                   \EXPL{121 = 11 \cdot 11}
        \\
		           &           &            &           & 7         & \PLUS{} & 4
        \\
        \cline{1-7}
		\ORANGE{}3 &           & \ORANGE{}5 &           & \GREEN{}2 &         & \GREEN{}4   
                   &&
                   \EXPL{59 = 35 \mathbin{\smash{+}} 24}
        \\
                   &           & \RED{}6    &           & \RED{}2   &         & \phantom{8}
                   &&
                   \EXPL{62 = 121 \mathbin{\smash{-}} 59}
        \\
        \cline{1-7}
        4          &           & 1          &           & 4         &         & 4
        \\
		\CodeAfter
        \begin{tikzpicture}[shorten > = 1mm, shorten < = 1mm, line width=.75pt,>=stealth]
            \draw [orange, <->]        (1-5.south) -- (3-5.north) ;
            \draw [green!50!blue, <->] (1-7.south) -- (3-7.north) ;
            \draw [red, ->]            (1-7.east)  -- (2-9.west) ;
            \draw [red, ->]            (3-7.east)  -- (2-9.west) ;
            \draw [red, ->]            (4-7.east)  -- (4-9.west) ;
            \draw [red, <-]            (5-7.east)  -- (5-9.west) ;
        \end{tikzpicture}
    \end{NiceMatrix}$
\end{center}
