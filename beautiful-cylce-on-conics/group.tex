Dans les preuves des sections précédentes, on constate que la construction géométrique se traduit par des identités basiques qui la clé de la périodicité de la suite de points $\left( A_i \right)$ . 
\begin{enumerate}
	\item Pour le cercle, c'était $\theta_i + \theta_{i+1} = \theta_{i+3} + \theta_{i+4}$ avec $\theta_i = \vangleorient{OI}{OA_i}$ . Nous nous intéressons ici juste à la preuve \ref{circle-proof-2}. 

	\item Pour la parabole, c'était $x_i + x_{i+1} = x_{i+3} + x_{i+4}$ . 

	\item Pour l'hyperbole, c'était  $x_i x_{i+1} = x_{i+3} x_{i+4}$ . 
\end{enumerate}


\medskip


En fait, si $\Gamma$ désigne une ellipse, une parabole quelconque, ou une hyperbole quelconque, on peut démontrer que l'existence d'un point $E \in \Gamma$ tel que la construction qui à $A \in \Gamma$ et $B \in \Gamma$ associe le point $S$ tel que $\geoset*{C}{AB} \,/\!/\, \geoset*{C}{ES}$ définisse une loi de groupe $\star$ sur $\Gamma$ avec $E$ pour élément neutre. 
Les cas étudiés dans ce document n'en sont que des cas particuliers. 
Le lecteur intéressé peut se reporter à  
\emph{\og Faire des additions modulaires sur une ellipse \fg} ,
\emph{\og Faire des additions sur une parabole \fg} 
et
\emph{\og Faire des produits sur une hyperbole \fg}
qui ont été rédigés par l'auteur de ce document
\footnote{
	Voir \texttt{addition-on-ellipsis.pdf} , \texttt{addition-on-parabolas.pdf}  et \texttt{product-on-hyperbolas.pdf}  à l'adresse \url{https://github.com/bc-writing/drafts} .
}.


\medskip


Tout comme dans la preuve dans la section \ref{circle-proof-2} , on constate alors que la construction \emph{\og magique \fg} est telle que $\forall n \in \NN_{\geq 5}$ , $A_{n}$ est l'unique point de $\Gamma$ tel que $\geoset*{C}{A_{n} A_{n-1}} \,/\!/\, \geoset*{C}{A_{n-3} A_{n-4}}$ . On en déduit alors que $A_{n} \star A_{n-1} = A_{n-3} \star A_{n-4}$ . On retombe alors sur une identité similaire à celle vue précédemment mais avec la loi $\star$ au lieu des lois $+$ et $\times$ . On conclut alors de la même façon. Que les mathématiques sont belles quand on prend le temps de les écouter !
