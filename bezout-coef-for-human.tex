\documentclass[12pt]{amsart}
\usepackage[T1]{fontenc}
\usepackage[utf8]{inputenc}

\usepackage[top=1.95cm, bottom=1.95cm, left=2.35cm, right=2.35cm]{geometry}


\usepackage{hyperref}
\usepackage{enumitem}
\usepackage{tcolorbox}
\usepackage{cleveref}
\usepackage{multicol}
\usepackage{amsmath}
\usepackage{nicematrix}
\usepackage[french]{babel}
\usepackage[
    type={CC},
    modifier={by-nc-sa},
    version={4.0},
]{doclicense}

\usepackage{tikz}
\usetikzlibrary{arrows, matrix, positioning,fit}


\usepackage{lymath}
\usepackage{lyalgo}


\newtheorem{fact}{Fait}
\newtheorem*{fact*}{Fait}

\newtheorem{example}{Exemple}

\newtheorem{remark}{Remarque}
\newtheorem*{remark*}{Remarque}

\newtheorem*{proof*}{Preuve}

\setlength\parindent{0pt}


% -- NEW ENVIRONMENTS -- %




% -- NEW MACROS -- %

\newcommand\ph{\phantom{x}}

\newcommand\myquote[1]{\emph{\og #1 \fg}}

\newcommand\bb{Bachet-Bézout}

% Arguments
%    #1 Caption
%    #2 TiKZ file

\newcommand\showstep[2]{
    \begin{center}
        \input{bezout-coef-for-human/#2.tkz}
    
        \vfill
        
        \small \itshape #1
    \end{center}    
}


\newcommand\showstepnovfill[2]{
    \begin{center}
        \input{bezout-coef-for-human/#2.tkz}
  
        \small \itshape #1
    \end{center}    
}


% -- SETTINGS -- %

\tikzset{
    % Good spacing hack (the ghost mode)
     gs/.style={
        rectangle,
        thick,
        text width=3.5em,
        align=center,
        draw=white,
        rounded corners,
        minimum height=2em
    }, 
	% Common
    rc/.style={rectangle,
        thick,rounded corners,draw,
        minimum height=2em},
    % Operator in a circle
    oc/.style={
        circle,
        draw,
        fill=white,
        inner sep=1pt, 
        outer sep=1pt
    },
    % Blue frame
    bf/.style={
        rc,
        align=center,
        draw=blue,
        fill=blue!20,
        text width=3.5em,
    },
    % Explain remained
    er/.style={
        rc,
        align=center,
        text width=3.5em,
        draw=gray,
        text=darkgray
    },
    % Explain activated
    ea/.style={
        rc,
        align=center,
        text width=3.5em,
        draw=gray,
        text=darkgray
    },
    % Long explain activated
    le/.style={
        rc,
        align=center,
        text width=8em,
        draw=gray,
        text=darkgray
    },
    % Blur effect
    be/.style={
        rectangle,
        thick,rounded corners,
        minimum height=2em,
        align=center,
        opacity = 0.3,
        text width=3.5em,
    },
    % Focus effect
    fe/.style={
        rc,
        align=center,
        text width=8em,
        draw=red,
        text=magenta
    },
    % Focus effect Bis
    feb/.style={
        rc,
        align=center,
        text width=8em,
        draw=blue,
        text=violet
    },
    % Long focus effect
    lfe/.style={
        rc,
        align=center,
        text width=12em,
        draw=red,
        text=magenta
    },
    % Long focus effect bis
    lfeb/.style={
        rc,
        align=center,
        text width=12em,
        draw=blue,
        text=violet
    },
}
      

     
     
\begin{document}

\title{BROUILLON - Algorithme d'Euclide et coefficients de Bachet-Bézout pour les humains}
\author{Christophe BAL}
\date{10 Sept. 2019 - 13  Sept. 2019}

\maketitle

\begin{center}
    \itshape
    Document, avec son source \LaTeX, disponible sur la page
    
    \url{https://github.com/bc-writing/drafts}.
\end{center}


\bigskip


\begin{center}
    \hrule\vspace{.3em}
    {
        \fontsize{1.35em}{1em}\selectfont
        \textbf{Mentions \og légales \fg}
    }
            
    \vspace{0.45em}
    \doclicenseThis
    \hrule
\end{center}


\setcounter{tocdepth}{2}
\tableofcontents


% --------------- %


\section{Où allons-nous ?}

Un résultat classique d'arithmétique dit qu'étant donné $(a ; b) \in \NNs \!\times \NNs$, il existe $(u ; v) \in \ZZ \!\times \ZZ$ tel que $au + bv = \pgcd(a ; b)$. Les entiers $u$ et $v$ seront appelés \myquote{coefficients de \bb}.
Notons qu'il n'y a pas unicité car nous avons par exemple :
\[(-3) \times 12 + 1 \times 42 = 4 \times 12 + (-1) \times 42 = 6 = \pgcd(12 ; 42)\]

\medskip

Nous allons voir comment trouver de tels entiers $u$ et $v$ tout d'abord de façon humainement rapide puis ensuite via des algorithmes efficaces pour un ordinateur.


% --------------- %


\section{L'algorithme \og human friendly \fg{} appliqué de façon magique}

\subsection{Un exemple complet façon \myquote{diaporama}}

Sur le lieu de téléchargement de ce document se trluve un fichier \verb+PDF+ de chemin relatif \verb+bezout-coef-for-human/slide-version.pdf+ présentant la méthode sous la forme d'un diaporama. Nous vous conseillons de le regarder avant de lire les explications suivantes.


% --------------- %


\subsection{Phase 1 -- Au début était l'algorithme d'Euclide}

Pour chercher des coefficients de \bb pour $(a ; b) = (27 ; 141)$, on commence par appliquer l'algorithme d'Euclide \myquote{verticalement} comme suit.

\begin{multicols}{2}
	\showstep{Étape 1: le plus grand naturel est mis au-dessus.}{tikz/27-141[all]/down-1}

	\columnbreak
	
	\showstep{Étape 2: deux naturels à diviser.}{tikz/27-141[all]/down-2}
\end{multicols}

\begin{multicols}{2}
	\showstep{Étape 3: première division euclidienne.}{tikz/27-141[all]/down-3}

	\columnbreak
	
	\showstep{Étape 4: on passe aux deux naturels suivants.}{tikz/27-141[all]/down-4}
\end{multicols}


\medskip


En répétant ce processus, nous arrivons à la représentation suivante.

\showstepnovfill{Étape finale (1\iere phase): l'algorithme d'Euclide \myquote{vertical}.}{tikz/27-141[main]/down}


% --------------- %


\subsection{Phase 2 -- Remontée facile des étapes}

La méthode classique consiste à remonter les calculs. Mais comment faire cette remontée tout en évitant un claquage neuronal ? L'astuce est la suivante.

\vfill\newpage

\begin{multicols}{2}
	\showstep{Étape 1: ajout d'une nouvelle colonne.}{tikz/27-141[all]/up-1}

	\columnbreak
	
	\showstep{Étape 2: on n'utilise pas la colonne centrale.}{tikz/27-141[all]/up-2}
\end{multicols}

\showstepnovfill{Étape 3: on fait une sorte de division \myquote{inversée}.}{tikz/27-141[all]/up-3}

\showstepnovfill{Étape 4: on passe aux trois naturels suivants.}{tikz/27-141[all]/up-4}


\medskip


\vfill\newpage

En répétant ce processus, nous arrivons à la représentation suivante.

\showstepnovfill{Étape finale (2\ieme phase): remontée à mains nues des calculs.}{tikz/27-141[main]/up}


% --------------- %


\subsection{Et voilà comment conclure !}

\showstepnovfill{Étape finale (la vraie): on finit avec un produit en croix.}{tikz/27-141[main]/last}


\medskip


Des coefficients de \bb s'obtiennent sans souci via l'équivalence suivante où nous avons $3 = \mathrm{pgcd}(27 , 141)$.
\[141 \times 4 - 27 \times 21 = -3 \,\Longleftrightarrow\, 27 \times 21 - 141 \times 4 = 3\]


\medskip


Nous allons voir, dans la section qui suit, que l'on obtient forcément à la fin $\pm \mathrm{pgcd}(27 , 141)$. 


\medskip


En pratique, nous n'avons pas besoin de détailler les calculs comme nous l'avons fait à certains moments afin d'expliquer comment procéder.
Avec ceci en tête, on comprend toute l'efficacité de la méthode présentée, mais pas encore justifiée, car il suffit de garder une trace minimale, mais complète, des étapes tout en ayant à chaque étape des opérations assez simples à effectuer.
Il reste à démontrer que notre méthode marche à tous les coups. Ceci est le propos de la section suivante.
	


% --------------- %


\section{Pourquoi cela marche-t-il ?}

\subsection{Avec des arguments élémentaires} Commençons par une preuve explicative qui malheureusement ne nous permet pas de voir d'où vient l'astuce \emph{(nous explorerons ceci dans la sous-section suivante)}. 

\showstepnovfill{Calculs faits dans les deux phases.}{tikz/why/twophases-focus}


\bigskip


Par construction, nous avons $a = qb + r$ et $X = qY + Z$. Ceci nous donne :

\vspace{-1em}

\begin{flalign*}
	d &= aY - bX               & \\
	  &= (qb + r)Y - b(qY + Z) & \\
	  &= rY - bZ               & \\
	  &= -e                    & \\
\end{flalign*}

\vspace{-1em}


Donc si l'on fait \myquote{glisser} des carrés sur les deux colonnes de droite, les produits en croix dans ces carrés ne différeront que par leur signe. 


\medskip


La représentation symbolique \myquote{complète} ci-dessous donne $aY - bX = \pm \pgcd(a ; b)$ car le dernier reste non nul de l'algorithme d'Euclide est $\pgcd(a ; b)$. Ceci prouve la validité de la méthode dans le cas général. On comprend au passage l'ajout initial du $0$ et du $1$ dans la 3\ieme{} colonne \emph{(bien entendu, $(-1)$ aurait aussi pu convenir)}.

\showstepnovfill{Représentation symbolique au complet.}{tikz/why/twophases-all-decorated}
	

% --------------- %


\subsection{Avec des matrices pour y voir plus clair}

Oublions tout ce que nous avons vu précédemment.
Soit $(a ; b) \in \NNs \!\times \NNs$ avec $a > b$. Nous cherchons $(u ; v) \in \ZZ \!\times \ZZ$ tel que $au + bv = \pgcd(a ; b)$. La petite astuce est de noter que $au + bv = \det M$ où 
$M 
 =
 \begin{pmatrix}
	a & -v \\ 
	b & u
 \end{pmatrix}$.
Nous allons poser $X = -v$ et $Y = u$ de sorte que 
$M 
 =
 \begin{pmatrix}
	a & X \\ 
	b & Y
 \end{pmatrix}$
 et raisonner en supposant l'existence de $u$ et $v$
 \footnote{
 	Ce n'est qu'à la fin de la preuve que nous aurons effectivement prouver l'existence de $u$ et $v$. 
 }.
 

\medskip


Soit ensuite $a = qb + r$ la division euclidienne de $a$ par $b$. L'algorithme d'Euclide nous fait alors travailler avec $(b ; r)$ au lieu de $(a ; b)$. Comme $r = a - qb$, on peut considérer la matrice
$N 
 =
 \begin{pmatrix}
	a - qb & X - qY \\ 
	b      & Y
 \end{pmatrix}$
qui vérifie $\det N = \det M$ puis, afin d'avoir $b$ en haut, la matrice
$P 
 =
 \begin{pmatrix}
	b      & Y      \\
	a - qb & X - qY 
 \end{pmatrix}$
qui vérifie $\det P = -\det M$.
 

\medskip


Notant $Z = X - qY$, de sorte que 
$P 
 =
 \begin{pmatrix}
	b & Y \\ 
	r & Z
 \end{pmatrix}$,
nous avons $X = Z + qY$. Ceci justifie la construction utilisée lors de la phase de remontée.
 

\bigskip


Pour passer à une nouvelle preuve, notons que
$\begin{pmatrix}
	b & Y \\ 
	r & Z
 \end{pmatrix}
 =
 \begin{pmatrix}
	0 & 1  \\ 
	1 & -q
 \end{pmatrix}
 \cdot
 \begin{pmatrix}
	a & X \\ 
	b & Y
 \end{pmatrix}$
puis introduisons les notations suivantes.

\begin{itemize}[label=\small\textbullet]
	\item $r_0 = a$, $r_1 = b$, $Z_0 = X$ et $Z_1 = Y$ où $Z_0$ et $Z_1$ ne sont pas connus pour le moment.

	\item Pour $k \in \NNs$, on note $r_{k-1} = r_k q_k + r_{k+1}$ la division euclidienne de $r_{k-1}$ par $r_k$, puis ensuite on pose $Z_{k+1} = Z_{k-1} - Z_k q_k$ de sorte que $Z_{k-1} = Z_k q_k + Z_{k+1}$.

	\item On note enfin
	      $M_k 
           =
           \begin{pmatrix}
          	  r_k     & Z_k     \\ 
          	  r_{k+1} & Z_{k+1}
           \end{pmatrix}$
          pour $k \in \NN$ et
	      $Q_k 
           =
           \begin{pmatrix}
          	  0 & 1     \\ 
          	  1 & - q_k
           \end{pmatrix}$
          pour $k \in \NNs$ de sorte que nous avons $M_{k+1} = Q_{k+1} \cdot M_k$ pour $k \in \NN$. Il est immédiat que $Q_k$ est inversible d'inverse
	      $R_k 
           =
           \begin{pmatrix}
          	  q_k & 1 \\ 
          	  1   & 0
           \end{pmatrix}$
          avec $\det R_k = -1$.
\end{itemize}


L'algorithme d'Euclide nous donne l'existence de $n \in \NN$ un indice minimal tel que $r_{n+1} = 0$ et $r_n = \pgcd(a ; b)$.


Comme
$\displaystyle M_n = \prod_{k = n}^{1} Q_k \cdot M_0$,
nous avons
$\displaystyle M_0 = \prod_{k = 1}^{n} R_k \cdot M_n$
avec
$M_0
 =
 \begin{pmatrix}
    r_0 & Z_0 \\ 
    r_1 & Z_1
 \end{pmatrix}
 =
 \begin{pmatrix}
    a & X \\ 
    b & Y
 \end{pmatrix}$
et
$M_n
 =
 \begin{pmatrix}
    r_n & Z_n     \\ 
    0   & Z_{n+1}
 \end{pmatrix}
 =
 \begin{pmatrix}
    \pgcd(a ; b) & Z_n     \\ 
    0            & Z_{n+1}
 \end{pmatrix}$.
 
 
 \medskip
 
 
Comme $\det M_0 = \pm 1 \det M_n$ car $\det R_k = -1$, il suffit de choisir $Z_{n+1} = 1$, avec $Z_n$ quelconque, pour avoir $aY - bX = \pm \pgcd(a ; b)$. Voilà comment découvrir la méthode visuelle vue précédemment où le choix particulier $Z_n = 0$ simplifie les tous premiers calculs.


\begin{remark}
	Comme $Z_n$ peut être quelconque, nous pouvons produire une infinité de coefficients de \bb. Cele vient du fait que les étapes de \emph{\og remontée \fg} sont du type $Z_{k-1} = Z_k q_k + Z_{k+1}$ toujours avec $q_k > 0$.
\end{remark}



% --------------- %


%\section{Des algorithmes \myquote{computer friendly}, de grands classiques}

%\input{bezout-coef-for-human/computer}

\bigskip

\hrule

\section{AFFAIRE À SUIVRE...}

\bigskip

\hrule
\end{document}
