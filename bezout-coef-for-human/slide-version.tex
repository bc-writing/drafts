\documentclass[10pt]{beamer}

% -- PACKAGES -- %

\usepackage[french]{babel}
\usepackage[utf8x]{inputenc}	

\usepackage{tikz}
\usetikzlibrary{arrows, matrix, positioning}


% -- NEW MACROS -- %

\newcommand\ph{\phantom{x}}


% -- SETTINGS -- %

\mode<presentation>{
    \usetheme{default}
    \usecolortheme{default}
    \usefonttheme{default} 
    \setbeamertemplate{caption}[numbered]	
}


\tikzset{
	% Common style
	common/.style={
		rectangle, 
		thick,
		rounded corners,
		align=center,
		minimum height=2em
	},
    % Operator in a circle
    oc/.style={
        circle,
        draw,
        fill=white,
        inner sep=1pt, 
        outer sep=1pt
    },
    % Good spacing hack (the ghost mode)
    gs/.style={
        common,
        text width=3.5em,
        draw=white
    },
    % Blue frame
    bf/.style={
        common,
        draw=blue,
        fill=blue!20,
        text width=3.5em,
    },
    % Explain remained
    er/.style={
        common,
        text width=3.5em,
        draw=gray,
        text=darkgray
    },
    % Explain activated
    ea/.style={
        common,
        text width=3.5em,
        draw=gray,
        text=darkgray
    },
    % Long explain activated
    le/.style={
        common,
        text width=8em,
        draw=gray,
        text=darkgray
    },
    % Blur effect
    be/.style={
        common,
        opacity = 0.3,
        text width=3.5em,
    },
    % Focus effect
    fe/.style={
        common,
        text width=8em,
        draw=red,
        text=magenta
    },
    % Focus effect Bis
    feb/.style={
        common,
        text width=8em,
        draw=blue,
        text=violet
    },
    % Long focus effect
    lfe/.style={
        common,
        text width=12em,
        draw=red,
        text=magenta
    },
    % Long focus effect bis
    lfeb/.style={
        common,
        text width=12em,
        draw=blue,
        text=violet
    },
}

      
\begin{document}

\begin{frame}
	\frametitle{Coefficients de Bachet-Bézout pour les humains}
	
	\centering
	
	Un moyen efficace de trouver des coefficients de Bachet-Bézout.
	
	\bigskip
	
	Un exemple avec $a= 141$ et $b = 27$.
\end{frame}

	
\foreach \n in {1,...,9}{	
	\begin{frame}[t]
		\frametitle{Algorithme d'Euclide posé verticalement}
	
		\phantom{X}
	
		\input{tikz/27-141[all]/down-\n.tkz}
	\end{frame}
}

	
\foreach \n in {1,...,8}{	
	\begin{frame}[t]
		\frametitle{Remontée astucieuse des calculs}
	
		\phantom{X}
	
		\input{tikz/27-141[all]/up-\n.tkz}
	\end{frame}
}

	
\begin{frame}[t]
	\frametitle{Le tour est joué}
	
	\phantom{X}
	
	\input{tikz/27-141[all]/last.tkz}
\end{frame}


\begin{frame}
	\frametitle{Conclusion}
	
	\centering
	
	Finalement nous avons :
	
	\bigskip
	
	$141 \times 4 - 27 \times 21 = -3$
	
	\medskip
	
	$\Updownarrow$
	
	\medskip
	
	$27 \times 21 - 141 \times 4 = 3$

	\medskip
	
	$\Updownarrow$
	
	\medskip
	
	\fcolorbox{red}{white}{$27 \times 21 - 141 \times 4 = \mathrm{pgcd}(27 , 141)$}
	
	\pause
	
	\bigskip
	
	\textbf{\Large Et cela marche à tous les coups !}
\end{frame}

\end{document}
