Les deux algorithmes \ref{algo-human-paper-bis} et  \ref{algo-human-matrix} sont informatiquement très maladroits. Voici pourquoi.

\begin{enumerate}
	\item L'algorithme \ref{algo-human-paper-bis} utilise une liste de taille la somme de $1$ et du nombre d'étapes de l'algo\-rithme d'Euclide.
	Dans la section \ref{euclide-complexity}, nous verrons que ce nombre d'étapes est environ égal à $5$ fois le nombre de chiffres de l'écriture décimale du plus petit des deux entiers.
	Donc si l'on travaille avec des entiers de tailles assez grande, la taille de la liste risque de devenir problématique.


	\item Le problème avec l'algorithme \ref{algo-human-matrix} est le produit cumulé des matrices. Il serait bien de pouvoir s'en passer !
\end{enumerate}

Dans la section qui suit nous allons voir que l'on peut chercher plus efficacement des coefficients de \bb{}.
Nous expliquerons que cette nouvelle approche a une mécanique cachée très proche de celle de l'algorithme \ref{algo-human-paper}.
