\subsection{Avec des arguments élémentaires} Commençons par une preuve explicative qui malheureusement ne nous permet pas de voir d'où vient l'astuce \emph{(nous explorerons ceci dans la sous-section suivante)}. 

\showstepnovfill{Calculs faits dans les deux phases.}{tikz/why/twophases-focus}


\bigskip


Par construction, nous avons $a = qb + r$ et $X = qY + Z$. Ceci nous donne :

\vspace{-1em}

\begin{flalign*}
	d &= aY - bX               & \\
	  &= (qb + r)Y - b(qY + Z) & \\
	  &= rY - bZ               & \\
	  &= -e                    & \\
\end{flalign*}

\vspace{-1em}


Donc si l'on fait \myquote{glisser} des carrés sur les deux colonnes de droite, les produits en croix dans ces carrés ne différeront que par leur signe. 


\medskip


La représentation symbolique \myquote{complète} ci-dessous donne $aY - bX = \pm \pgcd(a ; b)$ car le dernier reste non nul de l'algorithme d'Euclide est $\pgcd(a ; b)$. Ceci prouve la validité de la méthode dans le cas général. On comprend au passage l'ajout initial du $0$ et du $1$ dans la 3\ieme{} colonne. Bien entendu, $(-1)$ aurait pu convenir à la place de $1$, et l'on constate que $0$ peut être remplacé par n'importe quelle valeur entière.

\showstepnovfill{Représentation symbolique au complet.}{tikz/why/twophases-all-decorated}
	

% --------------- %


\subsection{Avec des matrices pour y voir plus clair}

Oublions tout ce que nous avons vu précédemment.
Soit $(a ; b) \in \NNs \!\times \NNs$ avec $a > b$. Nous cherchons $(u ; v) \in \ZZ \!\times \ZZ$ tel que $au + bv = \pgcd(a ; b)$. La petite astuce est de noter que $au + bv = \det M$ où 
$M 
 =
 \begin{pmatrix}
	a & -v \\ 
	b & u
 \end{pmatrix}$.
Nous allons poser $X = -v$ et $Y = u$ de sorte que 
$M 
 =
 \begin{pmatrix}
	a & X \\ 
	b & Y
 \end{pmatrix}$
 et raisonner en supposant l'existence de $u$ et $v$
 \footnote{
 	Ce n'est qu'à la fin de la preuve que nous aurons effectivement prouver l'existence de $u$ et $v$. 
 }.
 

\medskip


Soit ensuite $a = qb + r$ la division euclidienne de $a$ par $b$. L'algorithme d'Euclide nous fait alors travailler avec $(b ; r)$ au lieu de $(a ; b)$. Comme $r = a - qb$, on peut considérer la matrice
$N 
 =
 \begin{pmatrix}
	a - qb & X - qY \\ 
	b      & Y
 \end{pmatrix}$
qui vérifie $\det N = \det M$ puis, afin d'avoir $b$ en haut, la matrice
$P 
 =
 \begin{pmatrix}
	b      & Y      \\
	a - qb & X - qY 
 \end{pmatrix}$
qui vérifie $\det P = -\det M$.
 

\medskip


Notant $Z = X - qY$, de sorte que 
$P 
 =
 \begin{pmatrix}
	b & Y \\ 
	r & Z
 \end{pmatrix}$,
nous avons $X = Z + qY$. Ceci justifie la construction utilisée lors de la phase de remontée.
 

\bigskip


Pour passer à une nouvelle preuve, notons que
$\begin{pmatrix}
	b & Y \\ 
	r & Z
 \end{pmatrix}
 =
 \begin{pmatrix}
	0 & 1  \\ 
	1 & -q
 \end{pmatrix}
 \cdot
 \begin{pmatrix}
	a & X \\ 
	b & Y
 \end{pmatrix}$
puis introduisons les notations suivantes.

\begin{itemize}[label=\small\textbullet]
	\item $r_0 = a$, $r_1 = b$, $Z_0 = X$ et $Z_1 = Y$ où $Z_0$ et $Z_1$ ne sont pas connus pour le moment.

	\item Pour $k \in \NNs$, on note $r_{k-1} = r_k q_k + r_{k+1}$ la division euclidienne de $r_{k-1}$ par $r_k$, puis ensuite on pose $Z_{k+1} = Z_{k-1} - Z_k q_k$ de sorte que $Z_{k-1} = Z_k q_k + Z_{k+1}$.

	\item On note enfin
	      $M_k 
           =
           \begin{pmatrix}
          	  r_k     & Z_k     \\ 
          	  r_{k+1} & Z_{k+1}
           \end{pmatrix}$
          pour $k \in \NN$ et
	      $Q_k 
           =
           \begin{pmatrix}
          	  0 & 1     \\ 
          	  1 & - q_k
           \end{pmatrix}$
          pour $k \in \NNs$ de sorte que nous avons $M_{k+1} = Q_{k+1} \cdot M_k$ pour $k \in \NN$. Il est immédiat que $Q_k$ est inversible d'inverse
	      $R_k 
           =
           \begin{pmatrix}
          	  q_k & 1 \\ 
          	  1   & 0
           \end{pmatrix}$
          avec $\det R_k = -1$.
\end{itemize}


L'algorithme d'Euclide nous donne l'existence de $n \in \NN$ un indice minimal tel que $r_{n+1} = 0$ et $r_n = \pgcd(a ; b)$.


Comme
$\displaystyle M_n = \prod_{k = n}^{1} Q_k \cdot M_0$,
nous avons
$\displaystyle M_0 = \prod_{k = 1}^{n} R_k \cdot M_n$
avec
$M_0
 =
 \begin{pmatrix}
    r_0 & Z_0 \\ 
    r_1 & Z_1
 \end{pmatrix}
 =
 \begin{pmatrix}
    a & X \\ 
    b & Y
 \end{pmatrix}$
et
$M_n
 =
 \begin{pmatrix}
    r_n & Z_n     \\ 
    0   & Z_{n+1}
 \end{pmatrix}
 =
 \begin{pmatrix}
    \pgcd(a ; b) & Z_n     \\ 
    0            & Z_{n+1}
 \end{pmatrix}$.
 
 
 \medskip
 
 
Comme $\det M_0 = \pm 1 \det M_n$ car $\det R_k = -1$, il suffit de choisir $Z_{n+1} = 1$, avec $Z_n$ quelconque, pour avoir $aY - bX = \pm \pgcd(a ; b)$. Voilà comment découvrir la méthode visuelle vue précédemment où le choix particulier $Z_n = 0$ simplifie les tous premiers calculs.


\begin{remark}
	Comme $Z_n$ peut être quelconque, nous pouvons produire une infinité de coefficients de \bb. En effet, les étapes de \emph{\og remontée \fg} sont du type $Z_{k-1} = Z_k q_k + Z_{k+1}$ avec toujours $q_k > 0$, donc des valeurs naturelles non nulles
	\footnote{
		En fait, il n'est pas besoin d'imposer une condition particulière à $Z_n$. A vous de voir pourquoi.
	}
	différentes de $Z_n$ produiront des valeurs différentes de $Z_0$.
\end{remark}
