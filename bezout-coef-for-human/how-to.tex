\subsection{Un exemple complet façon \myquote{diaporama}}

Sur le lieu de téléchargement de ce document se trouve un fichier \verb+PDF+ de chemin relatif \verb+bezout-coef-for-human/slide-version.pdf+ présentant la méthode sous la forme d'un diaporama. Nous vous conseillons de le regarder avant de lire les explications ci-après.


% --------------- %


\subsection{Phase 1 -- Au début était l'algorithme d'Euclide...}

Pour chercher des coefficients de \bb{} pour $(a ; b) = (27 ; 141)$, on commence par appliquer l'algorithme d'Euclide \myquote{verticalement} comme suit.

\begin{multicols}{2}
	\showstep{Étape 1: le plus grand naturel est mis au-dessus.}{tikz/27-141[all]/down-1}

	\columnbreak
	
	\showstep{Étape 2: deux naturels à diviser.}{tikz/27-141[all]/down-2}
\end{multicols}

\begin{multicols}{2}
	\showstep{Étape 3: première division euclidienne.}{tikz/27-141[all]/down-3}

	\columnbreak
	
	\showstep{Étape 4: on passe aux deux naturels suivants.}{tikz/27-141[all]/down-4}
\end{multicols}


\medskip


En répétant ce processus, nous arrivons à la représentation suivante où nous n'avons pas besoin de garder la trace des divisions euclidiennes.

\showstepnovfill{Étape finale (1\iere phase): l'algorithme d'Euclide \myquote{vertical}.}{tikz/27-141[main]/down}



% --------------- %


\subsection{Phase 2 -- Remontée facile des étapes}

La méthode classique consiste à remonter les calculs. Mais comment faire cette remontée tout en évitant un claquage neuronal ? L'astuce est la suivante.

\vfill\newpage

\begin{multicols}{2}
	\showstep{Étape 1: ajout d'une nouvelle colonne.}{tikz/27-141[all]/up-1}

	\columnbreak
	
	\showstep{Étape 2: on n'utilise pas la colonne centrale.}{tikz/27-141[all]/up-2}
\end{multicols}

\showstepnovfill{Étape 3: on fait une sorte de division \myquote{inversée}.}{tikz/27-141[all]/up-3}

\showstepnovfill{Étape 4: on passe aux trois naturels suivants.}{tikz/27-141[all]/up-4}


\medskip


\vfill\newpage

En répétant ce processus, nous arrivons à la représentation suivante.

\showstepnovfill{Étape finale (2\ieme phase): remontée en voie libre des calculs.}{tikz/27-141[main]/up}


\begin{remark}
	En remontant les calculs sur la colonne centrale, on dispose d'un moyen simple de construire deux entiers $a$ et $b$ de $\pgcd$ fixé et avec des valeurs des quotients intermédiaires $q_k$ choisis.
\end{remark}


% --------------- %


\subsection{Et voilà comment conclure !}

\showstepnovfill{Étape finale (la vraie): on finit avec un produit en croix.}{tikz/27-141[main]/last}


\medskip


Des coefficients de \bb{} s'obtiennent sans souci via l'équivalence suivante où nous avons $3 = \mathrm{pgcd}(27 , 141)$.
\[141 \times 4 - 27 \times 21 = -3 \,\Longleftrightarrow\, 27 \times 21 - 141 \times 4 = 3\]


\medskip


Nous allons voir, dans la section qui suit, que l'on obtient forcément à la fin $\pm \mathrm{pgcd}(27 , 141)$. 


\medskip


En pratique, nous n'avons pas besoin de détailler les calculs comme nous l'avons fait à certains moments afin d'expliquer comment procéder.
Avec ceci en tête, on comprend toute l'efficacité de la méthode présentée, mais pas encore justifiée, car il suffit de garder une trace minimale, mais complète, des étapes tout en ayant à chaque étape des opérations assez simples à effectuer.
Il reste à démontrer que notre méthode marche à tous les coups. Ceci est le propos de la section suivante.
	