Les algorithmes \ref{algo-human-paper-bis} et \ref{algo-human-matrix} vus ci-dessus sont informatiquement très maladroits. Voici pourquoi.

\begin{enumerate}
	\item L'algorithme \ref{algo-human-paper-bis} utilise une liste de taille la somme de $1$ et du nombre d'étapes de l'algo\-rithme d'Euclide pour calculer $\pgcd(a ; b)$.
	Dans la section \ref{euclide-complexity}, nous verrons que ce nombre d'étapes est environ égal à $5$ fois le nombre de chiffres de l'écriture décimale du plus petit des deux entiers $a$ et $b$.
	Donc si l'on travaille avec des entiers de tailles assez grandes, la taille de la liste risque de devenir problématique sur du matériel où l'usage de la mémoire est critique \emph{(penser aux objets connectés)}.


	\item Le problème avec l'algorithme \ref{algo-human-matrix} est le produit cumulé des matrices qui cache beaucoup d'opérations intermédiaires. Il serait bien de pouvoir s'en passer !
\end{enumerate}

Dans la section qui suit nous allons voir que l'on peut chercher plus efficacement des coefficients de \bb{}, et ceci sans faire appel ni à des raisonnements avancés, ni à un algorithme complexe dans sa structure.
