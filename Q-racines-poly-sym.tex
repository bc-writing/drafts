\documentclass[12pt]{amsart}
\usepackage[T1]{fontenc}
\usepackage[utf8]{inputenc}

\usepackage[top=1.95cm, bottom=1.95cm, left=2.35cm, right=2.35cm]{geometry}

\usepackage{hyperref}
\usepackage[french]{babel}

\usepackage{lymath}

\DeclareMathOperator{\taille}{\tau}

\newtheorem{fact}{Fait}
\newtheorem*{proof*}{Preuve}

\setlength\parindent{0pt}


\begin{document}

\title{BROUILLON - Racines rationnelles d'un polynôme symétrique de degré 4}
\author{Christophe BAL}
\date{6 Décembre 2018}
\maketitle

$P(X) = a X^4 + b x^3 + c X^2 + b X + a$, un polynôme symétrique de degré 4, peut-il n'avoir que des racines entières ? Que des racines rationnelles ?



\section{Constatations générales}

On peut supposer que $a = 1$, i.e. $P(X) = X^4 + b X^3 + c X^2 + b X + 1$.

Dès lors si $P(r) = 0$ alors $r \neq 0$ et $P\left( \dfrac1r \right) = 0$.

Ensuite, nous avons :

\medskip

$P(X) = X^4 P\left( \dfrac1X \right)$ : caractérisation des polynômes symétriques de degré $4$

\medskip

$P\,^{\prime}(X) = 4 X^3 P\left( \dfrac1X \right) 
            - X^2 P\,^{\prime}\left( \dfrac1X \right)$

%\medskip
%
%$P\,^{\prime\prime}(X) = 12 X^2 P\left( \dfrac1X \right)  - 4 X P\,^{\prime}\left( \dfrac1X \right)
%		    - 2 X P\,^{\prime}\left( \dfrac1X \right) + P\,^{\prime\prime}\left( \dfrac1X \right)$
%
%$P\,^{\prime\prime}(X) = 12 X^2 P\left( \dfrac1X \right)  
%            - 6 X P\,^{\prime}\left( \dfrac1X \right) 
%            + P\,^{\prime\prime}\left( \dfrac1X \right)$

On en déduit que si $r$ est une racine d'ordre au moins $2$, il en est de même pour $\dfrac1r$.


\section{Uniquement des racines entières ?}

Si $P$ n'a que des racines entières, alors ces racines ne peuvent être que $\pm 1$ qui sont les seuls entiers ayant un inverse entier. Ceci donne les uniques  possibilités suivantes :

\begin{enumerate}
	\item Pour $P(X) = (X + 1)^4$
	      comme $X^4 \left( \dfrac1X + 1\right)^4 = (1 + X)^4$ 
	      on a : $P(X) = X^4 P\left( \dfrac1X \right)$ (on évite ainsi la tache ingrate de développer ceci même si les logiciels de calcul formel, comme  \url{https://www.wolframalpha.com},  le font sans sourciller).
	      Le polynôme est ok.

	\item $P(X) = (X - 1)^4$ est ok (comme ci-dessus et merci à l'exposant pair).
	
	\item Pour $P(X) = (X - 1)^3 (X + 1)$
	      comme $X^4 \left( \dfrac1X - 1\right)^3 \left( \dfrac1X + 1\right)
	          = (1 - X)^3 (1 + X)$ ,
	      on a : $P(X) = - X^4 P\left( \dfrac1X \right)$.
	      Le polynôme est rejeté.

	      \noindent En fait $P(X) = (X - 1)^3 (X + 1) = X^4 - 2 X^3 + 2 X - 1$ est anti-symétrique. 
	      Un polynôme de degré $4$ est anti-symétrique ssi  $P(X) = - X^4 P\left( \dfrac1X \right)$.
	
	\item $P(X) = (X + 1)^3 (X - 1)$ est rejeté (comme ci-dessus).

	\item $P(X) = (X + 1)^2 (X - 1)^2$ est ok car il vérifie $P(X) = X^4 P\left( \dfrac1X \right)$.
\end{enumerate}


\section{Uniquement des racines rationnelles ?}

Supposons que $r \in \QQ - \NN$ soit une racine de $P$.

\medskip

Le résultat sur la multiplicité supérieure ou égale à $2$ nous donne que si $r$ est de multiplicité au moins $2$ alors $\dfrac1r \neq r$ est aussi de multiplicité au moins $2$.
Ceci implique que $r$ est de multiplicité $1$ ou $2$.



\subsection*{$r$ est de multiplicité $1$}

Si $P$ admet une autre racine $s \in \QQ - \NN$ avec $s \neq r$ et $s \neq \dfrac1r$ alors nécessairement $P(X) = (X - r) \left( X - \dfrac1r \right) (X - s) \left( X - \dfrac1s \right)$.


D'où

$X^4 P\left( \dfrac1X \right)
= X^4 
  \left( \dfrac1X - r \right) \left( \dfrac1X - \dfrac1r \right) 
  \left( \dfrac1X - s \right) \left( \dfrac1X - \dfrac1s \right)$

$X^4 P\left( \dfrac1X \right)
= ( 1 - r X) \left( 1 - \dfrac{X}{r} \right) 
  ( 1 - s X) \left( 1 - \dfrac{X}{s} \right)$

$X^4 P\left( \dfrac1X \right)
= r \left( \dfrac1r - X \right) \times \dfrac1r ( r - X) 
  \times s \left( \dfrac1s - X \right) \times \dfrac1s ( s - X)$

$X^4 P\left( \dfrac1X \right)
= P(X)$

Ce polynôme est donc ok.


\bigskip

Il reste à étudier les cas suivants.

\begin{enumerate}
	\item $P(X) = (X - r) \left( X - \dfrac1r \right) (X + 1)^2$ et $P(X) = (X - r) \left( X - \dfrac1r \right) (X - 1)^2$ sont ok car il suffit de reprendre le calcul précédent avec $s = \pm 1$.
	

	\item $P(X) = (X - r) \left( X - \dfrac1r \right) (X + 1) (X - 1)$ vérifie $P(X) = - X^4 P\left( \dfrac1X \right)$ donc on a un polynôme anti-symétrique que l'on rejette.
\end{enumerate}


\subsection*{$r$ est de multiplicité $2$}

Dans ce cas, $P(X) = (X - r)^2 \left( X - \dfrac1r \right)^2$ nécessairement !

Il est immédiat que $P(X) = X^4 P\left( \dfrac1X \right)$ donc ce polynôme est ok.



\section{Calcul formel. Bon ou mauvais choix ?}

Par flemme, l'auteur avait d'abord raisonner avec un logiciel de calcul formel comme suit où des lettres différentes indiquent des racines différentes.

\begin{enumerate}
	\item $P(X) = (X - r)^4
	            = X^4
	            - 4 r X^3
	            + 6 r^2 X^2
	            - 4 r^3 X
	            + r^4$
	       est symétrique
	       ssi
	       $r^4 = 1$ et $r^3 = r$.
	       Si $r\in \QQ$ alors nécessairement $r = \pm 1$.
	       C'est alors ok via le développement de $(X - r)^4$.

	\item $P(X) = (X - r)^3 (X - s)
	            = X^4
	            - (3 r + s) X^3
	            + (3 r^2 + 3 r s) X^2
	            - (r^3 + 3 r^2 s) X
	            + r^3 s$
	       est symétrique
	       ssi
	       $r^3 s = 1$ et $3 r + s = r^3 + 3 r^2 s$.
	       On en déduit $3 r^4 + 1 = r^6 + 3 r^2$
	       d'où $T^3 - 3 T^2 + 3 T - 1 = 0$
	       i.e. $(T - 1)^3 = 1$
	       en posant $T = r^2$.
	       On en déduit que $r = \pm 1$ mais dans ce cas $s = r$ !
	       Donc on rejette. 

	\item $P(X) = (X - r)^2 (X - s)^2
	            = X^4
	            - (2 r  + 2 s) X^3
	            + (r^2 + 4 r s + s^2) X^2
	            - (2 r^2 s + 2 r s^2) X
	            + r^2 s^2
	            $
	       est symétrique
	       ssi
	       $r^2 s^2 = 1$ et $r + s = r^2 s + r s^2$ soit
	       $r s = \pm 1$ et $r + s = rs(r + s)$.
	       
	\begin{enumerate}
		\item $r s = 1$ donne $r  + s = r + s$ et surtout $s = \dfrac1r$.
		C'est alors ok via le développement de $(X - r)^2 \left( X - \dfrac1r \right)^2$.

		\item $r s = -1$ donne $r  + s = 0$ i.e. $s = - r$ d'où $r = \pm 1$.
		C'est alors ok via le développement de $(X - 1)^2 ( X + 1)^2$.
	\end{enumerate}
	
	\item $P(X) = (X - r)^2 (X - s) (X - t)
	            = X^4
	            - (s + t + 2 r) X^3
	            + (s t + r^2 + 2 r s + 2 r t) X^2
	            - (r^2 s + r^2 t + 2 r s t) X 
	            + r^2 s t$
	       est symétrique
	       ssi
	       $r^2 s t = 1$ et $s + t + 2 r = r^2 s + r^2 t + 2 r s t$. Que faire ?
\end{enumerate}

A partir de là, on est bloqué avec trois inconnues et seulement deux équations ! De plus, pour ce qui précède, on n'a aucun recul sur ce que l'on fait. C'est moche !



\section{Et les polynômes anti-symétriques ?}

En fait la preuve nous donne pour $P$ un polynôme anti-symétrique de coefficient dominant $1$ (toujours possible) :

\begin{enumerate}
	\item $P$ n'a que des racines entières ssi $P(X) = (X - 1)^3 (X + 1)$ ou $P(X) = (X - 1) (X + 1)^3$.

	\item $P$ n'a que des racines rationnelles dont une au moins non entière ssi $P(X) = (X - r) \left( X - \dfrac1r \right) (X - 1) (X + 1)$ où $r \in \QQ - \NN$.
\end{enumerate}




\section{À explorer...}

Comment généraliser à d'autres degrés ?

\medskip

Mais surtout, blanquette de veau ou moussaka ?

\end{document}
