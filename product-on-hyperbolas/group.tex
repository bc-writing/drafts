Le procédé de construction que nous venons de prouver se \emph{\og conserve \fg} par translations, et aussi par dilatations verticales et horizontales.
Il se trouve que ce sont ces transformations qui permettent d'obtenir une hyperbole $\geoset{H}\,^\prime : y = \frac{a x + b}{c x + d}$ , où $c \neq 0$ et $ad - bc \neq 0$ 
\footnote{
	La condition $ad - bc \neq 0$ évite d'avoir une simplification de $\frac{a x + b}{c x + d}$ en une fonction constante comme on peut le constater en considérant les vecteurs $\vect{u}(a ; b)$ et $\vect{v}(c ; d)$ , puis en notant que $ad - bc = \det \left( \vect{u} ; \vect{v} \right)$ .  
},
à partir de celle de l'hyperbole $\geoset{H} : y = \frac{1}{x}$ .
Nous pouvons donc munir toute hyperbole $\geoset{H}\,^\prime : y = \frac{a x + b}{c x + d}$ d'une structure de groupe isomorphe à celle de $(\RRs ; \times)$ , et ceci avec un procédé géométrique simple pour \emph{\og multiplier \fg} sur $\geoset{H}\,^\prime$ . Que c'est joli !
