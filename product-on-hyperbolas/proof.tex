\textbf{Cas 1.} \emph{Supposons que $x_A \neq \pm \, x_B$ de sorte que $x_S \neq 0$ .}

\medskip

La droite $(AB)$ a pour pente
$\frac{y_A - y_B}{x_A - x_B} = \frac{a^2 - b^2}{a - b} = a + b$ .
De plus, la droite $(OS)$ qui passe par l'origine $O$ du repère a pour pente
$\frac{y_S}{x_S} = \frac{x_S^2}{x_S} = x_S = a + b$ .
Les droites $(AB)$ et $(OS)$ sont bien parallèles comme nous l'avons affirmé.


\bigskip

\textbf{Cas 2.} \emph{Supposons que $x_A = - x_B$ .}

\medskip

Comme $x_S = a + b = 0$ , nous avons bien $S = O$ .


\bigskip

\textbf{Cas 3.} \emph{Supposons que $x_A = x_B \neq 0$ .}

\medskip

Dans ce cas, $x_S = 2a \neq 0$ donc la droite $(OS)$ a pour pente
$x_S = 2a$ qui est bien la pente de la tangente en $A$ à la parabole $\geoset{P}$ .