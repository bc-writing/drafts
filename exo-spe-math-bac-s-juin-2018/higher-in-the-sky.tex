Ce qui suit reprend les excellentes indications données par Jérôme Germoni dans une discussion au bas de cette page :
\url{http://images.math.cnrs.fr/+Nombres-puissants-au-bac-S+}
\textit{(chercher les messages de l'utilisateur projetmbc)}.


\bigskip

Dans la suite, nous ne considérons que l'ensemble des matrices $A$ telles que $A^T Q A = Q$ et $\det A = 1$ afin de pouvoir les multiplier entre elles.


\medskip

Nous savons que $A$ est du type 
$A
=
\begin{pmatrix} 
  a & 8c \\ 
  c & a
\end{pmatrix}$ 
et que la contrainte $a^2 - 8c^2 = 1$ est cachée dans la relation $A^T Q A = Q$ .


\medskip

Ensuite si
$\begin{pmatrix} 
  x \\ 
  y 
\end{pmatrix}$
verifie $x^2 - 8 y^2 = 1$ , il en sera de même pour
$\begin{pmatrix} 
  X \\ 
  Y 
\end{pmatrix}
=
A
\begin{pmatrix} 
  x \\ 
  y 
\end{pmatrix}$ .


\medskip

Notant $\probaset{S}$ l'ensemble des solutions entières
\footnote{
	Ce qui suit se généralise aux solutions rationnelles, à celles réelles ou encore à celles complexes mais ne sortons pas du cadre de ce modeste document.
}
de \textbf{[ED]} : $x^2 - 8 y^2 = 1$ , ce qui précède motive la définition de la loi $\star$ sur $\probaset{S}$ via
$\begin{pmatrix} 
  a \\ 
  c 
\end{pmatrix}
\star
\begin{pmatrix} 
  x \\ 
  y 
\end{pmatrix}
=
\begin{pmatrix} 
  a x + 8c y \\ 
  c x + a y
\end{pmatrix}$ .
Nous allons démontrer que cette loi $\star$ est bien définie et qu'elle fait de $\probaset{S}$ un groupe. Très joli ! Non ?


\medskip

Les raisonnements suivants se font en faisant le parallèle entre 
$\begin{pmatrix} 
  a \\ 
  c 
\end{pmatrix}
\star
\begin{pmatrix} 
  x \\ 
  y 
\end{pmatrix}$
et la matrice produit
$\begin{pmatrix} 
  a & 8c \\ 
  c & a 
\end{pmatrix}
\begin{pmatrix} 
  x & 8y \\ 
  y & x 
\end{pmatrix}$
dont on garde juste la première colonne
\footnote{
	Les résultats présentés ici se vérifient et se trouvent directement à la main sans passer par les matrices mais ceci est bien moins élégant.
} .
Rappelons que les deux matrices précédentes sont entières et du type $A$ vérifiant $A^T Q A = Q$ et $\det A = 1$ .


\begin{itemize}[label=\small\textbullet]
	\item La loi $\star$ est bien interne à $\probaset{S}$ car si $A^T Q A = Q$ avec $\det A = 1$ , et aussi $B^T Q B = Q$ avec $\det B = 1$ alors nous avons :
	
	\begin{itemize}[label=\raisebox{.3ex}{$\centerdot$}]
		\item $(AB)^T Q (AB) = B^T A^T Q A B = B^T Q B = Q$
		
		\medskip
		\item $\det(AB) = \det A \cdot \det B = 1$
	\end{itemize}


	\medskip
	\item
	$\begin{pmatrix} 
	  1 \\ 
	  0 
	\end{pmatrix}$
	est l'élément neutre de $\star$ car sa matrice associée est la matrice identité.


	\medskip
	\item
	$\begin{pmatrix} 
	  a  \\ 
	  -c 
	\end{pmatrix}$
	est l'inverse de
	$\begin{pmatrix} 
	  a \\ 
	  c 
	\end{pmatrix}$
	\textit{(revoir si besoin le cas 1 de la section précédente)}.


	\medskip
	\item Pour finir, l'associativité de $\star$ découle de celle de la multiplication des matrices.
\end{itemize}
