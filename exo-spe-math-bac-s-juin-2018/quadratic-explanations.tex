Une idée élémentaire pour découvrir la matrice \squote{magique} $A$ est de noter que nous pouvons écrire
$x^2 - 8 y^2
=
\begin{pmatrix} 
  x & y 
\end{pmatrix}
Q
\begin{pmatrix} 
  x \\ 
  y 
\end{pmatrix}$
en posant
$Q
=
\begin{pmatrix} 
  1 & 0  \\ 
  0 & -8 
\end{pmatrix}$ 
\textit{
	(le lecteur connaissant les formes quadratiques ne sera pas surpris par cette réécriture) 
}.


\medskip

En posant 
$\begin{pmatrix} 
  X \\ 
  Y 
\end{pmatrix}
=
A
\begin{pmatrix} 
  x \\ 
  y 
\end{pmatrix}$ ,
comme nous avons
$\begin{pmatrix} 
  X & Y 
\end{pmatrix}
Q
\begin{pmatrix} 
  X \\ 
  Y 
\end{pmatrix}
=
\begin{pmatrix} 
  x & y 
\end{pmatrix}
A^T Q A
\begin{pmatrix} 
  x \\ 
  y 
\end{pmatrix}$ ,
il est alors naturel de chercher $A$ vérifiant $A^T Q A = Q$ car d'une solution
$\begin{pmatrix} 
  x \\ 
  y 
\end{pmatrix}$
on pourra passer à une \squote{autre} solution
$\begin{pmatrix} 
  X \\ 
  Y 
\end{pmatrix}
=
A \begin{pmatrix} 
  x \\ 
  y 
\end{pmatrix}$
comme dans le sujet du BAC S.


\bigskip

Utilisant le déterminant, nous avons comme contrainte immédiate que $\det A = \pm 1$ \textit{($A$ doit donc être inversible)}. 


% ------------------ %


\bigskip

\textit{\textbf{Cas 1 :} supposons d'abord que $\det A = 1$ .}

\medskip

Notant
$A = \begin{pmatrix} 
  a & b \\ 
  c & d
\end{pmatrix}$ ,
nous avons alors
$A^{-1} = \begin{pmatrix} 
  d  & -b \\ 
  -c & a
\end{pmatrix}$
d'où :

\begin{flalign*}
	A^T Q A = Q & \Longleftrightarrow  A^T Q = Q A^{-1} & \\
	            & \Longleftrightarrow 
	\begin{pmatrix} 
	  a & c \\ 
	  b & d
	\end{pmatrix}
	\begin{pmatrix} 
	  1 & 0  \\ 
	  0 & -8 
	\end{pmatrix}
	=
	\begin{pmatrix} 
	  1 & 0  \\ 
	  0 & -8 
	\end{pmatrix}
	\begin{pmatrix} 
	  d  & -b \\ 
	  -c & a
	\end{pmatrix}
	                                                    & \\
	            & \Longleftrightarrow 
	\begin{pmatrix} 
	  a & -8 c \\ 
	  b & -8 d
	\end{pmatrix}
	=
	\begin{pmatrix} 
	  d  & -b \\ 
	  8c & -8a
	\end{pmatrix}
	                                                    & \\
	            & \Longleftrightarrow a = d \text{ et } b = 8c 
\end{flalign*}


\medskip

La condition $\det A = 1$ pour
$A
=
\begin{pmatrix} 
  a & b \\ 
  c & d
\end{pmatrix}
=
\begin{pmatrix} 
  a & 8c \\ 
  c & a
\end{pmatrix}$
nous donne
$a^2 - 8c^2 = 1$ . Que c'est joli !


\medskip

Notons que l'ensemble des matrices de ce type est stable par multiplication, et que la matrice du sujet de BAC n'est autre que celle correspondant à la solution élémentaire
$(a \,; c) = (3 \,; 1)$.


% ------------------ %


\bigskip

\textit{\textbf{Cas 2 :} supposons maintenant que $\det A = -1$ .}

\medskip

Notant
$A = \begin{pmatrix} 
  a & b \\ 
  c & d
\end{pmatrix}$ ,
nous avons alors
$A^{-1} = \begin{pmatrix} 
  -d & b  \\ 
   c & -a
\end{pmatrix}$
d'où comme dans les calculs précédents :
$A^T Q A = Q \Longleftrightarrow a = -d \text{ et } b = -8c$ .


\bigskip

La condition $\det A = -1$ pour
$A
=
\begin{pmatrix} 
  a & b \\ 
  c & d
\end{pmatrix}
=
\begin{pmatrix} 
  a & -8c \\ 
  c & -a
\end{pmatrix}$
nous redonne
$a^2 - 8c^2 = 1$ mais avec un autre ensemble de matrices qui n'est pas stable par multiplication.

