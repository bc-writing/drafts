A ce stade, il est temps d'expérimenter via un petit code nommé \verb+last-digits-big-tests.py+ disponible sur \url{https://github.com/bc-writing/drafts}.
Ce petit programme, absolument pas optimisé, fonctionne comme suit.

\begin{itemize}[label=\small\textbullet]
	\item \verb+NMIN+ et \verb+NMAX+ définissent l'intervalle où seront pris les nombres premiers.

	\item \verb+MAX_POWER+ donne le nombre maximal de chiffres à droite que l'on va tester.
\end{itemize}

Voici quelques résultats obtenus où pour \verb+MAX_POWER = 7+ il a fallu être très, très patient.

\begin{center}
\begin{tabular}{|c|c|c|c|}
	\hline
	\verb+NMIN+ & \verb+NMAX+ & \verb+MAX_POWER+
    & Message affiché
	\\
	\hline
    \verb+1+ & \verb+500+ & \verb+2+
    & \verb+19 tests failed : 29 , 41 , 61 , 89 , 101 , 109 , ...  +
	\\
	\hline
    \verb+1+ & \verb+500+ & \verb+3+
    & \verb+ 7 tests failed : 41 , 89 , 241 , 281 , 401 , 409 , 449+
	\\
	\hline
    \verb+1+ & \verb+500+ & \verb+4+
    & \verb+ 7 tests failed : 41 , 89 , 241 , 281 , 401 , 409 , 449+
	\\
	\hline
    \verb+1+ & \verb+500+ & \verb+7+
    & \verb+ 7 tests failed : 41 , 89 , 241 , 281 , 401 , 409 , 449+
	\\
	\hline
\end{tabular}
\end{center}


\medskip

Nous perdons là tout espoir de généralisation mais du coup un autre problème s'offre à nous à savoir : \emph{\og Peut-on caractériser les mauvais candidats à notre analayse via les derniers chiffres décimaux ? \fg}.
