\begin{fact}
	$\sqrt{11} \not\in \QQ$
\end{fact}

\begin{proof}
	Ici la 1\iere{} technique ne donnera rien, c'est immédiat.
	Utilisons donc la technique vue pour le fait \ref{sqrt-5-not-in-Q}. La 1\iere{} version du programme, avec \verb+N = 11+ mais sans personnalisation de la fonction \verb+keep+, nous donne :

	\begin{rawcode}
N = 11
r possibles :
0, 2, 4, 6, 8, 10, 12, 14, 16, 18, 20, 22, 24, 26, 28, 30, 32, 34, 36, 38, 40, 
42, 44, 46, 48, 50, 52, 54, 56, 58, 60, 62, 64, 66, 68, 70, 72, 74, 76, 78, 80, 
82, 84, 86, 88, 90, 92, 94, 96, 98,
s possibles :
0, 2, 4, 6, 8, 10, 12, 14, 16, 18, 20, 22, 24, 26, 28, 30, 32, 34, 36, 38, 40, 
42, 44, 46, 48, 50, 52, 54, 56, 58, 60, 62, 64, 66, 68, 70, 72, 74, 76, 78, 80, 
82, 84, 86, 88, 90, 92, 94, 96, 98,
	\end{rawcode}
	
	En utilisant \verb+return i % 2 != 0+
	dans la fonction \verb+keep+ pour n'afficher que les non multiples de $2$, nous obtenons :

	\begin{rawcode}
N = 11
r possibles :

s possibles :

	\end{rawcode}
	
	Ceci nous démontre que $r$ et $s$ sont tous les deux multiples de $2$ ce qui donne la contradiction pour conclure. 
\end{proof}