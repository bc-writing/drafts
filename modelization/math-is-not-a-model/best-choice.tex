La modélisation précédente est en fait très critiquable et dangereuse pour les raisons suivantes.

\begin{enumerate}
	\item Comme nous l'avons déjà souligné, la solution géométrique n'est pas la plus pratique. Il vaut mieux marcher un peu plus avec un seau vide qu'avec un seau plein. Or la solution géométrique est d'une grande élégance et c'est là tout le danger que de se laisser aveugler par de beaux raisonements. Ne pas confondre beauté mathématique et solution concrètement utile.


	\item Autre chose. Qui nous dit que l'on peut accéder à la rivière par l'un des chemins \emph{\og mathématiques \fg} trouvés ? Nous avons concrètement des contraintes géométriques fortes dues à la topologie du terrain que nous n'avons pas prises en compte.
\end{enumerate}


En résumé, une belle solution mathématique ne fait pas forcément un beau modèle. Nous avons constaté ici certains soucis très facilement. Mais qu'en sera-t-il pour des modèles plus complexes d'un point de vue mathématique ? Pour ne pas se laisser aveugler, il faudra toujours confronter son modèle à la réalité et accepter de devoir garder un modèle \emph{\og moche \fg} mathématiquement mais efficace concrètement.


\bigskip


\begin{center}
	\scalebox{2.5}{$\therefore$}

	\vspace{1em}
	
	\textbf{Quand on modélise, le contexte est essentiel !}
\end{center}
