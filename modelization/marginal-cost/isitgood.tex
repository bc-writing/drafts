Il se trouve que la modélisation du coût de production par un polynôme de degré 2 ou 3 apparait souvent dans des livres présentant des mathématiques élémentaires pour l'Économie. Cela montre que ces modèles ont leur utilité mais si l'on est honnête, les raisonnements précédents sont fragiles car il repose sur des \emph{\og approximations de courbes \fg} , une technique qui reste tout de même intéressante.