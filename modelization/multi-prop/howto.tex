Les sciences du réel sont avant tout des sciences de la mesure. Il peut arriver que des phénomènes fassent apparaître, via de simples mesures, de la proportionnalité sur plusieurs variables étudiées indépendamment les unes des autres : l'auteur de ces lignes parlent de \emph{\og multi-proportionnalité \fg}. Voici des exemples classiques de ce type fournis par les Sciences Physiques
\footnote{
	L'auteur n'affirme pas que ce qui suit est effectivement ce qui a guidé l'établissement des lois présentées mais cette approche de modélisation reste concrètement très acceptable.
	Afin de préciser le propos, une étude historique serait nécessaire mais le temps manque au papa qui écrit ces lignes...
	Toute contribution est bienvenue !
}.


% ---------------- %


\begin{itemize}[label=\small\textbullet]
	\item En Électricité, en régime continu, la tension et l'intensité du courant aux bornes d'une resistance vérifient $U = RI$ . Cette loi est due à Georg
	\footnote{
		L'absence d'un \emph{\og e \fg} final n'est pas une faute de frappe.
	}
	Simon Ohm (1789 -- 1854). On imagine bien un ensemble de mesures de $U$ en fonction de $I$ faisant apparaître une relation de quasi proportionnalité du fait de certaines imprécisions de mesure. Une fois cette observation faite, il devient \emph{\og naturel \fg} de poser la loi $U = RI$ . 


% ---------------- %


	\medskip
	\item Amedeo Avogadro (1776 -- 1846) a énoncé la loi des gaz parfaits : $PV = nRT$ .
	Dans cette relation, $V$ est un volume de gaz en $m^3$ , $P$ une pression en pascal, $T$ une température en $K$ , $n$ la quantité de matière en $mol$ , et $R \approx 8,314$ la constante des gaz parfait en $J \cdot mol^{-1} \cdot K^{-1}$ .
	On imagine bien ici aussi une étude, à quantité de matière fixée, d'une relation entre $P$ , $V$ et $T$ en s'intéressant à deux variables, la troisième restant constante. On observerait alors les phénomènes suivants, sous des conditions physiques acceptables correspondant en fait au domaine de validité de la loi.
	
	\begin{enumerate}[label=(\alph*)]
		\item $P$ étant fixé, on observe que la température est proportionnelle à $V$ d'où l'on pose $T = a(P) V$ avec $a(P)$ une constante dépendant a priori de $P$ .

		\item $V$ étant fixé, on observe que la température est proportionnelle à $P$ d'où $T = b(V) P$ avec $b(V)$ une constante dépendant a priori de $V$ .
	\end{enumerate}

	\noindent
	À ce stade, le physicien aguerri propose une formule du type $T = k P V$ avec $k$ une constante ne dépendant pas de $T$ , $P$ et $V$ .
	Intuitivement c'est facile à comprendre mais est-ce mathématiquement correct ? Nous verrons dans la section suivante que oui !
	
	\smallskip
	\noindent
	Le passage de $T = k P V$ à la loi des gaz parfaits s'obtiendrait de façon analogue via la prise en compte en plus du paramètre $n$ .
	Le choix et la signification de la constante $R$ sont motivés par le physicien qui veut rendre sa formule la plus expressive possible.


% ---------------- %


	\medskip
	\item Nous devons à Isaac Newton (1642 -- 1727) la loi de l'attraction universelle qui dit que l'intensité en $N$ de la force exercée entre deux corps $A$ et $B$ est $F_{A \leftrightarrow B} = G \frac{m_A m_B}{d^2}$ avec $m_A$ et $m_B$ les masses en $kg$ de chacun des deux corps, $d$ la distance en $m$ les séparant et $G \approx 6,674$ la constante gravitationnelle en $N \cdot m^2 \cdot kg^{-2}$ .
	Tout comme pour la loi des gaz parfaits, on pourrait imaginer étudier l'intensité relativement à $m_A$ , $m_B$ et l'inverse de $d^2$ respectivement, les deux autres paramètres étant fixés à chaque fois.
	
	\smallskip
	\noindent
	Pourquoi l'inverse de $d^2$ ? On peut imaginer une intuition relativement à l'inverse de $d$ , puis lors de mesures on s'aperçoit qu'il y a en fait une relation qui semble quadratique relativement à $\frac{1}{d}$ . Dès lors, il suffit de tenter sa chance avec l'inverse de $d^2$ .
\end{itemize}


% ---------------- %


\medskip

Dans chacun des cas proposés, nous avons rencontré une fonction $f$ à $n$ variables $x_1$ , ... , $x_n$ qui est proportionnelle relativement à chaque variable $x_i$ dès que l'on fixe les autres,
puis nous sommes arrivés à une relation du type $f(x_1 ; ... ; x_n) = k x_1 \cdots x_n$ avec $k$ une constante. Bien que cela semble intuitivement clair, il serait bien de prouver la validité de cette affirmation. Ceci va être fait dans la section suivante.


% ---------------- %


\begin{remark}
	Bien entendu une fois ces lois établies, il va falloir les valider par des mesures et étudier leur domaine d'applicabilité. Ceci se fait de façon directe et indirecte : par exemple, la loi d'attraction de Newton permet de valider les observations empiriques de Johannes Kepler (1571 -- 1630) sur les orbites coniques des astres
	\footnote{
		L'auteur conseille vivement au lecteur que les questions de géométrie passionnent la lecture de l'ouvrage \emph{\og Newton implique Kepler : méthodes géométriques élémentaires pour l'enseignement supérieur en mathématiques \fg} de Benoît Rittaud aux éditions ellipses.
	}. 
	N'est-ce pas joli ? C'est là tout le beau travail du scientifique du réel.
\end{remark}