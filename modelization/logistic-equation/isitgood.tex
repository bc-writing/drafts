Revenons sur la toute \emph{\og petite \fg} liste d'hypothèses de modélisation que nous avons faites.

\begin{enumerate}
	\item La probabilité $q_I$ d'être infecté dans l'année, et celle $q_V$ de mourrir dans l'année une fois infecté sont supposées indépendantes de l'année $t$ considérée.

	\item On a supposé $S$ , $P$ et $N$ définies et dérivables sur $\RRp$ tout entier.

	\item Enfin nous avions besoin que $S$ , $P$ et $N$ ne s'annulent jamais.
\end{enumerate}


% ----------- %


\paragraph{Hypothèse 1.} Ceci sous-entend un type particulier de propagation.
Par exemple, cette première hypothèse reste-t-elle valable si les personnes malades sont mises en quarantaine ? Non. C'est ce qui permet d'éradiquer des maladies très virulentes.
Une autre critique : plus il y a de malades contagieux, plus grande devient la probabilité de tomber malade. 
Du point de vue de la modélisation, pour une maladie pas \emph{\og trop violente \fg} et en début d'épidémie, on peut tout de même accepter la première hypothèse
\footnote{
	Des règles de quarantaines ne sont pas justifiables politiquement.
}.


% ----------- %


\paragraph{Hypothèse 2.} Le passage de fonctions définies sur $\NN$ à des fonctions définies sur $\RRp$ ne pose pas de difficultés conceptuelle et concrète. 

\smallskip

Ensuite arrive un grand classique des sciences du réel : les fonctions concrètes étudiées sont supposées suffisamment régulières. Pourquoi fait-on ceci ? Comme nous l'avons déjà indiqué, ceci permet de faire appel à des outils mathématiques puissants du calcul différentiel car l'on dispose de moins d'outils efficaces pour étudier les suites. 

\smallskip

Ceci étant dit, supposer la dérivabilité sur $\RRp$ tout entier nous amène à étudier des fonctions au comportement très lisse.
Ainsi les solutions des équations différentielles, dites logistiques, $y\,' = a y (1 - b y)$ sont très régulières, tandis que l'équation logistique discrète $u_{n+1} = a u_n (1 - b u_n)$ peut produire des suites au comportement chaotique. 

\smallskip

De plus, comment passer d'une étude entre $t$ et $t + \delta t$ avec $\delta t$ aussi petit que nécessaire, au cas où $\delta t$ tend vers $0$ ? Rien ne justifie ceci sérieusement d'un point de vue concret. Nous sommes là face à un choix très fort de modélisation.

\smallskip

L'hypothèse 2 est donc lourde de conséquences... Ceci étant dit, cela reste un classique de la modélisation et l'histoire des sciences du réel prouve que ce type d'hypothèse est féconde à produire des modèles utiles.


% ----------- %


\paragraph{Hypothèse 3.} Ceci ne pose aucun souci sauf à vouloir étudier une population sans aucune personne saine, ou bien sans personne vivante.


% ----------- %


\paragraph{Conclusion.} Bien que nous ayons fait un très joli raisonnement mathématique, nous venons de voir que concrètement il y a de réelles failles de modélisation mais comme toujours en modélisation, c'est l'affrontement du modèle et des vraies données qui servira de juge d'utilité.