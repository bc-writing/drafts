\documentclass[12pt]{amsart}
\usepackage[T1]{fontenc}
\usepackage[utf8]{inputenc}

\usepackage[top=1.95cm, bottom=1.95cm, left=2.35cm, right=2.35cm]{geometry}

\usepackage{hyperref}
\usepackage{import}

\usepackage{enumitem}
\usepackage{tcolorbox}
\usepackage{float}
\usepackage{cleveref}
\usepackage{multicol}
\usepackage{fancyvrb}
\usepackage{enumitem}
\usepackage{amsmath}
\usepackage{textcomp}
\usepackage{numprint}
\usepackage[french]{babel}
\usepackage[
    type={CC},
    modifier={by-nc-sa},
	version={4.0},
]{doclicense}

\newcommand\floor[1]{\left\lfloor #1 \right\rfloor}

\usepackage{lymath}


\newtheorem{fact}{Fait}[section]
\newtheorem{example}{Exemple}[section]
\newtheorem{definition}{Définition}[section]
\newtheorem{remark}{Remarque}[section]
\newtheorem*{proof*}{Preuve}

\setlength\parindent{0pt}

\floatstyle{boxed} 
\restylefloat{figure}


\DeclareMathOperator{\taille}{\text{\normalfont\texttt{taille}}}

\newcommand\sqseq[2]{\fbox{$#1$}_{\,\,#2}}


\DefineVerbatimEnvironment{rawcode}%
	{Verbatim}%
	{tabsize=4,%
	 frame=lines, framerule=0.3mm, framesep=2.5mm}
	 

%\let\oldsection\section
%\renewcommand\section[1]{\vfill\pagebreak\oldsection{#1}}

\let\oldsubsection\subsection
\renewcommand\subsection[1]{\bigskip\oldsubsection{#1}}

\newcommand\squote[1]{\og #1 \fg{}}


	 
\begin{document}

\title{BROUILLON - Ellipses, paraboles, hyperboles et constructions cycliques}
\author{Christophe BAL}
\date{27 Juillet 2019 - 1\ier Août 2019}

\maketitle

\begin{center}
	\itshape
	Document, avec son source \LaTeX, disponible sur la page
	
	\url{https://github.com/bc-writing/drafts}.
\end{center}


\bigskip


\begin{center}
	\hrule\vspace{.3em}
	{
		\fontsize{1.35em}{1em}\selectfont
		\textbf{Mentions \og légales \fg}
	}
			
	\vspace{0.45em}
	\doclicenseThis
	\hrule
\end{center}


\setcounter{tocdepth}{2}
\tableofcontents


Nous allons présenter une construction récursive très simple de points qui produit une suite périodique lorsque l'on est sur le cercle trigonométrique, la parabole $\geoset{P} : y = x^2$ ou l'hyperbole $\geoset{H} : y = \frac{1}{x}$ .
Même si ces courbes font partie de la belle famille des coniques du plan réel, lesquelles partagent des propriétés géométriques communes, la construction reste magique quand on la découvre.
Nous finirons ce document en utilisant un point de vue plus technique qui expliquera pourquoi la construction fonctionne plus généralement avec n'importe quelle ellipse, parabole ou hyperbole. 


% ------------------ %


\section{Une construction naturelle sur un cercle}
\label{circle}

\subimport{beautiful-cylce-on-conics/}{circle}


% ------------------ %


\section{\texorpdfstring{La construction reste valable sur la parabole d'équation $y = x^2$}%
                        {La construction reste valable sur la parabole d'équation y = x**2}}
\label{parabola}

\subimport{beautiful-cylce-on-conics/}{parabola}


% ------------------ %


\section{\texorpdfstring{Encore mieux, cela marche aussi avec l'hyperbole d'équation $y = \frac{1}{x}$}%
                        {Encore mieux, cela marche aussi avec l'hyperbole d'équation y = 1 / x}}
\label{hyperbola}
                        
\subimport{beautiful-cylce-on-conics/}{hyperbola}


% ------------------ %


\section{Tout ceci est facile si l'on prend un peu de recul}
\label{group}

\subimport{beautiful-cylce-on-conics/}{group}


\end{document}
