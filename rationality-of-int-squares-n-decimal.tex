\documentclass[12pt]{amsart}
\usepackage[T1]{fontenc}
\usepackage[utf8]{inputenc}

\usepackage[top=1.95cm, bottom=1.95cm, left=2.35cm, right=2.35cm]{geometry}

\usepackage{hyperref}
\usepackage{enumitem}
\usepackage{tcolorbox}
\usepackage{float}
\usepackage{cleveref}
\usepackage{multicol}
\usepackage{fancyvrb}
\usepackage{enumitem}
\usepackage{amsmath}
\usepackage{textcomp}
\usepackage{numprint}
\usepackage[french]{babel}
\usepackage[
    type={CC},
    modifier={by-nc-sa},
	version={4.0},
]{doclicense}

\newcommand\floor[1]{\left\lfloor #1 \right\rfloor}

\usepackage{lymath}


\newtheorem{fact}{Fait}[section]
\newtheorem{example}{Exemple}[section]
\newtheorem{remark}{Remarque}[section]
\newtheorem*{proof*}{Preuve}

\npthousandsep{.}
\setlength\parindent{0pt}

\floatstyle{boxed} 
\restylefloat{figure}


\DeclareMathOperator{\taille}{\text{\normalfont\texttt{taille}}}

\newcommand{\logicneg}{\text{\normalfont non \!}}

\newcommand\sqseq[2]{\fbox{$#1$}_{\,\,#2}}


\DefineVerbatimEnvironment{rawcode}%
	{Verbatim}%
	{tabsize=4,%
	 frame=lines, framerule=0.3mm, framesep=2.5mm}
	 
	 
	 
\begin{document}

\title{BROUILLON - NON ACHEVÉ ! - Écriture décimale, racine carrée de naturels et irrationalité}
\author{Christophe BAL}
\date{27 Mars 2019 - 30 Mars 2019}

\maketitle

\begin{center}
	\itshape
	Document, avec son source \LaTeX, disponible sur la page
	
	\url{https://github.com/bc-writing/drafts}.
\end{center}


\bigskip


\begin{center}
	\hrule\vspace{.3em}
	{
		\fontsize{1.35em}{1em}\selectfont
		\textbf{Mentions \og légales \fg}
	}
			
	\vspace{0.45em}
	\doclicenseThis
	\hrule
\end{center}


\setcounter{tocdepth}{2}
\tableofcontents



\section{$\sqrt2$ n'est pas rationnel, les décimaux nous le disent}

\begin{fact} \label{sqrt-2-not-in-Q}
	$\sqrt2 \not\in \QQ$
\end{fact}

\begin{proof}
	Nous allons donner une preuve élémentaire de ce fait inspiré de la preuve n\textdegree{}12 proposée dans cette page : \url{https://www.cut-the-knot.org/proofs/sq_root.shtml} 
	\footnote{
		Page disponible le 30 mars 2019.
	}.
	
	\medskip
	
	Regardons ce qu'il se passe si nous supposons l'existence de $(r ; s) \in \QQ \times \QQs$ tel que $\sqrt2 = \dfrac{r}{s}$. Nous pouvons supposer que $(r ; s) \in \QQp \times \QQsp$ et $\pgcd(r ; s) = 1$ \emph{(nous prenons la fraction la plus simple possible)}.
	Dès lors nous raisonnons comme suit.
	
	\begin{itemize}[label=\small\textbullet]
		\item $\sqrt2 = \dfrac{r}{s} \, \Leftrightarrow \, 2 s^2 = r^2$ car $r \geq 0$ et $s > 0$.


		\item Intéressons nous juste aux chiffres des unités de $r^2$ et $2s^2$ en fonction de ceux de $r$ et $s$ respectivement. Les résultats ci-dessous viennent de $(10d + u)^2 = 100d^2 + 20du + u^2$ où $(d ; u) \in \NN^2$.
		En effet, ceci démontre que le chiffre des unités de $(10d + u)^2$ est identique à celui de $u^2$ ce qui facilite notre travail
		\footnote{
			Pour ceux qui connaissent, on reconnait là les techniques de calculs modulo $10$.
		}.
		\begin{center}
			\begin{tabular}{|r|c|c|c|c|c|c|c|c|c|c|}
				\hline
				Chiffre des unités de $r$
				  & $0$  &  $1$  &  $2$  &  $3$  &  $4$  &  $5$  &  $6$  &  $7$  &  $8$  &  $9$
				\\ \hline
				\phantom{$2$}Chiffre des unités de $r^2$
				  & $0$  &  $1$  &  $4$  &  $9$  &  $6$  &  $5$  &  $6$  &  $9$  &  $4$  &  $1$
				\\ \hline
			\end{tabular}

			\medskip
			
			\begin{tabular}{|r|c|c|c|c|c|c|c|c|c|c|}
				\hline
				Chiffre des unités de $s$
				  & $0$  &  $1$  &  $2$  &  $3$  &  $4$  &  $5$  &  $6$  &  $7$  &  $8$  &  $9$
				\\ \hline
				Chiffre des unités de $2s^2$
				  & $0$  &  $2$  &  $8$  &  $8$  &  $2$  &  $0$  &  $2$  &  $8$  &  $8$  &  $2$
				\\ \hline
			\end{tabular}
		\end{center}


		\item Les résultas précédents montrent que $r^2$ et $2 s^2$ ne peuvent avoir que $0$ comme chiffre commun des unités.
		Ceci n'est possible que si $r$ est un multiple de $10$ et $s$ un multiple de $5$ d'où $r$ et $s$ sont des multiples de $5$.
		Or ceci contredit $\pgcd(r ; s) = 1$ 
		et donc il n'est pas possible d'avoir $(r ; s) \in \QQp \times \QQsp$ tel que $\sqrt2 = \dfrac{r}{s}$.
	\end{itemize}
\end{proof}


Au passage, nous avons une condition nécessaire pour qu'un naturel soit un carré parfait. Par exemple, $\numprint{1324535464654653524327}$ n'est pas un carré parfait. En effet, ce naturel se finit par $7$. Nous avons plus précisément le fait suivant.


\begin{fact} \label{perfect-square-last-digit}
	Un naturel $n \in \NN$ est un carré parfait s'il peut s'écrire $n = k^2$ avec $k \in \NN$.
	
	\medskip
	
	Si $n \in \NN$ est un carré parfait alors son chiffre des unités est forcément $0$ , $1$ , $4$ , $5$  , $6$ ou $9$ .
\end{fact}


\begin{remark}
	Comment peut-on avoir l'idée de la preuve du fait \ref{sqrt-2-not-in-Q} ? 
	On sait facilement caractériser les nombres pairs par leur dernier chiffre. Dès lors il semble naturel de tenter sa chance en analysant les chiffres des unités comme dans la preuve ci-dessus.
	Ceci étant dit, nous allons voir dans la section suivante que la démonstration s'adapte sans souci au cas de l'irrationalité de $\sqrt3$ or le critère de divisibilité par $3$ ne fait pas du tout référence au chiffre des unités !
\end{remark}


\begin{remark}
	Une petite question de logique pour conclure cette section.
	A-t-on fait une démonstration par l'absurde ? 
	De \textbf{façon informelle}, lorsque l'on fait une démonstration, on part d'un ensemble $\probaset{H}$ de faits vrais
	\footnote{
		On parle d'hypothèses en logique.
	}.
	On déduit de $\probaset{H}$ d'autres faits vrais pour arriver à un fait que l'on veut démontrer.
	
	\smallskip
	
	Démontrer une proposition $A$ par l'absurde c'est supposer que $\logicneg A$ est un fait vrai pour en déduire quelque chose de faux
	\footnote{
		Que les logiciens m'excusent mais ce n'est pas le propos ici de parler du calcul des prédicats de la logique du 1\ier{} ordre, un sujet très passionnant !
	}.
	Notre proposition ici est
	$\left[ \forall (r ; s) \in \QQ \times \QQs , \sqrt2 \neq \dfrac{r}{s} \right]$
	dont la négation est 
	$\left[ \exists (r ; s) \in \QQ \times \QQs , \sqrt2 = \dfrac{r}{s} \right]$.
	Comme de cette hypothèse, nous sommes arrivé à quelque chose de faux, nous avons fait une démonstration par l'absurde.	
\end{remark}





\section{$\sqrt3$ n'est pas rationnel}

\begin{fact}
	$\sqrt3 \not\in \QQ$
\end{fact}

\begin{proof}
	La démarche est la même que pour le fait \ref{sqrt-2-not-in-Q} via $r^2 = 3 s^2$ et les tableaux suivants.
	
	\begin{center}
		\begin{tabular}{|r|c|c|c|c|c|c|c|c|c|c|}
			\hline
			Chiffre des unités de $r$
			  & $0$  &  $1$  &  $2$  &  $3$  &  $4$  &  $5$  &  $6$  &  $7$  &  $8$  &  $9$
			\\ \hline
			\phantom{$3$}Chiffre des unités de $r^2$
			  & $0$  &  $1$  &  $4$  &  $9$  &  $6$  &  $5$  &  $6$  &  $9$  &  $4$  &  $1$
			\\ \hline
		\end{tabular}

		\medskip
		
		\begin{tabular}{|r|c|c|c|c|c|c|c|c|c|c|}
			\hline
			Chiffre des unités de $s$
			  & $0$  &  $1$  &  $2$  &  $3$  &  $4$  &  $5$  &  $6$  &  $7$  &  $8$  &  $9$
			\\ \hline
			Chiffre des unités de $3 s^2$
			  & $0$  &  $3$  &  $2$  &  $7$  &  $8$  &  $5$  &  $8$  &  $7$  &  $2$  &  $3$
			\\ \hline
		\end{tabular}
	\end{center}
\end{proof}



\section{$\sqrt5$ n'est pas rationnel -- Besoin d'une nouvelle preuve}

\begin{fact} \label{sqrt-5-not-in-Q}
	$\sqrt5 \not\in \QQ$
\end{fact}

\begin{proof}
	De nouveau, nous partons de $r^2 = 5s^2$ mais ici nous ne pouvons plus faire comme avant car nous avons les tableaux suivants qui nous disent que $r$ peut avoir n'importe quel chiffre des unités, ceci ne nous permettant pas d'avoir une contradiction.

	\begin{center}
		\begin{tabular}{|r|c|c|c|c|c|c|c|c|c|c|}
			\hline
			Chiffre des unités de $r$
			  & $0$  &  $1$  &  $2$  &  $3$  &  $4$  &  $5$  &  $6$  &  $7$  &  $8$  &  $9$
			\\ \hline
			\phantom{$5$}Chiffre des unités de $r^2$
			  & $0$  &  $1$  &  $4$  &  $9$  &  $6$  &  $5$  &  $6$  &  $9$  &  $4$  &  $1$
			\\ \hline
		\end{tabular}

		\medskip

		\begin{tabular}{|r|c|c|c|c|c|c|c|c|c|c|}
			\hline
			Chiffre des unités de $s$
			  & $0$  &  $1$  &  $2$  &  $3$  &  $4$  &  $5$  &  $6$  &  $7$  &  $8$  &  $9$
			\\ \hline
			Chiffre des unités de $5s^2$
			  & $0$  &  $5$  &  $0$  &  $5$  &  $0$  &  $5$  &  $0$  &  $5$  &  $0$  &  $5$
			\\ \hline
		\end{tabular}
	\end{center}

    \medskip

    Tentons tout simplement notre chance avec les deux derniers chiffres décimaux et non juste le dernier.
    À ce stade, nous allons utiliser un programme écrit en Python dont le code est le suivant
    \footnote{
    	Le fichier Python est disponible sur \url{https://github.com/bc-writing/drafts}.
		Chercher le fichier \texttt{last-two-digits.py} .
	}
	où la fonction \verb+keep+ nous sera utile après pour analyser rapidement les résultats.

	\begin{rawcode}
N = 5

def keep(i):
    return True

twolastdigits       = []
twolastdigits_for_N = []

for i in range(100):
    square   = (i**2)%100
    N_square = (N*square)%100

    twolastdigits.append(square)
    twolastdigits_for_N.append(N_square)

common = set(twolastdigits).intersection(twolastdigits_for_N)


print(f"N = {N}")
print("r possibles :")

for i in range(100):
    if twolastdigits[i] in common and keep(i):
        print(i, end = ", ")

print()
print("s possibles :")

for i in range(100):
    if twolastdigits_for_N[i] in common and keep(i):
        print(i, end = ", ")

print()
	\end{rawcode}


	\medskip

	Une fois lancé dans un terminal, ce programme nous affiche :

	\begin{rawcode}
N = 5
r possibles :
0, 5, 10, 15, 20, 25, 30, 35, 40, 45, 50, 55, 60, 65, 70, 75, 80, 85, 90, 95,
s possibles :
0, 5, 10, 15, 20, 25, 30, 35, 40, 45, 50, 55, 60, 65, 70, 75, 80, 85, 90, 95,
	\end{rawcode}

	Nous constatons que $r$ et $s$ sont tous les deux multiples de $5$ ce qui établit une contradiction permettant de conclure.
	Ceci est confirmé par le programme.
	En effet, dans la fonction \verb+keep+, utilisons \verb+return i % 5 != 0+
	à la place de \verb+return True+ afin de n'afficher que les non multiples de $5$.
	On obtient alors la sortie suivante.

	\begin{rawcode}
N = 5
r possibles :

s possibles :

	\end{rawcode}
\end{proof}


\begin{remark}
	Notons que notre programme fournit bien une argumentation totalement rigoureuse car il travaille uniquement sur des valeurs naturelles exactes.
\end{remark}




\section{$\sqrt7$ et $\sqrt{11}$ ne sont pas rationnels}

\begin{fact}
	$\sqrt7 \not\in \QQ$
\end{fact}

\begin{proof}
	Rien de nouveau. 
	On raisonne comme pour le fait \ref{sqrt-2-not-in-Q} via $r^2 = 7s^2$ et les tableaux suivants.
	
	\begin{center}
		\begin{tabular}{|r|c|c|c|c|c|c|c|c|c|c|}
			\hline
			Chiffre des unités de $r$
			  & $0$  &  $1$  &  $2$  &  $3$  &  $4$  &  $5$  &  $6$  &  $7$  &  $8$  &  $9$
			\\ \hline
			\phantom{$2$}Chiffre des unités de $r^2$
			  & $0$  &  $1$  &  $4$  &  $9$  &  $6$  &  $5$  &  $6$  &  $9$  &  $4$  &  $1$
			\\ \hline
		\end{tabular}

		\medskip
		
		\begin{tabular}{|r|c|c|c|c|c|c|c|c|c|c|}
			\hline
			Chiffre des unités de $s$
			  & $0$  &  $1$  &  $2$  &  $3$  &  $4$  &  $5$  &  $6$  &  $7$  &  $8$  &  $9$
			\\ \hline
			Chiffre des unités de $7s^2$
			  & $0$  &  $7$  &  $8$  &  $3$  &  $2$  &  $5$  &  $2$  &  $3$  &  $8$  &  $7$
			\\ \hline
		\end{tabular}
	\end{center}
\end{proof}

\begin{fact}
	$\sqrt{11} \not\in \QQ$
\end{fact}

\begin{proof}
	Ici la 1\iere{} technique ne donnera rien, c'est immédiat.
	Utilisons donc la technique vue pour le fait \ref{sqrt-5-not-in-Q}. La 1\iere{} version du programme, avec \verb+N = 11+ mais sans personnalisation de la fonction \verb+keep+, nous donne :

	\begin{rawcode}
N = 11
r possibles :
0, 2, 4, 6, 8, 10, 12, 14, 16, 18, 20, 22, 24, 26, 28, 30, 32, 34, 36, 38, 40, 
42, 44, 46, 48, 50, 52, 54, 56, 58, 60, 62, 64, 66, 68, 70, 72, 74, 76, 78, 80, 
82, 84, 86, 88, 90, 92, 94, 96, 98,
s possibles :
0, 2, 4, 6, 8, 10, 12, 14, 16, 18, 20, 22, 24, 26, 28, 30, 32, 34, 36, 38, 40, 
42, 44, 46, 48, 50, 52, 54, 56, 58, 60, 62, 64, 66, 68, 70, 72, 74, 76, 78, 80, 
82, 84, 86, 88, 90, 92, 94, 96, 98,
	\end{rawcode}
	
	En utilisant \verb+return i % 2 != 0+
	dans la fonction \verb+keep+ pour n'afficher que les non multiples de $2$, nous obtenons :

	\begin{rawcode}
N = 11
r possibles :

s possibles :

	\end{rawcode}
	
	Ceci nous démontre que $r$ et $s$ sont tous les deux multiples de $2$ ce qui donne la contradiction pour conclure. 
\end{proof}



\section{Prenons un peu de recul} \label{general-case}

\subsection{Un grand classique}

Le lecteur attentif aura noté que tous nos exemples ont porté sur des nombres premiers.
En fait, nous avons le fait suivant qui se démontre aisément avec la décomposition d'un entier en facteurs premiers, une preuve facile à trouver sur le web
\footnote{
	Cette preuve ne nécessite absolument pas l'unicité d'une telle décomposition, une unicité qui n'est pas immédiate à prouver proprement.
}.


\begin{fact}  \label{sqrt-p-not-in-Q}
	Pour tout nombre premier $p$, $\sqrt{p} \not\in \QQ$ .
\end{fact}


Nous allons essayer de voir si cette information n'est pas \emph{\og cachée \fg} juste dans les derniers chiffres de certaines écritures décimales.
Par exemple, pour $2$ , $3$ et $7$ , nous avons juste raisonné sur des chiffres des unités, tandis que pour $5$ et $11$, nous avons dû considérer les deux chiffres les moins significatifs, ceux à droite.
Une question vient naturellement : \emph{\og Peut-on généraliser les méthodes vues précédemment ? \fg}.


\subsection{Expérimentations}

A ce stade, il est temps d'expérimenter via un petit code nommé \verb+last-digits-big-tests.py+ disponible sur \url{https://github.com/bc-writing/drafts}.
Ce petit programme, absolument pas optimisé, fonctionne comme suit.

\begin{itemize}[label=\small\textbullet]
	\item \verb+NMIN+ et \verb+NMAX+ définissent l'intervalle où seront pris les nombres premiers.

	\item \verb+MAX_POWER+ donne le nombre maximal de chiffres à droite que l'on va tester.
\end{itemize}

Voici quelques résultats obtenus où pour \verb+MAX_POWER = 7+ il a fallu être très, très patient.

\begin{center}
\begin{tabular}{|c|c|c|c|}
	\hline
	\verb+NMIN+ & \verb+NMAX+ & \verb+MAX_POWER+
    & Message affiché
	\\
	\hline
    \verb+1+ & \verb+500+ & \verb+2+
    & \verb+19 tests failed : 29 , 41 , 61 , 89 , 101 , 109 , ...  +
	\\
	\hline
    \verb+1+ & \verb+500+ & \verb+3+
    & \verb+ 7 tests failed : 41 , 89 , 241 , 281 , 401 , 409 , 449+
	\\
	\hline
    \verb+1+ & \verb+500+ & \verb+4+
    & \verb+ 7 tests failed : 41 , 89 , 241 , 281 , 401 , 409 , 449+
	\\
	\hline
    \verb+1+ & \verb+500+ & \verb+7+
    & \verb+ 7 tests failed : 41 , 89 , 241 , 281 , 401 , 409 , 449+
	\\
	\hline
\end{tabular}
\end{center}


\medskip

Nous perdons là tout espoir de généralisation mais du coup un autre problème s'offre à nous à savoir : \emph{\og Peut-on caractériser les mauvais candidats à notre analayse via les derniers chiffres décimaux ? \fg}.



%\subsection{L'arithmétique modulaire}
%
%????


\end{document}
