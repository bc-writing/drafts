Il est facile de passer des arbres de l'exemple \ref{exa-tree} à ceux de l'exemple \ref{exa-tree-tree} : chaque \verb+Leaf(k)+ doit être remplacé par \verb+Leaf(Leaf(k))+ pour $k \in \NN$.

\medskip

De façon similaire, on peut passer des arbres de l'exemple \ref{exa-tree-tree} à ceux de l'exemple \ref{exa-tree}.

\medskip

Mais n'allons pas trop vite... En effet, pour tout arbre \verb+t+ de type $arbre(\NN)$, nous avons \verb+Leaf(t)+ qui est de type $arbre(arbre(\NN))$. Ceci a des implications fortes. Pour le voir, considérons l'arbre \verb+TEX+ suivant qui est de type $arbre(\NN)$.

\begin{pseudocode}
Node(
	Leaf(1), 
	Node(
		Node(
			Leaf(2),
			Leaf(3)
		), 
		Leaf(4)
	)
)
\end{pseudocode}

	
Voici un premier arbre de type $arbre(arbre(\NN))$ pouvant être \emph{\og interprété naturellement \fg} comme étant l'arbre \verb+TEX+. De nouveau, nous noterons \verb+Leaf''+ et \verb+Node''+ les applications feuille et noeud pour le type $arbre(arbre(\NN))$.

\begin{pseudocode}
Leaf''(
	Node(
		Leaf(1), 
		Node(
			Node(
				Leaf(2),
				Leaf(3)
			), 
			Leaf(4)
		)
	)
)
\end{pseudocode}


Voici un autre exemple.

\begin{pseudocode}
Node''(
	Leaf''(Leaf(1)), 
	Node''(
		Node''(
			Leaf''(Leaf(2)),
			Leaf''(Leaf(3))
		), 
		Leaf''(Leaf(4))
	)
)
\end{pseudocode}


On peut aussi proposer l'arbre ci-dessous.

\begin{pseudocode}
Node''(
	Leaf''(Leaf(1)), 
	Node''(
		Leaf''(
			Node(
				Leaf(2),
				Leaf(3)
			)
		), 
		Leaf''(Leaf(4))
	)
)
\end{pseudocode}


Pour finir, voici une dernière proposition.

\begin{pseudocode}
Node''(
	Leaf''(Leaf(1)), 
	Leaf''(
		Node(
			Node(
				Leaf(2),
				Leaf(3)
			), 
			Leaf(4)
		)
	)
)
\end{pseudocode}






Si l'on regarde bien chacun des exemples proposés, nous notons que les \verb+Leaf''+ \emph{\og polluent \fg} les arbres mais pas les \verb+Node''+.
En fait, en transformant syntaxiquement \verb+Node''(...)+ en \verb+Node(...)+ , puis  \verb+Leaf''(Leaf(...))+ en \verb+Leaf(...)+ et enfin \verb+Leaf''(Node(...))+ en \verb+Node(...)+ alors pour chaque exemple nous retombons sur notre arbre initial \verb+TEX+ de type $arbre(\NN)$.


\medskip

Intuitivement le type $arbre(arbre(\NN))$ ne contient pas plus d'information que le type $arbre(\NN)$. 
L'objectif de la section suivante va être de définir rigoureusement ce que nous entendons par \emph{\og information \fg} puis ensuite de voir qu'effectivement le type $arbre(arbre(\NN))$ ne donne pas plus d'information que le type $arbre(\NN)$.
