Sans donner de signification aux arbres que nous présentons, les arbres de type $arbre(arbre(\NN))$ et ceux de type $arbre(\NN)$ sont de natures structurelles totalement différentes comme le montrent les exemples précédents.


\medskip

D'un autre côté si nous nous intéressons non aux structures de nos arbres mais à ce qu'ils stockent, alors la section précédente nous pousse à trouver une grande similitude entre les arbres de type $arbre(arbre(\NN))$ et ceux de type $arbre(\NN)$. 


\medskip

Autrement dit, la comparaison des types $arbre(arbre(\NN))$ et $arbre(\NN)$ ne peut faire sens que du point de vie sémantique, et non simplement via une analyse syntaxique.
Nous devons donc arriver à donner une signification à nos arbres. C'est ce que proposent les deux définitions \emph{\og naturelles \fg} suivantes.


\begin{definition}
	KKKKK
\end{definition}



\begin{definition}
	KKKKK
\end{definition}



\begin{example}
	????    
	type $arbre(\NN)$
	\begin{pseudocode}
???
	\end{pseudocode}
	
	\medskip
	
	Ici nous avons ????
\end{example}



\begin{example}
	????    
	type $arbre(arbre(\NN))$
	\begin{pseudocode}
???
	\end{pseudocode}
	
	\medskip
	
	Ici nous avons ????
\end{example}



 de n souhaite définir l'ingformation contenue

on définit une fonciton $\infoarbre$ stockant profondeur usivi de la valeur d'un noued

l'info définit l'arbre de façon unique !

on montre que les type ont le smême info si on définit une fino récursivement sue arbre(arbre(N)) 


on crée une ou deux fnction qui fabrique une liste qui stocke -2 ou 2 si on va à gaiche ou à droite dans un arbre, et  si c'est une feuille on stocke valeur $10^n$ 

de la sorte on repère valeur et déplacement


pb pour arbre(arbre(N)), comment gfircer vélautaion de la feuille ???



\Huge

PLUS SIMPLE !

$\infoarbre$ et $\infoarbrearbre$ qui décrive le mêm enselble à carcatériser proprment !!!!