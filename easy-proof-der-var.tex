\documentclass[12pt]{amsart}
\usepackage[T1]{fontenc}
\usepackage[utf8]{inputenc}

\usepackage[top=1.95cm, bottom=1.95cm, left=2.35cm, right=2.35cm]{geometry}


\usepackage{wrapfig} 

\usepackage{hyperref}
\usepackage{enumitem}
\usepackage{tcolorbox}
\usepackage{float}
\usepackage{cleveref}
\usepackage{multicol}
\usepackage{fancyvrb}
\usepackage{enumitem}
\usepackage{amsmath}
\usepackage{textcomp}
\usepackage{numprint}
\usepackage[french]{babel}
\usepackage[
    type={CC},
    modifier={by-nc-sa},
	version={4.0},
]{doclicense}

\newcommand\floor[1]{\left\lfloor #1 \right\rfloor}

\usepackage{lymath}


\newtheorem{fact}{Fait}
\newtheorem*{fact*}{Fait}

\newtheorem{example}{Exemple}

\newtheorem{remark}{Remarque}
\newtheorem*{remark*}{Remarque}

\newtheorem*{proof*}{Preuve}

\setlength\parindent{0pt}

\floatstyle{boxed} 
\restylefloat{figure}


\DeclareMathOperator{\taille}{\text{\normalfont\texttt{taille}}}

\newcommand\sqseq[2]{\fbox{$#1$}_{\,\,#2}}


\DefineVerbatimEnvironment{rawcode}%
	{Verbatim}%
	{tabsize=4,%
	 frame=lines, framerule=0.3mm, framesep=2.5mm}
	 
	 
	 
\begin{document}

\title{BROUILLON - Dérivée et variations - Une preuve pour lycéen}
\author{Christophe BAL}
\date{16 Juillet 2019}

\maketitle

\begin{center}
	\itshape
	Document, avec son source \LaTeX, disponible sur la page
	
	\url{https://github.com/bc-writing/drafts}.
\end{center}


\bigskip


\begin{center}
	\hrule\vspace{.3em}
	{
		\fontsize{1.35em}{1em}\selectfont
		\textbf{Mentions \og légales \fg}
	}
			
	\vspace{0.45em}
	\doclicenseThis
	\hrule
\end{center}


\vspace{1em}


Voici un classique de l'analyse.


\begin{fact}
	Pour toute fonction $f$ définie et dérivable sur un intervalle non trivial $\intervalC{a}{b}$ , si $f\,' > 0$ sur $\intervalC{a}{b}$ alors $f$ est strictement croissante sur $\intervalC{a}{b}$ .
\end{fact}

La preuve standard de ce résultat passe par le célèbre théorème de Rolle appliqué à la fonction $g(x) = f(x) - \frac{f(a) - f(b)}{a - b} \cdot (x - a) - f(a)$ qui peut sembler un peu magique
\footnote{
	Pour $x \in \intervalC{a}{b}$ , $g(x)$ mesure l'écart entre la courbe $\geoset*{C}{f}$ et la droite $(AB) : y = \frac{f(a) - f(b)}{a - b} \cdot (x - a) + f(a)$ où $A(a ; f(a)) \in \geoset*{C}{f}$ et $B(b ; f(b)) \in \geoset*{C}{f}$ .
	La fonction $g$ n'est donc pas si magique que cela...
}.


\bigskip


Il se trouve que l'on peut démontrer \emph{\og plus simplement \fg} le résultat plus restrictif suivant.


\begin{fact}
	Pour toute fonction $f$ définie et dérivable sur un intervalle non trivial $\intervalC{a}{b}$ , et \textbf{de dérivée continue} 
	\footnote{
		On rencontre tout le temps cette situation au lycée.
	}
	sur $\intervalC{a}{b}$ , si $f\,' > 0$ sur $\intervalC{a}{b}$ alors $f$ est strictement croissante sur l'intervalle $\intervalC{a}{b}$ .
\end{fact}

La démonstration de ce résultat peut se faire via la stricte positivité de l'intégrale sur $\RR$ et l'identité $f(b) - f(a) = \int_a^b f\,'(x) \dd{x}$ .


\begin{remark*}
	Il est vrai que le calcul intégral n'est pas simple à définir proprement.
	D'un autre côté le théorème de Rolle s'appuie sur un résultat topologique puissant qui affirme que l'image continue d'un compact est un compact, mai aussi sur le fait que l'intervalle $\intervalC{a}{b}$ est compact.
	Malgré cela, l'auteur pense que d'un point de vue universitaire ce qui est affirmé dans ce document n'est qu'idioties.
	Par contre, un lycéen a accès intuitivement très facilement à un ensemble de résultats de base sur les intégrales et il semble dès lors que l'approche ci-dessus est bien plus simple que celle passant par le théorème de Rolle et la fonction un peu magique $g$ .
\end{remark*}

\end{document}
