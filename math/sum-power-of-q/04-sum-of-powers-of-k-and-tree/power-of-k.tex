\begin{frame}
	\begin{center}
		\Large
		Que se passe-t-il en général avec
		
		\bigskip
		
		les puissances successives de $k \in \mathbb{N}_{{}_{\geqslant 2}}$ ?
	\end{center}
\end{frame}


% -------------- %


\begin{frame}
	\begin{center}
		\Large
		Il est maintenant temps de démontrer
		
		\bigskip
		
		et non juste de constater.
	\end{center}
\end{frame}


% -------------- %


\begin{frame}
	\begin{center}
		\Large
		Peut-on trouver une numérotation faisant
		
		\bigskip
		
		apparaître les puissances de $k$ à gauche ?
	\end{center}
\end{frame}
   

% -------------- %


\begin{frame}
  \centering\Large
  Pour $k = 2$, on fait comme suit.		
  \bigskip
		
  \scriptsize
  \scalebox{0.925}{
      \begin{forest}
        /tikz/every label/.append style={text height=1ex, label distance=3pt},
        for tree={
          circle,
          draw,
          very thick,
          edge={very thick},
          s sep+=20pt,
          fill=blue!15,
          minimum size=40pt,
          bottom up,
        }
        [1
          [2]
          [3]
        ]
      \end{forest}
  }
\end{frame}
   

% -------------- %


\begin{frame}
  \centering\Large
  Pour $k = 3$, on fait comme suit.		
  \bigskip
		
  \scriptsize
  \scalebox{0.925}{
      \begin{forest}
        /tikz/every label/.append style={text height=1ex, label distance=3pt},
        for tree={
          circle,
          draw,
          very thick,
          edge={very thick},
          s sep+=20pt,
          fill=blue!15,
          minimum size=40pt,
          bottom up,
        }
        [\updown{1}{2}
          [\updown{3}{4}]
          [\updown{5}{6}]
          [\updown{7}{8}]
        ]
      \end{forest}
  }
\end{frame}
   

% -------------- %


\begin{frame}
  \centering\Large
  Pour $k = 4$, on fait comme suit.		
  \bigskip
		
  \scriptsize
  \scalebox{0.925}{
      \begin{forest}
        /tikz/every label/.append style={text height=1ex, label distance=3pt},
        for tree={
          circle,
          draw,
          very thick,
          edge={very thick},
          s sep+=20pt,
          fill=blue!15,
          minimum size=40pt,
          bottom up,
        }
        [\updownB123
          [\updownB456]
          [\updownB789]
          [\updownB{10}{11}{12}]
          [\updownB{13}{14}{15}]
        ]
      \end{forest}
  }
\end{frame}
   

% -------------- %


\begin{frame}
  \centering\Large
  Pour $k = 5$, on fait comme suit.		
  \bigskip
		
  \tiny
  \scalebox{0.925}{
      \begin{forest}
        /tikz/every label/.append style={text height=1ex, label distance=3pt},
        for tree={
          circle,
          draw,
          very thick,
          edge={very thick},
          s sep+=20pt,
          fill=blue!15,
          minimum size=40pt,
          bottom up,
        }
        [\updownBB1234
          [\updownBB5678]
          [\updownBB9{10}{11}{12}]
          [\updownBB{13}{14}{15}{16}]
          [\updownBB{17}{18}{19}{20}]
          [\updownBB{21}{22}{23}{24}]
        ]
      \end{forest}
  }
\end{frame}


% -------------- %


\begin{frame}
	\begin{center}
		\Large
		En fait la numérotation n'est pas importante !
	
		\uncover<2->{
			\bigskip
			
			Le plus utile est le nombre d'entiers par noeud.
    	}
	
		\uncover<3->{
			\bigskip
			
			La numérotation est là pour le prestige...
    	}
	
		\uncover<4->{
			\bigskip
			
			Voici pourquoi.
    	}
	\end{center}
\end{frame}
   

% -------------- %


\begin{frame}
  \centering\Large
  \only<1>{
  		Niveau 1 -- Nombre de nouveaux entiers
		
		\bigskip
		
		$\,\,\,\,\,k-1$ 
  }		
  \only<2>{
  		Niveau 2 -- Nombre de nouveaux entiers
		
		\bigskip
		
		$k(k-1) = k^2 - k\,\,$ 
  }		
  \only<3>{
  		Niveau 3 -- Nombre de nouveaux entiers
		
		\bigskip
		
		$k^2(k-1) = k^3 - k^2\,\,\,\,$ 
  }

  \bigskip
		
  \scriptsize
  \scalebox{0.925}{
      \begin{forest}
        /tikz/every label/.append style={text height=1ex, label distance=3pt},
        for tree={
          circle,
          draw,
          very thick,
          edge={very thick},
          s sep+=1pt,
          fill=blue!15,
          minimum size=25pt,
          bottom up,
        }
        [
          [, not before=2
          	[, not before=3]
          	[\dots, not before=3]
          	[, not before=3]
          ]
          [\dots, not before=2
          	[, not before=3]
          	[\dots, not before=3]
          	[, not before=3]
          ]
          [, not before=2
          	[, not before=3]
          	[\dots, not before=3]
          	[, not before=3]
          ]
        ]
      \end{forest}
  }
\end{frame}
   

% -------------- %


\begin{frame}
  \begin{center}
  	\Large
	Faisons un 1ier bilan.
  \end{center}
  
  \uncover<2->{
  	\bigskip
	
	\,\,\,\,\,\,\,\,
	Niveau 1 : $k - 1$ 
  }
  
  \uncover<3->{
  	\bigskip
	
	\,\,\,\,\,\,\,\,
	Niveau 2 : $k^2 - k$ 
  }
  
  \uncover<4->{
  	\bigskip
	
	\,\,\,\,\,\,\,\,
	Niveau 3 : $k^3 - k^2$ 
  }
  
  \uncover<5->{
  	\bigskip
	
	\,\,\,\,\,\,\,\,
	\phantom{Niveau 3} $\vdots$
  }
  
  \uncover<6->{
  	\bigskip
	
	\,\,\,\,\,\,\,\,
	Niveau $n$ : $k^n - k^{n-1}$ 
  }
  
  \uncover<7->{
  	\bigskip
	
	Niveau $n+1$ : $k^{n+1} - k^n$ 
  }
  
  \uncover<8->{
  	\bigskip
	
	\,\,\,\,\,\,\,\,
	$\,\,TOTAL = k^{n+1} - 1$ 
  }
\end{frame}
   

% -------------- %


\begin{frame}
  \begin{center}
  	\Large
	Faisons un 2nd bilan.
  
	\normalsize
	\uncover<3->{
  		\bigskip
	
		On multiplie le nombre de noeuds par $(k - 1)$.
 	}
  
	\uncover<4->{
  		\bigskip
	
		Jusqu'au niveau $n+1$ :
		
		\bigskip
	
		$TOTAL = (k - 1)(1 + k + k^2 + \cdots + k^n)$
 	}
	
	\uncover<2->{
	    \bigskip
	    
		  \scriptsize
  \scalebox{0.925}{
      \begin{forest}
        /tikz/every label/.append style={text height=1ex, label distance=3pt},
        for tree={
          circle,
          draw,
          very thick,
          edge={very thick},
          s sep+=1pt,
          fill=blue!15,
          minimum size=25pt,
          bottom up,
        }
        [
          [
          	[]
          	[\dots]
          	[]
          ]
          [\dots
          	[]
          	[\dots]
          	[]
          ]
          [
          	[]
          	[\dots]
          	[]
          ]
        ]
      \end{forest}
  }
	}
  \end{center}
\end{frame}
   

% -------------- %


\begin{frame}
	\begin{center}
		\Large
		Pour $k \in \mathbb{N}_{{}_{\geqslant 2}}$ , nous avons finalement démontré :
				
		\bigskip
		
		$(k-1) \, (1 + k + k^2 + \cdots + k^n) = k^{n+1} - 1$.
	\end{center}
\end{frame}
   

% -------------- %


\begin{frame}
	\begin{center}
		\Large
		Plus généralement, pour $q \in \mathbb{R}$ , nous avons :
				
		\bigskip
		
		$(q - 1) \, (1 + q + q^2 + \cdots + q^n) = q^{n+1} - 1$.
	\end{center}
\end{frame}