\begin{frame}
	\begin{center}
		\Large
		Pouvons-nous généraliser au cas
		
		\bigskip
		
		des puissances successives de $3$ ?
	\end{center}
\end{frame}
   

% -------------- %


\begin{frame}
  \centering
  \begin{forest}
    /tikz/every label/.append style={text height=1ex, label distance=3pt},
    for tree={
      circle,
      draw,
      very thick,
      edge={very thick},
      s sep+=2pt,
      fill=red!15,
      minimum size=23pt,
      bottom up,
    }
    [{1}\vphantom{$1_x^M$}
      [{2}\vphantom{$1_x^M$}, not before=2,
        [{5}\vphantom{$1_x^M$}, not before=3]
        [{6}\vphantom{$1_x^M$}, not before=3]
        [{7}\vphantom{$1_x^M$}, not before=3]
      ]
      [{3}\vphantom{$1_x^M$}, not before=2,
        [{8}\vphantom{$1_x^M$}, not before=3]
        [{9}\vphantom{$1_x^M$}, not before=3]
        [{10}, not before=3]
      ]
      [{4}\vphantom{$1_x^M$}, not before=2,
        [{11}, not before=3]
        [{12}, not before=3]
        [{13}, not before=3]
      ]
    ]
  \end{forest}
\end{frame}
   

% -------------- %


\begin{frame}
  \centering
  \begin{forest}
    /tikz/every label/.append style={text height=1ex, label distance=3pt},
    for tree={
      circle,
      draw,
      very thick,
      edge={very thick},
      s sep+=2pt,
      fill=red!15,
      minimum size=23pt,
      bottom up,
    }
    [\phantom{11}
      [\phantom{11}
        [\phantom{11}]
        [\phantom{11}]
        [\phantom{11}]
      ]
      [\phantom{11}
        [\phantom{11}]
        [\phantom{11}]
        [\phantom{11}]
      ]
      [\phantom{11}
        [\phantom{11}]
        [\phantom{11}]
        [\phantom{11}]
      ]
    ]
  \end{forest}
\end{frame}
   

% -------------- %


\begin{frame}
  \centering\scriptsize
  \scalebox{0.925}{
      \begin{forest}
        /tikz/every label/.append style={text height=1ex, label distance=3pt},
        for tree={
          circle,
          draw,
          very thick,
          edge={very thick},
          s sep+=2pt,
          fill=blue!15,
          minimum size=23pt,
          bottom up,
        }
        [\updown{1}{2}\vphantom{\updown{11}{12}}
          [\updown{3}{4}\vphantom{\updown{11}{12}}
            [\updown{9}{10}\vphantom{\updown{11}{12}}]
            [\updown{11}{12}]
            [\updown{13}{14}]
          ]
          [\updown{5}{6}
            [\updown{15}{16}]
            [\updown{17}{18}]
            [\updown{19}{20}]
          ]
          [\updown{7}{8}
            [\updown{21}{22}]
            [\updown{23}{24}]
            [\updown{25}{26}]
          ]
        ]
      \end{forest}
  }
\end{frame}


% -------------- %


\begin{frame}
	\begin{center}
		\Large
		Nous avons découvert assez vite la formule :
		
		\bigskip
		
		$2 \, (1 + 3 + 3^2 + \cdots + 3^n) = 3^{n+1} - 1$.
	\end{center}
\end{frame}

