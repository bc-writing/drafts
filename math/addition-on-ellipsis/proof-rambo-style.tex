Dans le repère $\paxes{O | I | J}$ supposé orthonormé, $A = (\cos a ; \sin a)$ et $B = (\cos b ; \sin b)$ avec $(a ; b) \in \RR^2$ .
Nous avons alors $S = (\cos(a + b) ; \sin(a + b))$ .
Pour faciliter les calculs, nous posons $A = (c_A ; s_A)$ et $B = (c_B ; s_B)$ de sorte que $S = (c_A c_B - s_A s_B ; c_A s_B + s_A c_B)$ d'après les formules trigonométriques d'addition.


\medskip


\textbf{Cas 1.} \emph{Supposons que $A \neq B$ .}

\medskip

La colinéarité des vecteurs $\vect{AB}$ et $\vect{IS}$ est justifiée par les calculs brutaux suivants \emph{(on pourrait faire appel à un logiciel de calcul formel qui ici peut être utilisé en toute confiance)}.
\begin{flalign*}
	\det\left ( \vect{AB} ; \vect{IS} \right)
		&=
		\left|\begin{NiceArray}{CC} 
			c_B - c_A  &  c_A c_B - s_A s_B - 1 \\ 
			s_B - s_A  &  c_A s_B + s_A c_B
		\end{NiceArray}\right|
		& \\
		&=
		(c_B - c_A)( c_A s_B + s_A c_B)
		-
		(s_B - s_A)(c_A c_B - s_A s_B - 1)
		& \\
		&=
		c_B c_A s_B + s_A c_B^2
		- c_A^2 s_B - c_A s_A c_B
		& \\
		&
		- s_B c_A c_B + s_A s_B^2 + s_B
		+ s_A c_A c_B - s_A^2 s_B - s_A
		& \\
		&=
		s_A c_B^2 - c_A^2 s_B + s_A s_B^2 + s_B - s_A^2 s_B - s_A
		& \\
		&=
		s_A (c_B^2 + s_B^2) - (c_A^2 + s_A^2) s_B + s_B - s_A
		& \\
		&=
		s_A - s_B + s_B - s_A
		& \\
		&=
		0
		& \\
\end{flalign*}

\vspace{-1em}


\textbf{Cas 2.} \emph{Supposons que $A = B$ .}

\medskip

En notant qu'ici $S = (c_A^2 - s_A^2 ; 2 c_A s_A)$ , l'orthogonalité des vecteurs $\vect{OA}$ et $\vect{IS}$ est justifiée par les calculs suivants \emph{(l'usage d'un logiciel de calcul formel serait un peu excessif)}.
\begin{flalign*}
	\dotprod{\vect{OA}}{\vect{IS}}
		&=
		c_A \cdot (c_A^2 - s_A^2 - 1) + s_A \cdot 2 c_A s_A
		& \\
		&=
		c_A \cdot (1 - s_A^2 - s_A^2 - 1) + 2 c_A s_A^2
		& \\
		&=
		- 2 c_A s_A^2 + 2 c_A s_A^2
		& \\
		&=
		0
		& \\
\end{flalign*}

\vspace{-1em}


