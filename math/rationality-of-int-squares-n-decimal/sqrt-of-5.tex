\begin{fact} \label{sqrt-5-not-in-Q}
	$\sqrt5 \not\in \QQ$
\end{fact}

\begin{proof}
	De nouveau, nous partons de $r^2 = 5s^2$ mais ici nous ne pouvons plus faire comme avant car nous avons les tableaux suivants qui nous disent que $r$ peut avoir n'importe quel chiffre des unités, ceci ne nous permettant pas d'avoir une contradiction.

	\begin{center}
		\begin{tabular}{|r|c|c|c|c|c|c|c|c|c|c|}
			\hline
			Chiffre des unités de $r$
			  & $0$  &  $1$  &  $2$  &  $3$  &  $4$  &  $5$  &  $6$  &  $7$  &  $8$  &  $9$
			\\ \hline
			\phantom{$5$}Chiffre des unités de $r^2$
			  & $0$  &  $1$  &  $4$  &  $9$  &  $6$  &  $5$  &  $6$  &  $9$  &  $4$  &  $1$
			\\ \hline
		\end{tabular}

		\medskip

		\begin{tabular}{|r|c|c|c|c|c|c|c|c|c|c|}
			\hline
			Chiffre des unités de $s$
			  & $0$  &  $1$  &  $2$  &  $3$  &  $4$  &  $5$  &  $6$  &  $7$  &  $8$  &  $9$
			\\ \hline
			Chiffre des unités de $5s^2$
			  & $0$  &  $5$  &  $0$  &  $5$  &  $0$  &  $5$  &  $0$  &  $5$  &  $0$  &  $5$
			\\ \hline
		\end{tabular}
	\end{center}

    \medskip

    Tentons tout simplement notre chance avec les deux derniers chiffres décimaux et non juste le dernier.
    À ce stade, nous allons utiliser un programme écrit en Python dont le code est le suivant
    \footnote{
    	Le fichier Python est disponible sur \url{https://github.com/bc-writing/drafts}.
		Chercher le fichier \texttt{last-two-digits.py} .
	}
	où la fonction \verb+keep+ nous sera utile après pour analyser rapidement les résultats.

	\begin{rawcode}
N = 5

def keep(i):
    return True

twolastdigits       = []
twolastdigits_for_N = []

for i in range(100):
    square   = (i**2)%100
    N_square = (N*square)%100

    twolastdigits.append(square)
    twolastdigits_for_N.append(N_square)

common = set(twolastdigits).intersection(twolastdigits_for_N)


print(f"N = {N}")
print("r possibles :")

for i in range(100):
    if twolastdigits[i] in common and keep(i):
        print(i, end = ", ")

print()
print("s possibles :")

for i in range(100):
    if twolastdigits_for_N[i] in common and keep(i):
        print(i, end = ", ")

print()
	\end{rawcode}


	\medskip

	Une fois lancé dans un terminal, ce programme nous affiche :

	\begin{rawcode}
N = 5
r possibles :
0, 5, 10, 15, 20, 25, 30, 35, 40, 45, 50, 55, 60, 65, 70, 75, 80, 85, 90, 95,
s possibles :
0, 5, 10, 15, 20, 25, 30, 35, 40, 45, 50, 55, 60, 65, 70, 75, 80, 85, 90, 95,
	\end{rawcode}

	Nous constatons que $r$ et $s$ sont tous les deux multiples de $5$ ce qui établit une contradiction permettant de conclure.
	Ceci est confirmé par le programme.
	En effet, dans la fonction \verb+keep+, utilisons \verb+return i % 5 != 0+
	à la place de \verb+return True+ afin de n'afficher que les non multiples de $5$.
	On obtient alors la sortie suivante.

	\begin{rawcode}
N = 5
r possibles :

s possibles :

	\end{rawcode}
\end{proof}


\begin{remark}
	Notons que notre programme fournit bien une argumentation totalement rigoureuse car il travaille uniquement sur des valeurs naturelles exactes.
\end{remark}
