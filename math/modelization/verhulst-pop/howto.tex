La grosse faiblesse du modèle de Malthus, lequel produit une fonction à la croissance exponentielle dès que le taux de natalité est supérieur à celui de la mortalité, est que ces deux taux sont supposés constants.
On peut alors considérer ces taux comme étant des fonctions.
La question se posant ensuite est de savoir de quelle variables dépendent ces taux. Du temps uniquement ? De la taille de la population uniquement ? Des deux 
\footnote{
	\dots mon capitaine.
} ?
Ayant noté les faiblesses du modèle de Malthus, le mathématicien belge Pierre François Verhulst (1804 -- 1849) propose les fonctions suivantes où $P(t)$ est le nombre d'individus à la fin de l'année $t$ .

\begin{itemize}[label=\small\textbullet]
	\item Le taux de natalité est une fonction affine croissante dépendant du nombre d'individus.
	Donc pour l'année $t$ , ce taux sera du type $f(t) = \alpha P(t) + \beta$ avec $(\alpha ; \beta) \in \left( \RRsp \right)^2$ .

	\item Le taux de mortalité une fonction affine décroissante dépendant du nombre d'individus.
	Donc pour l'année $t$ , ce taux sera du type $m(t) = q P(t) + r$ avec $(q ; r) \in \RRsn \times \RRsp$ .
\end{itemize}

Dès lors, le taux d'évolution $P(t+1) - P(t)$ est 
$f(t) P(t) - m(t) P(t) = P(t) \left( A - B P(t) \right)$ en posant $A = \alpha - q \in \RRp$ et $B = -\beta + r \in \RR$ \emph{(la justification du choix d'un signe moins apparaitra plus bas)}.


\medskip


À ce stade, nous avons une équation de récurrence du type $u_{n+1} = u_n (A - B u_n)$ . Il se trouve que le comportement de ces suites peut être de type chaotique. Il devient alors plus facile de passer par un modèle différentielle, c'est à dire de passer de suites à des fonctions dérivables comme dans le modèle de Malthus. Ceci nous donne : $P\,'(t) = P(t) \left( A - B P(t) \right)$ . La fonction $P$ vérifie donc l'équation différentielle $y\,' = y (A - B y)$ .


\medskip

Nous allons supposer $A > 0$ , le cas $A = 0$ sera examiné plus bas en remarque. Sous cette hypothèse, $y\,' = y (A - B y)$ se réécrit $y\,' = a y (1 - b y)$ avec $a = A > 0$ et $b = \frac{B}{A}$ . 
Nous allons établir juste après que $b > 0$ pour le modèle étudié
\footnote{
	Ceci justifie ce signe moins a priori étrange.
}.
Pour $(a ; b) \in \left( \RRsp \right)^2$ , Pierre François Verhulst qualifie $y\,' = a y (1 - b y)$ d'équation \emph{\og logistique \fg} sans que l'on sache vraiment pourquoi.


\medskip

Justifions que $b > 0$ pour notre modèle. Pour cela, nous allons tout simplement résoudre $y\,' = a y (1 - b y)$ , ce qui se fait facilement comme suit en supposant que $P$ ne s'annule jamais \emph{(hypothèse qui concrètement ne pose pas de problème)}.

\vspace{-1em}

\begin{flalign*}
	P\,'(t) = a P(t) \left(1 - b P(t) \right)
		& \Longleftrightarrow  - \frac{P\,'(t)}{P^2(t)} = - \frac{a}{P(t)} + ab
		& \\
		& \Longleftrightarrow  f\,'(t) = - a f(t) + ab  
				\,\, \text{où on a posé $f(t) = \frac{1}{P(t)}$ .}
		& \\
\end{flalign*}

\vspace{-1em}

Nous aboutissons à une simple équation différentielle linéaire du 1\ier ordre et il est connu que nécessairement $f(t) = b + k \ee^{-a t}$ où $k \in \RR$ est une constante dépendant des conditions réelles.
Nous avons finalement : $P(t) = \frac{1}{b + k \ee^{-a t}}$ . Faisant tendre $t$ vers $+\infty$ , nous avons que $\frac{1}{b}$ correspond au nombre d'individus lorsque la population se stabilise
\footnote{
	Attention car cette stabilisation à long terme est due au modèle. Rien ne nous dit que ce phénomène sera concrètement observable.
}
d'où $b > 0$ comme annoncé.


\medskip

Pierre François Verhulst rend l'équation plus expressive en écrivant $y\,' = a y \left( 1 - \frac{y}{K} \right)$ avec $K = \frac{1}{b}$ qui a la même dimension que $P(t)$ , de sorte que $P(t) = \frac{K}{1 + k K \ee^{-a t}}$ peut se réécrire $P(t) = \frac{K}{1 + \left( \frac{K}{P(0)} - 1 \right) \ee^{-a t}}$ .



\begin{remark}
	Dans  $y\,' = y (A - B y)$ , si $A = 0$ , nous obtenons $y\,' = - B y^2$ dont les solutions ne s'annulant pas sont du type $\frac{1}{k + B t}$ avec $k$ une constante. Appliqué à notre contexte, ceci fournirait une population $P(t) = k + B t$ évoluant de façon linéaire, une modélisation qui n'est pas du tout crédible. 
\end{remark}
