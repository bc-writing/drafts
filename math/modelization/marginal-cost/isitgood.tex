Il se trouve que la modélisation du coût de production par un polynôme de degré 2 ou 3 apparait souvent dans des livres présentant des mathématiques élémentaires pour l'Économie. Cela montre que ces modèles ont leur utilité et ceci même si les raisonnements précédents sont fragiles car ils reposent sur des \emph{\og approximations de courbes \fg} . La technique d'approximation de formules, ou plus précisément d'interpolation, reste tout de même intéressante.