\documentclass[12pt]{amsbook}
\usepackage[T1]{fontenc}
\usepackage[utf8]{inputenc}

\usepackage[top=1.95cm, bottom=1.95cm, left=2.35cm, right=2.35cm]{geometry}

\usepackage{amsmath}
\usepackage{amssymb}
\usepackage{enumitem}
\usepackage{multicol}
\usepackage[french]{babel}
\usepackage[
    type={CC},
    modifier={by-nc-sa},
	version={4.0},
]{doclicense}

\usepackage{tnsmath}

\DeclareMathOperator{\taille}{\tau}

\newtheorem{fact}{Fait}
\newtheorem*{proof*}{Preuve}

\newtheorem{remark}{Remarque}[section]

\setlength\parindent{0pt}


\newcommand\squote[1]{\og #1 \fg{}}

\newcommand\myabrev[1]{\textbf{\emph{#1}}}

% Source: https://tex.stackexchange.com/a/568582/6880
\let\olditem\item
\newlist{methods}{itemize}{1}
\setlist[methods]{%
    align=right,
    before=\changeitem,
    font=\bfseries,
    after=\let\item\olditem,
    leftmargin=2.35cm
}
\newcommand*{\changeitem}{%
    \renewcommand*{\item}[1][]{%
        \olditem[##1 \emph{:}]
    }%
}

\newcommand\methodo[2]{méthode de type \uppercase{#1}-#2}


\begin{document}

\title{BROUILLON - Sommer des puissances de différentes façons \\{\small(27/10/2020 -- 29/10/2020)} \\ MANQUE DES DESSINS !}

\author{Christophe BAL}

\maketitle


% ------------- %

\begin{tcolorbox}
	\small\itshape
	Ce document s'intéresse à différents moyens de trouver les formules de sommation de puissances successives d'un réel $q \neq 1$ .
	Nous commencerons par étudier les cas très particuliers des puis\-sances de $q = 2$ et de celles de $q = \dfrac12$ pour passer ensuite au cas général.
	
	\medskip
	
	Chaque section a été rédigée pour être lue indépendamment des autres même si cela implique de répéter certains calculs ou raisonnements que l'on trouve ailleurs dans le document.
	
	\bigskip
	
	\begin{center}
		\textbf{Abréviations utilisées pour les titres des sections} 
	\end{center}
	
	\smallskip
	
	\begin{methods}
		\item[ALG]
		      \methodo{alg}{ébrique}

		\item[ARI]
		      \methodo{ari}{thmétique}

		\item[EXP]
		      \methodo{exp}{érimental}

		\item[GÉO]
		      \methodo{géo}{métrique}

		\item[INFO]
		      \methodo{info}{rmatique}
	\end{methods}
\end{tcolorbox}


\bigskip
\bigskip
\bigskip


\begin{center}
	\hrule\vspace{.3em}
	{
		\fontsize{1.35em}{1em}\selectfont
		\textbf{Mentions \og légales \fg}
	}
			
	\vspace{0.45em}
	\doclicenseThis
	\hrule
\end{center}


% ------------- %


\setcounter{tocdepth}{1}
\tableofcontents


% ------------- %


\chapter{Sommer les puissances de $2$}

%\newpage
\section{[ALG] Jouons avec les écritures} \label{power-2:rewriting}

	\subsection{La preuve}

Les calculs suivants sont très simples à suivre mais malheureusement peu éclairants d'un point de vue conceptuel. Le cas où $n = 0$ étant évident, on suppose par la suite que $n \in \NNs$.

\medskip

\begin{explain}[style = sar, ope = \iff]
	S_n \eqdef \dsum_{k = 0}^{n} 2^k
		\explnext{}
	S_n = 1 + \dsum_{k = 1}^{n} 2^k
		\explnext{\footnotesize $k = i+1 \iff i = k-1$}
	S_n = 1 + \dsum_{i = 0}^{n-1} 2^{i + 1}
		\explnext{}
	S_n = 1 + 2 \dsum_{i = 0}^{n-1} 2^i
		\explnext*{\footnotesize Faisons apparaître la somme \og réduite\fg{} à gauche. }{}
	\dsum_{k = 0}^{n-1} 2^k + 2^n = 1 + 2 \dsum_{i = 0}^{n-1} 2^i
		\explnext{}
	S_{n-1} + 2^n = 1 + 2 S_{n-1}
		\explnext{}
	2^n - 1 = S_{n-1}
		\explnext{}
	S_{n-1} = 2^n - 1
\end{explain}

\bigskip

Donc $S_0 = 1$ et $\forall n \in \NNs$ , $S_{n-1} = 2^n - 1$ .
Ceci peut se réécrire :
$\forall n \in \NN$ , $S_n = 2^{n+1} - 1$ .


% ------------- %


\subsection{Commentaires}

La méthode précédente se généralise sans souci aux puissances de $q$ comme nous le verrons dans la section \ref{power-q:rewriting} page \pageref{power-q:rewriting}.
Par contre elle n'éclaire en rien sur la signification de la formule trouvée mais a l'avantage d'être prouvable via un ordinateur.


% ------------- %


\subsection{D'autres applications}

La réécriture de sommes peut permettre de trouver des sommes du type $\dsum_{k = 0}^{n} k^p$ .
Montrons par exemple comment trouver une formule explicite de la somme
\footnote{
	La lettre $G$ faire référence à Gauss à qui l'on attribue une méthode très astucieuse pour calculer cette somme en la réordonnant.
}
$G_n \eqdef \dsum_{k = 0}^{n} k$ où nous laissons de nouveau de côté le cas trivial où $n = 0$.
L'idée est de réécrire $C_n \eqdef \dsum_{k = 0}^{n} k^2$ et non $G_n$ car nous allons voir que la réécriture va éliminer les carrés
\footnote{
	Ce principe d'élimination se repère vite si l'on raisonne directement en réécrivant la somme cherchée $G_n$.
}.

\medskip

\begin{explain}[style = sar, ope = \iff]
	C_n \eqdef  \dsum_{k = 0}^{n} k^2
		\explnext{}
	C_n = 0 + \dsum_{k = 1}^{n} k^2
		\explnext{\footnotesize $k = i+1 \iff i = k-1$}
	C_n = \dsum_{i = 0}^{n-1} (i + 1)^2
		\explnext{}
	C_n = \dsum_{i = 0}^{n-1} (i^2 + 2 i + 1)
		\explnext{}
	C_n = \dsum_{i = 0}^{n-1} i^2 + 2 \dsum_{i = 0}^{n-1} i + \dsum_{i = 0}^{n-1} 1
		\explnext{}
	C_n = C_{n-1} + 2 G_{n-1} + n
		\explnext*{\footnotesize Apparition de la somme \og réduite\fg{} à gauche. }{}
	C_{n-1} + n^2 = C_{n-1} + 2 G_{n-1} + n
		\explnext{}
	n^2 = 2 G_{n-1} + n
\end{explain}


\begin{explain}[style = sar, ope = \iff]
	C_n \eqdef  \dsum_{k = 0}^{n} k^2
		\explnext{}
	n^2 - n = 2 G_{n-1}
		\explnext{}
	G_{n-1} = \dfrac{n(n - 1)}{2}
\end{explain}

\bigskip

Donc $G_0 = 0$ et $\forall n \in \NNs$ , $G_{n-1} = \dfrac{n(n - 1)}{2}$ .
Ceci se réécrit :
$\forall n \in \NN$ , $\dsum_{k = 0}^{n} k = \dfrac{n(n + 1)}{2}$ .

\medskip

On peut continuer de façon analogue pour obtenir une formule explicite de $C_n$ via la somme des cubes d'entiers successifs, puis ensuite on en aura une pour la somme des cubes elle-même mais les calculs deviennent vite pénibles
\footnote{
	En fait il existe une formulation générale faisant intervenir les nombres de Bernoulli qui ont de jolies propriétés.
}...
Voici la partie importante pour découvrir que 
$\forall n \in \NN$ , $\dsum_{k = 0}^{n} k^2 = \dfrac{n (n + 1) (2n + 1)}{6} = \dfrac{n^3}{3} + \dfrac{n^2}{2} + \dfrac{n}{6}$ .

\medskip

\begin{explain}[style = sar, ope = \iff]
	D_n \eqdef  \dsum_{k = 0}^{n} k^3
		\explnext{}
	D_n = \dsum_{i = 0}^{n-1} (i + 1)^3
		\explnext{}
	D_n = \dsum_{i = 0}^{n-1} (i^3 + 3 i^2 + 3 i + 1)
		\explnext{}
	D_n = D_{n-1} + 3 C_{n-1} + 3 G_{n-1} + n
		\explnext{}
	n^3 = 3 C_{n-1} + 3 G_{n-1} + n
		\explnext{}
	3 C_{n-1} = n^3 - 3 \cdot \dfrac{n(n - 1)}{2} - n
		\explnext{}
	6 C_{n-1} = 2 n^3 - 3n(n - 1) - 2n
		\explnext{}
	6 C_{n-1} = n (2 n^2 - 3n + 1)
		\explnext*{\footnotesize $1$ est une racine évidente de $2 X^2 - 3X + 1$.}{}
	6 C_{n-1} = n (n - 1) (2n - 1)
%		\explnext{}
%	C_{n-1} = \dfrac{n (n - 1) (2n - 1)}{6}
\end{explain}


\medskip


Finissons en montrant que
$\forall n \in \NN$ , $\dsum_{k = 0}^{n} k^3 = \dfrac{n^2 (n + 1)^2}{4} = \left( \, \dsum_{k = 0}^{n} k \, \right)^2$ . Que c'est joli !
Notez que les calculs se compliquent vite et rendent la preuve très inélégante.

\medskip

\begin{explain}[style = sar, ope = \iff]
	E_n \eqdef  \dsum_{k = 0}^{n} k^4
		\explnext{}
	E_n = \dsum_{i = 0}^{n-1} (i^4 + 4 i^3 + 6 i^2 + 4 i + 1)
		\explnext{}
	n^4 = 4 D_{n-1} + 6 C_{n-1} + 4 G_{n-1} + n
		\explnext{}
	4 D_{n-1} = n^4 - n (n - 1) (2n - 1) - 2n(n - 1) - n
		\explnext{}
	4 D_{n-1} = n \cdot \big[ \, \textcolor{blue}{n^3} - (n - 1) (2n - 1) - 2(n - 1) - \textcolor{blue}{1} \, \big]
		\explnext*{%
			\tiny%
			\begin{explain}[style = ar]
				n^3 - 1
					\explnext{}	
				(n - 1)(n^2 + n + 1)
			\end{explain}%
		}{}
	4 D_{n-1} = n (n - 1) \cdot \big[ \, \textcolor{blue}{n^2 + n + 1} - (2n - 1) - 2 \, \big]
		\explnext{}
	4 D_{n-1} = n (n - 1) (n^2 - n)
		\explnext{}
	4 D_{n-1} = n^2 (n - 1)^2
\end{explain}





\newpage
\section{[ALG] Sommes télescopiques} \label{power-2:telescopic}

	\subsection{La preuve}

Pour trouver une formule explicite de $S_n \eqdef \dsum_{k = 0}^{n} 2^k$, on peut noter que
$2^k = 2 \cdot 2^{k-1} = 2^{k-1} + 2^{k-1}$
donne
$2^{k-1} = 2^k - 2^{k-1}$
d'où l'on déduit que
$\forall k \in \NN$, $2^k = 2^{k+1} - 2^k$.
Ceci nous conduit aux calculs suivants.

\medskip

\begin{explain}[style = sar, ope = \iff]
	S_n \eqdef \dsum_{k = 0}^{n} 2^k
		\explnext{}
	S_n = \dsum_{k = 0}^{n} (2^{k+1} - 2^k)
		\explnext{}
	S_n = \dsum_{k = 0}^{n} 2^{k+1} - \dsum_{k = 0}^{n} 2^k
		\explnext{\footnotesize $i = k+1 \iff k = i-1$}
	S_n = \dsum_{i = 1}^{n+1} 2^{i} - \dsum_{k = 0}^{n} 2^k
		\explnext*{\footnotesize On fait apparaître des sommes identiques.}{}
	S_n = \dsum_{i = 1}^{n} 2^{i} + 2^{n+1} - 2^0 - \dsum_{k = 1}^{n} 2^k
		\explnext{}
	S_n = 2^{n+1} - 1
\end{explain}


% ------------- %


\subsection{Commentaires}

Les simplifications du type
$\dsum_{k = 0}^{n} (2^{k+1} - 2^k) = 2^{n+1} - 2^0$ ,
ou plus généralement du type
$\dsum_{k = 0}^{n} (u_{k+1} - u_{k}) = u_{n+1} - u_0$
ou
$\dsum_{k = 0}^{n} (u_{k} - u_{k+1}) = u_0 - u_{n+1}$ ,
sont un grand classique : on parle de \og sommes télescopiques \fg{}
\footnote{
	Cette technique permet par exemple de ramener l'étude d'une suite à celle d'une série.
	Or il se trouve que l'on dispose d'outils très pratiques pour étudier les séries.
}.
Cette astuce se généralise sans souci aux puissances de $q$ comme nous le verrons dans la section \ref{power-q:telescopic} page \pageref{power-q:telescopic}.
Bien qu'élégants du point de vue algébrique, les calculs ci-dessus ne donnent aucune information sur la signification de la formule trouvée.


% ------------- %


\subsection{D'autres applications}

Une application rigolote est l'obtention d'une formule explicite de la somme $I_n \eqdef \dsum_{k = 1}^{n} \dfrac{1}{k(k+1)}$. 
Une fois noté que $\dfrac{1}{k(k+1)} = \dfrac{1}{k} -  \dfrac{1}{k+1}$
le calcul est très aisé
\footnote{
	Cet exemple est conçu comme un cas typique d'usage de sommes télescopiques.
}.

\medskip

\begin{explain}[style = sar]
	I_n \explnext[\eqdef]{}
	\dsum_{k = 1}^{n} \dfrac{1}{k(k+1)}
		\explnext{}
	\dsum_{k = 1}^{n} \left( \dfrac{1}{k} - \dfrac{1}{k+1} \right)
		\explnext{\footnotesize Usage de sommes télescopiques.}
	\dfrac{1}{1} - \dfrac{1}{n+1}
		\explnext{}
	\dfrac{n}{n+1}
\end{explain}


\bigskip

On peut aussi utiliser des sommes télescopiques pour expliciter $\dsum_{k = 0}^{n} k^p$ .
Montrons par exemple comment trouver une formule explicite de la somme
\footnote{
	La lettre $G$ faire référence à Gauss à qui l'on attribue une méthode très astucieuse pour calculer cette somme en la réordonnant.
}
suivante $G_n \eqdef \dsum_{k = 0}^{n} k$.
L'idée astucieuse consiste à noter que $(k+1)^2 - k^2 = 2 k + 1$ puis à procéder comme suit.

\medskip

\begin{explain}[style = sar]
	2 G_n
		\explnext{}
	\dsum_{k = 0}^{n} 2 k
		\explnext{}
	\dsum_{k = 0}^{n} ((k+1)^2 - k^2 - 1)
		\explnext{}
	\dsum_{k = 0}^{n} ((k+1)^2 - k^2) - \dsum_{k = 0}^{n} 1
		\explnext{\footnotesize Usage de sommes télescopiques.}
	(n + 1)^2 - 0^2 - (n+1)
		\explnext{}
	(n + 1)((n + 1) - 1)
		\explnext{}
	n (n + 1)
\end{explain}

\bigskip

Donc
$\forall n \in \NN$ , $\dsum_{k = 0}^{n} k = \dfrac{n(n + 1)}{2}$ .
La preuve précédente est relativement élégante : vous pourrez comparer avec celle proposée dans la section \ref{power-2:rewriting} page \pageref{power-2:rewriting}.
Continuons pour obtenir une formule explicite de
$C_n \eqdef \dsum_{k = 0}^{n} k^2$
via $(k+1)^3 - k^3 = 3 k^2 + 3 k + 1$
et la formule précédente.

\medskip

\begin{explain}[style = sar]
	3 C_n
		\explnext{}
	\dsum_{k = 0}^{n} 3 k^2
		\explnext{}
	\dsum_{k = 0}^{n} ((k+1)^3 - k^3 - 3 k - 1)
		\explnext{}
	\dsum_{k = 0}^{n} ((k+1)^2 - k^2) - 3 \dsum_{k = 0}^{n} k - \dsum_{k = 0}^{n} 1
		\explnext{\footnotesize Usage de sommes télescopiques.}
	(n + 1)^3 - 0^3 - 3 \cdot \dfrac{n(n + 1)}{2} - (n+1)
		\explnext{}
	\dfrac{1}{2} \left[ \, 2 (n + 1)^3 - 3n (n + 1) - 2(n+1) \, \right]
		\explnext{}
	\dfrac{(n+1)}{2} \cdot (2 (n + 1)^2 - 3n - 2)
		\explnext{}
	\dfrac{(n+1)}{2} \cdot (2 (n^2 + 2 n + 1) - 3n - 2)
		\explnext{}
	\dfrac{(n+1)(2 n^2 + n)}{2}
		\explnext{}
	\dfrac{n(n+1)(2 n + 1)}{2}
\end{explain}


\medskip

Donc
$\forall n \in \NN$ , $\dsum_{k = 0}^{n} k^2 = \dfrac{n(n + 1)(2n + 1)}{6}$ .


\medskip

Finissons en montrant que
$\forall n \in \NN$ , $\dsum_{k = 0}^{n} k^3 = \dfrac{n^2 (n + 1)^2}{4} = \left( \, \dsum_{k = 0}^{n} k \, \right)^2$ . Un très joli résultat
\footnote{
	Une excellente preuve visuelle de ce résultat existe mais ceci est une autre histoire...
} !
Nous allons utiliser de $(k+1)^4 - k^4 = 4 k^3 + 6 k^2 + 4 k + 1$.
Notez que les calculs se compliquent vite et rendent la preuve de moins en moins élégante
\footnote{
	Un bon cadre d'étude pour des puissances plus élevées est celui utilisant les nombres de Bernoulli qui ont de jolies propriétés.
}.

\medskip

\begin{explain}[style = sar]
	4 \dsum_{k = 0}^{n} k^4
		\explnext{}
	\dsum_{k = 0}^{n} ((k+1)^4 - k^4 - 6 k^2 - 4 k - 1)
\end{explain}

\begin{explain}[style = sar]
	4 \dsum_{k = 0}^{n} k^4
		\explnext{}
	\dsum_{k = 0}^{n} ((k+1)^4 - k^4) - 6 \dsum_{k = 0}^{n} k^2 - 4 \dsum_{k = 0}^{n} k - \dsum_{k = 0}^{n} 1
		\explnext{}
	(n+1)^4 - 0^4 - 6 \cdot \dfrac{n(n + 1)(2n + 1)}{6} + 4 \cdot \dfrac{n(n + 1)}{2} - (n+1)
		\explnext{}
	(n+1) \, ((n+1)^3 - n(2n + 1) - 2n - 1)
		\explnext{}
	(n+1) \, ((n+1)^3 - 2n^2 - 3n - 1)
		\explnext*{\footnotesize $(-1)$ est une racine évidente de $2 X^2 + 3X + 1$.}{}
	(n+1) \, ((n+1)^3 - (n+1)(2n + 1))
		\explnext{}
	(n+1)^2 \, ((n+1)^2 - 2n - 1)
		\explnext{}
	n^2 (n+1)^2
\end{explain}







% ------------- %
%
%
%\chapter{Sommer les puissances de $q$}
%
%%\newpage
\section{[INFO] Comptons les feuilles d'arbres binaires complets} \label{power-q:p-ary-tree}

	\subsection{La preuve}

???


% ------------- %


\subsection{Commentaires}

???




\newpage
\section{[ALG] Jouons avec les écritures} \label{power-q:rewriting}

	\subsection{La preuve}

Les calculs suivants sont très simples à suivre mais malheureusement peu éclairants d'un point de vue conceptuel. Le cas où $n = 0$ étant évident, on suppose par la suite que $n \in \NNs$.

\medskip

\begin{explain}[style = sar, ope = \iff]
	S_n \eqdef \dsum_{k = 0}^{n} 2^k
		\explnext{}
	S_n = 1 + \dsum_{k = 1}^{n} 2^k
		\explnext{\footnotesize $k = i+1 \iff i = k-1$}
	S_n = 1 + \dsum_{i = 0}^{n-1} 2^{i + 1}
		\explnext{}
	S_n = 1 + 2 \dsum_{i = 0}^{n-1} 2^i
		\explnext*{\footnotesize Faisons apparaître la somme \og réduite\fg{} à gauche. }{}
	\dsum_{k = 0}^{n-1} 2^k + 2^n = 1 + 2 \dsum_{i = 0}^{n-1} 2^i
		\explnext{}
	S_{n-1} + 2^n = 1 + 2 S_{n-1}
		\explnext{}
	2^n - 1 = S_{n-1}
		\explnext{}
	S_{n-1} = 2^n - 1
\end{explain}

\bigskip

Donc $S_0 = 1$ et $\forall n \in \NNs$ , $S_{n-1} = 2^n - 1$ .
Ceci peut se réécrire :
$\forall n \in \NN$ , $S_n = 2^{n+1} - 1$ .


% ------------- %


\subsection{Commentaires}

La méthode précédente se généralise sans souci aux puissances de $q$ comme nous le verrons dans la section \ref{power-q:rewriting} page \pageref{power-q:rewriting}.
Par contre elle n'éclaire en rien sur la signification de la formule trouvée mais a l'avantage d'être prouvable via un ordinateur.


% ------------- %


\subsection{D'autres applications}

La réécriture de sommes peut permettre de trouver des sommes du type $\dsum_{k = 0}^{n} k^p$ .
Montrons par exemple comment trouver une formule explicite de la somme
\footnote{
	La lettre $G$ faire référence à Gauss à qui l'on attribue une méthode très astucieuse pour calculer cette somme en la réordonnant.
}
$G_n \eqdef \dsum_{k = 0}^{n} k$ où nous laissons de nouveau de côté le cas trivial où $n = 0$.
L'idée est de réécrire $C_n \eqdef \dsum_{k = 0}^{n} k^2$ et non $G_n$ car nous allons voir que la réécriture va éliminer les carrés
\footnote{
	Ce principe d'élimination se repère vite si l'on raisonne directement en réécrivant la somme cherchée $G_n$.
}.

\medskip

\begin{explain}[style = sar, ope = \iff]
	C_n \eqdef  \dsum_{k = 0}^{n} k^2
		\explnext{}
	C_n = 0 + \dsum_{k = 1}^{n} k^2
		\explnext{\footnotesize $k = i+1 \iff i = k-1$}
	C_n = \dsum_{i = 0}^{n-1} (i + 1)^2
		\explnext{}
	C_n = \dsum_{i = 0}^{n-1} (i^2 + 2 i + 1)
		\explnext{}
	C_n = \dsum_{i = 0}^{n-1} i^2 + 2 \dsum_{i = 0}^{n-1} i + \dsum_{i = 0}^{n-1} 1
		\explnext{}
	C_n = C_{n-1} + 2 G_{n-1} + n
		\explnext*{\footnotesize Apparition de la somme \og réduite\fg{} à gauche. }{}
	C_{n-1} + n^2 = C_{n-1} + 2 G_{n-1} + n
		\explnext{}
	n^2 = 2 G_{n-1} + n
\end{explain}


\begin{explain}[style = sar, ope = \iff]
	C_n \eqdef  \dsum_{k = 0}^{n} k^2
		\explnext{}
	n^2 - n = 2 G_{n-1}
		\explnext{}
	G_{n-1} = \dfrac{n(n - 1)}{2}
\end{explain}

\bigskip

Donc $G_0 = 0$ et $\forall n \in \NNs$ , $G_{n-1} = \dfrac{n(n - 1)}{2}$ .
Ceci se réécrit :
$\forall n \in \NN$ , $\dsum_{k = 0}^{n} k = \dfrac{n(n + 1)}{2}$ .

\medskip

On peut continuer de façon analogue pour obtenir une formule explicite de $C_n$ via la somme des cubes d'entiers successifs, puis ensuite on en aura une pour la somme des cubes elle-même mais les calculs deviennent vite pénibles
\footnote{
	En fait il existe une formulation générale faisant intervenir les nombres de Bernoulli qui ont de jolies propriétés.
}...
Voici la partie importante pour découvrir que 
$\forall n \in \NN$ , $\dsum_{k = 0}^{n} k^2 = \dfrac{n (n + 1) (2n + 1)}{6} = \dfrac{n^3}{3} + \dfrac{n^2}{2} + \dfrac{n}{6}$ .

\medskip

\begin{explain}[style = sar, ope = \iff]
	D_n \eqdef  \dsum_{k = 0}^{n} k^3
		\explnext{}
	D_n = \dsum_{i = 0}^{n-1} (i + 1)^3
		\explnext{}
	D_n = \dsum_{i = 0}^{n-1} (i^3 + 3 i^2 + 3 i + 1)
		\explnext{}
	D_n = D_{n-1} + 3 C_{n-1} + 3 G_{n-1} + n
		\explnext{}
	n^3 = 3 C_{n-1} + 3 G_{n-1} + n
		\explnext{}
	3 C_{n-1} = n^3 - 3 \cdot \dfrac{n(n - 1)}{2} - n
		\explnext{}
	6 C_{n-1} = 2 n^3 - 3n(n - 1) - 2n
		\explnext{}
	6 C_{n-1} = n (2 n^2 - 3n + 1)
		\explnext*{\footnotesize $1$ est une racine évidente de $2 X^2 - 3X + 1$.}{}
	6 C_{n-1} = n (n - 1) (2n - 1)
%		\explnext{}
%	C_{n-1} = \dfrac{n (n - 1) (2n - 1)}{6}
\end{explain}


\medskip


Finissons en montrant que
$\forall n \in \NN$ , $\dsum_{k = 0}^{n} k^3 = \dfrac{n^2 (n + 1)^2}{4} = \left( \, \dsum_{k = 0}^{n} k \, \right)^2$ . Que c'est joli !
Notez que les calculs se compliquent vite et rendent la preuve très inélégante.

\medskip

\begin{explain}[style = sar, ope = \iff]
	E_n \eqdef  \dsum_{k = 0}^{n} k^4
		\explnext{}
	E_n = \dsum_{i = 0}^{n-1} (i^4 + 4 i^3 + 6 i^2 + 4 i + 1)
		\explnext{}
	n^4 = 4 D_{n-1} + 6 C_{n-1} + 4 G_{n-1} + n
		\explnext{}
	4 D_{n-1} = n^4 - n (n - 1) (2n - 1) - 2n(n - 1) - n
		\explnext{}
	4 D_{n-1} = n \cdot \big[ \, \textcolor{blue}{n^3} - (n - 1) (2n - 1) - 2(n - 1) - \textcolor{blue}{1} \, \big]
		\explnext*{%
			\tiny%
			\begin{explain}[style = ar]
				n^3 - 1
					\explnext{}	
				(n - 1)(n^2 + n + 1)
			\end{explain}%
		}{}
	4 D_{n-1} = n (n - 1) \cdot \big[ \, \textcolor{blue}{n^2 + n + 1} - (2n - 1) - 2 \, \big]
		\explnext{}
	4 D_{n-1} = n (n - 1) (n^2 - n)
		\explnext{}
	4 D_{n-1} = n^2 (n - 1)^2
\end{explain}





%\newpage
\section{[ALG] Sommes télescopiques} \label{power-q:telescopic}

	\subsection{La preuve}

Pour trouver une formule explicite de $S_n \eqdef \dsum_{k = 0}^{n} 2^k$, on peut noter que
$2^k = 2 \cdot 2^{k-1} = 2^{k-1} + 2^{k-1}$
donne
$2^{k-1} = 2^k - 2^{k-1}$
d'où l'on déduit que
$\forall k \in \NN$, $2^k = 2^{k+1} - 2^k$.
Ceci nous conduit aux calculs suivants.

\medskip

\begin{explain}[style = sar, ope = \iff]
	S_n \eqdef \dsum_{k = 0}^{n} 2^k
		\explnext{}
	S_n = \dsum_{k = 0}^{n} (2^{k+1} - 2^k)
		\explnext{}
	S_n = \dsum_{k = 0}^{n} 2^{k+1} - \dsum_{k = 0}^{n} 2^k
		\explnext{\footnotesize $i = k+1 \iff k = i-1$}
	S_n = \dsum_{i = 1}^{n+1} 2^{i} - \dsum_{k = 0}^{n} 2^k
		\explnext*{\footnotesize On fait apparaître des sommes identiques.}{}
	S_n = \dsum_{i = 1}^{n} 2^{i} + 2^{n+1} - 2^0 - \dsum_{k = 1}^{n} 2^k
		\explnext{}
	S_n = 2^{n+1} - 1
\end{explain}


% ------------- %


\subsection{Commentaires}

Les simplifications du type
$\dsum_{k = 0}^{n} (2^{k+1} - 2^k) = 2^{n+1} - 2^0$ ,
ou plus généralement du type
$\dsum_{k = 0}^{n} (u_{k+1} - u_{k}) = u_{n+1} - u_0$
ou
$\dsum_{k = 0}^{n} (u_{k} - u_{k+1}) = u_0 - u_{n+1}$ ,
sont un grand classique : on parle de \og sommes télescopiques \fg{}
\footnote{
	Cette technique permet par exemple de ramener l'étude d'une suite à celle d'une série.
	Or il se trouve que l'on dispose d'outils très pratiques pour étudier les séries.
}.
Cette astuce se généralise sans souci aux puissances de $q$ comme nous le verrons dans la section \ref{power-q:telescopic} page \pageref{power-q:telescopic}.
Bien qu'élégants du point de vue algébrique, les calculs ci-dessus ne donnent aucune information sur la signification de la formule trouvée.


% ------------- %


\subsection{D'autres applications}

Une application rigolote est l'obtention d'une formule explicite de la somme $I_n \eqdef \dsum_{k = 1}^{n} \dfrac{1}{k(k+1)}$. 
Une fois noté que $\dfrac{1}{k(k+1)} = \dfrac{1}{k} -  \dfrac{1}{k+1}$
le calcul est très aisé
\footnote{
	Cet exemple est conçu comme un cas typique d'usage de sommes télescopiques.
}.

\medskip

\begin{explain}[style = sar]
	I_n \explnext[\eqdef]{}
	\dsum_{k = 1}^{n} \dfrac{1}{k(k+1)}
		\explnext{}
	\dsum_{k = 1}^{n} \left( \dfrac{1}{k} - \dfrac{1}{k+1} \right)
		\explnext{\footnotesize Usage de sommes télescopiques.}
	\dfrac{1}{1} - \dfrac{1}{n+1}
		\explnext{}
	\dfrac{n}{n+1}
\end{explain}


\bigskip

On peut aussi utiliser des sommes télescopiques pour expliciter $\dsum_{k = 0}^{n} k^p$ .
Montrons par exemple comment trouver une formule explicite de la somme
\footnote{
	La lettre $G$ faire référence à Gauss à qui l'on attribue une méthode très astucieuse pour calculer cette somme en la réordonnant.
}
suivante $G_n \eqdef \dsum_{k = 0}^{n} k$.
L'idée astucieuse consiste à noter que $(k+1)^2 - k^2 = 2 k + 1$ puis à procéder comme suit.

\medskip

\begin{explain}[style = sar]
	2 G_n
		\explnext{}
	\dsum_{k = 0}^{n} 2 k
		\explnext{}
	\dsum_{k = 0}^{n} ((k+1)^2 - k^2 - 1)
		\explnext{}
	\dsum_{k = 0}^{n} ((k+1)^2 - k^2) - \dsum_{k = 0}^{n} 1
		\explnext{\footnotesize Usage de sommes télescopiques.}
	(n + 1)^2 - 0^2 - (n+1)
		\explnext{}
	(n + 1)((n + 1) - 1)
		\explnext{}
	n (n + 1)
\end{explain}

\bigskip

Donc
$\forall n \in \NN$ , $\dsum_{k = 0}^{n} k = \dfrac{n(n + 1)}{2}$ .
La preuve précédente est relativement élégante : vous pourrez comparer avec celle proposée dans la section \ref{power-2:rewriting} page \pageref{power-2:rewriting}.
Continuons pour obtenir une formule explicite de
$C_n \eqdef \dsum_{k = 0}^{n} k^2$
via $(k+1)^3 - k^3 = 3 k^2 + 3 k + 1$
et la formule précédente.

\medskip

\begin{explain}[style = sar]
	3 C_n
		\explnext{}
	\dsum_{k = 0}^{n} 3 k^2
		\explnext{}
	\dsum_{k = 0}^{n} ((k+1)^3 - k^3 - 3 k - 1)
		\explnext{}
	\dsum_{k = 0}^{n} ((k+1)^2 - k^2) - 3 \dsum_{k = 0}^{n} k - \dsum_{k = 0}^{n} 1
		\explnext{\footnotesize Usage de sommes télescopiques.}
	(n + 1)^3 - 0^3 - 3 \cdot \dfrac{n(n + 1)}{2} - (n+1)
		\explnext{}
	\dfrac{1}{2} \left[ \, 2 (n + 1)^3 - 3n (n + 1) - 2(n+1) \, \right]
		\explnext{}
	\dfrac{(n+1)}{2} \cdot (2 (n + 1)^2 - 3n - 2)
		\explnext{}
	\dfrac{(n+1)}{2} \cdot (2 (n^2 + 2 n + 1) - 3n - 2)
		\explnext{}
	\dfrac{(n+1)(2 n^2 + n)}{2}
		\explnext{}
	\dfrac{n(n+1)(2 n + 1)}{2}
\end{explain}


\medskip

Donc
$\forall n \in \NN$ , $\dsum_{k = 0}^{n} k^2 = \dfrac{n(n + 1)(2n + 1)}{6}$ .


\medskip

Finissons en montrant que
$\forall n \in \NN$ , $\dsum_{k = 0}^{n} k^3 = \dfrac{n^2 (n + 1)^2}{4} = \left( \, \dsum_{k = 0}^{n} k \, \right)^2$ . Un très joli résultat
\footnote{
	Une excellente preuve visuelle de ce résultat existe mais ceci est une autre histoire...
} !
Nous allons utiliser de $(k+1)^4 - k^4 = 4 k^3 + 6 k^2 + 4 k + 1$.
Notez que les calculs se compliquent vite et rendent la preuve de moins en moins élégante
\footnote{
	Un bon cadre d'étude pour des puissances plus élevées est celui utilisant les nombres de Bernoulli qui ont de jolies propriétés.
}.

\medskip

\begin{explain}[style = sar]
	4 \dsum_{k = 0}^{n} k^4
		\explnext{}
	\dsum_{k = 0}^{n} ((k+1)^4 - k^4 - 6 k^2 - 4 k - 1)
\end{explain}

\begin{explain}[style = sar]
	4 \dsum_{k = 0}^{n} k^4
		\explnext{}
	\dsum_{k = 0}^{n} ((k+1)^4 - k^4) - 6 \dsum_{k = 0}^{n} k^2 - 4 \dsum_{k = 0}^{n} k - \dsum_{k = 0}^{n} 1
		\explnext{}
	(n+1)^4 - 0^4 - 6 \cdot \dfrac{n(n + 1)(2n + 1)}{6} + 4 \cdot \dfrac{n(n + 1)}{2} - (n+1)
		\explnext{}
	(n+1) \, ((n+1)^3 - n(2n + 1) - 2n - 1)
		\explnext{}
	(n+1) \, ((n+1)^3 - 2n^2 - 3n - 1)
		\explnext*{\footnotesize $(-1)$ est une racine évidente de $2 X^2 + 3X + 1$.}{}
	(n+1) \, ((n+1)^3 - (n+1)(2n + 1))
		\explnext{}
	(n+1)^2 \, ((n+1)^2 - 2n - 1)
		\explnext{}
	n^2 (n+1)^2
\end{explain}














\bigskip

\hrule

\section{AFFAIRE À SUIVRE...}

\bigskip

\hrule

\end{document}
