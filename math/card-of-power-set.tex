\documentclass[12pt]{amsart}
\usepackage[T1]{fontenc}
\usepackage[utf8]{inputenc}

\usepackage[top=1.95cm, bottom=1.95cm, left=2.35cm, right=2.35cm]{geometry}

\usepackage{import} 
\usepackage{wrapfig}



\usepackage{hyperref}
\usepackage{enumitem}
\usepackage{tcolorbox}
\usepackage{multicol}
\usepackage{fancyvrb}
\usepackage{amsmath}
\usepackage[french]{babel}
\usepackage[
    type={CC},
    modifier={by-nc-sa},
	version={4.0},
]{doclicense}
\usepackage{textcomp}
\usepackage{tcolorbox}
\usepackage{tnsmath}
    
\newtheorem{fact}{Fait}%[section]

\newtheorem*{theorem}{Théorème}

\newtheorem{example}{Exemple}[section]

\newtheorem{remark}{Remarque}[section]

\newtheorem*{proof*}{Preuve}

\usepackage{chemist}


\setlength\parindent{0pt}


\DeclareMathOperator{\taille}{\text{\normalfont\texttt{taille}}}

\newcommand\sqseq[2]{\fbox{$#1$}_{\,\,#2}}

\newcommand\floor[1]{\left\lfloor #1 \right\rfloorx}


\DefineVerbatimEnvironment{rawcode}%
	{Verbatim}%
	{tabsize=4,%
	 frame=lines, framerule=0.3mm, framesep=2.5mm}


\let\oldparagraph\paragraph
\renewcommand\paragraph[1]{\bigskip\oldparagraph{{\bfseries #1}}}
 
	 
\let\oldsection\section
\renewcommand\section[1]{\vfill\pagebreak\oldsection{#1}}
 

\newcommand\deltaval[3]{%
	\Delta{#1}^{\,#2}_{\,#3}%
}
	 
\begin{document}

\title{Nombre de sous-ensembles d'un ensemble fini\\Une preuve très élémentaire.}
\author{Christophe BAL}
\date{19 Octobre 2020}


\maketitle

\begin{center}
	\itshape
	Document, avec son source \LaTeX, disponible sur la page
	
	\url{https://github.com/bc-writing/drafts}.
\end{center}


\bigskip


\begin{center}
	\hrule\vspace{.3em}
	{
		\fontsize{1.35em}{1em}\selectfont
		\textbf{Mentions \og légales \fg}
	}
			
	\vspace{0.45em}
	\doclicenseThis
	\hrule
\end{center}


\bigskip


% ----------- %

Voici comment expliquer rapidement à un lycéen
\footnote{
	Bien entendu on adaptera le formalisme pour rendre digeste les principes élémentaires utilisés.
}
qu'un ensemble $\setgeo{E}$ de $n$ éléments permet de fabriquer $2^n$ sous-ensembles ayant de $0$ à $n$ éléments. Sans entrer dans le détail d'une récurrence, voici comment faire où si besoin le cas 1 peut être omis.

\begin{enumerate}
	\medskip
	\item $\emptyset$ est le seul ensemble à $0$ élément. Son seul sous-ensemble est $\emptyset$. On a bien $1 = 2^0$.


	\medskip
	\item Soit $\setgeo{E} = \setgene{\spadesuit}$ un ensemble avec un seul élément. Cet ensemble contient $2$ sous-ensembles, à savoir $\emptyset$ et $\setgene{\spadesuit}$. On a bien $2 = 2^1$.


	\medskip
	\item Soit $\setgeo{E} = \setgene{\spadesuit_1 ; \spadesuit_2}$ un ensemble avec deux éléments.
	      Cet ensemble contient les $4 = 2^2$ sous-ensembles
	      $\emptyset$ , $\setgene{\spadesuit_1}$ ,
	      $\setgene{\spadesuit_2}$ et $\setgene{\spadesuit_1 ; \spadesuit_2}$.
	
	      \noindent
	      Le pont clé est de noter que
	      $\setgene{\spadesuit_2} = \emptyset \cup \setgene{\spadesuit_2}$ et
	      $\setgene{\spadesuit_1 ; \spadesuit_2} = \setgene{\spadesuit_1} \cup \setgene{\spadesuit_2}$.

	\medskip
	\item Soit maintenant $\setgeo{E} = \setgene{\spadesuit_1 ; \dots ; \spadesuit_{n+1}}$ un ensemble avec $(n+1)$ éléments.
		  Les sous-ensembles de $\setgeo{E}$ se classent en deux catégories.
		  
		  \begin{enumerate}
		  		\smallskip
				\item \textbf{Catégorie 1 :} les sous-ensembles ne contenant pas $\spadesuit_{n+1}$.
				      Il y en a autant que de sous-ensembles de $\setgeo{F} = \setgene{\spadesuit_1 ; \dots ; \spadesuit_{n}}$.
		  		
				\smallskip
				\item \textbf{Catégorie 2 :} les sous-ensembles contenant $\spadesuit_{n+1}$.
				      De tels sous-ensembles s'obtiennent à partir de sous-ensembles de $\setgeo{F}$ en leur adjoignant $\spadesuit_{n+1}$.
				      Ceci démontre qu'il y a autant de sous-ensembles de catégorie 2 que de sous-ensembles de $\setgeo{F}$.
		  \end{enumerate}

		  \noindent
		  Finalement le nombre de sous-ensembles de $\setgeo{E}$ est deux fois plus grand que celui de $\setgeo{F}$.


	\medskip
	\item Finalement de proche en proche, ou plus rigoureusement via une récurrence, nous avons sans effort qu'un ensemble $\setgeo{E}$ de $n$ éléments permet de fabriquer $2^n$ sous-ensembles ayant de $0$ à $n$ éléments.
\end{enumerate}

\end{document}
