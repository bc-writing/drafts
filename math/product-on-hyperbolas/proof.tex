Rappelons que $\geoset{H} : y = \frac{1}{x}$ , $E(1 ; 1)$ , $A \left( a ; \frac{1}{a} \right)$ , $B \left( b ; \frac{1}{b} \right)$ et $P \left( p ; \frac{1}{p} \right)$ où $p = a b$ avec $(a ; b) \in \left( \RRs \right)^2$ .



\bigskip

\textbf{Cas 1.} \emph{Supposons que $x_A x_B = 1$ .}

\medskip

Il est clair que $P = E$ dans ce cas.



\bigskip

\textbf{Cas 2.} \emph{Supposons que $x_A x_B \neq 1$ et $A \neq B$ .}

\medskip

La droite $(AB)$ a pour pente
$\frac{y_A - y_B}{x_A - x_B} = \frac{1}{a - b} \left( \frac{1}{a} - \frac{1}{b} \right) = - \frac{1}{ab}$ .
De plus, la droite $(EP)$ , qui existe car $p \neq 1$ , a pour pente
$\frac{y_P - y_E}{x_P - x_E} = \frac{1}{p - 1} \left( \frac{1}{p} - 1 \right) = - \frac{1}{p} = - \frac{1}{ab}$ .
Les droites $(AB)$ et $(EP)$ sont bien parallèles comme nous l'avons affirmé.



\bigskip

\textbf{Cas 3.} \emph{Supposons que $x_A x_B \neq 1$ et $A = B$ .}

\medskip

Comme ici $p = a^2 \neq 1$ . la droite $(EP)$ a pour pente $\left( - \frac{1}{a^2} \right)$ qui est aussi la pente de la tangente en $A \left( a ; \frac{1}{a} \right)$ à l'hyperbole $\geoset{H}$ comme annoncé. 