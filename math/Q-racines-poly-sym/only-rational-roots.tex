Supposons que $r \in \QQ - \NN$ soit une racine de $P$.

\medskip

Le résultat sur la multiplicité supérieure ou égale à $2$ nous donne que si $r$ est de multiplicité au moins $2$ alors $\dfrac1r \neq r$ est aussi de multiplicité au moins $2$.
Ceci implique que $r$ est de multiplicité $1$ ou $2$.



\subsection*{$r$ est de multiplicité $1$}

Si $P$ admet une autre racine $s \in \QQ - \NN$ avec $s \neq r$ et $s \neq \dfrac1r$ alors nécessairement $P(X) = (X - r) \left( X - \dfrac1r \right) (X - s) \left( X - \dfrac1s \right)$.


D'où

$X^4 P\left( \dfrac1X \right)
= X^4 
  \left( \dfrac1X - r \right) \left( \dfrac1X - \dfrac1r \right) 
  \left( \dfrac1X - s \right) \left( \dfrac1X - \dfrac1s \right)$

$X^4 P\left( \dfrac1X \right)
= ( 1 - r X) \left( 1 - \dfrac{X}{r} \right) 
  ( 1 - s X) \left( 1 - \dfrac{X}{s} \right)$

$X^4 P\left( \dfrac1X \right)
= r \left( \dfrac1r - X \right) \times \dfrac1r ( r - X) 
  \times s \left( \dfrac1s - X \right) \times \dfrac1s ( s - X)$

$X^4 P\left( \dfrac1X \right)
= P(X)$

\medskip

Donc $P$ polynôme est symétrique mais il reste à étudier les cas suivants.

\begin{enumerate}
	\item $P(X) = (X - r) \left( X - \dfrac1r \right) (X + 1)^2$ et $P(X) = (X - r) \left( X - \dfrac1r \right) (X - 1)^2$ sont symétriques.
	Il suffit de reprendre le calcul précédent avec $s = \pm 1$.
	

	\item $P(X) = (X - r) \left( X - \dfrac1r \right) (X + 1) (X - 1)$ vérifie $P(X) = - X^4 P\left( \dfrac1X \right)$ donc $P$ est anti-symétrique.
\end{enumerate}


\subsection*{$r$ est de multiplicité $2$}

Dans ce cas, $P(X) = (X - r)^2 \left( X - \dfrac1r \right)^2$ nécessairement !
Il est immédiat que $P(X) = X^4 P\left( \dfrac1X \right)$ donc ce polynôme est symétrique.



