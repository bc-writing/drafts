Si $P$ n'a que des racines entières alors ces dernières ne peuvent être que $\pm 1$ qui sont les seuls entiers ayant un inverse entier. Ceci donne les uniques  possibilités suivantes :

\begin{enumerate}
	\item Pour $P(X) = (X + 1)^4$, nous avons
	      $X^4 \left( \dfrac1X + 1\right)^4 = (1 + X)^4$ 
	      d'où
	      $P(X) = X^4 P\left( \dfrac1X \right)$
	      donc $P$ est bien symétrique
	      \footnote{
	      	Ici nous aurions pu aussi développer $(X + 1)^4$ via le triangle de Pascal.
		  }.

	\item $P(X) = (X - 1)^4$ est aussi symétrique \emph{(on procède comme ci-dessus et en utilisant la parité de l'exposant)}.
	
	\item Pour $P(X) = (X - 1)^3 (X + 1)$, nous avons
	      $X^4 \left( \dfrac1X - 1\right)^3 \left( \dfrac1X + 1\right)
	      = (1 - X)^3 (1 + X)$
	      d'où
	      $P(X) = - X^4 P\left( \dfrac1X \right)$.
	      Le polynôme n'est pas symétrique.

	      \noindent En fait $P(X) = (X - 1)^3 (X + 1) = X^4 - 2 X^3 + 2 X - 1$ est anti-symétrique
	      \footnote{
	      	Le développement de $(X - 1)^3 (X + 1)$ a été obtenu sans effort via le service en ligne \url{https://www.wolframalpha.com}.
		  }. 
	      Un polynôme de degré $4$ est anti-symétrique si et seulement si  $P(X) = - X^4 P\left( \dfrac1X \right)$.
	
	\item $P(X) = (X + 1)^3 (X - 1)$ est anti-symétrique \emph{(comme ci-dessus)}.

	\item $P(X) = (X + 1)^2 (X - 1)^2$ est symétrique car il vérifie $P(X) = X^4 P\left( \dfrac1X \right)$.
\end{enumerate}


