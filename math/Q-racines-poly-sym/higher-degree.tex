\subsection*{Cas d'un polynôme symétrique de degré $n \geqslant 2$}

Soit $P$ un polynôme symétrique de degré $n \geqslant 2$ et de coefficient dominant $1$. On sait donc que $P(X) = X^n P\left( \dfrac1X \right)$. Notons que cette propriété est stable par multiplication. 

\medskip

De nouveau si $P(r) = 0$ alors $r \neq 0$ et $P\left( \dfrac1r \right) = 0$.
De plus, une récurrence facile nous donne :

\medskip

$\displaystyle
  P\,^{(k)}(X) 
=
  \epsilon_k X^{n - 2k} P\,^{(k)}\left( \dfrac1X \right)
+
  \sum_{i = n - 2k + 1}^{n - k} c_i X^i P\,^{(i)}\left( \dfrac1X \right)$
où $\epsilon_k = \pm 1$.


\medskip

On en déduit plus précisément que $r$ et $\dfrac1r$ ont la même multiplicité \emph{(de nouveau à l'aide d'une récurrence facile)}.


\medskip

On a donc
$\displaystyle P(X) = (X + 1)^a (X - 1)^b \prod_{fini} (X - r_i)^{k_i} \left( X - \dfrac1{r_i} \right)^{k_i}$ avec d'éventuels $r_i \in \QQ - \NN$, les exposants naturels pouvant être nuls.


\medskip

Dans ce produit, $(X + 1)$ et $(X - r_i) \left( X - \dfrac1{r_i} \right)$ sont symétriques tandis que $(X - 1)^b$ peut être anti-symétrique.
On a donc juste la contrainte $b \in 2 \NN$. Que c'est beau !


\subsection*{Cas d'un polynôme anti-symétrique de degré $n \geqslant 2$}

La démarche est similaire et on arrive à des polynômes du type
$\displaystyle P(X) = (X + 1)^a (X - 1)^b \prod_{fini} (X - r_i)^{k_i} \left( X - \dfrac1{r_i} \right)^{k_i}$
avec $b \in 2 \NN + 1$, les autres exposants pouvant être nuls avec d'éventuels $r_i \in \QQ - \NN$.



