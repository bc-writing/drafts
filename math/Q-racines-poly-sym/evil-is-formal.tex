Par flemme, l'auteur avait commencé par raisonner avec un logiciel de calcul formel, les neurones presque déconnectés. Voici ce qui avait été fait \emph{(l'idée était de commencer à étudier les cas avec le moins d'inconnues)}.

\begin{enumerate}
	\item On suppose que $P$ admet une seule racine $r$.
	Nous n'avons qu'une seule possibilité à savoir
	$P(X) = (X - r)^4
	      = X^4
	      - 4 r X^3
	      + 6 r^2 X^2
	      - 4 r^3 X
	      + r^4$
	qui est symétrique si et seulement si $r^4 = 1$ et $r^3 = r$.
	       
	       \noindent Si $r\in \QQ$ alors nécessairement $r = \pm 1$ et on tombe sur les polynômes symétriques $(X - 1)^4$ et $(X + 1)^4$. Pour vérifier que ces polynômes sont bien symétriques, il suffit de penser au triangle de Pascal. 


	\item On suppose que $P$ admet deux racines $r$ et $s$.
	Nous avons deux possibilités.
	\begin{enumerate}
		\item
		$P(X) = (X - r)^3 (X - s)
		      = X^4
		      - (3 r + s) X^3
		      + (3 r^2 + 3 r s) X^2
		      - (r^3 + 3 r^2 s) X
		      + r^3 s$
		est symétrique si et seulement si $r^3 s = 1$ et $3 r + s = r^3 + 3 r^2 s$.
	       
		\noindent
		On en déduit que $3 r^4 + 1 = r^6 + 3 r^2$
		d'où $T^3 - 3 T^2 + 3 T - 1 = 0$ i.e. $(T - 1)^3 = 1$
		en posant $T = r^2$.
		On a alors $r = \pm 1$ mais dans ce cas $s = r$ !
		Nous avons une contradiction. 


		\item
		$P(X) = (X - r)^2 (X - s)^2
		      = X^4
		      - (2 r  + 2 s) X^3
		      + (r^2 + 4 r s + s^2) X^2
		      - (2 r^2 s + 2 r s^2) X
		      + r^2 s^2$
		est symétrique si et seulement si
		$r^2 s^2 = 1$ et $r + s = r^2 s + r s^2$
		i.e. $r s = \pm 1$ et $r + s = rs(r + s)$. Nous avons deux sous-cas.
		\begin{enumerate}
			\item $r s = 1$ donne $r  + s = r + s$ et surtout $s = \dfrac1r$.
			On tombe sur le polynôme symétrique $(X - r)^2 \left( X - \dfrac1r \right)^2$.
			Ceci avait été vérifié sans effort, ni neurone, via un logiciel de calcul formel :
	 		$ X^4
			+ \left( \dfrac2r - 2 r \right) X^2
			+ \left( \dfrac{1}{r^2} + r^2 + 4 \right) X^2
			+ \left( \dfrac2r - 2 r \right) X
			+ 1$.


			\item $r s = -1$ donne $r  + s = 0$ i.e. $s = - r$ d'où $r = \pm 1$.
			Ceci nous donne le polynôme symétrique
			$ (X - 1)^2 ( X + 1)^2
			= (X^2 - 1)^2
			= X^4 - 2 X ^2 + 1$.
		\end{enumerate}
	\end{enumerate}

	
	\item Soyons fort et continuons en supposant que $P$ admet trois racines $r$, $s$ et $t$ i.e. que $P(X) = (X - r)^2 (X - s) (X - t)$.
	Le calcul formal nous donne :
	
	\noindent
	$P(X) = X^4
	      - (s + t + 2 r) X^3
	      + (s t + r^2 + 2 r s + 2 r t) X^2
	      - (r^2 s + r^2 t + 2 r s t) X 
	      + r^2 s t$
	       	
	\noindent
	$P$ est symétrique si et seulement si
	$r^2 s t = 1$ et $s + t + 2 r = r^2 s + r^2 t + 2 r s t$.
	Deux équations et trois inconnues\dots{} Que faire ? Considérons des sous-cas en activant quelques neurones
	\footnote{
		Mais pas tous car l'auteur bien qu'ayant l'intuition \og sûre \fg{} des multiplicités égales de $r$ et son inverse n'avait pas le recul nécessaire pour le démontrer !
	}
	pour noter que
	$a r^4 + b r^3 + c r^2 + b r + a = 0$
	si et seulement si
	$\dfrac{a r^4 + b r^3 + c r^2 + b r + a}{r^4} = 0$
	c'est à dire si et seulement si
	$a + b \dfrac1r + c \left( \dfrac1r \right)^2 + b \left( \dfrac1r \right)^3 + a \left( \dfrac1r \right)^4 = 0$
	\begin{enumerate}
		\item On suppose que ni $s$, ni $t$ n'est l'inverse de $r$. Dans ce cas nécessairement $r = \dfrac1r$ i.e. $r = \pm 1$.
		Nos équations deviennent
		$s t = 1$ et $s + t = s + t$.
		
		\noindent
		$P(X) = (X \pm 1)^2 (X - s) \left( X - \dfrac1s \right)$
		est symétrique puisque son développement est
		$ X^4
		+ \left( \dfrac1s - s \pm 2 \right) X^3
		+ \left( - \dfrac2s + 2 \pm 2 s \right) X^2
		+ \left( \dfrac1s - s \pm 2 \right) X
		+ 1$
		où tous les $\pm$ \og correspondent \fg{} au $\pm$ de la forme factorisée.	 


		\item On suppose maintenant que soit $s$, soit $t$ est l'inverse de $r$. Supposons par exemple que $s = \dfrac1r$.
		Dès lors $r^2 s t = 1$ implique que $\dfrac{t}{s} = 1$ !
		Nous avons une contradiction. 
	\end{enumerate}

	
	\item Persévérance ou acharnement ? Finissons avec le cas où $P$ admet quatre racines $r$, $s$, $t$ et $u$ c'est à dire $P(X) = (X - r) (X - s) (X - t) (X - u)$.
	
	\noindent
	On peut supposer que $s = \dfrac1r \neq \pm1$, et comme $u \neq r$ et $u \neq s$, nous avons forcément $u = \dfrac1t \neq \pm1$
	d'où
	$P(X) = (X - r) \left( X - \dfrac1r \right) (X - t) \left( X - \dfrac1t \right)$
	qui est symétrique comme le prouve le développement suivant :
		
	\noindent
	$ X^4
	- \left( \dfrac1r + r + \dfrac1t + t \right) X^3
	+ \left( r t + \dfrac{t}{r} + \dfrac{r}{t} + \dfrac{1}{rt} + 2 \right) X^2
	- \left( \dfrac1r + r + \dfrac1t + t \right) X 
	+ 1$
\end{enumerate}

Dans tout ce qui précède, on a très peu de recul sur ce que l'on fait. C'est moche !
De plus, passer au cas général devient juste ingérable\dots{}
Mais tout n'est pas si sombre puisque les deux derniers points montrent que l'usage du calcul formel est des plus abusifs ici tout en faisant apparaître l'efficacité de travailler avec une racine et son inverse.
La suite de l'histoire est au début de ce document.