\RequirePackage{snapshot}
\documentclass[12pt]{amsart}
\usepackage[T1]{fontenc}
\usepackage[utf8]{inputenc}

\usepackage[top=1.95cm, bottom=1.95cm, left=2.35cm, right=2.35cm]{geometry}

\usepackage{amsmath}
\usepackage{amssymb}
\usepackage{enumitem}
\usepackage{multicol}
\usepackage[french]{babel}
\usepackage[
    type={CC},
    modifier={by-nc-sa},
	version={4.0},
]{doclicense}

\usepackage{tnsmath}

\DeclareMathOperator{\taille}{\tau}

\newtheorem{fact}{Fait}
\newtheorem*{proof*}{Preuve}

\setlength\parindent{0pt}


\newcommand\squote[1]{\og #1 \fg{}}


\newcommand\RRinf{{\RR}_{\infty}}



\begin{document}

\title{BROUILLON - Suites homographiques sans mysticisme}
\author{Christophe BAL}
\date{?? Novembre 2020}
\maketitle


\vspace{-.9em}


\begin{center}
	\hrule\vspace{.3em}
	{
		\fontsize{1.35em}{1em}\selectfont
		\textbf{Mentions \og légales \fg}
	}
			
	\vspace{0.45em}
	\doclicenseThis
	\hrule
\end{center}



\setcounter{tocdepth}{2}
\tableofcontents



%\section{\texorpdfstring{Comment additionner des nombres grâce à la parabole d'équation $y = x^2$}%
%                        {Comment additionner des nombres grâce à la parabole d'équation y = x**2}}

\section{Fractions homographiques et matrices}

	Soit
$F(X) = \dfrac{a X + b}{c X + d}$
et
$G(X) = \dfrac{p X + q}{r X + s}$
non constantes c'est à dire telles que
$ad - bc \neq 0$
et
$ps - rq \neq 0$ .


Il est immédiat qu'il existe des paramètres $\alpha$, $\beta$, $\gamma$ et $\delta$ tels que
$F \compo G(X) = \dfrac{\alpha X + \beta}{\gamma X + \delta}$ .
Existe-t-il un moyen simple de calculer les paramètres de $F \compo G$ en fonction de ceux de $F$ et $G$ ?


\bigskip


\begin{explain}[style = sar]
	F \compo G(X) \,
		\explnext{}
	\dfrac{a(p X + q) + b(r X + s)}{c(p X + q) + d(r X + s)}
		\explnext{}
	\dfrac{(a p + b r) X + a q + bD}{(c p + d r) X + c q + d s}
\end{explain}


\medskip


En associant
$\begin{pmatrix}
    a & b \\
    c & d
\end{pmatrix}$ à $F(X)$ ,
$\begin{pmatrix}
    p & q \\
    r & s
\end{pmatrix}$ à $G(X)$
et
$\begin{pmatrix}
    \alpha & \beta  \\
    \gamma & \delta
\end{pmatrix}$ à $F \compo G(X)$ ,
nous venons de démontrer que
$\begin{pmatrix}
    \alpha & \beta  \\
    \gamma & \delta
\end{pmatrix}
=
\begin{pmatrix}
    a & b \\
    c & d
\end{pmatrix}
\begin{pmatrix}
    p & q \\
    r & s
\end{pmatrix}$ .


\medskip


Comme $\forall \lambda \in \CCs$ ,
$\dfrac{a X + b}{c X + d} = \dfrac{\lambda a X + \lambda b}{\lambda c X + \lambda d}$ , on va considérer l'écriture de $F$ telle que $ad- bc = 1$ et lui associer la matrice
$M_F \eqdef \begin{pmatrix}
    a & b \\
    c & d
\end{pmatrix}$ .
Par choix $\det M_F = 1$ , de sorte que l'on a un isomor\-phisme de groupes entre les fonctions homographiques
$F(X) = \dfrac{a X + b}{c X + d}$ avec $ad- bc = 1$
et le groupe $SL(2 , \CC)$ des matrices $2 \times 2$ de déterminant $1$ .




%\section{Compositions successives}
%
%	Soit $F(X) = \dfrac{a X + b}{c X + d}$ avec $ad- bc = 1$ .
Que peut-on dire de $\multicompo{F}{n} = \multicompo[dot]{F}{n}$ pour $n \in \NNs$ ?
%
%


%\section{Fonctions homographiques étendues}   OK pour définition de la suite ????
%
%	On note $\RRinf = \RR \cup \setgene{\infty}$ où $\infty$ est un symbole abstrait autre qu'un réel.
À une fraction homographique $F(X) = \dfrac{a X + b}{c X + d}$ avec $ad- bc = 1$ , on associe
$\funcdef[h]{f}{x}     {\dfrac{a x + b}{c x + d}}%
               {\RRinf}{\RRinf}$
de façon naturelle en traitant les cas particuliers comme suit.

\begin{enumerate}
	\item Si $c \neq 0$ alors $f\left(- \dfrac{d}{c} \right) \eqdef \infty$ .
	      Notons que si $a \neq 0$ , il n'est pas possible d'annuler $ax + b$ et $cx + b$ simultanément car $\dfrac{b}{a} \neq \dfrac{d}{c}$ .

	\item Si $c \neq 0$ alors $f(\infty) \eqdef \dfrac{a}{c}$ .

	\item Si $c = 0$ , de sorte que $a \neq 0$ , alors $f(\infty) \eqdef \infty$ .
\end{enumerate}




\bigskip

\hrule

\section{AFFAIRE À SUIVRE...}

\bigskip

\hrule


\end{document}
