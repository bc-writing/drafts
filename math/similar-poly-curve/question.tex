Il est connu ques les courbes des fonctions affines sont toutes des droites, et celles représentant des trinômes du 2\ieme{} degré sont toutes des paraboles. Quand on présente ce résultat au lycée, on n'a pas défini exactement ce qu'est une parabole
\footnote{
	La définition géométrique des grecques anciens restent la meilleure.
}.
On explique que l 'on peut passer de la représentation de la fonction carrée
$\funcdef[h]{f}{x}{x^2}{}{}$
à celle du trinôme
$\funcdef[h]{g}{x}{a \, x^2 + b \, x + c}{}{}$
via une translation, une dilatation verticales et/ou une dilatation horizontale.
Ceci nous amènes aux deux questions suivantes.
\begin{enumerate}	
	\item Peut-on passer de la courbe de
	      $\funcdef[h]{f}{x}{x^3}{}{}$
		  à celle du polynôme
		  $\funcdef[h]{g}{x}{a \, x^3 + b \, x^2 + c \, x + d}{}{}$
		  où $a\neq 0$
		  via une translation, une dilation verticales et/ou une dilatation horizontale.

	\item Que se passe-t-il pour les courbes des fonctions
	      $\funcdef[h]{f}{x}{x^k}{}{}$
		  pour $k \geq 4$ ?
\end{enumerate}

 