Soit $\setgeo*{C}{g}$ la courbe de la fonction
$\funcdef[h]{g}{x}{a \, x^3 + b \, x^2 + c \, x + d}{}{}$
où $a\neq 0$.
Nous allons démontrer que $\setgeo*{C}{g}$ s'obtient à partir de l'une des courbes suivantes en utilisant une translation horizontale, une translation verticale, une dilatation verticale et/ou une dilatation horizontale.

\begin{enumerate}
	\item $\Gamma_1$ représente $\funcdef[h]{f_1}{x}{x^3}{}{}$.

	\item $\Gamma_2$ représente $\funcdef[h]{f_2}{x}{x^3 - 3 x}{}{}$.

	\item $\Gamma_3$ représente $\funcdef[h]{f_3}{x}{x^3 + 3 x}{}{}$.
\end{enumerate}


\begin{proof}
	Distinguons trois cas en notant que l'on peut supposer que $a = 1$.
	
	\begin{enumerate}
		\item \textbf{$\der{g}{x}{1}(x)$ a une unique racine réelle.}

		      \smallskip

		      \noindent
		      Nous avons ici $\alpha \in \RR$ tel que $\der{g}{x}{1}(x) = 3(x - \alpha)^2$ et donc $g(x) = (x - \alpha)^3 + k$.
		      Il est immédiat que l'on peut passer de $\Gamma_1$ à $\setgeo*{C}{g}$ à l'aide des transformations autorisées.


		\medskip
		
		\item \textbf{$\der{g}{x}{1}(x)$ a deux racines réelles.}

		      \smallskip

		      \noindent
		      Nous avons ici $\alpha \neq \beta$ deux réels tels que $\der{g}{x}{1}(x) = 3 (x - \alpha) (x - \beta)$.
		      Les faits suivants montrent que l'on peut passer de $\setgeo*{C}{g}$ à $\Gamma_2$, et donc aussi de $\Gamma_2$ à $\setgeo*{C}{g}$, à l'aide des transformations autorisées.
		      \begin{enumerate}
		      		\item En posant $\delta = \dfrac{\alpha + \beta}{2}$,
						  $\der{g}{x}{1}(x + \delta) = 3 \left( x + \dfrac{\beta - \alpha}{2} \right) \left( x + \dfrac{\alpha - \beta}{2} \right)$.
				          Ceci nous fournit
				          $\der{g}{x}{1}(x + \delta) = 3 ( x - \lambda) (x + \lambda)$
				          avec $\lambda \neq 0$
				          puis ensuite
				          $\der{g}{x}{1}( \lambda \, x + \delta) = \lambda^2 \der{f_2}{x}{1}(x)$.

		      		\item En résumé,
				          $\der{f_2}{x}{1}(x) = \dfrac{1}{\lambda^2} \der{g}{x}{1}(\lambda \, x + \delta)$
				          puis par intégration
				          $f_2(x) = \dfrac{1}{\lambda^3} g(\lambda \, x + \delta) + k$.
		      \end{enumerate}


		
		\medskip
		
		\item \textbf{$\der{g}{x}{1}(x)$ n'a pas de racine réelle.}

		      \smallskip

		      \noindent
		      La forme canonique de $\der{g}{x}{1}(x)$ est ici
		      $\der{g}{x}{1}(x) = 3(x - p)^2 + m$ où les réels $p$ et $m$ sont tels que $m > 0$.
		      Les faits suivants montrent que l'on peut passer de $\setgeo*{C}{g}$ à $\Gamma_3$, et donc aussi de $\Gamma_3$ à $\setgeo*{C}{g}$, à l'aide des transformations autorisées.
		      \begin{enumerate}
		      		\item $\der{g}{x}{1}(x + p) = 3 x^2 + m$.
				          

		      		\item Notant $\mu = \sqrt{\frac{m}{3}}$, on a ensuite
				          $\der{g}{x}{1}(\mu x + p) = m x^2 + m$
				          soit
				          $\der{g}{x}{1}(\mu x + p) = \frac{m}{3} \der{f_3}{x}{1}(x)$.

		      		\item En résumé,
				          $\der{f_3}{x}{1}(x) = \frac{3}{m} \der{g}{x}{1}(\mu x + p)$
				          puis par intégration
				          $f_3(x) = \frac{3}{\mu m} g(\mu x + p) + k$.
		      \end{enumerate}
	\end{enumerate}
\end{proof}


Il est évident qu'il n'est pas possible de passer de $\Gamma_i$  à $\Gamma_j$ à l'aide des transformations autorisées.
On peut donc parler de trois types de courbe pour les polynômes de degré 3 contre un seul pour les fonctions affines et un seul pour les trinômes du 2\ieme{} degré.